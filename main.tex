\documentclass[10pt]{beamer}
\usetheme{metropolis}
\usepackage{FiraSans}
\usefonttheme{professionalfonts}

\usepackage{graphicx}
\usepackage{tikz}
\usepackage[spanish]{babel}
\usepackage{tcolorbox}
\usepackage{ragged2e}
\usepackage{pgfplots}
\pgfplotsset{compat=1.18}

\newcommand{\examplebox}[2]{
\begin{tcolorbox}[colframe=darkcardinal,colback=boxgray,title=#1]
#2
\end{tcolorbox}
}

% Personalización del pie de página
\setbeamertemplate{footline}{%
  \begin{beamercolorbox}[wd=\paperwidth,sep=2ex]{footline}%
    \usebeamerfont{structure}\textbf{I+E - Introducción} \hfill Profesor: Francisco Parra O. \hfill \textbf{Semestre 2, 2025}
  \end{beamercolorbox}%
}

% Definir cómo mostrar las secciones con bulletpoints
\setbeamertemplate{section in toc}{\leavevmode\leftskip=2em\rlap{\hskip-2em\textbullet}\inserttocsection\par}
\setbeamertemplate{subsection in toc}{\leavevmode\leftskip=4em\rlap{\hskip-4em\textbullet}\inserttocsubsection\par}

\title{Clase 01: Introducción}
\subtitle{Curso: Innovación y emprendimiento con casos reales}
\author{Profesor: Francisco Parra O.}
\institute{USACH}
\date{\today}

\titlegraphic{%
  \begin{tikzpicture}[overlay, remember picture]
    \node[anchor=north east, yshift=0cm] at (current page.north east) {
      \includegraphics[width=1.5cm]{Logo-Color-Usach-Web.jpg}
    };
  \end{tikzpicture}
}

\begin{document}

\maketitle

\begin{frame}{Contenido de la clase}
    \begin{itemize}
        \item Bienvenida y presentación del nuevo formato
        \item Contextualización del modelo de trabajo con casos reales
        \item Metodología y planificación del curso
        \item Mini-taller de preparación
        \item Próximos pasos
    \end{itemize}
\end{frame}

\begin{frame}{Evolución del curso}
  \begin{columns}
    \column{0.5\textwidth}
    \textbf{Formato anterior}
    \begin{itemize}
        \item Construcción de ideas desde cero
        \item Desarrollo teórico de emprendimientos
        \item Énfasis en cada etapa del proceso
        \item Evaluación de conceptos hipotéticos
    \end{itemize}
    
    \column{0.5\textwidth}
    \textbf{Nuevo formato}
    \begin{itemize}
        \item Resolución de problemas reales
        \item Desafíos basados en casos de industria
        \item Enfoque práctico y aplicado
        \item Soluciones implementables
    \end{itemize}
  \end{columns}
\end{frame}

\begin{frame}{Beneficios del nuevo enfoque}
    \begin{itemize}
        \item Experiencia práctica con problemas empresariales reales
        \item Desarrollo de habilidades valoradas en el mercado laboral
        \item Exploración de múltiples sectores e industrias
        \item Comprensión de restricciones y desafíos reales
        \item Flexibilidad para adaptarse a diferentes contextos
        \item Portfolio diversificado de soluciones
    \end{itemize}
\end{frame}

\begin{frame}{Contextualización: Casos de estudio}
    \begin{itemize}
        \item Los desafíos estarán basados en casos reales de empresas chilenas e internacionales
        \item Sectores diversos: tecnología, retail, servicios, manufactura, etc.
        \item Problemas actuales que enfrentan las organizaciones
        \item Adaptación de casos exitosos y análisis de fracasos
    \end{itemize}
    
    \vspace{1cm}
    \begin{center}
    \textbf{El profesor y ayudante presentarán los desafíos específicos a lo largo del semestre}
    \end{center}
\end{frame}

\begin{frame}{Modelo de trabajo académico-práctico}
    \begin{enumerate}
        \item Identificación de problemas inspirados en casos reales
        \item Análisis profundo del contexto empresarial
        \item Aplicación de metodologías de innovación
        \item Desarrollo de soluciones creativas y factibles
        \item Evaluación crítica y mejora continua
    \end{enumerate}
\end{frame}

\begin{frame}{Tipos de desafíos que abordaremos}
    \begin{description}
        \item[Optimización] Mejora de procesos existentes
        \item[Desarrollo] Nuevos productos o servicios
        \item[Expansión] Estrategias para nuevos mercados
        \item[Transformación] Digitalización y modernización
        \item[Sostenibilidad] Soluciones con impacto social/ambiental
    \end{description}
\end{frame}

\begin{frame}{Expectativas del curso}
    \begin{itemize}
        \item Compromiso con el análisis riguroso de problemas
        \item Aplicación sistemática de metodologías de innovación
        \item Trabajo colaborativo en equipos interdisciplinarios
        \item Propuestas viables y fundamentadas
        \item Comunicación profesional de ideas y soluciones
        \item Documentación completa del proceso creativo
    \end{itemize}
\end{frame}

\begin{frame}{Planificación del semestre}
    \begin{enumerate}
        \item \textbf{Semanas 1-3:} Contextualización y herramientas de análisis
        \item \textbf{Semanas 4-5:} Presentación del primer desafío
        \item \textbf{Semanas 6-8:} Investigación y definición del problema
        \item \textbf{Semanas 9-11:} Ideación y prototipado de soluciones
        \item \textbf{Semanas 12-14:} Validación y refinamiento
        \item \textbf{Semanas 15-16:} Presentación final y entrega
    \end{enumerate}
\end{frame}

\begin{frame}{Entregables principales}
    \begin{itemize}
        \item \textbf{E1:} Análisis del contexto y definición de problema (20\%)
        \item \textbf{E2:} Propuesta de solución y prototipo inicial (20\%)
        \item \textbf{E3:} Validación y resultados de pruebas (20\%)
        \item \textbf{E4:} Presentación y documentación final (30\%)
        \item \textbf{Participación:} Contribuciones en clase y actividades (10\%)
    \end{itemize}
\end{frame}

\begin{frame}{Sistema de evaluación}
    \begin{itemize}
        \item Evaluación por parte del profesor (70\%)
        \item Evaluación por parte del ayudante (15\%)
        \item Coevaluación entre pares del equipo (15\%)
    \end{itemize}
    
    \vspace{0.5cm}
    \textbf{Criterios de evaluación:}
    \begin{itemize}
        \item Comprensión del problema y contexto
        \item Aplicación de metodologías vistas en clase
        \item Originalidad y factibilidad de las soluciones
        \item Calidad de la investigación y validación
        \item Presentación y comunicación profesional
    \end{itemize}
\end{frame}

\begin{frame}{Formación de equipos}
    \begin{itemize}
        \item Equipos de 4-5 estudiantes
        \item Interdisciplinariedad recomendada
        \item Distribución de roles sugeridos:
        \begin{itemize}
            \item Coordinador/a de equipo
            \item Investigador/a y analista
            \item Desarrollador/a de conceptos
            \item Comunicador/a y visualizador/a
        \end{itemize}
        \item Proceso de formación: durante esta semana
    \end{itemize}
\end{frame}

\begin{frame}{Recursos disponibles}
    \begin{itemize}
        \item Mentoría directa por parte de profesores y ayudantes
        \item Acceso a casos de estudio y material de investigación
        \item Espacios de trabajo colaborativo
        \item Plataformas digitales para investigación y desarrollo
        \item Biblioteca de metodologías y herramientas
        \item Sesiones de retroalimentación periódicas
    \end{itemize}
\end{frame}

\begin{frame}{Mini-taller: Abordando problemas empresariales reales}
    \textbf{Actividad en grupos (20 minutos):}
    \begin{enumerate}
        \item Formen grupos de 4-5 personas
        \item Discutan las siguientes preguntas:
        \begin{itemize}
            \item ¿Cómo se aborda un problema real vs. uno teórico?
            \item ¿Qué habilidades serán cruciales para este formato?
            \item ¿Cuáles son las ventajas de trabajar con casos basados en situaciones reales?
        \end{itemize}
        \item Cada grupo compartirá brevemente sus principales conclusiones
    \end{enumerate}
\end{frame}

\begin{frame}{Estrategias para abordar problemas empresariales}
    \begin{itemize}
        \item Comprender el contexto completo del negocio
        \item Identificar a todos los stakeholders involucrados
        \item Evaluar impactos en diferentes áreas de la organización
        \item Considerar restricciones reales (tiempo, recursos, regulaciones)
        \item Buscar soluciones viables y escalables
        \item Priorizar según impacto y factibilidad
    \end{itemize}
\end{frame}

\begin{frame}{Habilidades clave para el éxito}
    \begin{columns}
        \column{0.5\textwidth}
        \textbf{Habilidades técnicas}
        \begin{itemize}
            \item Análisis de datos
            \item Metodologías de innovación
            \item Prototipado rápido
            \item Gestión de proyectos
            \item Evaluación financiera
        \end{itemize}
        
        \column{0.5\textwidth}
        \textbf{Habilidades blandas}
        \begin{itemize}
            \item Comunicación efectiva
            \item Trabajo en equipo
            \item Resolución de problemas
            \item Adaptabilidad
            \item Pensamiento crítico
        \end{itemize}
    \end{columns}
\end{frame}

\begin{frame}{Ventajas del trabajo con casos reales}
    \begin{columns}
        \column{0.5\textwidth}
        \textbf{Aprendizaje}
        \begin{itemize}
            \item Experiencia práctica aplicable
            \item Exposición a múltiples industrias
            \item Desarrollo de criterio empresarial
            \item Comprensión de complejidades reales
        \end{itemize}
        
        \column{0.5\textwidth}
        \textbf{Desarrollo profesional}
        \begin{itemize}
            \item Portfolio diversificado
            \item Habilidades transferibles
            \item Preparación para el mundo laboral
            \item Pensamiento estratégico
        \end{itemize}
    \end{columns}
\end{frame}

\begin{frame}{Próximos pasos}
    \begin{enumerate}
        \item \textbf{Para la próxima clase:}
        \begin{itemize}
            \item Revisar casos de innovación exitosos en Chile
            \item Identificar sectores de interés personal
            \item Formar equipos preliminares
        \end{itemize}
        
        \item \textbf{Preparación para los desafíos:}
        \begin{itemize}
            \item Familiarizarse con herramientas de análisis
            \item Practicar técnicas de investigación
            \item Desarrollar habilidades de trabajo en equipo
        \end{itemize}
    \end{enumerate}
\end{frame}

\begin{frame}{Reflexión final}
    \begin{center}
    \large{\textbf{"La innovación real no ocurre en el vacío, sino en la intersección entre los problemas reales y las soluciones creativas"}}
    \end{center}
    
    \vspace{1cm}
    ¿Preguntas?
\end{frame}

\end{document}
