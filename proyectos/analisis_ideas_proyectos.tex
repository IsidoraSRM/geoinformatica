\documentclass[11pt,a4paper]{article}
\usepackage[utf8]{inputenc}
\usepackage[spanish]{babel}
\usepackage{geometry}
\geometry{margin=2.5cm}
\usepackage{graphicx}
\usepackage{xcolor}
\usepackage{tcolorbox}
\usepackage{listings}
\usepackage{hyperref}
\usepackage{enumitem}
\usepackage{fancyhdr}
\usepackage{titlesec}
\usepackage{array}
\usepackage{tikz}
\usepackage{dirtytalk}
\usepackage{booktabs}
\usepackage{multicol}

% Colores personalizados
\definecolor{usachblue}{RGB}{0,121,192}
\definecolor{usachred}{RGB}{239,51,64}
\definecolor{verde}{RGB}{0,150,0}
\definecolor{amarillo}{RGB}{255,200,0}
\definecolor{rojo}{RGB}{220,0,0}

% Configuración de secciones
\titleformat{\section}[block]{\normalfont\Large\bfseries\color{usachblue}}{\thesection}{1em}{}
\titleformat{\subsection}[block]{\normalfont\large\bfseries\color{usachred}}{\thesubsection}{1em}{}

% Encabezado y pie
\pagestyle{fancy}
\fancyhf{}
\fancyhead[L]{\small Análisis de Proyectos}
\fancyhead[R]{\small Geoinformática 2025}
\fancyfoot[C]{\thepage}

% Comandos para evaluación
\newcommand{\alto}[1]{\textcolor{verde}{\textbf{#1}}}
\newcommand{\medio}[1]{\textcolor{amarillo}{\textbf{#1}}}
\newcommand{\bajo}[1]{\textcolor{rojo}{\textbf{#1}}}

\title{{\Huge \textbf{Análisis de Ideas de Proyecto}}\\[0.5cm]
{\Large Evaluación de Componente Geoespacial}\\[0.3cm]
{\large Curso: Geoinformática}}
\author{Profesor: Francisco Parra O.}
\date{Agosto 2025}

\begin{document}

\maketitle
\thispagestyle{empty}

\begin{tcolorbox}[colframe=usachblue,colback=blue!5]
\centering
\textbf{CRITERIO DE EVALUACIÓN}\\[0.3cm]
Este análisis evalúa cada proyecto según su potencial para incorporar\\
análisis geoespacial significativo, considerando que es un curso de Geoinformática.\\[0.2cm]
\alto{Alto}: Componente espacial esencial (70-100\%)\\
\medio{Medio}: Componente espacial relevante (30-70\%)\\
\bajo{Bajo}: Componente espacial marginal (<30\%)
\end{tcolorbox}

\newpage

\section{Resumen Ejecutivo}

\begin{table}[h]
\centering
\begin{tabular}{clcc}
\toprule
\textbf{\#} & \textbf{Proyecto} & \textbf{Potencial Geo} & \textbf{Recomendación} \\
\midrule
1 & Postulación a empleos & \medio{Medio} & Con ajustes \\
2 & Monitoreo de stock & \medio{Medio} & Con ajustes \\
3 & Reclutamiento & \bajo{Bajo} & No recomendado \\
4 & Comunicación interna & \bajo{Bajo} & No recomendado \\
5 & Análisis publicitario & \medio{Medio} & Con ajustes \\
6 & Entregas a domicilio & \alto{Alto} & Recomendado \\
\bottomrule
\end{tabular}
\end{table}

\section{Análisis Detallado}

\subsection{Proyecto 1: Postulación a Empleos según Destrezas}

\begin{tcolorbox}[colframe=amarillo,colback=yellow!10]
\textbf{Potencial Geoespacial:} \medio{MEDIO (40-50\%)}
\end{tcolorbox}

\subsubsection{Componente Espacial Posible}

\begin{itemize}
    \item \textbf{Matching geográfico}: Conectar candidatos con empleos cercanos
    \item \textbf{Análisis de commute}: Tiempo de viaje, rutas óptimas, transporte público
    \item \textbf{Mapas de calor}: Demanda laboral por zona y especialidad
    \item \textbf{Movilidad laboral}: Patrones de desplazamiento trabajo-hogar
    \item \textbf{Clusters de industrias}: Identificar polos de desarrollo por sector
\end{itemize}

\subsubsection{Propuesta de Mejora Geoespacial}

\say{Sistema de Matching Laboral Geointeligente para Santiago}

\begin{lstlisting}[language=Python,basicstyle=\tiny]
# Ejemplo: Score de match considerando distancia y transporte
def calcular_score_match(candidato, empleo):
    # Factor distancia
    dist_km = calcular_distancia(candidato.ubicacion, empleo.ubicacion)
    score_dist = 1 / (1 + dist_km/10)  # Decae con distancia
    
    # Factor transporte público
    tiempo_metro = calcular_tiempo_metro(candidato.ubicacion, empleo.ubicacion)
    score_transporte = 1 / (1 + tiempo_metro/30)  # Decae con tiempo
    
    # Factor habilidades
    score_skills = calcular_match_habilidades(candidato.skills, empleo.requerimientos)
    
    # Score ponderado
    score_final = (score_skills * 0.5 + 
                   score_dist * 0.3 + 
                   score_transporte * 0.2)
    
    return score_final
\end{lstlisting}

\subsubsection{Datos Geoespaciales Necesarios}
\begin{itemize}
    \item Red de transporte público (Metro, Transantiago)
    \item Ubicación de empresas por rubro
    \item Datos censales de población activa
    \item Tiempos de viaje reales (Google Maps API)
\end{itemize}

\subsubsection{Limitaciones}
\begin{itemize}
    \item El matching de habilidades no es inherentemente espacial
    \item Muchos trabajos son remotos/híbridos post-pandemia
    \item Privacidad de datos de candidatos
\end{itemize}

\newpage

\subsection{Proyecto 2: Monitoreo de Stock}

\begin{tcolorbox}[colframe=amarillo,colback=yellow!10]
\textbf{Potencial Geoespacial:} \medio{MEDIO (45-55\%)}
\end{tcolorbox}

\subsubsection{Componente Espacial Posible}

\begin{itemize}
    \item \textbf{Gestión multi-sucursal}: Optimización de inventario distribuido
    \item \textbf{Predicción espacial de demanda}: Por zona geográfica
    \item \textbf{Reabastecimiento inteligente}: Desde bodega más cercana
    \item \textbf{Análisis de competencia}: Stock según densidad de competidores
    \item \textbf{Logística de distribución}: Rutas óptimas de reposición
\end{itemize}

\subsubsection{Propuesta de Mejora Geoespacial}

\say{Sistema de Inventario Distribuido Geoptimizado}

\begin{lstlisting}[language=Python,basicstyle=\tiny]
# Ejemplo: Predicción de demanda por zona
import geopandas as gpd
from sklearn.ensemble import RandomForestRegressor

def predecir_demanda_espacial(producto, zona_gdf):
    features = []
    
    # Features espaciales
    features.append(zona_gdf['poblacion'])
    features.append(zona_gdf['ingreso_promedio'])
    features.append(zona_gdf['densidad_comercial'])
    features.append(calcular_distancia_competidor_cercano(zona_gdf))
    features.append(obtener_demanda_historica_vecinos(zona_gdf))
    
    # Features temporales
    features.append(dia_semana)
    features.append(es_quincena)
    features.append(temperatura_promedio)
    
    # Modelo predictivo
    modelo = RandomForestRegressor()
    demanda_predicha = modelo.predict(features)
    
    return demanda_predicha

def optimizar_reabastecimiento(sucursales_gdf, bodegas_gdf, demanda_predicha):
    # Problema de transporte con restricciones espaciales
    # Minimizar: costo_transporte + costo_stockout
    # Sujeto a: capacidad_bodegas, tiempos_entrega
    pass
\end{lstlisting}

\subsubsection{Valor Agregado Geoespacial}
\begin{itemize}
    \item Reducción de quiebres de stock por zona
    \item Optimización de rutas de distribución
    \item Predicción de demanda localizada
    \item Balanceo de inventario entre sucursales
\end{itemize}

\subsubsection{Limitaciones}
\begin{itemize}
    \item Requiere múltiples ubicaciones físicas
    \item El core del problema es más de optimización que espacial
    \item Necesita integración con sistemas ERP existentes
\end{itemize}

\newpage

\subsection{Proyecto 3: Automatización de Reclutamiento}

\begin{tcolorbox}[colframe=rojo,colback=red!10]
\textbf{Potencial Geoespacial:} \bajo{BAJO (10-20\%)}
\end{tcolorbox}

\subsubsection{Componente Espacial Limitado}

\begin{itemize}
    \item Filtro por ubicación de candidatos (muy básico)
    \item Programación de entrevistas considerando zonas horarias
    \item Análisis de procedencia de candidatos
\end{itemize}

\subsubsection{Por Qué No es Adecuado para Geoinformática}

\begin{enumerate}
    \item \textbf{Core no espacial}: El problema principal es procesamiento de texto y automatización
    \item \textbf{Geografía marginal}: La ubicación es solo un filtro, no análisis
    \item \textbf{Sin análisis espacial}: No requiere operaciones GIS complejas
    \item \textbf{Mejor como proyecto de NLP/ML}: Parsing de CVs, matching de skills
\end{enumerate}

\subsubsection{Alternativa Sugerida}
Si el grupo está interesado en RRHH, podrían pivotear a:
\begin{itemize}
    \item \say{Análisis espacial de talento tech en Santiago}: Mapear dónde viven los desarrolladores, diseñadores, etc.
    \item \say{Optimización de oficinas satélite}: Dónde ubicar oficinas para minimizar commute
\end{itemize}

\newpage

\subsection{Proyecto 4: Comunicación Interna Empresarial}

\begin{tcolorbox}[colframe=rojo,colback=red!10]
\textbf{Potencial Geoespacial:} \bajo{BAJO (5-15\%)}
\end{tcolorbox}

\subsubsection{Componente Espacial Muy Limitado}

\begin{itemize}
    \item Visualización de estructura organizacional por ubicación (si hay múltiples oficinas)
    \item Dashboard de métricas por región
    \item Coordinación entre oficinas en diferentes zonas horarias
\end{itemize}

\subsubsection{Por Qué No es Adecuado}

\begin{enumerate}
    \item \textbf{Problema organizacional}: No geográfico
    \item \textbf{Sin datos espaciales}: La comunicación es digital, no física
    \item \textbf{Herramientas existentes}: Slack, Teams, etc. ya resuelven esto
    \item \textbf{Fuera del scope}: Más apropiado para gestión de proyectos
\end{enumerate}

\subsubsection{Recomendación}
\textbf{NO recomendado} para este curso. El grupo debería considerar otros problemas con componente espacial real.

\newpage

\subsection{Proyecto 5: Análisis de Métricas Publicitarias}

\begin{tcolorbox}[colframe=amarillo,colback=yellow!10]
\textbf{Potencial Geoespacial:} \medio{MEDIO (35-45\%)}
\end{tcolorbox}

\subsubsection{Componente Espacial Posible}

\begin{itemize}
    \item \textbf{Geomarketing}: Análisis de efectividad por zona geográfica
    \item \textbf{Publicidad exterior}: Optimización de ubicación de vallas/pantallas
    \item \textbf{Segmentación geográfica}: Personalización de campañas por barrio
    \item \textbf{Attribution espacial}: Relacionar ventas físicas con ads digitales
    \item \textbf{Footfall analysis}: Medir tráfico en tiendas post-campaña
\end{itemize}

\subsubsection{Propuesta de Mejora Geoespacial}

\say{Sistema de Geomarketing y Attribution Espacial}

\begin{lstlisting}[language=Python,basicstyle=\tiny]
# Ejemplo: Attribution de campaña digital a ventas físicas
def analizar_impacto_espacial_campana(campana_data, ventas_data, tiendas_gdf):
    # Crear zonas de influencia por tienda
    for tienda in tiendas_gdf.iterrows():
        zona_influencia = tienda.geometry.buffer(2000)  # 2km radio
        
        # Usuarios impactados en la zona
        usuarios_zona = campana_data[
            campana_data.within(zona_influencia)
        ]
        
        # Correlacionar con ventas
        ventas_pre = ventas_data[
            (ventas_data.tienda_id == tienda.id) & 
            (ventas_data.fecha < campana.fecha_inicio)
        ].mean()
        
        ventas_post = ventas_data[
            (ventas_data.tienda_id == tienda.id) & 
            (ventas_data.fecha > campana.fecha_inicio)
        ].mean()
        
        lift = (ventas_post - ventas_pre) / ventas_pre
        
        # Modelo de attribution
        attribution_score = calcular_attribution(
            usuarios_impactados=len(usuarios_zona),
            lift_ventas=lift,
            distancia_promedio=usuarios_zona.distance(tienda.geometry).mean()
        )
    
    return attribution_scores
\end{lstlisting}

\subsubsection{Datos Necesarios}
\begin{itemize}
    \item Ubicación de usuarios (anonimizada)
    \item Puntos de venta físicos
    \item Datos de campañas digitales con geotargeting
    \item Ventas por ubicación y tiempo
\end{itemize}

\subsubsection{Limitaciones}
\begin{itemize}
    \item Privacidad de datos de usuarios
    \item Muchas métricas son puramente digitales (CTR, conversiones online)
    \item Requiere empresa con presencia física y digital
\end{itemize}

\newpage

\subsection{Proyecto 6: Optimización de Entregas a Domicilio}

\begin{tcolorbox}[colframe=verde,colback=green!10]
\textbf{Potencial Geoespacial:} \alto{ALTO (80-90\%)}
\end{tcolorbox}

\subsubsection{Componente Espacial Core}

\begin{itemize}
    \item \textbf{Ruteo optimizado}: VRP (Vehicle Routing Problem) con restricciones
    \item \textbf{Predicción de tiempos}: Basado en tráfico real-time
    \item \textbf{Zonificación dinámica}: Asignación de repartidores por zona
    \item \textbf{Last-mile optimization}: Rutas peatonales/ciclistas
    \item \textbf{Hub location}: Ubicación óptima de centros de distribución
    \item \textbf{Tracking en tiempo real}: Seguimiento GPS de entregas
\end{itemize}

\subsubsection{Propuesta Completa}

\say{Sistema Inteligente de Logística Última Milla con IA Espacial}

\begin{lstlisting}[language=Python,basicstyle=\tiny]
import osmnx as ox
import networkx as nx
from ortools.constraint_solver import pywrapcp

class SistemaEntregasGeoptimizado:
    def __init__(self, area_servicio):
        self.graph = ox.graph_from_place(area_servicio, network_type='drive')
        self.delivery_points = []
        self.vehicles = []
        
    def optimizar_rutas_dia(self, pedidos_gdf, vehiculos_disponibles):
        # 1. Clustering espacial de pedidos
        clusters = self.clusterizar_pedidos(pedidos_gdf)
        
        # 2. Asignación vehículo-cluster
        asignaciones = self.asignar_vehiculos(clusters, vehiculos_disponibles)
        
        # 3. Optimización de ruta por vehículo
        rutas = {}
        for vehiculo, pedidos in asignaciones.items():
            ruta_optima = self.resolver_vrp(vehiculo, pedidos)
            rutas[vehiculo] = ruta_optima
            
        # 4. Balanceo dinámico
        rutas_balanceadas = self.balancear_cargas(rutas)
        
        return rutas_balanceadas
    
    def predecir_tiempo_entrega(self, origen, destino, hora_salida):
        # Considerar tráfico histórico y en tiempo real
        trafico_historico = self.get_trafico_historico(hora_salida)
        trafico_actual = self.get_trafico_realtime()
        
        # Ruta más rápida con pesos dinámicos
        ruta = nx.shortest_path(self.graph, origen, destino, 
                               weight=lambda u,v,d: d['length'] * trafico_actual[u,v])
        
        tiempo_estimado = self.calcular_tiempo_ruta(ruta, trafico_actual)
        
        # Factor de incertidumbre
        buffer = self.calcular_buffer_tiempo(hora_salida, destino)
        
        return tiempo_estimado + buffer
    
    def monitorear_entregas_realtime(self):
        # Dashboard con mapa en vivo
        for vehiculo in self.vehicles:
            posicion_actual = vehiculo.get_gps_position()
            proxima_entrega = vehiculo.proxima_parada
            tiempo_restante = self.predecir_tiempo_entrega(
                posicion_actual, proxima_entrega, now()
            )
            
            # Alertar si hay retraso previsto
            if tiempo_restante > vehiculo.ventana_tiempo:
                self.notificar_cliente(proxima_entrega, nuevo_estimado)
                self.reasignar_si_necesario(vehiculo, proxima_entrega)
\end{lstlisting}

\subsubsection{Tecnologías y Datos}

\begin{itemize}
    \item \textbf{Routing}: OSRM, GraphHopper, OR-Tools
    \item \textbf{Tráfico real-time}: Google Maps API, Waze API
    \item \textbf{Optimización}: Algoritmos genéticos, Simulated Annealing
    \item \textbf{Tracking}: GPS/GNSS, WebSockets para real-time
    \item \textbf{Predicción}: ML con features espaciotemporales
\end{itemize}

\subsubsection{Métricas de Éxito}

\begin{itemize}
    \item Reducción tiempo promedio de entrega: 20-30\%
    \item Aumento entregas por ruta: 15-25\%
    \item Reducción km recorridos: 15-20\%
    \item Satisfacción cliente (entregas a tiempo): >95\%
    \item ROI: 6-12 meses
\end{itemize}

\subsubsection{Por Qué es Ideal para Geoinformática}

\begin{enumerate}
    \item \textbf{100\% espacial}: Todo el problema es geográfico
    \item \textbf{Datos reales disponibles}: OSM, APIs de tráfico
    \item \textbf{Impacto medible}: Métricas claras de mejora
    \item \textbf{Tecnologías GIS}: Routing, geocoding, spatial analysis
    \item \textbf{ML espacial}: Predicción con features geográficas
    \item \textbf{Visualización}: Mapas, dashboards, tracking
\end{enumerate}

\newpage

\section{Recomendaciones por Potencial}

\subsection{Proyectos Recomendados}

\begin{tcolorbox}[colframe=verde,colback=green!10,title=\alto{ALTO POTENCIAL}]
\textbf{Proyecto 6: Sistema de Entregas}
\begin{itemize}
    \item Problema 100\% espacial
    \item Datos disponibles y accesibles
    \item Alto impacto comercial
    \item Múltiples técnicas GIS aplicables
    \item Excelente para portfolio
\end{itemize}
\end{tcolorbox}

\subsection{Proyectos con Potencial (Requieren Ajustes)}

\begin{tcolorbox}[colframe=amarillo,colback=yellow!10,title=\medio{POTENCIAL MEDIO}]
\textbf{Proyecto 1: Matching Laboral}
\begin{itemize}
    \item Agregar análisis de movilidad urbana
    \item Incorporar factores de transporte público
    \item Mapeo de clusters industriales
\end{itemize}

\textbf{Proyecto 2: Gestión de Inventario}
\begin{itemize}
    \item Enfocar en cadena de suministro multi-local
    \item Predicción de demanda geolocalizada
    \item Optimización de rutas de reabastecimiento
\end{itemize}

\textbf{Proyecto 5: Geomarketing}
\begin{itemize}
    \item Centrarse en attribution offline-online
    \item Optimización de publicidad exterior
    \item Análisis de footfall y conversión
\end{itemize}
\end{tcolorbox}

\subsection{Proyectos No Recomendados}

\begin{tcolorbox}[colframe=rojo,colback=red!10,title=\bajo{BAJO POTENCIAL}]
\textbf{Proyecto 3: Reclutamiento}\\
\textbf{Proyecto 4: Comunicación Interna}
\begin{itemize}
    \item Componente espacial marginal o inexistente
    \item Mejor para otros cursos (IA, Sistemas de Información)
    \item No aprovechan las herramientas GIS del curso
\end{itemize}
\end{tcolorbox}

\newpage

\section{Guía para Fortalecer Componente Espacial}

\subsection{Preguntas Clave}

Para evaluar si un proyecto tiene suficiente componente espacial:

\begin{enumerate}
    \item ¿El \textbf{dónde} es fundamental para resolver el problema?
    \item ¿Se requieren operaciones espaciales (buffer, overlay, routing)?
    \item ¿Los datos principales tienen coordenadas o direcciones?
    \item ¿El análisis espacial agrega valor significativo a la solución?
    \item ¿Se pueden aplicar técnicas de GIS/Remote Sensing?
\end{enumerate}

\subsection{Cómo Agregar Componente Espacial}

\begin{table}[h]
\centering
\begin{tabular}{ll}
\toprule
\textbf{Componente Débil} & \textbf{Fortalecimiento} \\
\midrule
Filtro por ubicación & Análisis de accesibilidad multimodal \\
Lista de direcciones & Geocoding + análisis de patrones espaciales \\
Distancia simple & Routing con restricciones + tráfico \\
Mapa estático & Dashboard interactivo con análisis \\
Datos puntuales & Series espaciotemporales \\
Zonas fijas & Zonificación dinámica optimizada \\
\bottomrule
\end{tabular}
\end{table}

\subsection{Stack Tecnológico Recomendado}

\begin{multicols}{2}
\textbf{Backend:}
\begin{itemize}
    \item Python: GeoPandas, Shapely, Folium
    \item PostGIS para base de datos espacial
    \item QGIS para análisis exploratorio
    \item OR-Tools para optimización
\end{itemize}

\columnbreak

\textbf{Frontend:}
\begin{itemize}
    \item Leaflet/Mapbox para mapas web
    \item Streamlit para dashboards rápidos
    \item D3.js para visualizaciones custom
    \item WebSockets para real-time
\end{itemize}
\end{multicols}

\section{Conclusiones}

\begin{tcolorbox}[colframe=usachblue,colback=blue!5]
\textbf{Recomendación Final:}
\begin{enumerate}
    \item \textbf{Primera opción}: Proyecto 6 (Entregas) - Usar tal cual
    \item \textbf{Segunda opción}: Proyecto 1 (Empleo) - Con enfoque en movilidad urbana
    \item \textbf{Tercera opción}: Proyecto 2 (Inventario) - Como sistema multi-sucursal
    \item \textbf{Evitar}: Proyectos 3 y 4 - No tienen componente espacial significativo
\end{enumerate}

\vspace{0.3cm}
\textbf{Recordar}: Este es un curso de \textbf{Geoinformática}. El componente espacial debe ser central, no periférico. Si pueden resolver el problema sin mapas o análisis espacial, probablemente no es adecuado para este curso.
\end{tcolorbox}

\end{document}