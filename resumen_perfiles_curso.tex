\documentclass[11pt,a4paper]{article}
\usepackage[utf8]{inputenc}
\usepackage[spanish]{babel}
\usepackage{geometry}
\geometry{margin=2.5cm}
\usepackage{graphicx}
\usepackage{xcolor}
\usepackage{tcolorbox}
\usepackage{hyperref}
\usepackage{enumitem}
\usepackage{fancyhdr}
\usepackage{fontawesome5}
\usepackage{pgfplots}
\pgfplotsset{compat=1.18}
\usepackage{tikz}
\usepackage{array}
\usepackage{multicol}

% Colores personalizados
\definecolor{usachblue}{RGB}{0,121,192}
\definecolor{usachred}{RGB}{239,51,64}
\definecolor{darkgreen}{RGB}{0,100,0}
\definecolor{darkorange}{RGB}{255,140,0}

% Configuración de página
\pagestyle{fancy}
\fancyhf{}
\fancyhead[L]{\small Geoinformática 2025-2}
\fancyhead[R]{\small USACH}
\fancyfoot[C]{\thepage}

\title{{\Huge \textbf{Análisis de Perfiles del Curso}}\\[0.5cm]
{\Large Geoinformática - Semestre 2, 2025}\\[0.3cm]
{\large Resumen de Encuesta Diagnóstica}}
\author{Prof. Francisco Parra O.\\Geólogo, PhD en Informática}
\date{\today}

\begin{document}

\maketitle

\section{Resumen Ejecutivo}

\begin{tcolorbox}[colback=blue!5,colframe=usachblue,title=Datos Generales del Curso]
\begin{itemize}
    \item \textbf{Total de estudiantes:} 17
    \item \textbf{Con experiencia en programación espacial:} 8 (47\%)
    \item \textbf{Sin experiencia previa:} 1 (6\%)
    \item \textbf{Con ideas de proyecto definidas:} 10 (59\%)
    \item \textbf{Lenguaje más dominado:} Python y SQL (100\%)
\end{itemize}
\end{tcolorbox}

\section{Análisis de Experiencia}

\begin{center}
\begin{tikzpicture}
\begin{axis}[
    ybar,
    width=12cm,
    height=6cm,
    ylabel={Número de estudiantes},
    symbolic x coords={Nunca, Google Maps, PostGIS, Programado, SIG},
    xtick=data,
    nodes near coords,
    ymin=0,
    ymax=12,
    bar width=0.7cm,
    title={Nivel de Experiencia con Datos Espaciales}
]
\addplot[fill=usachblue] coordinates {
    (Nunca,1) 
    (Google Maps,8) 
    (PostGIS,2) 
    (Programado,7) 
    (SIG,2)
};
\end{axis}
\end{tikzpicture}
\end{center}

\section{Lenguajes de Programación}

\begin{center}
\begin{tikzpicture}
\begin{axis}[
    ybar,
    width=12cm,
    height=6cm,
    ylabel={Número de estudiantes},
    symbolic x coords={Python, SQL, Javascript, R},
    xtick=data,
    nodes near coords,
    ymin=0,
    ymax=18,
    bar width=0.8cm,
    title={Dominio de Lenguajes de Programación}
]
\addplot[fill=darkgreen] coordinates {
    (Python,17) 
    (SQL,17) 
    (Javascript,14) 
    (R,7)
};
\end{axis}
\end{tikzpicture}
\end{center}

\section{Áreas de Interés}

\begin{center}
\begin{tikzpicture}
\begin{axis}[
    ybar,
    width=14cm,
    height=7cm,
    ylabel={Número de estudiantes},
    symbolic x coords={Medio Ambiente, Transporte, Negocios, Urbanismo, Salud},
    xtick=data,
    x tick label style={rotate=45, anchor=east},
    nodes near coords,
    ymin=0,
    ymax=12,
    bar width=0.7cm,
    title={Áreas de Interés para Proyectos}
]
\addplot[fill=darkorange] coordinates {
    (Medio Ambiente,10) 
    (Transporte,9) 
    (Negocios,8) 
    (Urbanismo,5) 
    (Salud,2)
};
\end{axis}
\end{tikzpicture}
\end{center}

\section{Detalle de Estudiantes}

\subsection{Estudiantes con Experiencia Avanzada}
\begin{multicols}{2}
\begin{itemize}[leftmargin=*]
    \item \textbf{Branco García:} PostGIS, todos los lenguajes
    \item \textbf{Aracely Castro:} SIG + programación
    \item \textbf{Anael Guzmán:} PostGIS con rutas
    \item \textbf{Diego Hernández:} PostGIS
    \item \textbf{Lucas Contador:} SIG + programación
    \item \textbf{Felipe Baeza:} Programación espacial
    \item \textbf{Isidora Reveco:} Programación espacial
    \item \textbf{Roberto Galleguillos:} Programación espacial
\end{itemize}
\end{multicols}

\subsection{Ideas de Proyecto Destacadas}

\begin{enumerate}
    \item \textbf{Branco García:} Análisis de densidad poblacional y transporte público
    \item \textbf{Aracely Castro:} Tracking de animales en zonas rurales
    \item \textbf{Catalina López:} Sistema de reforestación inteligente
    \item \textbf{Diego Hernández:} Análisis de suelos para minería
    \item \textbf{Matías Vejar:} App de micros rurales
    \item \textbf{John Fernández:} Mapa del crimen para ubicación de comisarías
\end{enumerate}

\section{Recomendaciones para el Curso}

\begin{tcolorbox}[colback=green!5,colframe=darkgreen,title=Fortalezas del Grupo]
\begin{itemize}
    \item Dominio universal de Python y SQL facilita estandarización
    \item Alta proporción con experiencia previa en programación
    \item Interés fuerte en temas ambientales (oportunidad para proyectos de impacto)
    \item Varios estudiantes con ideas concretas de proyecto
\end{itemize}
\end{tcolorbox}

\begin{tcolorbox}[colback=yellow!5,colframe=darkorange,title=Áreas de Atención]
\begin{itemize}
    \item Solo 41\% conoce R - considerar tutoriales adicionales
    \item Un estudiante sin experiencia previa - necesita apoyo extra
    \item 41\% sin idea de proyecto - requieren más orientación
    \item Poca experiencia con SIG desktop - enfocarse en programación
\end{itemize}
\end{tcolorbox}

\section{Propuesta de Grupos de Trabajo}

Basándome en los perfiles, sugiero formar grupos equilibrados:

\subsection{Grupos Sugeridos (por afinidad y complementariedad)}

\begin{enumerate}
    \item \textbf{Grupo Transporte Urbano:}
    \begin{itemize}
        \item Branco García (líder técnico)
        \item Matías Vejar
        \item Isidora Reveco
    \end{itemize}
    
    \item \textbf{Grupo Medio Ambiente:}
    \begin{itemize}
        \item Aracely Castro (líder técnico)
        \item Catalina López
        \item Valentina Barría
    \end{itemize}
    
    \item \textbf{Grupo Business Intelligence:}
    \begin{itemize}
        \item Felipe Baeza (líder técnico)
        \item Fabián Ibarra
        \item Diego Hernández
    \end{itemize}
    
    \item \textbf{Grupo Smart Cities:}
    \begin{itemize}
        \item Lucas Contador (líder técnico)
        \item Byron Gracia
        \item John Fernández
    \end{itemize}
    
    \item \textbf{Grupo Análisis Multimodal:}
    \begin{itemize}
        \item Anael Guzmán (líder técnico)
        \item Roberto Galleguillos
        \item Bastián Guerrero
    \end{itemize}
    
    \item \textbf{Grupo Salud:}
    \begin{itemize}
        \item Valentina Campos
        \item Jaime Riquelme
    \end{itemize}
\end{enumerate}


\section{Recursos Adicionales Recomendados}

\subsection{Para el grupo mayoritario (Python + SQL)}
\begin{itemize}
    \item Crear repositorio con notebooks de ejemplo
    \item Datasets chilenos pre-procesados
    \item Template de proyecto en GeoPandas
\end{itemize}

\subsection{Para interesados en medio ambiente}
\begin{itemize}
    \item Acceso a Google Earth Engine
    \item Datos de estaciones SINCA
    \item Imágenes Sentinel-2 de Chile
\end{itemize}

\subsection{Para principiantes}
\begin{itemize}
    \item Tutorías con ayudantes
    \item Material de autoaprendizaje gradual
    \item Pair programming con compañeros avanzados
\end{itemize}

\end{document}