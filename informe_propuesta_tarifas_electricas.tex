\documentclass[12pt,a4paper]{article}
\usepackage[utf8]{inputenc}
\usepackage[spanish]{babel}
\usepackage{geometry}
\usepackage{graphicx}
\usepackage{xcolor}
\usepackage{listings}
\usepackage{float}
\usepackage{amsmath}
\usepackage{amssymb}
\usepackage{hyperref}
\usepackage{tikz}
\usepackage{pgfplots}
\usepackage{booktabs}
\usepackage{longtable}
\usepackage{array}
\usepackage{multirow}
\usepackage{enumitem}
\usepackage{tcolorbox}

\geometry{left=2.5cm, right=2.5cm, top=2.5cm, bottom=2.5cm}

% Colores institucionales USACH
\definecolor{usachblue}{RGB}{0,54,99}
\definecolor{usachorange}{RGB}{255,102,0}
\definecolor{darkgreen}{RGB}{0,128,0}
\definecolor{codebg}{RGB}{245,245,245}

% Configuración de listings para código
\lstset{
    backgroundcolor=\color{codebg},
    basicstyle=\footnotesize\ttfamily,
    breaklines=true,
    captionpos=b,
    commentstyle=\color{darkgreen},
    keywordstyle=\color{usachblue}\bfseries,
    stringstyle=\color{usachorange},
    numbers=left,
    numberstyle=\tiny\color{gray},
    stepnumber=1,
    numbersep=10pt,
    showspaces=false,
    showstringspaces=false,
    frame=single,
    rulecolor=\color{gray},
    xleftmargin=1cm,
    xrightmargin=1cm
}

% Configuración de hyperref
\hypersetup{
    colorlinks=true,
    linkcolor=usachblue,
    urlcolor=usachorange,
    citecolor=darkgreen
}

\title{
    \Large{\textbf{INFORME DE EVALUACIÓN DE PROYECTO}}\\
    \vspace{0.5cm}
    \huge{\textbf{Sistema de Predicción Geoespacial de\\Tarifas Eléctricas en Chile}}\\
    \vspace{0.5cm}
    \large{Curso de Geoinformática - 2° Semestre 2025}\\
    \large{Universidad de Santiago de Chile}
}

\author{
    \textbf{Profesor:} Dr. Francisco Parra O.\\
    \textbf{Tipo de Proyecto:} Comercial
}

\date{Agosto 2025}

\begin{document}

\maketitle
\newpage

\tableofcontents
\newpage

\section{Resumen Ejecutivo}

El proyecto propuesto representa una \textbf{excelente integración} de tecnologías geoespaciales avanzadas para resolver un problema crítico del mercado energético chileno. La propuesta combina:

\begin{itemize}[leftmargin=*]
    \item \textbf{Google Earth Engine} para obtención de variables climáticas
    \item \textbf{Machine Learning} con algoritmos de series temporales
    \item \textbf{Análisis espacial} del sistema eléctrico nacional
    \item \textbf{Visualización interactiva} de predicciones georreferenciadas
\end{itemize}

\subsection{Evaluación General}
\begin{center}
\begin{tabular}{|l|c|l|}
\hline
\textbf{Criterio} & \textbf{Puntuación} & \textbf{Observación} \\
\hline
Relevancia técnica & 9.5/10 & Uso avanzado de GEE y ML espacial \\
Viabilidad & 9.0/10 & Datos disponibles públicamente \\
Innovación & 8.5/10 & Enfoque novedoso para mercado chileno \\
Impacto comercial & 9.0/10 & Alto valor para múltiples stakeholders \\
Complejidad apropiada & 9.5/10 & Desafiante para 8° semestre \\
\hline
\textbf{TOTAL} & \textbf{91/100} & \textbf{PROYECTO SOBRESALIENTE} \\
\hline
\end{tabular}
\end{center}

\section{Análisis Técnico Detallado}

\subsection{Arquitectura del Sistema}

El proyecto propone una arquitectura de tres capas bien definida:

\subsubsection{Capa de Datos}
\begin{itemize}
    \item \textbf{Datos Satelitales (Google Earth Engine)}
    \begin{itemize}
        \item MODIS para temperatura superficial (LST)
        \item ERA5 para radiación solar
        \item Sentinel-2 para índices espectrales (NDWI, NDSI)
    \end{itemize}
    
    \item \textbf{Datos del Mercado Eléctrico}
    \begin{itemize}
        \item Costos marginales históricos (Coordinador Eléctrico Nacional)
        \item Precios de combustibles fósiles
        \item Capacidad instalada por tecnología
    \end{itemize}
\end{itemize}

\subsubsection{Capa de Procesamiento}
\begin{lstlisting}[language=Python, caption=Pipeline de procesamiento propuesto]
# 1. Extraccion de variables GEE
import ee
ee.Initialize()

def extract_climate_variables(roi, date_range):
    # Temperatura superficial
    lst = ee.ImageCollection('MODIS/006/MOD11A1')\
        .filterDate(date_range[0], date_range[1])\
        .filterBounds(roi)\
        .select('LST_Day_1km')\
        .mean()
    
    # Radiacion solar
    solar = ee.ImageCollection('ECMWF/ERA5/DAILY')\
        .filterDate(date_range[0], date_range[1])\
        .select('surface_solar_radiation_downwards')\
        .mean()
    
    # Indices de sequia
    ndwi = calculate_ndwi(roi, date_range)
    
    return ee.Image.cat([lst, solar, ndwi])

# 2. Modelo predictivo
from sklearn.ensemble import RandomForestRegressor
import xgboost as xgb

def train_price_model(features, targets):
    # Features: variables climaticas + historicos
    # Targets: costos marginales por zona
    
    model = xgb.XGBRegressor(
        n_estimators=500,
        max_depth=8,
        learning_rate=0.01,
        objective='reg:squarederror'
    )
    
    # Incluir lags temporales
    features_with_lags = add_temporal_lags(features, [1, 7, 30, 365])
    
    # Validacion cruzada espacial
    cv_scores = spatial_cross_validation(model, features_with_lags, targets)
    
    return model.fit(features_with_lags, targets)
\end{lstlisting}

\subsubsection{Capa de Visualización}
\begin{itemize}
    \item Dashboard interactivo con Streamlit/Dash
    \item Mapas de calor con predicciones usando Folium
    \item API REST para integración con sistemas externos
\end{itemize}

\subsection{Variables Clave y Justificación}

\begin{longtable}{|p{3.5cm}|p{4cm}|p{5cm}|p{3cm}|}
\hline
\textbf{Variable} & \textbf{Fuente} & \textbf{Justificación} & \textbf{Resolución} \\
\hline
\endfirsthead
\multicolumn{4}{c}{\textit{Continuación de la tabla anterior}} \\
\hline
\textbf{Variable} & \textbf{Fuente} & \textbf{Justificación} & \textbf{Resolución} \\
\hline
\endhead
\hline
\multicolumn{4}{r}{\textit{Continúa en la siguiente página}} \\
\endfoot
\hline
\endlastfoot

NDWI (Índice de Agua) & Sentinel-2 & Nivel de embalses afecta generación hidroeléctrica (40\% matriz Chile) & 10m, quincenal \\
\hline
NDSI (Índice de Nieve) & MODIS/Sentinel-2 & Predictor de deshielo y disponibilidad hídrica futura & 500m, diaria \\
\hline
LST (Temperatura) & MODIS & Correlación con demanda por climatización & 1km, diaria \\
\hline
Radiación Solar & ERA5 & Predictor directo de generación solar (20\% matriz norte) & 30km, horaria \\
\hline
Velocidad Viento & ERA5 & Generación eólica en centro-sur & 30km, horaria \\
\hline
Precipitación & GPM/CHIRPS & Afluentes futuros para hidroeléctricas & 10km, diaria \\
\hline
CMg Histórico & CEN & Variable objetivo y feature lag & Por barra, horaria \\
\hline
Precio Combustibles & CNE & Costo generación térmica de respaldo & Nacional, semanal \\
\hline
\end{longtable}

\subsection{Metodología de Machine Learning}

\subsubsection{Preprocesamiento Espacial}
\begin{enumerate}
    \item \textbf{Zonificación del territorio:} División en 5 macrozonas eléctricas
    \item \textbf{Agregación temporal:} Promedios mensuales para predicción estratégica
    \item \textbf{Normalización:} StandardScaler para variables climáticas
    \item \textbf{Feature engineering:}
    \begin{itemize}
        \item Ratios inter-zonales de variables
        \item Anomalías respecto a promedios históricos
        \item Índices compuestos (ej: stress hídrico = f(NDWI, precipitación, temperatura))
    \end{itemize}
\end{enumerate}

\subsubsection{Modelo Predictivo}
\begin{lstlisting}[language=Python, caption=Implementación del modelo XGBoost]
import xgboost as xgb
from sklearn.model_selection import TimeSeriesSplit
from sklearn.metrics import mean_absolute_percentage_error
import geopandas as gpd

class SpatialElectricityPricePredictor:
    def __init__(self, zones_shapefile):
        self.zones = gpd.read_file(zones_shapefile)
        self.models = {}  # Un modelo por zona
        
    def train(self, climate_data, market_data):
        for zone_id in self.zones['id']:
            # Filtrar datos por zona
            zone_features = self.extract_zone_features(
                climate_data, zone_id
            )
            
            # Configurar modelo con hiperparametros optimizados
            self.models[zone_id] = xgb.XGBRegressor(
                n_estimators=1000,
                max_depth=10,
                learning_rate=0.01,
                subsample=0.8,
                colsample_bytree=0.8,
                gamma=0.1,
                reg_alpha=0.1,
                reg_lambda=1.0,
                objective='reg:squarederror',
                eval_metric='mape'
            )
            
            # Validacion temporal
            tscv = TimeSeriesSplit(n_splits=5)
            cv_scores = []
            
            for train_idx, val_idx in tscv.split(zone_features):
                X_train = zone_features.iloc[train_idx]
                y_train = market_data['cmg'].iloc[train_idx]
                X_val = zone_features.iloc[val_idx]
                y_val = market_data['cmg'].iloc[val_idx]
                
                self.models[zone_id].fit(
                    X_train, y_train,
                    eval_set=[(X_val, y_val)],
                    early_stopping_rounds=50,
                    verbose=False
                )
                
                predictions = self.models[zone_id].predict(X_val)
                mape = mean_absolute_percentage_error(y_val, predictions)
                cv_scores.append(mape)
            
            print(f"Zona {zone_id} - MAPE promedio: {np.mean(cv_scores):.2%}")
    
    def predict_spatial(self, future_climate):
        predictions = {}
        for zone_id in self.zones['id']:
            zone_features = self.extract_zone_features(
                future_climate, zone_id
            )
            predictions[zone_id] = self.models[zone_id].predict(zone_features)
        
        return self.create_prediction_map(predictions)
\end{lstlisting}

\section{Evaluación de Complejidad y Alcance}

\subsection{Aspectos Destacables}

\begin{enumerate}
    \item \textbf{Integración Multidisciplinaria}
    \begin{itemize}
        \item Teledetección satelital
        \item Ciencia de datos espaciales
        \item Economía energética
        \item Meteorología aplicada
    \end{itemize}
    
    \item \textbf{Desafíos Técnicos Apropiados}
    \begin{itemize}
        \item Manejo de big data satelital en GEE
        \item Sincronización de múltiples fuentes temporales
        \item Modelado de dependencias espaciales
        \item Optimización de hiperparámetros en ML
    \end{itemize}
    
    \item \textbf{Escalabilidad}
    \begin{itemize}
        \item Arquitectura cloud-ready
        \item Procesamiento paralelo por zonas
        \item API para integración empresarial
    \end{itemize}
\end{enumerate}

\subsection{Matriz de Riesgos y Mitigación}

\begin{center}
\begin{tabular}{|p{3cm}|c|c|p{5cm}|}
\hline
\textbf{Riesgo} & \textbf{Probabilidad} & \textbf{Impacto} & \textbf{Mitigación} \\
\hline
Calidad datos GEE & Baja & Alto & Validación con estaciones meteorológicas in-situ \\
\hline
Overfitting modelo & Media & Medio & Cross-validation espacial y temporal rigurosa \\
\hline
Cambios regulatorios & Baja & Medio & Diseño modular para adaptar nuevas variables \\
\hline
Latencia predicciones & Media & Bajo & Cache de resultados y procesamiento asíncrono \\
\hline
\end{tabular}
\end{center}

\section{Plan de Implementación Recomendado}

\subsection{Fase 1: Preparación de Datos (3 semanas)}
\begin{itemize}
    \item Configurar cuenta Google Earth Engine
    \item Desarrollar scripts de extracción de variables climáticas
    \item Obtener y limpiar datos históricos del CEN
    \item Crear base de datos espacial con PostGIS
\end{itemize}

\subsection{Fase 2: Desarrollo del Modelo (4 semanas)}
\begin{itemize}
    \item Análisis exploratorio de correlaciones
    \item Feature engineering espacial y temporal
    \item Entrenamiento y optimización de modelos
    \item Validación con métricas espaciales (Moran's I)
\end{itemize}

\subsection{Fase 3: Sistema de Visualización (3 semanas)}
\begin{itemize}
    \item Dashboard interactivo con Streamlit
    \item Mapas de predicción con Folium
    \item API REST con FastAPI
    \item Documentación técnica
\end{itemize}

\subsection{Fase 4: Validación y Ajustes (2 semanas)}
\begin{itemize}
    \item Backtesting con datos históricos
    \item Análisis de sensibilidad
    \item Optimización de performance
    \item Presentación final
\end{itemize}

\section{Mejoras y Extensiones Sugeridas}

\subsection{Mejoras Técnicas Inmediatas}

\begin{enumerate}
    \item \textbf{Incorporar Autocorrelación Espacial}
    \begin{lstlisting}[language=Python]
from pysal.lib import weights
from pysal.model import spreg

# Crear matriz de pesos espaciales
w = weights.Queen.from_dataframe(zones_gdf)

# Modelo con lag espacial
model = spreg.ML_Lag(
    y, X, w,
    name_y='cmg',
    name_x=feature_names
)
    \end{lstlisting}
    
    \item \textbf{Análisis de Incertidumbre}
    \begin{itemize}
        \item Intervalos de confianza con bootstrap
        \item Propagación de error desde inputs satelitales
        \item Escenarios probabilísticos
    \end{itemize}
    
    \item \textbf{Deep Learning Espacial}
    \begin{lstlisting}[language=Python]
import torch
import torch_geometric

class SpatialGNN(torch.nn.Module):
    """Graph Neural Network para prediccion
    considerando conectividad de la red electrica"""
    
    def __init__(self, num_features, hidden_dim):
        super().__init__()
        self.conv1 = GCNConv(num_features, hidden_dim)
        self.conv2 = GCNConv(hidden_dim, hidden_dim)
        self.fc = torch.nn.Linear(hidden_dim, 1)
        
    def forward(self, x, edge_index):
        x = self.conv1(x, edge_index).relu()
        x = self.conv2(x, edge_index).relu()
        return self.fc(x)
    \end{lstlisting}
\end{enumerate}

\subsection{Extensiones Comerciales}

\begin{enumerate}
    \item \textbf{Módulo de Optimización para Generadoras}
    \begin{itemize}
        \item Recomendaciones de despacho óptimo
        \item Análisis what-if para mantenimientos
        \item Alertas de oportunidades de arbitraje
    \end{itemize}
    
    \item \textbf{Integración con Trading Energético}
    \begin{itemize}
        \item Señales de compra/venta en mercado spot
        \item Valorización de contratos PPA
        \item Gestión de riesgo de portafolio
    \end{itemize}
    
    \item \textbf{Servicios para Consumidores Industriales}
    \begin{itemize}
        \item Optimización de consumo por horario
        \item Evaluación de autogeneración
        \item Negociación de contratos
    \end{itemize}
\end{enumerate}

\section{Impacto Esperado}

\subsection{Beneficiarios Directos}

\begin{itemize}
    \item \textbf{Generadoras eléctricas:} Optimización de estrategias de oferta
    \item \textbf{Comercializadoras:} Mejor gestión de riesgo de precio
    \item \textbf{Grandes consumidores:} Planificación de consumo y costos
    \item \textbf{Reguladores:} Monitoreo de comportamiento del mercado
    \item \textbf{Inversionistas:} Evaluación de proyectos renovables
\end{itemize}

\subsection{Métricas de Éxito}

\begin{center}
\begin{tabular}{|l|c|c|}
\hline
\textbf{Métrica} & \textbf{Objetivo} & \textbf{Método de Medición} \\
\hline
MAPE predicción & < 15\% & Backtesting 2023-2024 \\
Cobertura espacial & 95\% territorio & Zonas con predicción \\
Latencia API & < 500ms & Monitoring en producción \\
Usuarios activos & > 50/mes & Google Analytics \\
ROI para clientes & > 20\% & Encuesta post-implementación \\
\hline
\end{tabular}
\end{center}

\section{Conclusiones y Recomendaciones}

\subsection{Fortalezas del Proyecto}

\begin{enumerate}
    \item \textbf{Alta relevancia:} Aborda problema crítico del sector energético chileno
    \item \textbf{Innovación técnica:} Integración novedosa de GEE + ML + mercado eléctrico
    \item \textbf{Viabilidad demostrada:} Todos los datos necesarios son accesibles
    \item \textbf{Escalabilidad comercial:} Múltiples modelos de negocio viables
    \item \textbf{Aprendizaje integral:} Cubre todos los objetivos del curso de Geoinformática
\end{enumerate}

\subsection{Recomendaciones Finales}

\begin{tcolorbox}[colback=blue!5!white,colframe=usachblue,title=\textbf{RECOMENDACIÓN DEL PROFESOR}]
\textbf{El proyecto debe ser APROBADO y recibir máximo apoyo.}

Representa exactamente el tipo de aplicación innovadora de geoinformática que el curso busca promover. La combinación de:
\begin{itemize}
    \item Procesamiento masivo de datos satelitales
    \item Machine learning espacial avanzado
    \item Problema de alto impacto económico
    \item Solución técnicamente sofisticada
\end{itemize}

Lo convierte en un proyecto ejemplar que podría:
\begin{enumerate}
    \item Publicarse en conferencias de energía y geoinformática
    \item Convertirse en startup o spin-off universitario
    \item Servir como caso de estudio para futuras generaciones
\end{enumerate}

\textbf{Calificación sugerida: 7.0} (con potencial de nota máxima según ejecución)
\end{tcolorbox}

\subsection{Siguientes Pasos}

\begin{enumerate}
    \item \textbf{Inmediato:} Reunión con el estudiante para afinar alcance
    \item \textbf{Semana 1:} Configurar accesos a GEE y datos del CEN
    \item \textbf{Semana 2:} Definir arquitectura técnica detallada
    \item \textbf{Semana 3:} Iniciar desarrollo con prototipo mínimo viable
    \item \textbf{Mensual:} Revisiones de avance con feedback técnico
\end{enumerate}

\section{Referencias Técnicas Sugeridas}

\begin{enumerate}
    \item Gorelick, N. et al. (2017). Google Earth Engine: Planetary-scale geospatial analysis for everyone. \textit{Remote Sensing of Environment}.
    
    \item Pérez-Arriaga, I. (2013). Regulation of the Power Sector. Springer.
    
    \item Hong, T. et al. (2016). Probabilistic electric load forecasting: A tutorial review. \textit{International Journal of Forecasting}.
    
    \item Coordinador Eléctrico Nacional (2024). Informe de Operación del Sistema Eléctrico Nacional.
    
    \item Zhang, Y. et al. (2023). Deep learning for renewable energy forecasting: A taxonomy, and systematic literature review. \textit{Journal of Cleaner Production}.
    
    \item Chilean Energy Ministry (2024). Política Energética Nacional 2050.
\end{enumerate}

\vspace{2cm}

\begin{flushright}
\textbf{Dr. Francisco Parra O.}\\
Profesor Curso Geoinformática\\
Universidad de Santiago de Chile\\
\texttt{francisco.parra.o@usach.cl}\\
\vspace{0.5cm}
Agosto 2025
\end{flushright}

\end{document}