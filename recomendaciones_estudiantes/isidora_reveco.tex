\documentclass[11pt,a4paper]{article}
\usepackage[utf8]{inputenc}
\usepackage[spanish]{babel}
\usepackage{geometry}
\geometry{margin=2.5cm}
\usepackage{graphicx}
\usepackage{xcolor}
\usepackage{tcolorbox}
\usepackage{hyperref}
\usepackage{enumitem}
\usepackage{fancyhdr}
\usepackage{fontawesome5}
\usepackage{multicol}

% Colores personalizados
\definecolor{usachblue}{RGB}{0,121,192}
\definecolor{usachred}{RGB}{239,51,64}
\definecolor{darkgreen}{RGB}{0,100,0}
\definecolor{darkorange}{RGB}{255,140,0}

% Configuración de página
\pagestyle{fancy}
\fancyhf{}
\fancyhead[L]{\small Geoinformática 2025-2}
\fancyhead[R]{\small USACH}
\fancyfoot[C]{\thepage}

\begin{document}

\begin{center}
    {\Huge \textbf{Recomendaciones de Proyecto}}\\[0.5cm]
    {\Large \textcolor{usachblue}{Geoinformática - Semestre 2, 2025}}\\[0.3cm]
    \rule{\textwidth}{0.5pt}\\[0.3cm]
    {\LARGE \textbf{Isidora Reveco}}\\[0.2cm]
    {\large Fecha: \today}
\end{center}

\vspace{0.5cm}

\section*{\faIcon{user-circle} Perfil del Estudiante}

\begin{tcolorbox}[colback=blue!5,colframe=usachblue,title=Resumen de tu Perfil]
\begin{itemize}[leftmargin=*]
    \item \textbf{Nivel de experiencia:} Avanzado
    \item \textbf{Lenguajes dominados:} Python, Javascript, SQL
    \item \textbf{Áreas de interés:} Negocios, urbanismo, medioambiente, transporte
    \item \textbf{Stack tecnológico recomendado:} GeoPandas, Folium, Rasterio + Leaflet, Mapbox, Turf.js + PostGIS
\end{itemize}
\end{tcolorbox}


\section*{\faIcon{lightbulb} Tu Idea de Proyecto}

\begin{tcolorbox}[colback=yellow!10,colframe=darkorange,title=Idea Original]
\textit{"Algo relacionado al tema transporte que sea comercial"}

\vspace{0.3cm}
\textbf{Comentario del profesor:} Esta es una excelente idea que puedes desarrollar. Te sugiero considerar los siguientes aspectos técnicos:
\begin{itemize}
    \item Fuentes de datos disponibles (INE, municipalidades, APIs públicas)
    \item Metodología de análisis espacial apropiada
    \item Herramientas específicas de tu stack tecnológico
\end{itemize}
\end{tcolorbox}

\section*{\faIcon{flask} Proyectos Científicos Recomendados}

Basándome en tus intereses y experiencia, estos proyectos científicos serían ideales para ti:


\begin{tcolorbox}[colback=green!5,colframe=darkgreen,title={\small Proyecto Científico \#1}]
\textbf{Detección de Asentamientos Informales}\\[0.2cm]
\textcolor{gray}{\small Área: Urbanismo}\\[0.2cm]
Usar deep learning en imágenes satelitales para identificar campamentos
\end{tcolorbox}


\begin{tcolorbox}[colback=green!5,colframe=darkgreen,title={\small Proyecto Científico \#2}]
\textbf{Análisis de Gentrificación}\\[0.2cm]
\textcolor{gray}{\small Área: Urbanismo}\\[0.2cm]
Detectar cambios socioeconómicos usando datos de arriendo y censo
\end{tcolorbox}


\begin{tcolorbox}[colback=green!5,colframe=darkgreen,title={\small Proyecto Científico \#3}]
\textbf{Planificación de Áreas Verdes}\\[0.2cm]
\textcolor{gray}{\small Área: Urbanismo}\\[0.2cm]
Optimizar ubicación de parques según déficit y densidad poblacional
\end{tcolorbox}


\begin{tcolorbox}[colback=green!5,colframe=darkgreen,title={\small Proyecto Científico \#4}]
\textbf{Análisis de Islas de Calor Urbanas en Santiago}\\[0.2cm]
\textcolor{gray}{\small Área: Medioambiente}\\[0.2cm]
Identificar zonas con mayor temperatura usando imágenes satelitales Landsat y correlacionar con cobertura vegetal
\end{tcolorbox}


\section*{\faIcon{building} Proyectos Comerciales Recomendados}

Si te interesa un enfoque más aplicado a la industria, considera estos proyectos:


\begin{tcolorbox}[colback=orange!5,colframe=darkorange,title={\small Proyecto Comercial \#1}]
\textbf{Geomarketing para Retail}\\[0.2cm]
\textcolor{gray}{\small Área: Negocios}\\[0.2cm]
Identificar ubicaciones óptimas para nuevas tiendas usando análisis de competencia y demografía
\end{tcolorbox}


\begin{tcolorbox}[colback=orange!5,colframe=darkorange,title={\small Proyecto Comercial \#2}]
\textbf{Segmentación Espacial de Clientes}\\[0.2cm]
\textcolor{gray}{\small Área: Negocios}\\[0.2cm]
Crear perfiles de clientes por zona geográfica para marketing dirigido
\end{tcolorbox}


\begin{tcolorbox}[colback=orange!5,colframe=darkorange,title={\small Proyecto Comercial \#3}]
\textbf{Análisis de Competencia}\\[0.2cm]
\textcolor{gray}{\small Área: Negocios}\\[0.2cm]
Mapear competidores y analizar áreas de influencia usando polígonos de Voronoi
\end{tcolorbox}


\begin{tcolorbox}[colback=orange!5,colframe=darkorange,title={\small Proyecto Comercial \#4}]
\textbf{Agricultura de Precisión}\\[0.2cm]
\textcolor{gray}{\small Área: Agricultura}\\[0.2cm]
Optimizar uso de recursos usando índices de vegetación y datos de suelo
\end{tcolorbox}


\section*{\faIcon{graduation-cap} Recursos Recomendados}

\subsection*{Recursos según tu nivel}

\begin{itemize}[leftmargin=*]
    \item Libro: Geocomputation with Python/R
    \item Paper: Recent advances in GeoAI
    \item Google Earth Engine tutorials
    \item Kaggle competitions geoespaciales
\end{itemize}


\vspace{0.5cm}

\begin{tcolorbox}[colback=gray!10,colframe=gray!50]
\centering
\textbf{Contacto}\\[0.2cm]
\faIcon{envelope} francisco.parra.o@usach.cl
\end{tcolorbox}

\end{document}