\documentclass[10pt]{beamer}
\usetheme{metropolis}
\usepackage{FiraSans}
\usefonttheme{professionalfonts}

\usepackage{graphicx}
\usepackage{tikz}
\usepackage[spanish]{babel}
\usepackage{tcolorbox}
\usepackage{ragged2e}
\usepackage{pgfplots}
\pgfplotsset{compat=1.18}
\usepackage{listings}
\usepackage{xcolor}
\usepackage{hyperref}
\usepackage{fontawesome5}
\usepackage{multicol}
\usepackage{array}
\usetikzlibrary{shapes.geometric, arrows, positioning, calc, mindmap}

\definecolor{codegreen}{rgb}{0,0.6,0}
\definecolor{codegray}{rgb}{0.5,0.5,0.5}
\definecolor{codepurple}{rgb}{0.58,0,0.82}
\definecolor{backcolour}{rgb}{0.95,0.95,0.92}
\definecolor{usachblue}{RGB}{0,121,192}
\definecolor{usachred}{RGB}{239,51,64}
\definecolor{pythonblue}{RGB}{55,118,171}
\definecolor{rblue}{RGB}{33,102,172}

\lstdefinestyle{mystyle}{
    backgroundcolor=\color{backcolour},   
    commentstyle=\color{codegreen},
    keywordstyle=\color{magenta},
    numberstyle=\tiny\color{codegray},
    stringstyle=\color{codepurple},
    basicstyle=\ttfamily\footnotesize,
    breakatwhitespace=false,         
    breaklines=true,                 
    captionpos=b,                    
    keepspaces=true,                 
    numbers=left,                    
    numbersep=5pt,                  
    showspaces=false,                
    showstringspaces=false,
    showtabs=false,                  
    tabsize=2
}

\lstset{style=mystyle}

\newcommand{\examplebox}[2]{
\begin{tcolorbox}[colframe=usachblue,colback=white,title=#1]
#2
\end{tcolorbox}
}

% Personalización del pie de página
\setbeamertemplate{footline}{%
  \begin{beamercolorbox}[wd=\paperwidth,sep=2ex]{footline}%
    \usebeamerfont{structure}\textbf{Geoinformática - Clase 2} \hfill Profesor: Francisco Parra O. \hfill \textbf{Semestre 2, 2025}
  \end{beamercolorbox}%
}

% Configuración de bullets
\setbeamertemplate{itemize items}[circle]
\setbeamertemplate{enumerate items}[default]

\title{Clase 02: Fundamentos de Geocomputación}
\subtitle{Historia, herramientas y ecosistema de desarrollo}
\author{Prof. Francisco Parra O. \\ \small{Geólogo, PhD en Informática}}
\institute{USACH - Ingeniería Civil en Informática}
\date{\today}

\titlegraphic{%
  \begin{tikzpicture}[overlay, remember picture]
    \node[anchor=north east, yshift=0cm] at (current page.north east) {
      \includegraphics[width=1.5cm]{../../logo.jpg}
    };
  \end{tikzpicture}
}

\begin{document}

\maketitle

% SLIDE 2 - Agenda (2 min)
\begin{frame}{Agenda de hoy}
    \tableofcontents
    
    \vspace{0.5cm}
    \begin{center}
        \textcolor{usachblue}{\faIcon{clock}} \textbf{Duración: 80 minutos}
    \end{center}
\end{frame}

% SLIDE 3 - Repaso clase anterior (3 min)
\begin{frame}{Repaso clase anterior}
    \begin{columns}
        \column{0.5\textwidth}
        \textbf{Conceptos clave:}
        \begin{itemize}
            \item Definición de Geocomputación
            \item Tipos de datos espaciales
            \item Aplicaciones en Chile
            \item Proyecto semestral
        \end{itemize}
        
        \column{0.5\textwidth}
        \textbf{Recordatorio:}
        \begin{itemize}
            \item \textcolor{usachred}{\faIcon{exclamation-triangle}} Instalar software antes del lab
            \item \textcolor{usachblue}{\faIcon{users}} Pensar en grupos de proyecto
            \item \textcolor{green}{\faIcon{check}} Completar encuesta diagnóstica
        \end{itemize}
    \end{columns}
    
    \vspace{0.5cm}
    \begin{center}
        \textit{Hoy profundizaremos en los fundamentos y herramientas}
    \end{center}
\end{frame}

% SECCIÓN 1: HISTORIA Y EVOLUCIÓN (20 minutos)
\section{Historia y evolución de la Geocomputación}

% SLIDE 4 - Precursores (3 min)
\begin{frame}{Los precursores: Cartografía y computación}
    \begin{columns}
        \column{0.5\textwidth}
        \textbf{Antes de los SIG (pre-1960):}
        \begin{itemize}
            \item Mapas en papel y overlay manual
            \item Primeros intentos de cuantificación espacial
            \item John Snow (1854): Mapa del cólera en Londres
            \item Von Thünen (1826): Modelo de uso del suelo
        \end{itemize}
        
        \column{0.5\textwidth}
        \begin{center}
            \begin{tikzpicture}[scale=0.8]
                % Representación del mapa de John Snow
                \draw[thick] (0,0) rectangle (4,4);
                \foreach \x in {0.5,1.5,2.5,3.5} {
                    \foreach \y in {0.5,1.5,2.5,3.5} {
                        \draw[gray] (\x-0.2,\y) -- (\x+0.2,\y);
                        \draw[gray] (\x,\y-0.2) -- (\x,\y+0.2);
                    }
                }
                % Pozo de agua (fuente del brote)
                \fill[blue] (2,2) circle (0.2);
                % Casos de cólera
                \foreach \i in {1,...,8} {
                    \fill[red] (2+rand*0.8,2+rand*0.8) circle (0.05);
                }
                \node[below] at (2,-0.5) {\small Mapa de John Snow (1854)};
            \end{tikzpicture}
        \end{center}
    \end{columns}
\end{frame}

% SLIDE 5 - Era de los SIG (4 min)
\begin{frame}{La era de los SIG (1960-1990)}
    \begin{tikzpicture}[scale=0.9, every node/.style={font=\small}]
        % Timeline
        \draw[thick,->] (0,0) -- (11,0) node[right] {};
        
        % Decades
        \foreach \x/\year in {0/1960,3.5/1970,7/1980,10.5/1990} {
            \draw (\x,0.1) -- (\x,-0.1) node[below] {\textbf{\year}};
        }
        
        % Events
        \node[above, text width=3cm, align=center, fill=blue!20, rounded corners] at (1.5,0.5) {
            \textbf{1963: CGIS}\\
            Primer SIG operacional\\
            (Canadá)
        };
        
        \node[below, text width=3cm, align=center, fill=green!20, rounded corners] at (3.5,-0.5) {
            \textbf{1969: ESRI}\\
            Fundación de Environmental Systems Research Institute
        };
        
        \node[above, text width=3cm, align=center, fill=yellow!20, rounded corners] at (7,0.5) {
            \textbf{1982: ArcInfo}\\
            Primer SIG comercial exitoso
        };
        
        \node[below, text width=3cm, align=center, fill=orange!20, rounded corners] at (9,-0.5) {
            \textbf{1985: GRASS GIS}\\
            Primer SIG open source\\
            (US Army)
        };
    \end{tikzpicture}
    
    \vspace{0.5cm}
    \textbf{Características de esta era:}
    \begin{itemize}
        \item Digitalización de mapas analógicos
        \item Mainframes y minicomputadoras
        \item Alto costo y especialización
        \item Enfoque en gestión de datos más que análisis
    \end{itemize}
\end{frame}

% SLIDE 6 - Nacimiento de la Geocomputación (4 min)
\begin{frame}{El nacimiento de la Geocomputación (1990-2000)}
    \examplebox{Conferencia inaugural de Geocomputación (1996)}{
        Universidad de Leeds, Reino Unido - Primera conferencia internacional que acuñó el término
    }
    
    \textbf{¿Qué diferencia la Geocomputación del SIG tradicional?}
    
    \begin{columns}
        \column{0.5\textwidth}
        \textbf{Cambio de paradigma:}
        \begin{itemize}
            \item De gestión a análisis
            \item De datos a conocimiento
            \item De mapas a modelos
            \item De estático a dinámico
        \end{itemize}
        
        \column{0.5\textwidth}
        \textbf{Nuevas técnicas:}
        \begin{itemize}
            \item Autómatas celulares
            \item Redes neuronales espaciales
            \item Algoritmos genéticos
            \item Simulación basada en agentes
        \end{itemize}
    \end{columns}
    
    \vspace{0.3cm}
    \begin{center}
        \textit{"Geocomputation is about using the various different types of geodata and\\
        about developing relevant geo-tools within an 'intelligent' IT framework"}\\
        \small{- Stan Openshaw, 1996}
    \end{center}
\end{frame}

% SLIDE 7 - Era Web GIS (4 min)
\begin{frame}{La revolución Web GIS (2000-2010)}
    \begin{center}
        \begin{tikzpicture}[mindmap, grow cyclic, every node/.style=concept, concept color=usachblue!40,
            level 1/.style={level distance=3.5cm,sibling angle=90},
            level 2/.style={level distance=2cm,sibling angle=45}]
            
            \node[text width=2cm, align=center] {Web GIS\\2000s}
                child { node[concept color=green!40] {Google Maps\\(2005)}
                    child { node[concept color=green!20, text width=1.5cm] {API pública} }
                    child { node[concept color=green!20, text width=1.5cm] {Street View} }
                }
                child { node[concept color=blue!40] {OpenStreetMap\\(2004)}
                    child { node[concept color=blue!20, text width=1.5cm] {Datos libres} }
                }
                child { node[concept color=red!40] {Google Earth\\(2005)}
                    child { node[concept color=red!20, text width=1.5cm] {3D global} }
                }
                child { node[concept color=orange!40] {Web 2.0}
                    child { node[concept color=orange!20, text width=1.5cm] {Mashups} }
                    child { node[concept color=orange!20, text width=1.5cm] {AJAX} }
                };
        \end{tikzpicture}
    \end{center}
    
    \textbf{Impacto:} Democratización del acceso a datos y herramientas geoespaciales
\end{frame}

% SLIDE 8 - Era actual (5 min)
\begin{frame}{Era actual: Big Data y AI Geoespacial (2010-presente)}
    \begin{columns}
        \column{0.6\textwidth}
        \textbf{Características actuales:}
        \begin{itemize}
            \item \textcolor{blue}{\faIcon{database}} \textbf{Big Data:} Petabytes de datos satelitales
            \item \textcolor{green}{\faIcon{cloud}} \textbf{Cloud Computing:} Google Earth Engine, AWS
            \item \textcolor{red}{\faIcon{brain}} \textbf{AI/ML:} Deep Learning para imágenes
            \item \textcolor{orange}{\faIcon{mobile-alt}} \textbf{Ubicuidad:} GPS en cada smartphone
            \item \textcolor{purple}{\faIcon{chart-line}} \textbf{Real-time:} Análisis en tiempo real
        \end{itemize}
        
        \column{0.4\textwidth}
        \begin{tikzpicture}[scale=0.7]
            \begin{axis}[
                ybar,
                ylabel={Datos (PB/año)},
                xlabel={Año},
                symbolic x coords={2010,2015,2020,2025},
                xtick=data,
                nodes near coords,
                ymin=0,
                bar width=0.5cm
            ]
            \addplot coordinates {(2010,0.5) (2015,5) (2020,50) (2025,200)};
            \end{axis}
        \end{tikzpicture}
        \small{Crecimiento exponencial de datos geoespaciales}
    \end{columns}
    
    \vspace{0.3cm}
    \textbf{Ejemplo Chile:} Plataforma IDE Chile integra 50+ servicios geoespaciales
\end{frame}

% SECCIÓN 2: SOFTWARE PARA ANÁLISIS (20 minutos)
\section{Software para análisis geoespacial}

% SLIDE 9 - Panorama del software (3 min)
\begin{frame}{Panorama del software geoespacial}
    \begin{center}
        \begin{tikzpicture}[scale=0.9]
            % Ejes
            \draw[->] (0,0) -- (8,0) node[right] {Facilidad de uso};
            \draw[->] (0,0) -- (0,6) node[above] {Capacidad analítica};
            
            % Software placement
            \node[draw, fill=green!20] at (6,2) {Google Earth};
            \node[draw, fill=blue!20] at (5,3.5) {QGIS};
            \node[draw, fill=red!20] at (4,4) {ArcGIS};
            \node[draw, fill=yellow!20] at (2,5) {Python};
            \node[draw, fill=orange!20] at (2.5,4.5) {R};
            \node[draw, fill=purple!20] at (3,3) {GRASS};
            \node[draw, fill=cyan!20] at (1,5.5) {PostGIS};
            
            % Categorías
            \draw[dashed, gray] (4,0) -- (4,6);
            \node[rotate=90, gray] at (-0.5,3) {Programación};
            \node[rotate=90, gray] at (7.5,3) {GUI};
        \end{tikzpicture}
    \end{center}
\end{frame}

% SLIDE 10 - Software Desktop (4 min)
\begin{frame}{Software Desktop GIS}
    \begin{columns}
        \column{0.5\textwidth}
        \textbf{\textcolor{green}{Open Source:}}
        \begin{itemize}
            \item \textbf{QGIS} (Quantum GIS)
                \begin{itemize}
                    \item[$\bullet$] Más popular
                    \item[$\bullet$] Interfaz amigable
                    \item[$\bullet$] Plugins extensivos
                \end{itemize}
            \item \textbf{GRASS GIS}
                \begin{itemize}
                    \item[$\bullet$] Análisis avanzado
                    \item[$\bullet$] Desde 1985
                \end{itemize}
            \item \textbf{SAGA GIS}
                \begin{itemize}
                    \item[$\bullet$] Geomorfología
                    \item[$\bullet$] Análisis de terreno
                \end{itemize}
        \end{itemize}
        
        \column{0.5\textwidth}
        \textbf{\textcolor{blue}{Comercial:}}
        \begin{itemize}
            \item \textbf{ArcGIS Pro} (ESRI)
                \begin{itemize}
                    \item[$\bullet$] Estándar industria
                    \item[$\bullet$] Integración completa
                    \item[$\bullet$] Alto costo
                \end{itemize}
            \item \textbf{MapInfo}
                \begin{itemize}
                    \item[$\bullet$] Business intelligence
                \end{itemize}
            \item \textbf{Global Mapper}
                \begin{itemize}
                    \item[$\bullet$] LiDAR processing
                \end{itemize}
        \end{itemize}
    \end{columns}
    
    \vspace{0.5cm}
    \examplebox{Para este curso:}{
        Usaremos QGIS como complemento visual (10\% del tiempo)
    }
\end{frame}

% SLIDE 11 - Comparación QGIS vs ArcGIS (3 min)
\begin{frame}{QGIS vs ArcGIS Pro: Comparación}
    \begin{table}[h]
        \centering
        \small
        \begin{tabular}{|l|c|c|}
            \hline
            \textbf{Característica} & \textbf{QGIS} & \textbf{ArcGIS Pro} \\
            \hline
            Costo & Gratis & \$700/año (estudiante) \\
            \hline
            Sistema Operativo & Win/Mac/Linux & Solo Windows \\
            \hline
            Curva aprendizaje & Moderada & Empinada \\
            \hline
            Documentación & Comunitaria & Profesional \\
            \hline
            Soporte Python & \textcolor{green}{\checkmark} & \textcolor{green}{\checkmark} \\
            \hline
            Análisis 3D & Básico & Avanzado \\
            \hline
            Procesamiento & PyQGIS & ArcPy \\
            \hline
            Cloud Integration & Limitada & ArcGIS Online \\
            \hline
        \end{tabular}
    \end{table}
    
    \begin{center}
        \textcolor{usachblue}{\faIcon{lightbulb}} \textit{QGIS es perfectamente capaz para el 95\% de tareas GIS}
    \end{center}
\end{frame}

% SLIDE 12 - Servicios Cloud (4 min)
\begin{frame}{Plataformas Cloud para Geocomputación}
    \begin{columns}
        \column{0.5\textwidth}
        \textbf{\faIcon{google} Google Earth Engine}
        \begin{itemize}
            \item Petabytes de imágenes satelitales
            \item Procesamiento en la nube
            \item JavaScript/Python API
            \item Gratuito para académicos
        \end{itemize}
        
        \vspace{0.3cm}
        \textbf{\faIcon{amazon} AWS}
        \begin{itemize}
            \item S3 para almacenamiento
            \item EC2 para procesamiento
            \item SageMaker para ML
        \end{itemize}
        
        \column{0.5\textwidth}
        \textbf{\faIcon{microsoft} Microsoft Azure}
        \begin{itemize}
            \item Azure Maps
            \item Planetary Computer
            \item AI for Earth
        \end{itemize}
        
        \vspace{0.3cm}
        \textbf{\faIcon{map} Mapbox}
        \begin{itemize}
            \item Mapas personalizados
            \item APIs de geocoding
            \item Navegación
        \end{itemize}
    \end{columns}
    
    \vspace{0.3cm}
    \begin{center}
        \textcolor{usachred}{\faIcon{exclamation-circle}} Cloud computing es esencial para Big Data geoespacial
    \end{center}
\end{frame}

% SLIDE 13 - Bases de datos espaciales (4 min)
\begin{frame}{Bases de Datos Espaciales}
    \textbf{¿Por qué bases de datos espaciales?}
    \begin{itemize}
        \item Consultas espaciales SQL
        \item Índices espaciales para performance
        \item Integridad referencial
        \item Multi-usuario concurrente
    \end{itemize}
    
    \vspace{0.3cm}
    \begin{columns}
        \column{0.5\textwidth}
        \textbf{PostGIS (PostgreSQL)}
        \begin{itemize}
            \item Open source líder
            \item Completo soporte OGC
            \item Funciones espaciales avanzadas
        \end{itemize}
        
        \column{0.5\textwidth}
        \textbf{Otras opciones:}
        \begin{itemize}
            \item Oracle Spatial
            \item MySQL Spatial
            \item SQLite/SpatiaLite
            \item MongoDB (NoSQL)
        \end{itemize}
    \end{columns}
    
    \vspace{0.3cm}
    \begin{tcolorbox}[colframe=blue!50]
        \small
        \texttt{SELECT nombre, ST\_Area(geom)/10000 as hectareas\\
        FROM parcelas\\
        WHERE ST\_Within(geom, (SELECT geom FROM comunas WHERE nombre='Santiago'))}
    \end{tcolorbox}
\end{frame}

% SLIDE 14 - Herramientas de línea de comandos (2 min)
\begin{frame}{Herramientas de línea de comandos}
    \textbf{GDAL/OGR: La navaja suiza del GIS}
    
    \begin{columns}
        \column{0.5\textwidth}
        \textbf{GDAL (Raster)}
        \begin{itemize}
            \item \texttt{gdalinfo}: Información
            \item \texttt{gdal\_translate}: Conversión
            \item \texttt{gdalwarp}: Reproyección
            \item \texttt{gdal\_calc.py}: Álgebra
        \end{itemize}
        
        \column{0.5\textwidth}
        \textbf{OGR (Vector)}
        \begin{itemize}
            \item \texttt{ogrinfo}: Información
            \item \texttt{ogr2ogr}: Conversión
            \item \texttt{ogrmerge}: Fusión
            \item \texttt{ogr\_layer\_algebra}: Operaciones
        \end{itemize}
    \end{columns}
    
    \vspace{0.3cm}
    \begin{tcolorbox}[colframe=gray!50]
        \small
        \texttt{\# Convertir shapefile a GeoJSON\\
        ogr2ogr -f "GeoJSON" output.json input.shp\\
        \\
        \# Reproyectar raster a UTM 19S\\
        gdalwarp -t\_srs EPSG:32719 input.tif output.tif}
    \end{tcolorbox}
\end{frame}

% SECCIÓN 3: ECOSISTEMA PYTHON Y R (25 minutos)
\section{Ecosistema Python y R para geodatos}

% SLIDE 15 - Por qué programar GIS (3 min)
\begin{frame}{¿Por qué programar para GIS?}
    \begin{columns}
        \column{0.5\textwidth}
        \textbf{Ventajas de programar:}
        \begin{itemize}
            \item \textcolor{green}{\faIcon{redo}} Reproducibilidad
            \item \textcolor{blue}{\faIcon{robot}} Automatización
            \item \textcolor{orange}{\faIcon{expand}} Escalabilidad
            \item \textcolor{red}{\faIcon{code-branch}} Versionado
            \item \textcolor{purple}{\faIcon{share-alt}} Compartir código
        \end{itemize}
        
        \column{0.5\textwidth}
        \begin{center}
            \begin{tikzpicture}[scale=0.8]
                % Manual vs Programmatic
                \node[draw, fill=red!20, text width=3cm, align=center] at (0,3) {Proceso Manual\\100 archivos\\8 horas};
                \node[draw, fill=green!20, text width=3cm, align=center] at (0,0) {Script Python\\100 archivos\\10 minutos};
                \draw[->, thick, red] (0,2.3) -- (0,0.7);
            \end{tikzpicture}
        \end{center}
    \end{columns}
    
    \vspace{0.5cm}
    \examplebox{Regla de oro:}{
        Si lo haces más de 3 veces → automatízalo con código
    }
\end{frame}

% SLIDE 16 - Ecosistema Python (5 min)
\begin{frame}{Ecosistema Python para Geoinformática}
    \begin{center}
        \begin{tikzpicture}[scale=0.85]
            % Core
            \node[draw, fill=pythonblue!30, minimum width=2cm, minimum height=1cm] at (0,0) {Python Core};
            
            % Data manipulation
            \node[draw, fill=green!20, minimum width=2cm] at (-3,2) {pandas};
            \node[draw, fill=green!20, minimum width=2cm] at (-3,1) {numpy};
            
            % Geo libraries
            \node[draw, fill=blue!20, minimum width=2.5cm] at (0,2) {\textbf{geopandas}};
            \node[draw, fill=blue!20, minimum width=2cm] at (0,3) {shapely};
            \node[draw, fill=blue!20, minimum width=2cm] at (0,1) {fiona};
            
            % Raster
            \node[draw, fill=orange!20, minimum width=2cm] at (3,2) {\textbf{rasterio}};
            \node[draw, fill=orange!20, minimum width=2cm] at (3,1) {xarray};
            
            % Visualization
            \node[draw, fill=purple!20, minimum width=2cm] at (-3,-1.5) {matplotlib};
            \node[draw, fill=purple!20, minimum width=2cm] at (0,-1.5) {\textbf{folium}};
            \node[draw, fill=purple!20, minimum width=2cm] at (3,-1.5) {plotly};
            
            % Arrows
            \draw[->] (-2,1.5) -- (-0.5,1.5);
            \draw[->] (0.5,1.5) -- (2,1.5);
            \draw[->] (0,-0.5) -- (0,-1);
            \draw[->] (-1.5,0.5) -- (-1.5,-1);
        \end{tikzpicture}
    \end{center}
    
    \textbf{Bibliotecas clave:}
    \begin{multicols}{3}
        \small
        \begin{itemize}
            \item \texttt{geopandas}: DataFrames espaciales
            \item \texttt{rasterio}: Lectura/escritura raster
            \item \texttt{shapely}: Geometrías
            \item \texttt{folium}: Mapas web
            \item \texttt{contextily}: Mapas base
            \item \texttt{osmnx}: Redes OSM
        \end{itemize}
    \end{multicols}
\end{frame}

% SLIDE 17 - Ejemplo Python (4 min)
\begin{frame}[fragile]{Ejemplo: Análisis espacial con Python}
    \begin{lstlisting}[language=Python]
import geopandas as gpd
import matplotlib.pyplot as plt
from shapely.geometry import Point

# Leer datos de comunas de Santiago
comunas = gpd.read_file('santiago_comunas.shp')

# Crear puntos de interés (hospitales)
hospitales = gpd.GeoDataFrame(
    {'nombre': ['Hospital 1', 'Hospital 2'],
     'geometry': [Point(-70.65, -33.45), 
                  Point(-70.60, -33.42)]},
    crs='EPSG:4326')

# Buffer de 2km alrededor de hospitales
areas_servicio = hospitales.buffer(0.02)

# Encontrar comunas servidas
comunas_servidas = comunas[comunas.intersects(
    areas_servicio.unary_union)]

print(f"Comunas con cobertura: {len(comunas_servidas)}")
    \end{lstlisting}
\end{frame}

% SLIDE 18 - Ecosistema R (5 min)
\begin{frame}{Ecosistema R para Geoinformática}
    \begin{center}
        \begin{tikzpicture}[scale=0.85]
            % Core
            \node[draw, fill=rblue!30, minimum width=2cm, minimum height=1cm] at (0,0) {R Base};
            
            % Tidyverse
            \node[draw, fill=green!20, minimum width=2cm] at (-3,2) {dplyr};
            \node[draw, fill=green!20, minimum width=2cm] at (-3,1) {ggplot2};
            
            % Geo libraries
            \node[draw, fill=blue!20, minimum width=2.5cm] at (0,2) {\textbf{sf}};
            \node[draw, fill=blue!20, minimum width=2cm] at (0,3) {sp (legacy)};
            \node[draw, fill=blue!20, minimum width=2cm] at (0,1) {rgeos};
            
            % Raster
            \node[draw, fill=orange!20, minimum width=2cm] at (3,2) {\textbf{terra}};
            \node[draw, fill=orange!20, minimum width=2cm] at (3,1) {stars};
            
            % Visualization
            \node[draw, fill=purple!20, minimum width=2cm] at (-3,-1.5) {\textbf{tmap}};
            \node[draw, fill=purple!20, minimum width=2cm] at (0,-1.5) {leaflet};
            \node[draw, fill=purple!20, minimum width=2cm] at (3,-1.5) {mapview};
            
            % Arrows
            \draw[->] (-2,1.5) -- (-0.5,1.5);
            \draw[->] (0.5,1.5) -- (2,1.5);
            \draw[->] (0,-0.5) -- (0,-1);
            \draw[->] (-1.5,0.5) -- (-1.5,-1);
        \end{tikzpicture}
    \end{center}
    
    \textbf{Paquetes clave:}
    \begin{multicols}{3}
        \small
        \begin{itemize}
            \item \texttt{sf}: Simple Features
            \item \texttt{terra}: Datos raster
            \item \texttt{tmap}: Mapas temáticos
            \item \texttt{leaflet}: Mapas interactivos
            \item \texttt{rayshader}: 3D
            \item \texttt{gstat}: Geoestadística
        \end{itemize}
    \end{multicols}
\end{frame}

% SLIDE 19 - Ejemplo R (4 min)
\begin{frame}[fragile]{Ejemplo: Análisis espacial con R}
    \begin{lstlisting}[language=R]
library(sf)
library(tmap)
library(dplyr)

# Leer datos de comunas
comunas <- st_read("santiago_comunas.shp")

# Crear puntos de hospitales
hospitales <- data.frame(
  nombre = c("Hospital 1", "Hospital 2"),
  lon = c(-70.65, -70.60),
  lat = c(-33.45, -33.42)
) %>%
  st_as_sf(coords = c("lon", "lat"), crs = 4326)

# Buffer de 2km
areas_servicio <- st_buffer(hospitales, dist = 2000)

# Intersección con comunas
comunas_servidas <- comunas[areas_servicio, ]

# Visualizar
tm_shape(comunas) + tm_polygons() +
  tm_shape(areas_servicio) + tm_borders("red", lwd = 2)
    \end{lstlisting}
\end{frame}

% SLIDE 20 - Python vs R (4 min)
\begin{frame}{Python vs R para Geoinformática}
    \begin{table}[h]
        \centering
        \small
        \begin{tabular}{|l|c|c|}
            \hline
            \textbf{Aspecto} & \textbf{Python} & \textbf{R} \\
            \hline
            Sintaxis & Más general & Más estadística \\
            \hline
            Curva aprendizaje & Más suave & Más empinada \\
            \hline
            Velocidad & \textcolor{green}{\checkmark} Más rápido & Más lento \\
            \hline
            Visualización estática & Buena & \textcolor{green}{\checkmark} Excelente \\
            \hline
            Machine Learning & \textcolor{green}{\checkmark} scikit-learn & Bueno \\
            \hline
            Deep Learning & \textcolor{green}{\checkmark} TensorFlow & Limitado \\
            \hline
            Estadística espacial & Buena & \textcolor{green}{\checkmark} Excelente \\
            \hline
            Integración web & \textcolor{green}{\checkmark} Django/Flask & Shiny \\
            \hline
            Comunidad GIS & Grande & Grande \\
            \hline
        \end{tabular}
    \end{table}
    
    \vspace{0.3cm}
    \examplebox{Recomendación del curso:}{
        \textbf{Python} para producción y escalabilidad\\
        \textbf{R} para exploración y visualización estadística
    }
\end{frame}

% SECCIÓN 4: EJEMPLOS PRÁCTICOS (15 minutos)
\section{Primeros ejemplos prácticos}

% SLIDE 21 - Flujo de trabajo típico (3 min)
\begin{frame}{Flujo de trabajo típico en Geocomputación}
    \begin{center}
        \begin{tikzpicture}[scale=0.9, 
            box/.style={draw, rounded corners, fill=blue!20, text width=2.5cm, align=center, minimum height=1cm},
            arrow/.style={->, thick, >=stealth}]
            
            % Boxes
            \node[box, fill=green!20] (data) at (0,4) {1. Obtención\\de Datos};
            \node[box, fill=yellow!20] (clean) at (3,4) {2. Limpieza\\Preparación};
            \node[box, fill=orange!20] (analysis) at (6,4) {3. Análisis\\Espacial};
            \node[box, fill=red!20] (viz) at (9,4) {4. Visualización};
            
            \node[box, fill=purple!20] (model) at (3,1) {5. Modelado\\Predicción};
            \node[box, fill=cyan!20] (share) at (6,1) {6. Compartir\\Publicar};
            
            % Arrows
            \draw[arrow] (data) -- (clean);
            \draw[arrow] (clean) -- (analysis);
            \draw[arrow] (analysis) -- (viz);
            \draw[arrow] (analysis) -- (model);
            \draw[arrow] (model) -- (share);
            \draw[arrow] (viz) -- (share);
            
            % Tools annotations
            \node[below, font=\tiny] at (0,3.3) {APIs, Web scraping};
            \node[below, font=\tiny] at (3,3.3) {pandas, dplyr};
            \node[below, font=\tiny] at (6,3.3) {geopandas, sf};
            \node[below, font=\tiny] at (9,3.3) {folium, tmap};
            \node[below, font=\tiny] at (3,0.3) {scikit-learn};
            \node[below, font=\tiny] at (6,0.3) {GitHub, web};
        \end{tikzpicture}
    \end{center}
\end{frame}

% SLIDE 22 - Caso práctico: COVID-19 (5 min)
\begin{frame}{Caso práctico: Análisis COVID-19 en RM}
    \textbf{Objetivo:} Identificar zonas de alto riesgo en Santiago
    
    \begin{columns}
        \column{0.5\textwidth}
        \textbf{Datos necesarios:}
        \begin{itemize}
            \item Casos por comuna (MINSAL)
            \item Población (INE)
            \item Geometrías comunas
            \item Centros de salud
            \item Movilidad (opcional)
        \end{itemize}
        
        \column{0.5\textwidth}
        \textbf{Pasos del análisis:}
        \begin{enumerate}
            \item Calcular tasa incidencia
            \item Detectar clusters (Moran's I)
            \item Analizar accesibilidad salud
            \item Crear mapa de riesgo
            \item Dashboard interactivo
        \end{enumerate}
    \end{columns}
    
    \vspace{0.3cm}
    \begin{tcolorbox}[colframe=red!50]
        \small
        \textbf{Resultado:} Mapa de calor identificando Las Condes, Vitacura y Providencia como clusters iniciales (marzo 2020)
    \end{tcolorbox}
\end{frame}

% SLIDE 23 - Demo en vivo setup (3 min)
\begin{frame}[fragile]{Demo: Configuración del ambiente}
    \textbf{1. Verificar instalación Python:}
    \begin{lstlisting}[language=bash]
python --version  # Debe ser 3.8+
pip --version
    \end{lstlisting}
    
    \textbf{2. Crear ambiente virtual:}
    \begin{lstlisting}[language=bash]
python -m venv geo_env
source geo_env/bin/activate  # Linux/Mac
# o
geo_env\Scripts\activate  # Windows
    \end{lstlisting}
    
    \textbf{3. Instalar bibliotecas geoespaciales:}
    \begin{lstlisting}[language=bash]
pip install geopandas folium matplotlib
pip install rasterio contextily
pip install jupyter notebook
    \end{lstlisting}
    
    \begin{center}
        \textcolor{usachred}{\faIcon{exclamation-triangle}} Esto lo haremos juntos en el laboratorio
    \end{center}
\end{frame}

% SLIDE 24 - Primer script completo (4 min)
\begin{frame}[fragile]{Tu primer script geoespacial}
    \begin{lstlisting}[language=Python, basicstyle=\tiny\ttfamily]
# archivo: mi_primer_mapa.py
import geopandas as gpd
import matplotlib.pyplot as plt

# 1. Descargar datos de ejemplo
world = gpd.read_file(gpd.datasets.get_path('naturalearth_lowres'))

# 2. Filtrar América del Sur
sudamerica = world[world['continent'] == 'South America']

# 3. Calcular centroides
sudamerica['centroid'] = sudamerica.geometry.centroid

# 4. Crear figura con subplots
fig, (ax1, ax2) = plt.subplots(1, 2, figsize=(15, 8))

# 5. Mapa 1: Población
sudamerica.plot(column='pop_est', 
                ax=ax1, 
                legend=True,
                cmap='YlOrRd',
                edgecolor='black')
ax1.set_title('Población de Sudamérica')

# 6. Mapa 2: PIB per cápita  
sudamerica.plot(column='gdp_md_est',
                ax=ax2,
                legend=True, 
                cmap='Greens',
                edgecolor='black')
ax2.set_title('PIB de países sudamericanos')

plt.tight_layout()
plt.savefig('sudamerica_analisis.png', dpi=300)
plt.show()
    \end{lstlisting}
\end{frame}

% SECCIÓN 5: PREPARACIÓN Y CIERRE (10 minutos)
\section{Preparación para el laboratorio}

% SLIDE 25 - Qué haremos en el lab (3 min)
\begin{frame}{Laboratorio 1: Configuración del ambiente}
    \textbf{Objetivos del laboratorio de hoy:}
    
    \begin{enumerate}
        \item \textcolor{blue}{\faIcon{download}} \textbf{Instalación completa}
        \begin{itemize}
            \item Python + Anaconda/Miniconda
            \item R + RStudio
            \item QGIS (opcional)
            \item Git
        \end{itemize}
        
        \item \textcolor{green}{\faIcon{check-circle}} \textbf{Verificación}
        \begin{itemize}
            \item Importar bibliotecas sin errores
            \item Ejecutar script de prueba
        \end{itemize}
        
        \item \textcolor{orange}{\faIcon{map}} \textbf{Primer mapa}
        \begin{itemize}
            \item Cargar datos de Chile
            \item Visualización básica
            \item Exportar resultado
        \end{itemize}
    \end{enumerate}
    
    \vspace{0.3cm}
    \examplebox{Importante:}{
        Traer computador con permisos de administrador para instalar software
    }
\end{frame}

% SLIDE 26 - Recursos de aprendizaje (3 min)
\begin{frame}{Recursos para profundizar}
    \begin{columns}
        \column{0.5\textwidth}
        \textbf{Libros online gratuitos:}
        \begin{itemize}
            \item \href{https://py.geocompx.org}{Geocomputation with Python}
            \item \href{https://r.geocompx.org}{Geocomputation with R}
            \item \href{https://geographicdata.science/book}{Geographic Data Science}
        \end{itemize}
        
        \vspace{0.3cm}
        \textbf{Documentación oficial:}
        \begin{itemize}
            \item \href{https://geopandas.org}{GeoPandas docs}
            \item \href{https://r-spatial.github.io/sf/}{sf package docs}
            \item \href{https://docs.qgis.org}{QGIS docs}
        \end{itemize}
        
        \column{0.5\textwidth}
        \textbf{Cursos online:}
        \begin{itemize}
            \item Coursera: GIS Specialization
            \item edX: Spatial Data Science
            \item YouTube: GeoDelta Labs
        \end{itemize}
        
        \vspace{0.3cm}
        \textbf{Comunidades:}
        \begin{itemize}
            \item \faIcon{stack-overflow} Stack Overflow GIS
            \item \faIcon{reddit} r/gis
            \item \faIcon{twitter} \#gischat
            \item \faIcon{github} awesome-gis
        \end{itemize}
    \end{columns}
\end{frame}

% SLIDE 27 - Proyecto: primeras ideas (2 min)
\begin{frame}{Proyecto semestral: Empezar a pensar}
    \textbf{Brainstorming de ideas para proyecto:}
    
    \begin{columns}
        \column{0.5\textwidth}
        \textcolor{blue}{\faIcon{building} \textbf{Comercial:}}
        \begin{itemize}
            \item Optimización de rutas delivery
            \item Análisis inmobiliario Santiago
            \item Ubicación óptima de sucursales
            \item Segmentación de clientes
        \end{itemize}
        
        \column{0.5\textwidth}
        \textcolor{green}{\faIcon{flask} \textbf{Científico:}}
        \begin{itemize}
            \item Contaminación del aire RM
            \item Expansión urbana Santiago
            \item Riesgo de incendios V Región
            \item Acceso a áreas verdes
        \end{itemize}
    \end{columns}
    
    \vspace{0.5cm}
    \begin{center}
        \begin{tcolorbox}[colframe=usachblue, width=10cm]
            \centering
            \textbf{Semana 4:} Formación de grupos (1-3 personas)\\
            \textcolor{gray}{\small Empiecen a conversar con compañeros sobre intereses comunes}
        \end{tcolorbox}
    \end{center}
\end{frame}

% SLIDE 28 - Resumen y conceptos clave (2 min)
\begin{frame}{Resumen: Conceptos clave de hoy}
    \begin{columns}
        \column{0.5\textwidth}
        \textbf{Historia:}
        \begin{itemize}
            \item 1960s: Primeros SIG
            \item 1996: Nace Geocomputación
            \item 2005: Revolución Web GIS
            \item Hoy: Big Data + AI
        \end{itemize}
        
        \textbf{Software:}
        \begin{itemize}
            \item Desktop: QGIS vs ArcGIS
            \item Cloud: Google Earth Engine
            \item BD: PostGIS
            \item CLI: GDAL/OGR
        \end{itemize}
        
        \column{0.5\textwidth}
        \textbf{Programación:}
        \begin{itemize}
            \item Python: geopandas, rasterio
            \item R: sf, terra, tmap
            \item Reproducibilidad es clave
            \item Automatización ahorra tiempo
        \end{itemize}
        
        \textbf{Para recordar:}
        \begin{itemize}
            \item Open source es viable
            \item Python líder en la industria
            \item R excelente para estadística
            \item Práctica hace al maestro
        \end{itemize}
    \end{columns}
\end{frame}

% SLIDE 29 - Tarea (1 min)
\begin{frame}{Para la próxima clase}
    \begin{enumerate}
        \item \textcolor{red}{\faIcon{exclamation-circle}} \textbf{URGENTE:} Instalar software antes del laboratorio
        \begin{itemize}
            \item Python (Anaconda) + bibliotecas
            \item R + RStudio + paquetes
            \item QGIS (opcional pero recomendado)
        \end{itemize}
        
        \item \textcolor{blue}{\faIcon{book}} \textbf{Lectura:} Capítulo 1 de Geocomputation with Python
        \begin{itemize}
            \item \url{https://py.geocompx.org/01-introduction}
        \end{itemize}
        
        \item \textcolor{green}{\faIcon{lightbulb}} \textbf{Reflexionar:} 
        \begin{itemize}
            \item ¿Qué problema espacial te gustaría resolver?
            \item ¿Qué datos necesitarías?
        \end{itemize}
    \end{enumerate}
    
    \vspace{0.5cm}
    \begin{center}
        \textcolor{gray}{\faIcon{envelope} Dudas: francisco.parra.o@usach.cl}
    \end{center}
\end{frame}

% SLIDE 30 - Cierre (1 min)
\begin{frame}{¡Nos vemos en el laboratorio!}
    \begin{center}
        \Large{\textbf{¿Preguntas?}}
        
        \vspace{1cm}
        
        \begin{tikzpicture}
            \node[draw=usachblue, line width=2pt, rounded corners, fill=blue!10, text width=8cm, align=center, minimum height=2cm] {
                \textbf{A continuación:}\\
                Laboratorio 1 - Configuración del ambiente\\
                \small{Mismo salón, 10 minutos de break}
            };
        \end{tikzpicture}
        
        \vspace{1cm}
        
        \textcolor{usachred}{\faIcon{coffee}} \textit{Pueden ir por un café mientras preparamos el laboratorio}
    \end{center}
\end{frame}

\end{document}