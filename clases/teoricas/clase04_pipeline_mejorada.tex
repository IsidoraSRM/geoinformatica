\documentclass[10pt,aspectratio=169]{beamer}
\usetheme{metropolis}
\usepackage{FiraSans}
\usefonttheme{professionalfonts}

\usepackage{graphicx}
\usepackage{tikz}
\usepackage[spanish]{babel}
\usepackage{tcolorbox}
\usepackage{ragged2e}
\usepackage{pgfplots}
\pgfplotsset{compat=1.18}
\usepackage{listings}
\usepackage{xcolor}
\usepackage{array}
\usepackage{booktabs}
\usepackage{multirow}
\usepackage{hyperref}
\usepackage{fontawesome5}
\usetikzlibrary{shapes,arrows,positioning,calc,backgrounds,fit}

\definecolor{codegreen}{rgb}{0,0.6,0}
\definecolor{codegray}{rgb}{0.5,0.5,0.5}
\definecolor{codepurple}{rgb}{0.58,0,0.82}
\definecolor{backcolour}{rgb}{0.95,0.95,0.92}
\definecolor{usachblue}{rgb}{0,0.4,0.7}
\definecolor{usachred}{rgb}{0.8,0,0}

\lstdefinestyle{mystyle}{
    backgroundcolor=\color{backcolour},   
    commentstyle=\color{codegreen},
    keywordstyle=\color{magenta},
    numberstyle=\tiny\color{codegray},
    stringstyle=\color{codepurple},
    basicstyle=\ttfamily\tiny,
    breakatwhitespace=false,         
    breaklines=true,                 
    captionpos=b,                    
    keepspaces=true,                 
    numbers=left,                    
    numbersep=3pt,                  
    showspaces=false,                
    showstringspaces=false,
    showtabs=false,                  
    tabsize=2
}

\lstset{style=mystyle}

\newcommand{\conceptbox}[2]{
\begin{tcolorbox}[colframe=usachblue,colback=blue!5,title=#1,fonttitle=\bfseries]
#2
\end{tcolorbox}
}

\newcommand{\alertbox}[1]{
\begin{tcolorbox}[colframe=red!70,colback=red!5]
\centering\faExclamationTriangle\space #1
\end{tcolorbox}
}

\setbeamertemplate{footline}{%
  \begin{beamercolorbox}[wd=\paperwidth,sep=2ex]{footline}%
    \usebeamerfont{structure}\textbf{Geoinformática - Clase 4} \hfill Profesor: Francisco Parra O. \hfill \textbf{Semestre 2, 2025}
  \end{beamercolorbox}%
}

\title{Clase 04: Pipeline de Desarrollo Geoespacial}
\subtitle{Arquitectura y mejores prácticas para soluciones geoespaciales}
\author{Profesor: Francisco Parra O.}
\institute{USACH - Ingeniería Civil en Informática}
\date{\today}

\titlegraphic{%
  \begin{tikzpicture}[overlay, remember picture]
    \node[anchor=north east, yshift=0cm] at (current page.north east) {
      \includegraphics[width=1.5cm]{../../logo.jpg}
    };
  \end{tikzpicture}
}

\begin{document}

\maketitle

\begin{frame}{Agenda}
    \tableofcontents[hideallsubsections]
\end{frame}

\section{Introducción: ¿Qué es un Pipeline Geoespacial?}

\begin{frame}{¿Qué es un Pipeline Geoespacial?}
    \begin{columns}[T]
        \column{0.5\textwidth}
        \conceptbox{Definición}{
            Un \textbf{pipeline geoespacial} es una secuencia automatizada de procesos que transforman datos geográficos crudos en información accionable.
        }
        
        \vspace{0.3cm}
        \textbf{Características clave:}
        \begin{itemize}
            \item \faDatabase\space Manejo de grandes volúmenes
            \item \faSync\space Procesamiento continuo
            \item \faCheckCircle\space Validación automática
            \item \faChartLine\space Escalabilidad
        \end{itemize}
        
        \column{0.5\textwidth}
        \begin{tikzpicture}[scale=0.7, every node/.style={scale=0.7}]
            \node[draw, rectangle, fill=blue!20] (raw) at (0,0) {Datos Crudos};
            \node[draw, rectangle, fill=green!20] (clean) at (0,-1.5) {Limpieza};
            \node[draw, rectangle, fill=yellow!20] (transform) at (0,-3) {Transformación};
            \node[draw, rectangle, fill=orange!20] (analysis) at (0,-4.5) {Análisis};
            \node[draw, rectangle, fill=red!20] (viz) at (0,-6) {Visualización};
            
            \draw[->, thick] (raw) -- (clean);
            \draw[->, thick] (clean) -- (transform);
            \draw[->, thick] (transform) -- (analysis);
            \draw[->, thick] (analysis) -- (viz);
            
            % Anotaciones
            \node[right=1cm of raw] {\small OSM, Sensores, APIs};
            \node[right=1cm of clean] {\small Validación, Filtrado};
            \node[right=1cm of transform] {\small Proyección, Agregación};
            \node[right=1cm of analysis] {\small Clustering, Routing};
            \node[right=1cm of viz] {\small Mapas, Dashboards};
        \end{tikzpicture}
    \end{columns}
\end{frame}

\begin{frame}{¿Por qué necesitamos pipelines?}
    \begin{columns}[T]
        \column{0.5\textwidth}
        \textbf{Problemas sin pipeline:}
        \begin{itemize}
            \item[\faTimesCircle] Procesos manuales repetitivos
            \item[\faTimesCircle] Errores humanos frecuentes
            \item[\faTimesCircle] Difícil reproducibilidad
            \item[\faTimesCircle] Escalabilidad limitada
            \item[\faTimesCircle] Falta de trazabilidad
        \end{itemize}
        
        \column{0.5\textwidth}
        \textbf{Beneficios con pipeline:}
        \begin{itemize}
            \item[\faCheckCircle] Automatización completa
            \item[\faCheckCircle] Consistencia garantizada
            \item[\faCheckCircle] Reproducibilidad total
            \item[\faCheckCircle] Escalabilidad horizontal
            \item[\faCheckCircle] Auditoría y monitoreo
        \end{itemize}
    \end{columns}
    
    \vspace{0.5cm}
    \alertbox{Un pipeline bien diseñado puede reducir el tiempo de procesamiento de días a minutos}
\end{frame}

\section{Arquitectura de un Pipeline Geoespacial}

\begin{frame}{Principios SOLID en Pipelines Geoespaciales}
    \begin{columns}[T]
        \column{0.5\textwidth}
        \textbf{S - Single Responsibility}
        \begin{itemize}
            \item Cada módulo una tarea
            \item Geocoder solo geocodifica
            \item Router solo calcula rutas
        \end{itemize}
        
        \textbf{O - Open/Closed}
        \begin{itemize}
            \item Extensible para nuevas fuentes
            \item Cerrado para modificaciones core
        \end{itemize}
        
        \textbf{L - Liskov Substitution}
        \begin{itemize}
            \item Interfaces consistentes
            \item Proveedores intercambiables
        \end{itemize}
        
        \column{0.5\textwidth}
        \textbf{I - Interface Segregation}
        \begin{itemize}
            \item APIs específicas por dominio
            \item No forzar dependencias innecesarias
        \end{itemize}
        
        \textbf{D - Dependency Inversion}
        \begin{itemize}
            \item Depender de abstracciones
            \item Inyección de dependencias
            \item Configuración externa
        \end{itemize}
        
        \vspace{0.3cm}
        \alertbox{Aplicar SOLID reduce acoplamiento y mejora mantenibilidad}
    \end{columns}
\end{frame}

\begin{frame}{Arquitectura en Capas}
    \begin{tikzpicture}[scale=0.9, every node/.style={scale=0.8}]
        % Capas
        \node[draw, rectangle, minimum width=10cm, minimum height=1cm, fill=blue!20] (data) at (0,0) {\textbf{Capa de Datos}};
        \node[draw, rectangle, minimum width=10cm, minimum height=1cm, fill=green!20] (process) at (0,1.5) {\textbf{Capa de Procesamiento}};
        \node[draw, rectangle, minimum width=10cm, minimum height=1cm, fill=yellow!20] (business) at (0,3) {\textbf{Capa de Lógica de Negocio}};
        \node[draw, rectangle, minimum width=10cm, minimum height=1cm, fill=orange!20] (api) at (0,4.5) {\textbf{Capa de API/Servicios}};
        \node[draw, rectangle, minimum width=10cm, minimum height=1cm, fill=red!20] (ui) at (0,6) {\textbf{Capa de Presentación}};
        
        % Tecnologías
        \node[right=0.5cm of data, text=gray] {\small PostGIS, MongoDB, S3};
        \node[right=0.5cm of process, text=gray] {\small Dask, Spark, Airflow};
        \node[right=0.5cm of business, text=gray] {\small Python, R, Julia};
        \node[right=0.5cm of api, text=gray] {\small FastAPI, GraphQL, REST};
        \node[right=0.5cm of ui, text=gray] {\small React, Streamlit, Dash};
        
        % Flechas bidireccionales
        \draw[<->, thick] (data) -- (process);
        \draw[<->, thick] (process) -- (business);
        \draw[<->, thick] (business) -- (api);
        \draw[<->, thick] (api) -- (ui);
    \end{tikzpicture}
    
    \vspace{0.3cm}
    \conceptbox{Principio de Separación de Responsabilidades}{
        Cada capa tiene una responsabilidad específica y se comunica solo con capas adyacentes
    }
\end{frame}

\begin{frame}{Patrones de Diseño para Pipelines}
    \begin{columns}[T]
        \column{0.33\textwidth}
        \textbf{\faStream\space ETL/ELT}
        \begin{itemize}
            \item Extract
            \item Transform/Load
            \item Load/Transform
        \end{itemize}
        \textit{Ideal para: Batch processing}
        
        \column{0.33\textwidth}
        \textbf{\faBolt\space Event-Driven}
        \begin{itemize}
            \item Triggers
            \item Webhooks
            \item Message Queues
        \end{itemize}
        \textit{Ideal para: Real-time}
        
        \column{0.34\textwidth}
        \textbf{\faCodeBranch\space Microservicios}
        \begin{itemize}
            \item Servicios independientes
            \item API Gateway
            \item Service Mesh
        \end{itemize}
        \textit{Ideal para: Escalabilidad}
    \end{columns}
    
    \vspace{0.5cm}
    \begin{tcolorbox}[colframe=gray,colback=gray!10]
        \centering
        \textbf{Regla de oro:} Elige el patrón según tu caso de uso, no por moda tecnológica
    \end{tcolorbox}
\end{frame}

\section{Adquisición de Datos Geoespaciales}

\begin{frame}{Fuentes de Datos Geoespaciales}
    \begin{columns}[T]
        \column{0.5\textwidth}
        \textbf{Fuentes Abiertas:}
        \begin{itemize}
            \item \faGlobe\space OpenStreetMap
            \item \faSatellite\space Sentinel Hub
            \item \faUniversity\space Datos gubernamentales
            \item \faCloud\space APIs meteorológicas
        \end{itemize}
        
        \vspace{0.3cm}
        \textbf{Fuentes Comerciales:}
        \begin{itemize}
            \item \faMapMarkedAlt\space Google Maps API
            \item \faRoute\space Mapbox
            \item \faBuilding\space HERE Technologies
            \item \faImage\space Planet Labs
        \end{itemize}
        
        \column{0.5\textwidth}
        \conceptbox{Consideraciones Clave}{
            \begin{itemize}
                \item \textbf{Licencias:} Revisa restricciones
                \item \textbf{Calidad:} Valida precisión
                \item \textbf{Actualización:} Frecuencia de cambios
                \item \textbf{Cobertura:} Área geográfica
                \item \textbf{Formato:} Vector vs Raster
            \end{itemize}
        }
    \end{columns}
\end{frame}

\begin{frame}{Estrategias de Adquisición}
    \begin{tikzpicture}[scale=0.8, every node/.style={scale=0.8}]
        % Nodo central
        \node[draw, circle, fill=yellow!30, minimum size=2cm] (center) at (0,0) {\textbf{Datos}};
        
        % Estrategias alrededor
        \node[draw, rectangle, fill=blue!20] (batch) at (0,3) {Batch Download};
        \node[draw, rectangle, fill=green!20] (stream) at (3,2) {Streaming};
        \node[draw, rectangle, fill=orange!20] (api) at (3,-2) {API REST};
        \node[draw, rectangle, fill=red!20] (scraping) at (0,-3) {Web Scraping};
        \node[draw, rectangle, fill=purple!20] (sync) at (-3,-2) {Sincronización};
        \node[draw, rectangle, fill=cyan!20] (webhook) at (-3,2) {Webhooks};
        
        % Conexiones
        \draw[->, thick] (batch) -- (center);
        \draw[->, thick] (stream) -- (center);
        \draw[->, thick] (api) -- (center);
        \draw[->, thick] (scraping) -- (center);
        \draw[->, thick] (sync) -- (center);
        \draw[->, thick] (webhook) -- (center);
        
        % Anotaciones
        \node[above=0.1cm of batch, text=gray] {\footnotesize Grandes volúmenes};
        \node[right=0.1cm of stream, text=gray] {\footnotesize Tiempo real};
        \node[right=0.1cm of api, text=gray] {\footnotesize On-demand};
        \node[below=0.1cm of scraping, text=gray] {\footnotesize Último recurso};
        \node[left=0.1cm of sync, text=gray] {\footnotesize Incremental};
        \node[left=0.1cm of webhook, text=gray] {\footnotesize Event-driven};
    \end{tikzpicture}
\end{frame}

\begin{frame}{Teoría de Grafos en Redes Viales}
    \begin{columns}[T]
        \column{0.5\textwidth}
        \conceptbox{Modelado como Grafo}{
            \begin{itemize}
                \item \textbf{Nodos:} Intersecciones
                \item \textbf{Aristas:} Segmentos de calle
                \item \textbf{Pesos:} Distancia, tiempo, costo
                \item \textbf{Dirección:} Sentido del tráfico
            \end{itemize}
        }
        
        \textbf{Métricas de Red:}
        \begin{itemize}
            \item \textbf{Centralidad:} Importancia del nodo
            \item \textbf{Conectividad:} Robustez de la red
            \item \textbf{Clustering:} Agrupación local
            \item \textbf{Shortest Path:} Ruta óptima
        \end{itemize}
        
        \column{0.5\textwidth}
        \begin{tikzpicture}[scale=0.8]
            % Grafo de ejemplo
            \node[circle, draw, fill=blue!30] (A) at (0,0) {A};
            \node[circle, draw, fill=blue!30] (B) at (2,1) {B};
            \node[circle, draw, fill=blue!30] (C) at (2,-1) {C};
            \node[circle, draw, fill=blue!30] (D) at (4,0) {D};
            \node[circle, draw, fill=blue!30] (E) at (4,2) {E};
            
            \draw[thick, ->] (A) -- node[above] {5} (B);
            \draw[thick, ->] (A) -- node[below] {3} (C);
            \draw[thick, ->] (B) -- node[right] {2} (D);
            \draw[thick, ->] (C) -- node[right] {4} (D);
            \draw[thick, ->] (B) -- node[above] {1} (E);
            \draw[thick, ->] (D) -- node[right] {3} (E);
            
            \node[below=1cm of C] {\small Grafo dirigido con pesos};
        \end{tikzpicture}
        
        \vspace{0.3cm}
        \conceptbox{Algoritmos Clave}{
            Dijkstra, A*, Bellman-Ford, Floyd-Warshall
        }
    \end{columns}
\end{frame}

\begin{frame}{OSMnx: Abstracción de Complejidad}
    \conceptbox{¿Qué es OSMnx?}{
        Framework que abstrae la complejidad de trabajar con datos de OpenStreetMap para análisis de redes urbanas
    }
    
    \begin{columns}[T]
        \column{0.5\textwidth}
        \textbf{Problemas que resuelve:}
        \begin{itemize}
            \item Descarga eficiente de datos OSM
            \item Limpieza de topología
            \item Simplificación de intersecciones
            \item Proyección automática
            \item Cálculo de métricas
        \end{itemize}
        
        \column{0.5\textwidth}
        \textbf{Flujo conceptual:}
        \begin{enumerate}
            \item \textbf{Definir} área de estudio
            \item \textbf{Filtrar} tipo de red
            \item \textbf{Construir} grafo topológico
            \item \textbf{Analizar} propiedades
            \item \textbf{Visualizar} resultados
        \end{enumerate}
    \end{columns}
    
    \vspace{0.3cm}
    \alertbox{OSMnx maneja automáticamente la complejidad geométrica y topológica}
\end{frame}

\section{Almacenamiento y Gestión de Datos}

\begin{frame}{PostGIS: El Corazón del Pipeline}
    \begin{columns}[T]
        \column{0.5\textwidth}
        \conceptbox{¿Qué es PostGIS?}{
            Extensión espacial de PostgreSQL que añade soporte para objetos geográficos, permitiendo consultas espaciales SQL.
        }
        
        \textbf{Ventajas clave:}
        \begin{itemize}
            \item \faDatabase\space ACID compliance
            \item \faSearch\space Índices espaciales (R-tree)
            \item \faTools\space +3000 funciones espaciales
            \item \faLock\space Seguridad empresarial
            \item \faUsers\space Multi-usuario
        \end{itemize}
        
        \column{0.5\textwidth}
        \textbf{Operaciones fundamentales:}
        \begin{itemize}
            \item \textbf{ST\_Contains:} ¿A contiene B?
            \item \textbf{ST\_Distance:} Distancia entre objetos
            \item \textbf{ST\_Buffer:} Área de influencia
            \item \textbf{ST\_Intersection:} Intersección
            \item \textbf{ST\_Union:} Unión de geometrías
            \item \textbf{ST\_DWithin:} Objetos cercanos
        \end{itemize}
        
        \vspace{0.3cm}
        \alertbox{PostGIS es el estándar de facto para bases de datos espaciales}
    \end{columns}
\end{frame}

\begin{frame}{Arquitectura de Almacenamiento}
    \begin{tikzpicture}[scale=0.7, every node/.style={scale=0.7}]
        % Base de datos principal
        \node[draw, cylinder, shape aspect=2, minimum width=2cm, minimum height=2cm, fill=blue!30] (postgis) at (0,0) {PostGIS};
        
        % Caché
        \node[draw, cylinder, shape aspect=2, minimum width=1.5cm, minimum height=1.5cm, fill=red!30] (redis) at (4,0) {Redis};
        
        % Archivos
        \node[draw, rectangle, minimum width=2cm, minimum height=1cm, fill=green!30] (s3) at (-4,0) {S3/MinIO};
        
        % Data Lake
        \node[draw, rectangle, minimum width=2cm, minimum height=1cm, fill=yellow!30] (lake) at (0,-3) {Data Lake};
        
        % Conexiones
        \draw[<->, thick] (postgis) -- (redis) node[midway, above] {\small Caché};
        \draw[<->, thick] (postgis) -- (s3) node[midway, above] {\small Archivos};
        \draw[->, thick] (postgis) -- (lake) node[midway, right] {\small Histórico};
        
        % Anotaciones
        \node[below=0.5cm of postgis] {\textbf{Transaccional}};
        \node[below=0.5cm of redis] {\textbf{Temporal}};
        \node[below=0.5cm of s3] {\textbf{Objetos}};
        \node[below=0.2cm of lake] {\textbf{Analítico}};
    \end{tikzpicture}
    
    \vspace{0.5cm}
    \conceptbox{Estrategia Híbrida}{
        Combina diferentes tecnologías según el tipo de dato y patrón de acceso
    }
\end{frame}

\section{Procesamiento y Análisis}

\begin{frame}{Paradigmas de Procesamiento Geoespacial}
    \begin{tikzpicture}[scale=0.8, every node/.style={scale=0.8}]
        % Batch Processing
        \node[draw, rectangle, fill=blue!20, minimum width=3cm, minimum height=1.5cm] (batch) at (0,0) {\textbf{Batch}\\\small Grandes volúmenes\\Latencia alta OK};
        
        % Stream Processing
        \node[draw, rectangle, fill=green!20, minimum width=3cm, minimum height=1.5cm] (stream) at (5,0) {\textbf{Stream}\\\small Datos continuos\\Baja latencia};
        
        % Micro-batch
        \node[draw, rectangle, fill=yellow!20, minimum width=3cm, minimum height=1.5cm] (micro) at (10,0) {\textbf{Micro-batch}\\\small Híbrido\\Balance};
        
        % Ejemplos
        \node[below=0.5cm of batch] {\footnotesize Apache Spark\\PostGIS bulk ops};
        \node[below=0.5cm of stream] {\footnotesize Kafka Streams\\Apache Flink};
        \node[below=0.5cm of micro] {\footnotesize Spark Streaming\\Storm Trident};
        
        % Casos de uso
        \node[above=0.5cm of batch] {\textit{Análisis histórico}};
        \node[above=0.5cm of stream] {\textit{Tracking GPS}};
        \node[above=0.5cm of micro] {\textit{Alertas tráfico}};
    \end{tikzpicture}
    
    \vspace{0.5cm}
    \conceptbox{Criterio de Selección}{
        Volumen × Velocidad × Variedad = Paradigma adecuado
    }
\end{frame}

\begin{frame}{Técnicas de Procesamiento Espacial}
    \begin{columns}[T]
        \column{0.33\textwidth}
        \textbf{Geocoding}
        \begin{itemize}
            \item Dirección → Coordenadas
            \item Nominatim
            \item Google Geocoding API
            \item Pelias
        \end{itemize}
        
        \vspace{0.3cm}
        \textit{Caso de uso:} Localizar clientes
        
        \column{0.33\textwidth}
        \textbf{Clustering Espacial}
        \begin{itemize}
            \item DBSCAN
            \item K-means espacial
            \item Hierarchical clustering
            \item OPTICS
        \end{itemize}
        
        \vspace{0.3cm}
        \textit{Caso de uso:} Zonas comerciales
        
        \column{0.34\textwidth}
        \textbf{Análisis de Redes}
        \begin{itemize}
            \item Shortest path
            \item Isócronas
            \item Service areas
            \item Network flow
        \end{itemize}
        
        \vspace{0.3cm}
        \textit{Caso de uso:} Rutas óptimas
    \end{columns}
    
    \vspace{0.5cm}
    \begin{tcolorbox}[colframe=green!70,colback=green!10]
        \centering\faLightbulb\space \textbf{Tip:} Siempre valida tus resultados con visualización
    \end{tcolorbox}
\end{frame}

\begin{frame}{Optimización y Escalabilidad}
    \begin{columns}[T]
        \column{0.5\textwidth}
        \textbf{Estrategias de Optimización:}
        \begin{itemize}
            \item \textbf{Índices espaciales:} R-tree, Quadtree
            \item \textbf{Particionamiento:} Por región, tiempo
            \item \textbf{Materialización:} Vistas precalculadas
            \item \textbf{Simplificación:} Douglas-Peucker
            \item \textbf{Paralelización:} Dask, Ray
        \end{itemize}
        
        \column{0.5\textwidth}
        \begin{tikzpicture}[scale=0.6, every node/.style={scale=0.6}]
            \begin{axis}[
                xlabel={Volumen de datos (GB)},
                ylabel={Tiempo (segundos)},
                legend pos=north west,
                grid=major,
                width=8cm,
                height=5cm
            ]
            \addplot[color=red, thick] coordinates {
                (1,10) (10,100) (100,1000) (1000,10000)
            };
            \addplot[color=blue, thick] coordinates {
                (1,5) (10,20) (100,80) (1000,320)
            };
            \addplot[color=green, thick] coordinates {
                (1,3) (10,10) (100,35) (1000,120)
            };
            \legend{Sin optimizar, Con índices, Con Dask}
            \end{axis}
        \end{tikzpicture}
    \end{columns}
\end{frame}

\section{APIs y Servicios}

\begin{frame}{Arquitectura de Microservicios Geoespaciales}
    \begin{tikzpicture}[scale=0.7, every node/.style={scale=0.7}]
        % API Gateway
        \node[draw, rectangle, fill=orange!30, minimum width=10cm, minimum height=1cm] (gateway) at (0,4) {\textbf{API Gateway}};
        
        % Microservicios
        \node[draw, rectangle, fill=blue!20, minimum width=2cm] (geocoding) at (-3,2) {Geocoding\\Service};
        \node[draw, rectangle, fill=green!20, minimum width=2cm] (routing) at (0,2) {Routing\\Service};
        \node[draw, rectangle, fill=yellow!20, minimum width=2cm] (tiles) at (3,2) {Tile\\Service};
        
        % Bases de datos
        \node[draw, cylinder, shape aspect=2, fill=gray!20] (db1) at (-3,0) {PostGIS};
        \node[draw, cylinder, shape aspect=2, fill=gray!20] (db2) at (0,0) {pgRouting};
        \node[draw, cylinder, shape aspect=2, fill=gray!20] (db3) at (3,0) {TileCache};
        
        % Message Queue
        \node[draw, rectangle, fill=purple!20, minimum width=8cm, minimum height=0.8cm] (queue) at (0,-2) {Message Queue (RabbitMQ/Kafka)};
        
        % Conexiones
        \draw[<->, thick] (gateway) -- (geocoding);
        \draw[<->, thick] (gateway) -- (routing);
        \draw[<->, thick] (gateway) -- (tiles);
        \draw[<->, thick] (geocoding) -- (db1);
        \draw[<->, thick] (routing) -- (db2);
        \draw[<->, thick] (tiles) -- (db3);
        \draw[<->, thick] (geocoding) -- (queue);
        \draw[<->, thick] (routing) -- (queue);
        \draw[<->, thick] (tiles) -- (queue);
    \end{tikzpicture}
    
    \conceptbox{Ventajas}{
        Escalabilidad independiente • Despliegue granular • Tolerancia a fallos • Tecnología heterogénea
    }
\end{frame}

\begin{frame}{Diseño de APIs Geoespaciales}
    \conceptbox{Principios REST para Geo-APIs}{
        \begin{itemize}
            \item \textbf{Recursos claros:} /api/v1/places, /api/v1/routes
            \item \textbf{Filtros espaciales:} ?within=polygon\&distance=1000
            \item \textbf{Formatos estándar:} GeoJSON, WKT, KML
            \item \textbf{Paginación:} Especialmente importante con geometrías
        \end{itemize}
    }
    
    \begin{columns}[T]
        \column{0.5\textwidth}
        \textbf{Endpoints típicos:}
        \begin{itemize}
            \item GET /pois?type=hospital
            \item POST /geocode
            \item GET /route?from=A\&to=B
            \item GET /isochrone?point=x,y
            \item POST /spatial-query
        \end{itemize}
        
        \column{0.5\textwidth}
        \textbf{Consideraciones:}
        \begin{itemize}
            \item Rate limiting
            \item Caché de resultados
            \item Compresión gzip
            \item CORS headers
            \item API keys/OAuth
        \end{itemize}
    \end{columns}
\end{frame}

\begin{frame}{Estándares OGC}
    \conceptbox{Open Geospatial Consortium}{
        Organización que desarrolla estándares abiertos para datos y servicios geoespaciales
    }
    
    \begin{columns}[T]
        \column{0.5\textwidth}
        \textbf{Servicios Web OGC:}
        \begin{itemize}
            \item \textbf{WMS:} Web Map Service
            \item \textbf{WFS:} Web Feature Service
            \item \textbf{WCS:} Web Coverage Service
            \item \textbf{WPS:} Web Processing Service
            \item \textbf{CSW:} Catalog Service
        \end{itemize}
        
        \column{0.5\textwidth}
        \textbf{Formatos OGC:}
        \begin{itemize}
            \item \textbf{GML:} Geography Markup Language
            \item \textbf{KML:} Keyhole Markup Language
            \item \textbf{GeoPackage:} SQLite espacial
            \item \textbf{CityGML:} Modelos 3D urbanos
        \end{itemize}
    \end{columns}
    
    \vspace{0.3cm}
    \alertbox{Usar estándares OGC garantiza interoperabilidad entre sistemas}
\end{frame}

\section{Visualización y Dashboards}

\begin{frame}{Estrategias de Visualización}
    \begin{tikzpicture}[scale=0.7, every node/.style={scale=0.7}]
        % Tipos de visualización
        \node[draw, rectangle, fill=blue!20, minimum width=3cm] (static) at (0,0) {Mapas Estáticos};
        \node[draw, rectangle, fill=green!20, minimum width=3cm] (interactive) at (4,0) {Mapas Interactivos};
        \node[draw, rectangle, fill=yellow!20, minimum width=3cm] (dashboard) at (8,0) {Dashboards};
        \node[draw, rectangle, fill=orange!20, minimum width=3cm] (3d) at (0,-2) {Visualización 3D};
        \node[draw, rectangle, fill=red!20, minimum width=3cm] (ar) at (4,-2) {Realidad Aumentada};
        \node[draw, rectangle, fill=purple!20, minimum width=3cm] (real) at (8,-2) {Tiempo Real};
        
        % Tecnologías
        \node[below=0.2cm of static, text=gray] {\footnotesize Matplotlib, QGIS};
        \node[below=0.2cm of interactive, text=gray] {\footnotesize Folium, Leaflet};
        \node[below=0.2cm of dashboard, text=gray] {\footnotesize Streamlit, Dash};
        \node[below=0.2cm of 3d, text=gray] {\footnotesize Cesium, Deck.gl};
        \node[below=0.2cm of ar, text=gray] {\footnotesize ARCore, ARKit};
        \node[below=0.2cm of real, text=gray] {\footnotesize WebSockets, SSE};
    \end{tikzpicture}
    
    \vspace{0.5cm}
    \conceptbox{Principio de Visualización}{
        La mejor visualización es aquella que comunica la información de manera clara y permite tomar decisiones informadas
    }
\end{frame}

\begin{frame}{Mejores Prácticas de Visualización}
    \begin{columns}[T]
        \column{0.5\textwidth}
        \textbf{Do's:} \faCheckCircle
        \begin{itemize}
            \item Usa proyecciones apropiadas
            \item Incluye leyendas claras
            \item Optimiza para el dispositivo
            \item Permite interacción
            \item Muestra contexto
            \item Usa colores accesibles
        \end{itemize}
        
        \column{0.5\textwidth}
        \textbf{Don'ts:} \faTimesCircle
        \begin{itemize}
            \item Sobrecargar con información
            \item Ignorar la escala
            \item Usar proyecciones incorrectas
            \item Olvidar la fuente de datos
            \item Abusar de efectos 3D
            \item Ignorar el rendimiento
        \end{itemize}
    \end{columns}
    
    \vspace{0.5cm}
    \begin{tcolorbox}[colframe=blue!70,colback=blue!10]
        \centering\faInfo\space \textbf{Regla 5-segundo:} El usuario debe entender el mapa en 5 segundos
    \end{tcolorbox}
\end{frame}

\section{Deployment y DevOps}

\begin{frame}{Estrategias de Escalamiento}
    \begin{columns}[T]
        \column{0.5\textwidth}
        \textbf{Escalamiento Vertical:}
        \begin{itemize}
            \item \faArrowUp\space Más CPU/RAM
            \item \faServer\space Servidor más potente
            \item \faDollarSign\space Costo exponencial
            \item \faExclamationTriangle\space Límite físico
        \end{itemize}
        
        \textit{Cuándo usar:}
        \begin{itemize}
            \item PostGIS principal
            \item Cálculos complejos
            \item Datos correlacionados
        \end{itemize}
        
        \column{0.5\textwidth}
        \textbf{Escalamiento Horizontal:}
        \begin{itemize}
            \item \faArrowsAltH\space Más nodos
            \item \faClone\space Replicación
            \item \faChartLine\space Costo lineal
            \item \faInfinity\space Sin límite teórico
        \end{itemize}
        
        \textit{Cuándo usar:}
        \begin{itemize}
            \item APIs stateless
            \item Procesamiento paralelo
            \item Cache distribuido
        \end{itemize}
    \end{columns}
    
    \vspace{0.5cm}
    \begin{tcolorbox}[colframe=blue!70,colback=blue!10]
        \centering\faLightbulb\space \textbf{Patrón híbrido:} Escala vertical para BD, horizontal para servicios
    \end{tcolorbox}
\end{frame}

\begin{frame}{Containerización con Docker}
    \conceptbox{¿Por qué Docker para Geo?}{
        \begin{itemize}
            \item \textbf{Reproducibilidad:} Mismo ambiente en desarrollo y producción
            \item \textbf{Dependencias complejas:} GDAL, GEOS, PROJ fácilmente instaladas
            \item \textbf{Escalabilidad:} Kubernetes para orquestación
            \item \textbf{Aislamiento:} Cada servicio en su contenedor
        \end{itemize}
    }
    
    \textbf{Stack típico con Docker Compose:}
    \begin{columns}[T]
        \column{0.5\textwidth}
        \begin{itemize}
            \item PostGIS database
            \item Redis cache
            \item API backend (FastAPI)
            \item Frontend (React/Vue)
            \item Nginx proxy
        \end{itemize}
        
        \column{0.5\textwidth}
        \begin{itemize}
            \item GeoServer
            \item Jupyter notebooks
            \item pgAdmin
            \item Grafana monitoring
            \item ElasticSearch logs
        \end{itemize}
    \end{columns}
\end{frame}

\begin{frame}{CI/CD para Pipelines Geoespaciales}
    \begin{tikzpicture}[scale=0.7, every node/.style={scale=0.7}]
        % Pipeline CI/CD
        \node[draw, rectangle, fill=blue!20] (dev) at (0,0) {Desarrollo};
        \node[draw, rectangle, fill=green!20] (test) at (3,0) {Testing};
        \node[draw, rectangle, fill=yellow!20] (build) at (6,0) {Build};
        \node[draw, rectangle, fill=orange!20] (deploy) at (9,0) {Deploy};
        \node[draw, rectangle, fill=red!20] (monitor) at (12,0) {Monitor};
        
        \draw[->, thick] (dev) -- (test);
        \draw[->, thick] (test) -- (build);
        \draw[->, thick] (build) -- (deploy);
        \draw[->, thick] (deploy) -- (monitor);
        \draw[->, thick, bend right=45] (monitor) to (dev);
        
        % Herramientas
        \node[below=0.5cm of dev] {\footnotesize Git, VS Code};
        \node[below=0.5cm of test] {\footnotesize Pytest, Coverage};
        \node[below=0.5cm of build] {\footnotesize Docker, GitHub Actions};
        \node[below=0.5cm of deploy] {\footnotesize K8s, AWS/GCP};
        \node[below=0.5cm of monitor] {\footnotesize Prometheus, Grafana};
    \end{tikzpicture}
    
    \vspace{0.3cm}
    \alertbox{Automatiza todo lo que puedas, especialmente validación de datos espaciales}
\end{frame}

\section{Casos de Uso Reales}

\begin{frame}{Patrones Arquitectónicos Comunes}
    \begin{columns}[T]
        \column{0.33\textwidth}
        \textbf{Lambda Architecture}
        \begin{tikzpicture}[scale=0.5]
            \node[draw, rectangle, fill=blue!20] (batch) at (0,2) {Batch};
            \node[draw, rectangle, fill=green!20] (speed) at (0,0) {Speed};
            \node[draw, rectangle, fill=yellow!20] (serve) at (2,1) {Serving};
            \draw[->] (batch) -- (serve);
            \draw[->] (speed) -- (serve);
        \end{tikzpicture}
        
        \textit{Uso:} Análisis histórico + real-time
        
        \column{0.33\textwidth}
        \textbf{Kappa Architecture}
        \begin{tikzpicture}[scale=0.5]
            \node[draw, rectangle, fill=green!20] (stream) at (0,1) {Stream};
            \node[draw, rectangle, fill=yellow!20] (serve) at (2,1) {Serving};
            \draw[->] (stream) -- (serve);
        \end{tikzpicture}
        
        \textit{Uso:} Solo streaming, reprocesar si necesario
        
        \column{0.34\textwidth}
        \textbf{Event Sourcing}
        \begin{tikzpicture}[scale=0.5]
            \node[draw, rectangle, fill=purple!20] (events) at (0,1) {Events};
            \node[draw, rectangle, fill=orange!20] (state) at (2,1) {State};
            \draw[->] (events) -- (state);
        \end{tikzpicture}
        
        \textit{Uso:} Auditoría completa, time-travel
    \end{columns}
    
    \vspace{0.5cm}
    \alertbox{Elige el patrón según requisitos de latencia y consistencia}
\end{frame}

\begin{frame}{Caso 1: Sistema de Routing Urbano}
    \begin{columns}[T]
        \column{0.5\textwidth}
        \textbf{Problema:}
        \begin{itemize}
            \item Optimizar rutas de delivery
            \item 1000+ pedidos diarios
            \item Ventanas de tiempo
            \item Tráfico en tiempo real
        \end{itemize}
        
        \vspace{0.3cm}
        \textbf{Solución:}
        \begin{itemize}
            \item OSM para red vial
            \item pgRouting para cálculo
            \item Redis para caché
            \item WebSocket para actualizaciones
        \end{itemize}
        
        \column{0.5\textwidth}
        \textbf{Pipeline implementado:}
        \begin{enumerate}
            \item Geocoding de direcciones
            \item Clustering por zonas
            \item Cálculo de rutas óptimas
            \item Asignación a conductores
            \item Tracking en tiempo real
            \item Analytics post-delivery
        \end{enumerate}
        
        \vspace{0.3cm}
        \conceptbox{Resultado}{
            30\% reducción en tiempo de entrega
        }
    \end{columns}
\end{frame}

\begin{frame}{Caso 2: Análisis de Mercado Inmobiliario}
    \begin{columns}[T]
        \column{0.5\textwidth}
        \textbf{Objetivo:}
        Predecir precios de propiedades basado en ubicación y amenidades cercanas
        
        \vspace{0.3cm}
        \textbf{Datos utilizados:}
        \begin{itemize}
            \item Propiedades históricas
            \item POIs (colegios, metro, parques)
            \item Demografía por zona
            \item Calidad del aire
            \item Ruido ambiental
        \end{itemize}
        
        \column{0.5\textwidth}
        \textbf{Técnicas aplicadas:}
        \begin{itemize}
            \item Buffer analysis (500m, 1km)
            \item Spatial join con amenidades
            \item Kriging para interpolación
            \item Random Forest con features espaciales
            \item Validación cruzada espacial
        \end{itemize}
        
        \vspace{0.3cm}
        \conceptbox{Precisión}{
            R² = 0.87 en predicción de precios
        }
    \end{columns}
\end{frame}

\section{Testing y Calidad}

\begin{frame}{Estrategias de Testing para Pipelines Geoespaciales}
    \begin{columns}[T]
        \column{0.5\textwidth}
        \textbf{Niveles de Testing:}
        \begin{itemize}
            \item \textbf{Unit Tests:} Funciones individuales
            \item \textbf{Integration:} Componentes conectados
            \item \textbf{System:} Pipeline completo
            \item \textbf{Acceptance:} Requisitos de negocio
        \end{itemize}
        
        \vspace{0.3cm}
        \textbf{Tests Específicos Geo:}
        \begin{itemize}
            \item Validación de geometrías
            \item Proyecciones correctas
            \item Topología consistente
            \item Precisión espacial
        \end{itemize}
        
        \column{0.5\textwidth}
        \begin{tikzpicture}[scale=0.7]
            % Pirámide de testing
            \draw[fill=green!30] (0,0) -- (4,0) -- (3,1) -- (1,1) -- cycle;
            \draw[fill=yellow!30] (1,1) -- (3,1) -- (2.5,2) -- (1.5,2) -- cycle;
            \draw[fill=orange!30] (1.5,2) -- (2.5,2) -- (2.2,2.8) -- (1.8,2.8) -- cycle;
            \draw[fill=red!30] (1.8,2.8) -- (2.2,2.8) -- (2,3.3) -- cycle;
            
            \node at (2,0.5) {\small Unit};
            \node at (2,1.5) {\small Integration};
            \node at (2,2.4) {\small System};
            \node at (2,3) {\tiny E2E};
            
            \node[right=0.5cm] at (4,0.5) {\small 70\%};
            \node[right=0.5cm] at (3,1.5) {\small 20\%};
            \node[right=0.5cm] at (2.5,2.4) {\small 8\%};
            \node[right=0.5cm] at (2.2,3) {\small 2\%};
        \end{tikzpicture}
        
        \vspace{0.3cm}
        \conceptbox{Cobertura Objetivo}{
            Mínimo 80\% para código crítico
        }
    \end{columns}
\end{frame}

\begin{frame}{Monitoreo y Observabilidad}
    \begin{tikzpicture}[scale=0.8, every node/.style={scale=0.8}]
        % Los tres pilares
        \node[draw, circle, fill=blue!30, minimum size=2.5cm] (logs) at (0,0) {\textbf{Logs}\\\small Eventos\\discretos};
        \node[draw, circle, fill=green!30, minimum size=2.5cm] (metrics) at (3,0) {\textbf{Métricas}\\\small Valores\\numéricos};
        \node[draw, circle, fill=yellow!30, minimum size=2.5cm] (traces) at (1.5,-2) {\textbf{Traces}\\\small Flujo de\\requests};
        
        % Intersección
        \node at (1.5,0) {\footnotesize Correlación};
        
        % Herramientas
        \node[below=0.3cm of logs] {\tiny ELK Stack};
        \node[below=0.3cm of metrics] {\tiny Prometheus};
        \node[below=0.3cm of traces] {\tiny Jaeger};
    \end{tikzpicture}
    
    \vspace{0.3cm}
    \conceptbox{Métricas Clave para Geo-Pipelines}{
        \begin{itemize}
            \item \textbf{Latencia:} Tiempo de procesamiento por geometría
            \item \textbf{Throughput:} Features procesadas/segundo
            \item \textbf{Error rate:} Geometrías inválidas
            \item \textbf{Uso de recursos:} CPU/RAM/Disco
        \end{itemize}
    }
\end{frame}

\section{Mejores Prácticas y Antipatrones}

\begin{frame}{Mejores Prácticas}
    \begin{columns}[T]
        \column{0.5\textwidth}
        \textbf{\faCheckCircle\space Datos:}
        \begin{itemize}
            \item Validar geometrías siempre
            \item Mantener CRS consistente
            \item Documentar fuentes
            \item Versionar cambios
            \item Implementar data lineage
        \end{itemize}
        
        \vspace{0.3cm}
        \textbf{\faCheckCircle\space Procesamiento:}
        \begin{itemize}
            \item Usar índices espaciales
            \item Cachear resultados costosos
            \item Paralelizar cuando sea posible
            \item Monitorear performance
            \item Implementar circuit breakers
        \end{itemize}
        
        \column{0.5\textwidth}
        \textbf{\faCheckCircle\space Arquitectura:}
        \begin{itemize}
            \item Separar responsabilidades
            \item Usar colas para async
            \item Implementar retry logic
            \item Logs estructurados
            \item Health checks
        \end{itemize}
        
        \vspace{0.3cm}
        \textbf{\faCheckCircle\space Seguridad:}
        \begin{itemize}
            \item Sanitizar inputs espaciales
            \item Rate limiting en APIs
            \item Encriptar datos sensibles
            \item Auditar accesos
            \item Backup regular
        \end{itemize}
    \end{columns}
\end{frame}

\begin{frame}{Antipatrones Comunes}
    \begin{columns}[T]
        \column{0.5\textwidth}
        \textbf{\faTimesCircle\space No hagas esto:}
        \begin{itemize}
            \item Ignorar proyecciones
            \item Procesar todo en memoria
            \item Hardcodear coordenadas
            \item Olvidar validación
            \item Mezclar CRS
            \item SQL injection con WKT
        \end{itemize}
        
        \column{0.5\textwidth}
        \textbf{\faExclamationTriangle\space Consecuencias:}
        \begin{itemize}
            \item Resultados incorrectos
            \item Out of memory
            \item Código no portable
            \item Datos corruptos
            \item Cálculos erróneos
            \item Vulnerabilidades
        \end{itemize}
    \end{columns}
    
    \vspace{0.5cm}
    \alertbox{El 80\% de los errores en geoinformática se deben a problemas con proyecciones y validación de datos}
\end{frame}

\section{Desafíos y Soluciones}

\begin{frame}{Desafíos Comunes en Pipelines Geoespaciales}
    \begin{columns}[T]
        \column{0.5\textwidth}
        \textbf{\faExclamationTriangle\space Desafío 1: Volumen de Datos}
        \begin{itemize}
            \item Terabytes de imágenes satelitales
            \item Millones de puntos GPS
        \end{itemize}
        \textit{Solución:} Particionamiento + Procesamiento paralelo
        
        \vspace{0.3cm}
        \textbf{\faExclamationTriangle\space Desafío 2: Heterogeneidad}
        \begin{itemize}
            \item Múltiples formatos
            \item Diferentes CRS
        \end{itemize}
        \textit{Solución:} ETL robusto + Estandarización
        
        \column{0.5\textwidth}
        \textbf{\faExclamationTriangle\space Desafío 3: Tiempo Real}
        \begin{itemize}
            \item Actualizaciones constantes
            \item Baja latencia requerida
        \end{itemize}
        \textit{Solución:} Stream processing + Caché
        
        \vspace{0.3cm}
        \textbf{\faExclamationTriangle\space Desafío 4: Calidad de Datos}
        \begin{itemize}
            \item Geometrías inválidas
            \item Datos faltantes
        \end{itemize}
        \textit{Solución:} Validación + Limpieza automática
    \end{columns}
    
    \vspace{0.3cm}
    \alertbox{El 60\% del tiempo en un proyecto geo se invierte en limpieza de datos}
\end{frame}

\begin{frame}{Matriz de Decisión Tecnológica}
    \begin{table}
        \centering
        \small
        \begin{tabular}{|l|c|c|c|c|}
            \hline
            \textbf{Criterio} & \textbf{PostGIS} & \textbf{MongoDB} & \textbf{Elasticsearch} & \textbf{BigQuery} \\
            \hline
            Consultas espaciales & \faCheckCircle & \faCircle & \faTimesCircle & \faCheckCircle \\
            \hline
            Escalabilidad & \faCircle & \faCheckCircle & \faCheckCircle & \faCheckCircle \\
            \hline
            ACID & \faCheckCircle & \faCircle & \faTimesCircle & \faCircle \\
            \hline
            Costo & \faCheckCircle & \faCheckCircle & \faCircle & \faTimesCircle \\
            \hline
            Tiempo real & \faCircle & \faCheckCircle & \faCheckCircle & \faCircle \\
            \hline
            Análisis complejo & \faCheckCircle & \faTimesCircle & \faCircle & \faCheckCircle \\
            \hline
        \end{tabular}
    \end{table}
    
    \vspace{0.3cm}
    \conceptbox{Recomendación}{
        No existe una solución única. Combina tecnologías según tus necesidades específicas.
    }
\end{frame}

\section{Herramientas y Recursos}

\begin{frame}{Stack Tecnológico Recomendado}
    \begin{columns}[T]
        \column{0.33\textwidth}
        \textbf{Backend:}
        \begin{itemize}
            \item Python 3.9+
            \item GeoPandas
            \item Shapely
            \item Rasterio
            \item PostGIS
            \item FastAPI
        \end{itemize}
        
        \column{0.33\textwidth}
        \textbf{Frontend:}
        \begin{itemize}
            \item Leaflet/OpenLayers
            \item Mapbox GL JS
            \item Deck.gl
            \item Folium
            \item Streamlit
            \item Dash
        \end{itemize}
        
        \column{0.34\textwidth}
        \textbf{DevOps:}
        \begin{itemize}
            \item Docker
            \item GitHub Actions
            \item Terraform
            \item Kubernetes
            \item Prometheus
            \item Grafana
        \end{itemize}
    \end{columns}
    
    \vspace{0.5cm}
    \conceptbox{Criterios de Selección}{
        Elige herramientas basándote en: comunidad activa, documentación, licencia, y curva de aprendizaje
    }
\end{frame}

\begin{frame}{Recursos de Aprendizaje}
    \begin{columns}[T]
        \column{0.5\textwidth}
        \textbf{Documentación:}
        \begin{itemize}
            \item \faBook\space PostGIS in Action
            \item \faGlobe\space geopandas.org
            \item \faGithub\space Awesome GIS
            \item \faYoutube\space PyGIS tutorials
            \item \faGraduationCap\space Coursera GIS
        \end{itemize}
        
        \vspace{0.3cm}
        \textbf{Comunidades:}
        \begin{itemize}
            \item \faUsers\space OSGeo
            \item \faReddit\space r/gis
            \item \faStackOverflow\space GIS Stack Exchange
            \item \faSlack\space GeoPython
        \end{itemize}
        
        \column{0.5\textwidth}
        \textbf{Datasets de práctica:}
        \begin{itemize}
            \item Natural Earth Data
            \item OpenStreetMap extracts
            \item GADM boundaries
            \item NASA Earthdata
            \item Copernicus Open Access Hub
        \end{itemize}
        
        \vspace{0.3cm}
        \textbf{Herramientas online:}
        \begin{itemize}
            \item geojson.io
            \item kepler.gl
            \item Overpass Turbo
            \item QGIS Cloud
        \end{itemize}
    \end{columns}
\end{frame}

\section{Conclusiones}

\begin{frame}{Resumen de la Clase}
    \begin{columns}[T]
        \column{0.5\textwidth}
        \textbf{Aprendimos:}
        \begin{itemize}
            \item[\faCheckCircle] Qué es un pipeline geoespacial
            \item[\faCheckCircle] Arquitectura en capas
            \item[\faCheckCircle] Fuentes de datos
            \item[\faCheckCircle] Almacenamiento con PostGIS
            \item[\faCheckCircle] Procesamiento escalable
            \item[\faCheckCircle] APIs y servicios
            \item[\faCheckCircle] Visualización efectiva
            \item[\faCheckCircle] Deployment con Docker
        \end{itemize}
        
        \column{0.5\textwidth}
        \textbf{Conceptos clave:}
        \begin{itemize}
            \item Automatización
            \item Reproducibilidad
            \item Escalabilidad
            \item Interoperabilidad
            \item Monitoreo
            \item Optimización
            \item Validación
            \item Documentación
        \end{itemize}
    \end{columns}
    
    \vspace{0.3cm}
    \conceptbox{Mensaje Final}{
        Un buen pipeline geoespacial no es el que usa más tecnología, sino el que resuelve el problema de manera eficiente y mantenible
    }
\end{frame}

\begin{frame}{Próximos Pasos}
    \textbf{Para consolidar lo aprendido:}
    \begin{enumerate}
        \item \faCode\space Completa los notebooks de práctica
        \item \faDatabase\space Experimenta con PostGIS localmente
        \item \faRocket\space Construye tu primer pipeline completo
        \item \faUsers\space Comparte tu proyecto con la comunidad
    \end{enumerate}
    
    \vspace{0.5cm}
    \textbf{Proyecto sugerido:}
    \conceptbox{Sistema de Análisis Urbano}{
        Construye un pipeline que:
        \begin{itemize}
            \item Descargue datos de OSM de tu ciudad
            \item Identifique zonas comerciales (clustering)
            \item Calcule accesibilidad a transporte público
            \item Genere un dashboard interactivo
            \item Exponga una API REST
        \end{itemize}
    }
\end{frame}

\begin{frame}[standout]
    \Huge ¿Preguntas?
    
    \vspace{1cm}
    \Large
    \faEnvelope\space fparra@usach.cl
    
    \faGithub\space github.com/fparrao
    
    \vspace{1cm}
    \normalsize
    Material disponible en el repositorio del curso
\end{frame}

\end{document}