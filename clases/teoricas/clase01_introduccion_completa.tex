\documentclass[10pt]{beamer}
\usetheme{metropolis}
\usepackage{FiraSans}
\usefonttheme{professionalfonts}

\usepackage{graphicx}
\usepackage{tikz}
\usepackage[spanish]{babel}
\usepackage{tcolorbox}
\usepackage{ragged2e}
\usepackage{pgfplots}
\pgfplotsset{compat=1.18}
\usepackage{listings}
\usepackage{xcolor}
\usepackage{hyperref}
\usepackage{fontawesome5}

\definecolor{codegreen}{rgb}{0,0.6,0}
\definecolor{codegray}{rgb}{0.5,0.5,0.5}
\definecolor{codepurple}{rgb}{0.58,0,0.82}
\definecolor{backcolour}{rgb}{0.95,0.95,0.92}
\definecolor{usachblue}{RGB}{0,121,192}
\definecolor{usachred}{RGB}{239,51,64}

\lstdefinestyle{mystyle}{
    backgroundcolor=\color{backcolour},   
    commentstyle=\color{codegreen},
    keywordstyle=\color{magenta},
    numberstyle=\tiny\color{codegray},
    stringstyle=\color{codepurple},
    basicstyle=\ttfamily\footnotesize,
    breakatwhitespace=false,         
    breaklines=true,                 
    captionpos=b,                    
    keepspaces=true,                 
    numbers=left,                    
    numbersep=5pt,                  
    showspaces=false,                
    showstringspaces=false,
    showtabs=false,                  
    tabsize=2
}

\lstset{style=mystyle}

\newcommand{\examplebox}[2]{
\begin{tcolorbox}[colframe=usachblue,colback=white,title=#1]
#2
\end{tcolorbox}
}

% Personalización del pie de página
\setbeamertemplate{footline}{%
  \begin{beamercolorbox}[wd=\paperwidth,sep=2ex]{footline}%
    \usebeamerfont{structure}\textbf{Geoinformática - Clase 1} \hfill Profesor: Francisco Parra O. \hfill \textbf{Semestre 2, 2025}
  \end{beamercolorbox}%
}

% Configuración de bullets
\setbeamertemplate{itemize items}[circle]
\setbeamertemplate{enumerate items}[default]

\title{Clase 01: Introducción a la Geoinformática}
\subtitle{Fundamentos y aplicaciones de la Geocomputación}
\author{Profesor: Francisco Parra O.}
\institute{USACH - Ingeniería Civil en Informática}
\date{\today}

\titlegraphic{%
  \begin{tikzpicture}[overlay, remember picture]
    \node[anchor=north east, yshift=0cm] at (current page.north east) {
      \includegraphics[width=1.5cm]{../../logo.jpg}
    };
  \end{tikzpicture}
}

\begin{document}

\maketitle

% SLIDE 2 - Agenda (2 min)
\begin{frame}{Agenda de hoy}
    \tableofcontents
    
    \vspace{0.5cm}
    \begin{center}
        \textcolor{usachblue}{\faIcon{clock}} \textbf{Duración: 80 minutos}
    \end{center}
\end{frame}

% SECCIÓN 1: PRESENTACIÓN (10 minutos)
\section{Presentación del curso}

% SLIDE 3 - Bienvenida (2 min)
\begin{frame}{Bienvenida al curso}
    \begin{columns}
        \column{0.6\textwidth}
        \textbf{Profesor:} Francisco Parra O.
        \begin{itemize}
            \item Geólogo
            \item Doctorado en Informática
            \item Especialización en Sistemas de Información Geográfica
            \item Consultor en proyectos de geoinformática
        \end{itemize}
        
        \vspace{0.5cm}
        \textbf{Contacto:}
        \begin{itemize}
            \item \faIcon{envelope} francisco.parra.o@usach.cl
            \item \faIcon{clock} Horario consultas: Por definir
        \end{itemize}
        
        \column{0.4\textwidth}
        \begin{center}
            \begin{tikzpicture}
                \node[draw=usachblue, line width=2pt, circle, minimum size=3cm, fill=blue!10] {
                    \begin{tabular}{c}
                        \Large\textbf{GEO} \\[0.2cm]
                        \Large\textbf{INFO} \\[0.2cm]
                        \Large\textbf{2025}
                    \end{tabular}
                };
            \end{tikzpicture}
        \end{center}
    \end{columns}
\end{frame}

% SLIDE 4 - Estructura del curso (3 min)
\begin{frame}{Estructura del curso}
    \begin{columns}
        \column{0.5\textwidth}
        \textbf{Modalidad semanal:}
        \begin{itemize}
            \item \textcolor{usachblue}{\faIcon{book}} \textbf{Martes:} Clase teórica (1:20 hrs)
            \item \textcolor{usachblue}{\faIcon{chalkboard-teacher}} \textbf{Jueves:} Clase teórica (1:20 hrs)
            \item \textcolor{usachred}{\faIcon{laptop-code}} \textbf{Jueves:} Laboratorio (1:20 hrs)
        \end{itemize}
        
        \vspace{0.3cm}
        \textbf{16 semanas efectivas}
        \begin{itemize}
            \item 30 clases teóricas
            \item 16 laboratorios prácticos
            \item 1 proyecto semestral
        \end{itemize}
        
        \column{0.5\textwidth}
        \begin{tikzpicture}[scale=0.8]
            \pie[text=pin, radius=2]{
                30/Unidad 1: Fundamentos,
                40/Unidad 2: Operaciones,
                30/Unidad 3: Aplicaciones
            }
        \end{tikzpicture}
    \end{columns}
\end{frame}

% SLIDE 5 - Resultados de aprendizaje (3 min)
\begin{frame}{Resultados de aprendizaje}
    \examplebox{Al finalizar el curso serás capaz de:}{
        \begin{enumerate}
            \item \textbf{Comprender} los fundamentos de datos geoespaciales
            \item \textbf{Manipular} datos vectoriales y raster usando R/Python
            \item \textbf{Analizar} fenómenos espaciales con técnicas computacionales
            \item \textbf{Visualizar} información geográfica de manera efectiva
            \item \textbf{Desarrollar} soluciones a problemas reales con componente espacial
            \item \textbf{Automatizar} flujos de trabajo geoespaciales
        \end{enumerate}
    }
\end{frame}

% SLIDE 6 - Evaluación (2 min)
\begin{frame}{Sistema de evaluación}
    \begin{center}
        \Large{\textcolor{usachred}{\textbf{100\% Proyecto Semestral}}}
    \end{center}
    
    \vspace{0.5cm}
    
    \begin{columns}
        \column{0.5\textwidth}
        \textbf{Opción 1: Proyecto Comercial}
        \begin{itemize}
            \item Solución para empresa/negocio
            \item Geomarketing
            \item Logística y rutas
            \item Análisis inmobiliario
        \end{itemize}
        
        \column{0.5\textwidth}
        \textbf{Opción 2: Proyecto Científico}
        \begin{itemize}
            \item Investigación aplicada
            \item Análisis ambiental
            \item Modelado de fenómenos
            \item Salud pública espacial
        \end{itemize}
    \end{columns}
    
    \vspace{0.5cm}
    \begin{center}
        \textcolor{gray}{\small Grupos de 1-3 personas | Presentación en diciembre}
    \end{center}
\end{frame}

% SECCIÓN 2: INTRODUCCIÓN A LA GEOCOMPUTACIÓN (25 minutos)
\section{¿Qué es la Geocomputación?}

% SLIDE 7 - Definición (3 min)
\begin{frame}{Definición de Geocomputación}
    \begin{tcolorbox}[colframe=usachblue,colback=blue!5]
        \textbf{Geocomputación:} Aplicación de técnicas computacionales avanzadas para resolver problemas espaciales complejos, integrando ciencias de la computación, geografía y estadística.
    \end{tcolorbox}
    
    \vspace{0.5cm}
    
    \textbf{Componentes clave:}
    \begin{columns}
        \column{0.33\textwidth}
        \begin{center}
            \textcolor{usachblue}{\faIcon{globe} \\ \textbf{Geografía}}\\
            Comprensión del espacio y lugar
        \end{center}
        
        \column{0.33\textwidth}
        \begin{center}
            \textcolor{usachred}{\faIcon{laptop-code} \\ \textbf{Computación}}\\
            Algoritmos y procesamiento
        \end{center}
        
        \column{0.33\textwidth}
        \begin{center}
            \textcolor{orange}{\faIcon{chart-line} \\ \textbf{Análisis}}\\
            Estadística y modelado
        \end{center}
    \end{columns}
\end{frame}

% SLIDE 8 - Evolución histórica (4 min)
\begin{frame}{Evolución de la Geocomputación}
    \begin{tikzpicture}[scale=0.9, every node/.style={font=\small}]
        % Timeline
        \draw[thick,->] (0,0) -- (11,0) node[right] {2025};
        
        % Decades
        \foreach \x/\year in {0/1960,2/1970,4/1980,6/1990,8/2000,10/2010} {
            \draw (\x,0.1) -- (\x,-0.1) node[below] {\year};
        }
        
        % Events
        \node[above, text width=2cm, align=center] at (0,0.5) {Primeros SIG\\ (CGIS Canadá)};
        \node[below, text width=2cm, align=center] at (2,-0.5) {Estructuras\\ espaciales};
        \node[above, text width=2cm, align=center] at (4,0.5) {ArcInfo\\ comercial};
        \node[below, text width=2.5cm, align=center] at (6,-0.5) {Término\\ "Geocomputación"};
        \node[above, text width=2cm, align=center] at (8,0.5) {Google Maps\\ Web GIS};
        \node[below, text width=2cm, align=center] at (10,-0.5) {Big Data\\ espacial};
    \end{tikzpicture}
    
    \vspace{0.5cm}
    \textbf{Hitos importantes:}
    \begin{itemize}
        \item \textbf{1996:} Primera conferencia internacional de Geocomputación (Leeds, UK)
        \item \textbf{2004:} Lanzamiento de Google Earth democratiza acceso a geodatos
        \item \textbf{2018:} Geocomputation with R - libro de referencia open source
        \item \textbf{2024:} IA generativa aplicada a análisis geoespacial
    \end{itemize}
\end{frame}

% SLIDE 9 - Diferencias con SIG tradicional (3 min)
\begin{frame}{Geocomputación vs SIG tradicional}
    \begin{columns}
        \column{0.5\textwidth}
        \textbf{SIG Tradicional}
        \begin{itemize}
            \item Software de escritorio
            \item Interfaz gráfica (GUI)
            \item Operaciones predefinidas
            \item Análisis manual
            \item Datos locales
            \item Usuarios especializados
        \end{itemize}
        
        \column{0.5\textwidth}
        \textbf{Geocomputación}
        \begin{itemize}
            \item Programación y scripts
            \item Automatización completa
            \item Algoritmos personalizados
            \item Machine Learning espacial
            \item Big Data y cloud
            \item Integración con Data Science
        \end{itemize}
    \end{columns}
    
    \vspace{0.5cm}
    \begin{center}
        \textcolor{usachblue}{\faIcon{arrow-right} La Geocomputación extiende las capacidades del SIG tradicional}
    \end{center}
\end{frame}

% SLIDE 10 - Tipos de datos geoespaciales (5 min)
\begin{frame}{Tipos de datos geoespaciales}
    \begin{columns}
        \column{0.5\textwidth}
        \textbf{Datos Vectoriales}
        \begin{itemize}
            \item \textcolor{blue}{\faIcon{map-pin}} Puntos (coordenadas X,Y)
            \item \textcolor{green}{\faIcon{route}} Líneas (secuencia de puntos)
            \item \textcolor{red}{\faIcon{draw-polygon}} Polígonos (áreas cerradas)
        \end{itemize}
        
        \vspace{0.3cm}
        \textbf{Ejemplos:}
        \begin{itemize}
            \item Ubicación de hospitales
            \item Red de calles
            \item Límites comunales
        \end{itemize}
        
        \column{0.5\textwidth}
        \textbf{Datos Raster}
        \begin{itemize}
            \item Grilla regular de celdas
            \item Cada celda con un valor
            \item Resolución espacial fija
        \end{itemize}
        
        \vspace{0.3cm}
        \textbf{Ejemplos:}
        \begin{itemize}
            \item Imágenes satelitales
            \item Modelos de elevación
            \item Mapas de temperatura
        \end{itemize}
    \end{columns}
    
    \vspace{0.3cm}
    \begin{center}
        \begin{tikzpicture}[scale=0.6]
            % Vector example
            \draw[blue, thick] (0,0) circle (0.05) node[below] {Punto};
            \draw[green, thick] (1,0) -- (2,0.5) -- (3,0) node[below] {Línea};
            \draw[red, thick, fill=red!20] (4,0) -- (5,0) -- (5,1) -- (4,1) -- cycle;
            \node[below] at (4.5,0) {Polígono};
            
            % Raster example
            \foreach \x in {7,7.3,7.6,7.9} {
                \foreach \y in {0,0.3,0.6,0.9} {
                    \draw[gray] (\x,\y) rectangle (\x+0.3,\y+0.3);
                }
            }
            \node[below] at (7.6,-0.3) {Raster};
        \end{tikzpicture}
    \end{center}
\end{frame}

% SLIDE 11 - Software y herramientas (5 min)
\begin{frame}{Ecosistema de herramientas}
    \begin{columns}
        \column{0.33\textwidth}
        \textbf{Software Desktop}
        \begin{itemize}
            \item QGIS (libre)
            \item ArcGIS Pro
            \item GRASS GIS
            \item SAGA GIS
        \end{itemize}
        
        \column{0.33\textwidth}
        \textbf{Lenguajes}
        \begin{itemize}
            \item \textcolor{blue}{R} + RStudio
            \item \textcolor{green}{Python} + Jupyter
            \item JavaScript
            \item SQL espacial
        \end{itemize}
        
        \column{0.33\textwidth}
        \textbf{Bibliotecas clave}
        \begin{itemize}
            \item \texttt{sf}, \texttt{terra} (R)
            \item \texttt{geopandas} (Python)
            \item \texttt{leaflet} (web)
            \item \texttt{PostGIS} (BD)
        \end{itemize}
    \end{columns}
    
    \vspace{0.5cm}
    \examplebox{En este curso usaremos:}{
        \begin{itemize}
            \item \textbf{Python} como lenguaje principal (60\%)
            \item \textbf{R} como complemento (30\%)
            \item \textbf{QGIS} para visualización (10\%)
        \end{itemize}
    }
\end{frame}

% SLIDE 12 - Por qué Python para geoinformática (5 min)
\begin{frame}[fragile]{¿Por qué Python para Geoinformática?}
    \begin{columns}
        \column{0.5\textwidth}
        \textbf{Ventajas de Python:}
        \begin{itemize}
            \item Lenguaje más usado en Data Science
            \item Integración con Machine Learning
            \item Ecosistema geoespacial maduro
            \item Sintaxis clara y legible
            \item Multiplataforma y versátil
        \end{itemize}
        
        \column{0.5\textwidth}
        \begin{lstlisting}[language=Python]
# Ejemplo simple
import geopandas as gpd
import matplotlib.pyplot as plt

# Leer datos
comunas = gpd.read_file("comunas.shp")

# Crear mapa
comunas.plot(column='poblacion',
             cmap='Blues',
             legend=True)
plt.show()
        \end{lstlisting}
    \end{columns}
    
    \vspace{0.3cm}
    \textbf{Bibliotecas principales que usaremos:}
    \begin{itemize}
        \item \texttt{geopandas}: Manejo de datos vectoriales
        \item \texttt{rasterio}: Procesamiento de datos raster
        \item \texttt{folium}: Mapas web interactivos
        \item \texttt{shapely}: Operaciones geométricas
    \end{itemize}
\end{frame}

% SECCIÓN 3: APLICACIONES (25 minutos)
\section{Aplicaciones en el mundo real}

% SLIDE 13 - Dominios de aplicación (3 min)
\begin{frame}{Dominios de aplicación}
    \begin{center}
        \begin{tikzpicture}[scale=0.8]
            % Centro
            \node[circle, draw=usachblue, fill=blue!20, minimum size=2cm] at (0,0) {\textbf{Geo-}\\\textbf{computación}};
            
            % Aplicaciones alrededor
            \node[draw, rounded corners, fill=green!20] at (3,2) {Medio Ambiente};
            \node[draw, rounded corners, fill=red!20] at (3,-2) {Emergencias};
            \node[draw, rounded corners, fill=yellow!20] at (-3,2) {Urbanismo};
            \node[draw, rounded corners, fill=orange!20] at (-3,-2) {Negocios};
            \node[draw, rounded corners, fill=purple!20] at (0,3) {Salud Pública};
            \node[draw, rounded corners, fill=cyan!20] at (0,-3) {Transporte};
            
            % Conectores
            \draw[->] (1,0.5) -- (2.5,1.5);
            \draw[->] (1,-0.5) -- (2.5,-1.5);
            \draw[->] (-1,0.5) -- (-2.5,1.5);
            \draw[->] (-1,-0.5) -- (-2.5,-1.5);
            \draw[->] (0,1) -- (0,2.5);
            \draw[->] (0,-1) -- (0,-2.5);
        \end{tikzpicture}
    \end{center}
\end{frame}

% SLIDE 14 - Caso Chile: Incendios forestales (5 min)
\begin{frame}{Caso Chile: Gestión de incendios forestales}
    \textbf{Contexto:} Chile enfrenta incendios forestales cada verano
    
    \begin{columns}
        \column{0.5\textwidth}
        \textbf{Problema:}
        \begin{itemize}
            \item 500,000+ hectáreas afectadas (2023)
            \item Pérdidas humanas y económicas
            \item Necesidad de respuesta rápida
        \end{itemize}
        
        \vspace{0.3cm}
        \textbf{Solución Geoinformática:}
        \begin{itemize}
            \item Modelos predictivos de propagación
            \item Análisis de riesgo por zona
            \item Optimización de recursos
            \item Rutas de evacuación
        \end{itemize}
        
        \column{0.5\textwidth}
        \textbf{Datos utilizados:}
        \begin{itemize}
            \item \faIcon{satellite} Imágenes satelitales (MODIS)
            \item \faIcon{mountain} Modelo digital de elevación
            \item \faIcon{wind} Datos meteorológicos
            \item \faIcon{tree} Cobertura vegetal
            \item \faIcon{home} Asentamientos humanos
        \end{itemize}
        
        \vspace{0.3cm}
        \textbf{Instituciones involucradas:}
        \begin{itemize}
            \item CONAF
            \item SENAPRED
            \item Universidades chilenas
        \end{itemize}
    \end{columns}
\end{frame}

% SLIDE 15 - Caso Santiago: Movilidad urbana (5 min)
\begin{frame}{Caso Santiago: Análisis de movilidad urbana}
    \textbf{Proyecto:} Optimización del transporte público en Gran Santiago
    
    \begin{columns}
        \column{0.6\textwidth}
        \textbf{Fuentes de datos:}
        \begin{itemize}
            \item GPS de buses del Transantiago
            \item Datos de tarjeta Bip! (6M viajes/día)
            \item Sensores de tráfico
            \item Datos de aplicaciones móviles
        \end{itemize}
        
        \textbf{Análisis realizados:}
        \begin{itemize}
            \item Matrices origen-destino
            \item Identificación de cuellos de botella
            \item Predicción de demanda
            \item Optimización de rutas y frecuencias
        \end{itemize}
        
        \column{0.4\textwidth}
        \examplebox{Resultados:}{
            \begin{itemize}
                \item 15\% reducción en tiempos de viaje
                \item Mejor distribución de flota
                \item Identificación de zonas mal servidas
            \end{itemize}
        }
        
        \textbf{Tecnologías:}
        \begin{itemize}
            \item Python + GeoPandas
            \item PostgreSQL/PostGIS
            \item Kepler.gl para visualización
        \end{itemize}
    \end{columns}
\end{frame}

% SLIDE 16 - Geomarketing retail (5 min)
\begin{frame}{Geomarketing: Localización de tiendas}
    \textbf{Cliente:} Cadena de retail expandiéndose en Chile
    
    \begin{columns}
        \column{0.5\textwidth}
        \textbf{Factores analizados:}
        \begin{enumerate}
            \item Demografía por manzana censal
            \item Poder adquisitivo (ABC1, C2, C3)
            \item Competencia (radio 1km)
            \item Accesibilidad (transporte público)
            \item Flujo peatonal
            \item Proyección de crecimiento
        \end{enumerate}
        
        \column{0.5\textwidth}
        \textbf{Metodología:}
        \begin{itemize}
            \item Análisis de áreas de influencia
            \item Modelo de Huff (gravedad)
            \item Machine Learning para predicción de ventas
            \item Optimización multicriterio
        \end{itemize}
        
        \vspace{0.3cm}
        \textbf{Herramientas:}
        \begin{itemize}
            \item Python + GeoPandas para análisis
            \item Folium para dashboard
            \item Streamlit para aplicación web
        \end{itemize}
    \end{columns}
    
    \vspace{0.3cm}
    \begin{center}
        \textcolor{usachred}{\faIcon{store} ROI: +25\% en ventas vs selección tradicional}
    \end{center}
\end{frame}

% SLIDE 17 - Agricultura de precisión (4 min)
\begin{frame}{Agricultura de precisión en Chile}
    \textbf{Sector:} Viñedos en Valle Central
    
    \begin{columns}
        \column{0.5\textwidth}
        \textbf{Tecnologías aplicadas:}
        \begin{itemize}
            \item Drones con cámaras multiespectrales
            \item Sensores IoT de humedad
            \item Imágenes satelitales Sentinel-2
            \item Estaciones meteorológicas
        \end{itemize}
        
        \textbf{Análisis geoespacial:}
        \begin{itemize}
            \item NDVI para vigor vegetativo
            \item Mapas de estrés hídrico
            \item Zonificación de parcelas
            \item Predicción de rendimiento
        \end{itemize}
        
        \column{0.5\textwidth}
        \begin{center}
            \begin{tikzpicture}[scale=0.7]
                % Parcela
                \draw[thick] (0,0) rectangle (4,3);
                
                % Zonas de vigor
                \fill[green!70] (0,0) rectangle (1.5,1.5);
                \fill[green!50] (1.5,0) rectangle (3,1.5);
                \fill[yellow!50] (3,0) rectangle (4,1.5);
                \fill[green!60] (0,1.5) rectangle (2,3);
                \fill[orange!40] (2,1.5) rectangle (4,3);
                
                \node at (2,-0.5) {Mapa de vigor (NDVI)};
                
                % Leyenda
                \fill[green!70] (5,2.5) rectangle (5.3,2.8);
                \node[right] at (5.3,2.65) {Alto};
                \fill[yellow!50] (5,2) rectangle (5.3,2.3);
                \node[right] at (5.3,2.15) {Medio};
                \fill[orange!40] (5,1.5) rectangle (5.3,1.8);
                \node[right] at (5.3,1.65) {Bajo};
            \end{tikzpicture}
        \end{center}
        
        \textbf{Beneficios:}
        \begin{itemize}
            \item 30\% ahorro en agua
            \item 20\% reducción en pesticidas
            \item Mejor calidad del producto
        \end{itemize}
    \end{columns}
\end{frame}

% SLIDE 18 - Salud pública (3 min)
\begin{frame}{Salud pública: Análisis epidemiológico}
    \textbf{Caso:} Análisis espacial de COVID-19 en Región Metropolitana (2020-2021)
    
    \begin{columns}
        \column{0.6\textwidth}
        \textbf{Análisis realizados:}
        \begin{itemize}
            \item Mapas de calor de casos por comuna
            \item Correlación con variables socioeconómicas
            \item Accesibilidad a centros de salud
            \item Predicción de zonas de riesgo
            \item Optimización de puntos de vacunación
        \end{itemize}
        
        \textbf{Datos integrados:}
        \begin{itemize}
            \item Casos confirmados (MINSAL)
            \item Densidad poblacional (INE)
            \item Movilidad (Google Mobility Reports)
            \item Infraestructura sanitaria
        \end{itemize}
        
        \column{0.4\textwidth}
        \examplebox{Impacto:}{
            Permitió focalizar recursos en comunas críticas y optimizar la estrategia de vacunación
        }
        
        \vspace{0.3cm}
        \textbf{Técnicas utilizadas:}
        \begin{itemize}
            \item Autocorrelación espacial
            \item Modelos SIR espaciales
            \item Análisis de clusters
        \end{itemize}
    \end{columns}
\end{frame}

% SECCIÓN 4: ASPECTOS PRÁCTICOS (15 minutos)
\section{Aspectos prácticos del curso}

% SLIDE 19 - Recursos y bibliografía (3 min)
\begin{frame}{Recursos del curso}
    \begin{columns}
        \column{0.5\textwidth}
        \textbf{Bibliografía principal:}
        \begin{itemize}
            \item \textcolor{blue}{\faIcon{book}} Geocomputation with R\\
            {\small Lovelace, Nowosad \& Muenchow}\\
            {\small \url{r.geocompx.org} (gratis)}
            
            \item \textcolor{green}{\faIcon{book}} Geocomputation with Python\\
            {\small Dorman, Graser, Nowosad \& Lovelace}\\
            {\small \url{py.geocompx.org} (gratis)}
        \end{itemize}
        
        \column{0.5\textwidth}
        \textbf{Recursos adicionales:}
        \begin{itemize}
            \item Repositorio GitHub del curso
            \item Datos geoespaciales de Chile
            \item Tutoriales y ejemplos
            \item Foro de consultas
            \item Videos complementarios
        \end{itemize}
        
        \textbf{Datasets chilenos:}
        \begin{itemize}
            \item IDE Chile
            \item Datos abiertos MINVU
            \item Centro de Información SERNAGEOMIN
        \end{itemize}
    \end{columns}
\end{frame}

% SLIDE 20 - Hitos del proyecto (3 min)
\begin{frame}{Calendario del proyecto semestral}
    \begin{center}
        \begin{tikzpicture}[scale=0.9]
            % Timeline
            \draw[thick, ->, usachblue] (0,0) -- (10,0) node[right] {Dic};
            
            % Meses
            \foreach \x/\mes in {0/Ago,2/Sep,4/Oct,6/Nov,8/Dic} {
                \draw[usachblue] (\x,0.1) -- (\x,-0.1) node[below] {\mes};
            }
            
            % Hitos
            \node[draw, fill=yellow!30, above] at (1.5,0.5) {Formar grupos};
            \node[draw, fill=orange!30, above] at (3.5,1) {Propuesta};
            \node[draw, fill=red!30, above] at (5,0.5) {Plan formal};
            \node[draw, fill=green!30, above] at (6.5,1) {Avance 1};
            \node[draw, fill=blue!30, above] at (7.5,0.5) {Avance 2};
            \node[draw, fill=purple!30, above] at (9,1) {Presentación};
            
            % Conectores
            \draw[dashed, gray] (1.5,0) -- (1.5,0.5);
            \draw[dashed, gray] (3.5,0) -- (3.5,1);
            \draw[dashed, gray] (5,0) -- (5,0.5);
            \draw[dashed, gray] (6.5,0) -- (6.5,1);
            \draw[dashed, gray] (7.5,0) -- (7.5,0.5);
            \draw[dashed, gray] (9,0) -- (9,1);
        \end{tikzpicture}
    \end{center}
    
    \vspace{0.5cm}
    \textbf{Recomendaciones para el proyecto:}
    \begin{itemize}
        \item Elegir un problema real y acotado
        \item Asegurar disponibilidad de datos
        \item Planificar iteraciones incrementales
        \item Documentar todo el proceso
        \item Usar control de versiones (Git)
    \end{itemize}
\end{frame}

% SLIDE 21 - Preparación para laboratorio (3 min)
\begin{frame}{Preparación para el primer laboratorio}
    \textbf{Jueves 21 de agosto - Laboratorio 1: Configuración del ambiente}
    
    \examplebox{Para instalar antes de la clase:}{
        \begin{enumerate}
            \item \textbf{R} (versión 4.3 o superior): \url{r-project.org}
            \item \textbf{RStudio}: \url{posit.co/download/rstudio-desktop}
            \item \textbf{Python} (Anaconda): \url{anaconda.com/download}
            \item Crear cuenta en \textbf{GitHub}
        \end{enumerate}
    }
    
    \textbf{Bibliotecas de Python para instalar:}
    \begin{columns}
        \column{0.5\textwidth}
        \begin{itemize}
            \item \texttt{pip install geopandas}
            \item \texttt{pip install rasterio}
            \item \texttt{pip install folium}
        \end{itemize}
        
        \column{0.5\textwidth}
        \begin{itemize}
            \item \texttt{pip install shapely}
            \item \texttt{pip install matplotlib}
            \item \texttt{pip install contextily}
        \end{itemize}
    \end{columns}
\end{frame}

% SECCIÓN 5: ACTIVIDAD Y CIERRE (5 minutos)
\section{Actividad diagnóstica}

% SLIDE 22 - Evaluación diagnóstica (3 min)
\begin{frame}{Evaluación diagnóstica (no calificada)}
    \textbf{Objetivo:} Conocer el nivel inicial del curso
    
    \begin{enumerate}
        \item ¿Has trabajado con datos geográficos anteriormente?
        \begin{itemize}
            \item[$\square$] Nunca
            \item[$\square$] He usado Google Maps/Earth
            \item[$\square$] He usado software SIG
            \item[$\square$] He programado con datos espaciales
        \end{itemize}
        
        \item ¿Qué lenguajes de programación conoces?
        \begin{itemize}
            \item[$\square$] R \quad $\square$ Python \quad $\square$ JavaScript \quad $\square$ SQL
        \end{itemize}
        
        \item ¿Qué aplicación geoespacial te interesa más?
        \begin{itemize}
            \item[$\square$] Medio ambiente \quad $\square$ Urbanismo \quad $\square$ Negocios
            \item[$\square$] Salud \quad $\square$ Transporte \quad $\square$ Otra: \_\_\_\_\_
        \end{itemize}
        
        \item ¿Tienes alguna idea para tu proyecto semestral?
    \end{enumerate}
    
    \begin{center}
        \textcolor{gray}{\small Completar en formulario online (link en el foro)}
    \end{center}
\end{frame}

% SLIDE 23 - Demo rápida (5 min)
\begin{frame}[fragile]{Demo: Tu primer mapa en Python}
    \begin{lstlisting}[language=Python]
# Instalar: pip install geopandas folium matplotlib
import geopandas as gpd
import matplotlib.pyplot as plt

# Cargar datos del mundo
world = gpd.read_file(gpd.datasets.get_path('naturalearth_lowres'))

# Filtrar Sudamérica
south_america = world[world['continent'] == 'South America']

# Crear mapa temático
fig, ax = plt.subplots(1, 1, figsize=(10, 10))
south_america.plot(column='gdp_md_est',
                   ax=ax,
                   legend=True,
                   cmap='RdYlGn',
                   edgecolor='black')

# Añadir títulos
ax.set_title('PIB de países sudamericanos', fontsize=16)
ax.set_axis_off()
plt.show()
    \end{lstlisting}
\end{frame}

% SLIDE 24 - Reflexión final (2 min)
\begin{frame}{Para reflexionar}
    \begin{center}
        \Large{\textcolor{usachblue}{"El 80\% de todos los datos tienen un componente geográfico"}}
        
        \vspace{0.5cm}
        \normalsize{- IBM Business Analytics}
        
        \vspace{1cm}
        
        \begin{tikzpicture}
            \node[draw=usachblue, line width=2pt, rounded corners, fill=blue!10, text width=10cm, align=center, minimum height=2cm] {
                \textbf{La Geoinformática nos permite extraer valor de esa dimensión espacial para tomar mejores decisiones}
            };
        \end{tikzpicture}
    \end{center}
\end{frame}

% SLIDE 25 - Cierre (1 min)
\begin{frame}{¡Bienvenidos a Geoinformática!}
    \begin{center}
        \Large{\textbf{¿Preguntas?}}
        
        \vspace{1cm}
        
        \begin{columns}
            \column{0.5\textwidth}
            \begin{center}
                \textcolor{usachblue}{\faIcon{calendar-alt} \textbf{Próxima clase:}}\\
                Jueves 21 de agosto\\
                Clase teórica + Laboratorio 1
            \end{center}
            
            \column{0.5\textwidth}
            \begin{center}
                \textcolor{usachred}{\faIcon{tasks} \textbf{Para hacer:}}\\
                $\checkmark$ Instalar software\\
                $\checkmark$ Completar encuesta\\
                $\checkmark$ Pensar ideas proyecto
            \end{center}
        \end{columns}
        
        \vspace{1cm}
        
        \textbf{Contacto:} francisco.parra.o@usach.cl
        
        \vspace{0.5cm}
        
        \textcolor{gray}{\small Material disponible en GitHub del curso}
    \end{center}
\end{frame}

\end{document}