\documentclass[10pt]{beamer}
\usetheme{metropolis}
\usepackage{FiraSans}
\usefonttheme{professionalfonts}

\usepackage{graphicx}
\usepackage{tikz}
\usepackage[spanish]{babel}
\usepackage{tcolorbox}
\usepackage{ragged2e}
\usepackage{pgfplots}
\pgfplotsset{compat=1.18}
\usepackage{listings}
\usepackage{xcolor}
\usepackage{array}
\usepackage{booktabs}
\usepackage{multirow}

\definecolor{codegreen}{rgb}{0,0.6,0}
\definecolor{codegray}{rgb}{0.5,0.5,0.5}
\definecolor{codepurple}{rgb}{0.58,0,0.82}
\definecolor{backcolour}{rgb}{0.95,0.95,0.92}
\definecolor{usachblue}{rgb}{0,0.4,0.7}
\definecolor{usachred}{rgb}{0.8,0,0}

\lstdefinestyle{mystyle}{
    backgroundcolor=\color{backcolour},   
    commentstyle=\color{codegreen},
    keywordstyle=\color{magenta},
    numberstyle=\tiny\color{codegray},
    stringstyle=\color{codepurple},
    basicstyle=\ttfamily\footnotesize,
    breakatwhitespace=false,         
    breaklines=true,                 
    captionpos=b,                    
    keepspaces=true,                 
    numbers=left,                    
    numbersep=5pt,                  
    showspaces=false,                
    showstringspaces=false,
    showtabs=false,                  
    tabsize=2
}

\lstset{style=mystyle}

\newcommand{\examplebox}[2]{
\begin{tcolorbox}[colframe=usachblue,colback=blue!5,title=#1]
#2
\end{tcolorbox}
}

\setbeamertemplate{footline}{%
  \begin{beamercolorbox}[wd=\paperwidth,sep=2ex]{footline}%
    \usebeamerfont{structure}\textbf{Geoinformática - Clase 3} \hfill Profesor: Francisco Parra O. \hfill \textbf{Semestre 2, 2025}
  \end{beamercolorbox}%
}

\title{Clase 03: Fundamentos de datos geoespaciales}
\subtitle{Tipos y estructuras de datos espaciales}
\author{Profesor: Francisco Parra O.}
\institute{USACH - Ingeniería Civil en Informática}
\date{\today}

\titlegraphic{%
  \begin{tikzpicture}[overlay, remember picture]
    \node[anchor=north east, yshift=0cm] at (current page.north east) {
      \includegraphics[width=1.5cm]{Logo-Color-Usach-Web.jpg}
    };
  \end{tikzpicture}
}

\begin{document}

\maketitle

\begin{frame}{Agenda}
    \tableofcontents
\end{frame}

\section{Datos vectoriales: puntos, líneas, polígonos}

\begin{frame}{Modelo Vectorial: Fundamentos}
    \begin{columns}
        \column{0.5\textwidth}
        \textbf{Características:}
        \begin{itemize}
            \item Representación discreta
            \item Geometrías precisas
            \item Topología explícita
            \item Atributos asociados
        \end{itemize}
        
        \vspace{0.5cm}
        
        \textbf{Ventajas:}
        \begin{itemize}
            \item Alta precisión
            \item Eficiente en almacenamiento
            \item Análisis topológico
        \end{itemize}
        
        \column{0.5\textwidth}
        \begin{tikzpicture}[scale=0.8]
            \draw[gray!30] (0,0) grid (5,5);
            \node[circle,fill=red,inner sep=2pt] at (1,1) {};
            \node[circle,fill=red,inner sep=2pt] at (2,3) {};
            \node[circle,fill=red,inner sep=2pt] at (4,2) {};
            \draw[thick,blue] (0.5,0.5) -- (1.5,2) -- (3,1.5) -- (4.5,3);
            \filldraw[fill=green!30,draw=green,thick] 
                (1,3.5) -- (2,4.5) -- (3.5,4) -- (3,2.5) -- cycle;
            \node at (2.5,0.3) {\small Puntos};
            \node at (2.5,0.8) {\small Líneas};
            \node at (2.5,1.3) {\small Polígonos};
        \end{tikzpicture}
    \end{columns}
\end{frame}

\begin{frame}[fragile]{Puntos: Localizaciones Discretas}
    \begin{columns}
        \column{0.5\textwidth}
        \textbf{Características:}
        \begin{itemize}
            \item Coordenadas X, Y (Z opcional)
            \item Sin dimensión espacial
            \item Múltiples atributos
        \end{itemize}
        
        \textbf{Aplicaciones:}
        \begin{itemize}
            \item Ubicación de sensores
            \item Puntos de interés (POI)
            \item Eventos geográficos
            \item Muestras de campo
        \end{itemize}
        
        \column{0.5\textwidth}
        \begin{lstlisting}[language=Python, caption=Trabajando con puntos]
from shapely.geometry import Point
import geopandas as gpd

# Crear punto
estacion = Point(-70.651, -33.438)
print(f"X: {estacion.x}")
print(f"Y: {estacion.y}")

# GeoDataFrame con puntos
puntos = gpd.GeoDataFrame({
    'id': [1, 2, 3],
    'tipo': ['sensor', 'muestra', 'POI'],
    'geometry': [
        Point(-70.651, -33.438),
        Point(-70.649, -33.437),
        Point(-70.652, -33.439)
    ]
})
        \end{lstlisting}
    \end{columns}
\end{frame}

\begin{frame}[fragile]{Líneas: Conexiones y Redes}
    \begin{columns}
        \column{0.5\textwidth}
        \textbf{Características:}
        \begin{itemize}
            \item Secuencia de vértices
            \item Longitud medible
            \item Dirección opcional
            \item Conectividad de red
        \end{itemize}
        
        \textbf{Aplicaciones:}
        \begin{itemize}
            \item Redes viales
            \item Ríos y cursos de agua
            \item Líneas de transmisión
            \item Rutas de transporte
        \end{itemize}
        
        \column{0.5\textwidth}
        \begin{lstlisting}[language=Python, caption=Análisis de líneas]
from shapely.geometry import LineString

# Crear línea
ruta = LineString([
    (-70.651, -33.438),
    (-70.649, -33.437),
    (-70.648, -33.439),
    (-70.650, -33.441)
])

# Propiedades
print(f"Longitud: {ruta.length}")
print(f"Vertices: {len(ruta.coords)}")

# Operaciones
punto_medio = ruta.interpolate(0.5, 
                               normalized=True)
buffer_ruta = ruta.buffer(0.001)
        \end{lstlisting}
    \end{columns}
\end{frame}

\begin{frame}[fragile]{Polígonos: Áreas y Regiones}
    \begin{columns}
        \column{0.5\textwidth}
        \textbf{Características:}
        \begin{itemize}
            \item Anillo exterior cerrado
            \item Posibles huecos internos
            \item Área y perímetro
            \item Relaciones topológicas
        \end{itemize}
        
        \textbf{Aplicaciones:}
        \begin{itemize}
            \item Límites administrativos
            \item Parcelas/predios
            \item Zonas de cobertura
            \item Áreas de influencia
        \end{itemize}
        
        \column{0.5\textwidth}
        \begin{lstlisting}[language=Python, caption=Polígonos y análisis]
from shapely.geometry import Polygon

# Crear polígono
manzana = Polygon([
    (-70.650, -33.440),
    (-70.648, -33.440),
    (-70.648, -33.438),
    (-70.650, -33.438),
    (-70.650, -33.440)
])

# Propiedades geométricas
print(f"Area: {manzana.area}")
print(f"Perimetro: {manzana.length}")

# Operaciones espaciales
edificio = Point(-70.649, -33.439)
print(manzana.contains(edificio))
        \end{lstlisting}
    \end{columns}
\end{frame}

\section{Datos raster: grillas y resolución}

\begin{frame}{Modelo Raster: Fundamentos}
    \begin{columns}
        \column{0.5\textwidth}
        \textbf{Características:}
        \begin{itemize}
            \item Matriz regular de celdas
            \item Resolución espacial fija
            \item Valores continuos o discretos
            \item Múltiples bandas
        \end{itemize}
        
        \vspace{0.3cm}
        \textbf{Ventajas:}
        \begin{itemize}
            \item Datos continuos
            \item Análisis de superficie
            \item Teledetección
        \end{itemize}
        
        \column{0.5\textwidth}
        \begin{tikzpicture}[scale=0.12]
            \foreach \x in {0,...,9} {
                \foreach \y in {0,...,9} {
                    \pgfmathsetmacro\val{int(mod(\x+\y*2,5)*40+50)}
                    \fill[gray!\val] (\x,\y) rectangle (\x+1,\y+1);
                }
            }
            \draw[step=1,black,thin] (0,0) grid (10,10);
            \node at (5,-1) {\small Matriz de píxeles};
        \end{tikzpicture}
    \end{columns}
\end{frame}

\begin{frame}[fragile]{Resolución Espacial}
    \begin{columns}
        \column{0.5\textwidth}
        \textbf{Concepto clave:}
        \begin{itemize}
            \item Tamaño del píxel en terreno
            \item Trade-off: detalle vs. tamaño
            \item Escala de análisis
        \end{itemize}
        
        \vspace{0.3cm}
        \examplebox{Resoluciones típicas}{
            \begin{itemize}
                \item Landsat: 30m
                \item Sentinel-2: 10m
                \item PlanetScope: 3m
                \item WorldView: 0.3m
            \end{itemize}
        }
        
        \column{0.5\textwidth}
        \begin{lstlisting}[language=Python, caption=Trabajando con rasters]
import rasterio
import numpy as np

# Abrir raster
with rasterio.open('imagen.tif') as src:
    # Metadatos
    print(f"CRS: {src.crs}")
    print(f"Res: {src.res}")
    print(f"Bounds: {src.bounds}")
    
    # Leer banda
    banda1 = src.read(1)
    
    # Estadísticas
    print(f"Min: {banda1.min()}")
    print(f"Max: {banda1.max()}")
    print(f"Mean: {banda1.mean()}")
        \end{lstlisting}
    \end{columns}
\end{frame}

\begin{frame}[fragile]{Bandas Espectrales}
    \begin{columns}
        \column{0.5\textwidth}
        \textbf{Imágenes multiespectrales:}
        \begin{itemize}
            \item RGB (visible)
            \item Infrarrojo cercano (NIR)
            \item Infrarrojo medio (SWIR)
            \item Térmico (TIR)
        \end{itemize}
        
        \textbf{Índices espectrales:}
        \begin{itemize}
            \item NDVI (vegetación)
            \item NDWI (agua)
            \item NDBI (construcciones)
        \end{itemize}
        
        \column{0.5\textwidth}
        \begin{lstlisting}[language=Python, caption=Cálculo de NDVI]
import rasterio
import numpy as np

# Abrir bandas
with rasterio.open('B4_red.tif') as red:
    b4 = red.read(1).astype(float)
    
with rasterio.open('B5_nir.tif') as nir:
    b5 = nir.read(1).astype(float)

# Calcular NDVI
ndvi = (b5 - b4) / (b5 + b4 + 1e-10)

# Clasificar vegetación
vegetacion = ndvi > 0.3
agua = ndvi < 0
suelo = (ndvi >= 0) & (ndvi <= 0.3)
        \end{lstlisting}
    \end{columns}
\end{frame}

\begin{frame}{Modelos de Elevación Digital (DEM)}
    \begin{columns}
        \column{0.5\textwidth}
        \textbf{Tipos de DEM:}
        \begin{itemize}
            \item DSM: Superficie con objetos
            \item DTM: Terreno sin objetos
            \item Derivados: pendiente, aspecto
        \end{itemize}
        
        \textbf{Aplicaciones:}
        \begin{itemize}
            \item Análisis hidrológico
            \item Visibilidad
            \item Modelado 3D
            \item Riesgo de inundación
        \end{itemize}
        
        \column{0.5\textwidth}
        \begin{tikzpicture}[scale=0.8]
            \begin{axis}[
                width=6cm,
                height=4cm,
                xlabel={\small X},
                ylabel={\small Y},
                zlabel={\small Elevación},
                mesh/cols=10,
                view={30}{30}
            ]
            \addplot3[
                mesh,
                samples=10,
                domain=-2:2,
                colormap/viridis
            ] {exp(-x^2-y^2)};
            \end{axis}
        \end{tikzpicture}
        
        \vspace{0.5cm}
        \small
        Fuentes de DEM:
        \begin{itemize}
            \item SRTM (30m global)
            \item ASTER GDEM (30m)
            \item LiDAR (alta precisión)
        \end{itemize}
    \end{columns}
\end{frame}

\section{Formatos de archivos comunes}

\begin{frame}{Formatos Vectoriales}
    \begin{table}
        \small
        \begin{tabular}{lll}
        \toprule
        \textbf{Formato} & \textbf{Características} & \textbf{Uso} \\
        \midrule
        Shapefile & Estándar ESRI, múltiples archivos & Universal \\
        GeoJSON & Texto JSON, web-friendly & Web mapping \\
        GeoPackage & SQLite, todo-en-uno & Moderno, móvil \\
        KML/KMZ & XML, Google Earth & Visualización \\
        PostGIS & Base de datos espacial & Producción \\
        \bottomrule
        \end{tabular}
    \end{table}
    
    \vspace{0.5cm}
    
    \examplebox{Recomendación}{
        \begin{itemize}
            \item Intercambio: GeoPackage > Shapefile
            \item Web: GeoJSON para datos pequeños
            \item Producción: PostGIS para grandes volúmenes
        \end{itemize}
    }
\end{frame}

\begin{frame}[fragile]{Shapefile: El Estándar Legacy}
    \begin{columns}
        \column{0.5\textwidth}
        \textbf{Componentes obligatorios:}
        \begin{itemize}
            \item .shp - geometrías
            \item .shx - índice espacial
            \item .dbf - atributos
        \end{itemize}
        
        \textbf{Opcionales:}
        \begin{itemize}
            \item .prj - proyección
            \item .cpg - encoding
            \item .qpj - QGIS projection
        \end{itemize}
        
        \textbf{Limitaciones:}
        \begin{itemize}
            \item Nombres campo: 10 chars
            \item Tamaño máximo: 2GB
            \item Sin topología
        \end{itemize}
        
        \column{0.5\textwidth}
        \begin{lstlisting}[language=Python, caption=Lectura de Shapefile]
import geopandas as gpd

# Leer shapefile
gdf = gpd.read_file('comunas.shp')

# Explorar estructura
print(gdf.head())
print(gdf.crs)
print(gdf.geometry.type.unique())

# Filtrar y guardar
santiago = gdf[gdf['COMUNA'] == 'Santiago']
santiago.to_file('santiago.shp')

# Convertir a otros formatos
gdf.to_file('comunas.geojson', 
            driver='GeoJSON')
gdf.to_file('comunas.gpkg', 
            driver='GPKG')
        \end{lstlisting}
    \end{columns}
\end{frame}

\begin{frame}[fragile]{GeoJSON: Intercambio Web}
    \begin{columns}
        \column{0.5\textwidth}
        \textbf{Ventajas:}
        \begin{itemize}
            \item Legible por humanos
            \item Soporte nativo web
            \item Un solo archivo
            \item Estándar RFC 7946
        \end{itemize}
        
        \textbf{Estructura:}
        \begin{itemize}
            \item FeatureCollection
            \item Features
            \item Geometry + Properties
            \item CRS:84 (WGS84)
        \end{itemize}
        
        \column{0.5\textwidth}
        \begin{lstlisting}[language=Python, caption=Ejemplo GeoJSON]
{
  "type": "FeatureCollection",
  "features": [
    {
      "type": "Feature",
      "geometry": {
        "type": "Point",
        "coordinates": [-70.651, -33.438]
      },
      "properties": {
        "nombre": "USACH",
        "tipo": "Universidad",
        "estudiantes": 22000
      }
    }
  ]
}
        \end{lstlisting}
    \end{columns}
\end{frame}

\begin{frame}{Formatos Raster}
    \begin{table}
        \small
        \begin{tabular}{lll}
        \toprule
        \textbf{Formato} & \textbf{Características} & \textbf{Uso} \\
        \midrule
        GeoTIFF & TIFF + georeferencia & Universal \\
        COG & Cloud Optimized GeoTIFF & Web/cloud \\
        NetCDF & Multidimensional & Clima/tiempo \\
        HDF5 & Jerárquico, comprimido & Satélites \\
        JP2000 & Compresión wavelet & Ortofotos \\
        \bottomrule
        \end{tabular}
    \end{table}
    
    \vspace{0.5cm}
    
    \begin{tcolorbox}[colframe=usachblue,colback=blue!5]
        \textbf{Consideraciones:}
        \begin{itemize}
            \item Compresión: sin pérdida vs con pérdida
            \item Pirámides: visualización multiescala
            \item Tiles: acceso eficiente a porciones
        \end{itemize}
    \end{tcolorbox}
\end{frame}

\begin{frame}[fragile]{Cloud Optimized GeoTIFF (COG)}
    \begin{columns}
        \column{0.5\textwidth}
        \textbf{Optimizaciones:}
        \begin{itemize}
            \item Tiles internos
            \item Overviews (pirámides)
            \item HTTP range requests
            \item Compresión eficiente
        \end{itemize}
        
        \textbf{Beneficios:}
        \begin{itemize}
            \item Streaming parcial
            \item Sin descarga completa
            \item Visualización rápida
            \item Cloud-native
        \end{itemize}
        
        \column{0.5\textwidth}
        \begin{lstlisting}[language=Python, caption=Crear y leer COG]
import rasterio
from rasterio.crs import CRS

# Convertir a COG
gdal_translate input.tif output_cog.tif \
  -of COG \
  -co COMPRESS=LZW \
  -co BLOCKSIZE=512

# Leer COG desde URL
import rasterio
from rasterio.windows import Window

url = 'https://example.com/cog.tif'
with rasterio.open(url) as src:
    # Leer solo una ventana
    window = Window(0, 0, 512, 512)
    data = src.read(1, window=window)
        \end{lstlisting}
    \end{columns}
\end{frame}

\section{Atributos y geometrías}

\begin{frame}{Integración Atributos-Geometría}
    \begin{columns}
        \column{0.5\textwidth}
        \textbf{Modelo relacional espacial:}
        \begin{itemize}
            \item Tabla = Feature Class
            \item Fila = Feature
            \item Columna geométrica especial
            \item Índice espacial
        \end{itemize}
        
        \vspace{0.3cm}
        
        \textbf{Tipos de atributos:}
        \begin{itemize}
            \item Identificadores
            \item Descriptivos
            \item Numéricos
            \item Temporales
            \item Relacionales
        \end{itemize}
        
        \column{0.5\textwidth}
        \begin{table}
            \footnotesize
            \begin{tabular}{|l|l|l|}
            \hline
            \textbf{ID} & \textbf{Nombre} & \textbf{Geometry} \\
            \hline
            1 & Parque & POLYGON(...) \\
            2 & Plaza & POLYGON(...) \\
            3 & Lago & POLYGON(...) \\
            \hline
            \end{tabular}
        \end{table}
        
        \vspace{0.5cm}
        
        \begin{tikzpicture}[scale=0.7]
            \draw[fill=green!30] (0,0) rectangle (2,1.5);
            \draw[fill=blue!30] (2.5,0) circle (0.8);
            \draw[fill=yellow!30] (4,0) -- (5,1) -- (4.5,1.5) -- (3.5,0.5) -- cycle;
            \node at (1,-0.5) {\tiny ID=1};
            \node at (2.5,-0.5) {\tiny ID=2};
            \node at (4.3,-0.5) {\tiny ID=3};
        \end{tikzpicture}
    \end{columns}
\end{frame}

\begin{frame}[fragile]{GeoPandas: DataFrames Espaciales}
    \begin{columns}
        \column{0.5\textwidth}
        \textbf{Extensión de Pandas:}
        \begin{itemize}
            \item GeoDataFrame
            \item GeoSeries
            \item Operaciones vectorizadas
            \item Integración con ecosistema
        \end{itemize}
        
        \textbf{Funcionalidades:}
        \begin{itemize}
            \item Joins espaciales
            \item Agregaciones
            \item Visualización
            \item I/O múltiples formatos
        \end{itemize}
        
        \column{0.5\textwidth}
        \begin{lstlisting}[language=Python, caption=GeoPandas workflow]
import geopandas as gpd
import matplotlib.pyplot as plt

# Cargar datos
comunas = gpd.read_file('comunas.shp')
puntos = gpd.read_file('colegios.geojson')

# Join espacial
colegios_por_comuna = gpd.sjoin(
    puntos, comunas, 
    predicate='within'
)

# Agregación
resumen = colegios_por_comuna.groupby(
    'COMUNA'
).size()

# Visualización
ax = comunas.plot(column='POBLACION',
                  legend=True)
puntos.plot(ax=ax, color='red', 
           markersize=2)
        \end{lstlisting}
    \end{columns}
\end{frame}

\begin{frame}[fragile]{Operaciones Espaciales Fundamentales}
    \begin{columns}
        \column{0.5\textwidth}
        \textbf{Operaciones geométricas:}
        \begin{itemize}
            \item Buffer
            \item Intersección
            \item Unión
            \item Diferencia
            \item Simplificación
        \end{itemize}
        
        \textbf{Relaciones espaciales:}
        \begin{itemize}
            \item Contains/Within
            \item Intersects
            \item Touches
            \item Overlaps
            \item Crosses
        \end{itemize}
        
        \column{0.5\textwidth}
        \begin{lstlisting}[language=Python, caption=Operaciones espaciales]
from shapely.ops import unary_union
import geopandas as gpd

# Buffer
zonas_influencia = gdf.buffer(100)

# Intersección
interseccion = gdf1.overlay(gdf2, 
                           how='intersection')

# Unión de geometrías
union_total = unary_union(gdf.geometry)

# Consultas espaciales
cerca_metro = puntos[
    puntos.distance(estacion) < 500
]

# Dissolve por atributo
regiones = comunas.dissolve(by='REGION')
        \end{lstlisting}
    \end{columns}
\end{frame}

\begin{frame}{Indexación Espacial}
    \begin{columns}
        \column{0.5\textwidth}
        \textbf{R-tree (Rectangle tree):}
        \begin{itemize}
            \item Estructura jerárquica
            \item Bounding boxes
            \item Búsqueda eficiente
            \item O(log n) promedio
        \end{itemize}
        
        \vspace{0.3cm}
        
        \textbf{Aplicaciones:}
        \begin{itemize}
            \item Consultas de proximidad
            \item Joins espaciales
            \item Intersecciones
            \item KNN espacial
        \end{itemize}
        
        \column{0.5\textwidth}
        \begin{tikzpicture}[scale=0.6]
            \draw[thick] (0,0) rectangle (6,6);
            \draw[blue,thick] (0.5,0.5) rectangle (3,3);
            \draw[red,thick] (3.5,3.5) rectangle (5.5,5.5);
            \draw[green,thick] (1,3.5) rectangle (3,5);
            
            \foreach \x/\y in {1/1,2/2,1.5/2.5,2.5/1.5} {
                \node[circle,fill=blue!50,inner sep=1pt] at (\x,\y) {};
            }
            \foreach \x/\y in {4/4,5/5,4.5/4.5} {
                \node[circle,fill=red!50,inner sep=1pt] at (\x,\y) {};
            }
            \foreach \x/\y in {2/4,1.5/4.5} {
                \node[circle,fill=green!50,inner sep=1pt] at (\x,\y) {};
            }
            
            \node at (3,-0.5) {\small R-tree structure};
        \end{tikzpicture}
        
        \vspace{0.3cm}
        \small
        Reduce búsquedas de O(n) a O(log n)
    \end{columns}
\end{frame}

\begin{frame}[fragile]{Validación y Limpieza Geométrica}
    \begin{columns}
        \column{0.5\textwidth}
        \textbf{Problemas comunes:}
        \begin{itemize}
            \item Self-intersections
            \item Duplicados
            \item Gaps/Overlaps
            \item Slivers
            \item Topología inválida
        \end{itemize}
        
        \textbf{Herramientas:}
        \begin{itemize}
            \item Shapely: is\_valid
            \item Buffer(0) trick
            \item Simplificación
            \item Snap to grid
        \end{itemize}
        
        \column{0.5\textwidth}
        \begin{lstlisting}[language=Python, caption=Validación y corrección]
import geopandas as gpd
from shapely.validation import explain_validity

# Verificar validez
gdf['valido'] = gdf.geometry.is_valid

# Explicar problemas
invalidos = gdf[~gdf['valido']]
for idx, row in invalidos.iterrows():
    print(explain_validity(row.geometry))

# Corregir con buffer(0)
gdf['geometry'] = gdf.geometry.buffer(0)

# Eliminar slivers
gdf = gdf[gdf.geometry.area > 0.001]

# Simplificar
gdf['geometry'] = gdf.geometry.simplify(
    tolerance=1.0, 
    preserve_topology=True
)
        \end{lstlisting}
    \end{columns}
\end{frame}

\begin{frame}{Sistemas de Referencia de Coordenadas (CRS)}
    \begin{columns}
        \column{0.5\textwidth}
        \textbf{Componentes:}
        \begin{itemize}
            \item Datum (modelo de la Tierra)
            \item Proyección cartográfica
            \item Unidades de medida
            \item Códigos EPSG
        \end{itemize}
        
        \textbf{CRS comunes:}
        \begin{itemize}
            \item WGS84 (EPSG:4326)
            \item Web Mercator (EPSG:3857)
            \item UTM zonas
            \item Lambert Conformal
        \end{itemize}
        
        \column{0.5\textwidth}
        \begin{tcolorbox}[colframe=usachred,colback=red!5]
            \textbf{Chile continental:}
            \begin{itemize}
                \item EPSG:32718 (UTM 18S)
                \item EPSG:32719 (UTM 19S)
                \item EPSG:5361 (SIRGAS-Chile)
            \end{itemize}
        \end{tcolorbox}
        
        \vspace{0.3cm}
        
        \textbf{Transformaciones:}
        \begin{itemize}
            \item Reproyección
            \item Cambio de datum
            \item Distorsiones inevitables
        \end{itemize}
    \end{columns}
\end{frame}

\begin{frame}[fragile]{Machine Learning Espacial}
    \begin{columns}
        \column{0.5\textwidth}
        \textbf{Particularidades:}
        \begin{itemize}
            \item Autocorrelación espacial
            \item Primera ley de Tobler
            \item Cross-validation espacial
            \item Feature engineering espacial
        \end{itemize}
        
        \textbf{Aplicaciones:}
        \begin{itemize}
            \item Predicción de precios
            \item Clasificación de cobertura
            \item Interpolación espacial
            \item Detección de patrones
        \end{itemize}
        
        \column{0.5\textwidth}
        \begin{lstlisting}[language=Python, caption=ML con features espaciales]
from sklearn.ensemble import RandomForestRegressor
import geopandas as gpd

# Features espaciales
gdf['dist_centro'] = gdf.distance(centro)
gdf['dist_metro'] = gdf.distance(metro)
gdf['n_vecinos'] = gdf.buffer(500).apply(
    lambda x: puntos.within(x).sum()
)

# Lag espacial
from libpysal.weights import KNN
w = KNN.from_dataframe(gdf, k=5)
gdf['precio_lag'] = w.lag(gdf['precio'])

# Modelo
X = gdf[['area', 'dist_centro', 
         'dist_metro', 'precio_lag']]
y = gdf['precio']

model = RandomForestRegressor()
model.fit(X, y)
        \end{lstlisting}
    \end{columns}
\end{frame}

\begin{frame}{Resumen: Vector vs Raster}
    \begin{table}
        \footnotesize
        \begin{tabular}{lll}
        \toprule
        \textbf{Aspecto} & \textbf{Vector} & \textbf{Raster} \\
        \midrule
        Estructura & Objetos discretos & Matriz continua \\
        Precisión & Alta & Depende de resolución \\
        Almacenamiento & Eficiente para objetos & Grande para áreas \\
        Topología & Explícita & Implícita \\
        Análisis & Redes, buffers & Álgebra de mapas \\
        Visualización & Escalable & Pixelada al zoom \\
        Casos de uso & Catastro, redes & Teledetección, DEM \\
        \bottomrule
        \end{tabular}
    \end{table}
    
    \vspace{0.5cm}
    
    \begin{tcolorbox}[colframe=usachblue,colback=blue!5]
        \textbf{Integración Vector-Raster:}
        \begin{itemize}
            \item Rasterización: vector → raster
            \item Vectorización: raster → vector
            \item Zonal statistics: resumen raster por polígono
            \item Point sampling: extracción de valores raster
        \end{itemize}
    \end{tcolorbox}
\end{frame}

\begin{frame}[fragile]{Ejercicio Práctico Integrador}
    \textbf{Tarea:} Analizar ubicación óptima para nuevo colegio
    
    \begin{columns}
        \column{0.5\textwidth}
        \textbf{Datos disponibles:}
        \begin{itemize}
            \item Manzanas censales (vector)
            \item Colegios existentes (puntos)
            \item Red vial (líneas)
            \item Población por edad (raster)
            \item Elevación (DEM)
        \end{itemize}
        
        \textbf{Criterios:}
        \begin{enumerate}
            \item Máxima población 5-18 años
            \item Mínimo 500m de otro colegio
            \item Máximo 100m de vía principal
            \item Pendiente menor a 5 grados
        \end{enumerate}
        
        \column{0.5\textwidth}
        \begin{lstlisting}[language=Python, caption=Análisis multicriterio]
# 1. Preparar datos
manzanas = gpd.read_file('manzanas.shp')
colegios = gpd.read_file('colegios.geojson')
vias = gpd.read_file('vias.shp')

# 2. Crear buffers
buffer_colegios = colegios.buffer(500)
buffer_vias = vias.buffer(100)

# 3. Áreas candidatas
candidatas = manzanas.copy()
candidatas = candidatas[
    ~candidatas.intersects(
        buffer_colegios.unary_union
    )
]
candidatas = candidatas[
    candidatas.intersects(
        buffer_vias.unary_union
    )
]

# 4. Score por población
# (continúa...)
        \end{lstlisting}
    \end{columns}
\end{frame}

\begin{frame}{Actividades Prácticas}
    \textbf{Para implementar en el laboratorio:}
    
    \begin{enumerate}
        \item \textbf{Exploración de datos vectoriales}
        \begin{itemize}
            \item Cargar shapefile de comunas de Santiago
            \item Calcular área y perímetro
            \item Identificar comuna más grande
        \end{itemize}
        
        \item \textbf{Análisis raster básico}
        \begin{itemize}
            \item Cargar imagen satelital
            \item Calcular NDVI
            \item Clasificar cobertura vegetal
        \end{itemize}
        
        \item \textbf{Operaciones espaciales}
        \begin{itemize}
            \item Crear buffer de 500m alrededor de estaciones de metro
            \item Contar puntos de interés dentro de cada buffer
            \item Hacer overlay de dos capas vectoriales
        \end{itemize}
    \end{enumerate}
    
    \vspace{0.3cm}
    
    \textbf{Próxima clase:} 
    \begin{tcolorbox}[colframe=usachblue,colback=blue!5]
        Jueves: Sistemas de Referencia Espacial (CRS) + Lab 1
    \end{tcolorbox}
\end{frame}

\begin{frame}{Cierre}
    \begin{center}
        \Large{¿Preguntas?}
        
        \vspace{1cm}
        
        francisco.parra.o@usach.cl
        
        \vspace{0.5cm}
        
        \textbf{Material disponible en:}\\
        Plataforma del curso
        
        \vspace{0.5cm}
        
        \textbf{Próxima sesión:}\\
        Jueves - CRS y Laboratorio 1
    \end{center}
\end{frame}

\end{document}