\documentclass[10pt]{beamer}
\usetheme{metropolis}
\usepackage[utf8]{inputenc}
\usepackage{FiraSans}
\usefonttheme{professionalfonts}

\usepackage{graphicx}
\usepackage{tikz}
\usepackage[spanish]{babel}
\usepackage{tcolorbox}
\usepackage{ragged2e}
\usepackage{pgfplots}
\pgfplotsset{compat=1.18}
\usepackage{listings}
\usepackage{xcolor}
\usepackage{fontawesome5}
\usepackage{amssymb}
\usepackage{amsmath}
\usepackage{array}
\usepackage{multirow}
\usepackage{booktabs}

\usetikzlibrary{shapes.geometric, arrows, positioning, calc}

\definecolor{codegreen}{rgb}{0,0.6,0}
\definecolor{codegray}{rgb}{0.5,0.5,0.5}
\definecolor{codepurple}{rgb}{0.58,0,0.82}
\definecolor{backcolour}{rgb}{0.95,0.95,0.92}
\definecolor{usachblue}{RGB}{0,54,99}
\definecolor{usachorange}{RGB}{255,102,0}

\lstdefinestyle{mystyle}{
    backgroundcolor=\color{backcolour},   
    commentstyle=\color{codegreen},
    keywordstyle=\color{magenta},
    numberstyle=\tiny\color{codegray},
    stringstyle=\color{codepurple},
    basicstyle=\ttfamily\tiny,
    breakatwhitespace=false,         
    breaklines=true,                 
    captionpos=b,                    
    keepspaces=true,                 
    numbers=left,                    
    numbersep=5pt,                  
    showspaces=false,                
    showstringspaces=false,
    showtabs=false,                  
    tabsize=2
}

\lstset{style=mystyle}

% Personalización del pie de página
\setbeamertemplate{footline}{%
  \begin{beamercolorbox}[wd=\paperwidth,sep=2ex]{footline}%
    \usebeamerfont{structure}\textbf{Geoinformática - Clase 3} \hfill Prof. Francisco Parra O. \hfill \textbf{USACH 2025}
  \end{beamercolorbox}%
}

\title{Clase 03: Fundamentos de Datos Geoespaciales}
\subtitle{Modelos Vectorial y Raster}
\author{Profesor: Francisco Parra O.}
\institute{USACH - Ingeniería Civil en Informática}
\date{20 de agosto, 2025}

\begin{document}

\maketitle

\begin{frame}{Objetivos de Aprendizaje}
    \begin{itemize}
        \item \textbf{Comprender} los modelos fundamentales de datos espaciales
        \item \textbf{Diferenciar} entre representaciones vectorial y raster
        \item \textbf{Identificar} formatos de archivos y sus aplicaciones
        \item \textbf{Manipular} datos geoespaciales con Python
        \item \textbf{Seleccionar} el modelo apropiado según el problema
    \end{itemize}
\end{frame}

\begin{frame}{Agenda}
    \tableofcontents
\end{frame}

\section{Introducción a los Modelos de Datos}

\begin{frame}{¿Cómo representamos el mundo real?}
    \begin{columns}
        \column{0.5\textwidth}
        \textbf{El desafío:}
        \begin{itemize}
            \item Mundo continuo e infinito
            \item Computadores discretos y finitos
            \item Múltiples fenómenos simultáneos
            \item Diferentes escalas y precisiones
        \end{itemize}
        
        \column{0.5\textwidth}
        \begin{tikzpicture}[scale=0.8]
            % Mundo real
            \node[draw, circle, fill=blue!20, minimum size=2cm] at (0,3) {Mundo Real};
            % Flecha
            \draw[->, thick] (0,1.5) -- (0,0.5);
            % Modelos
            \node[draw, rectangle, fill=green!20] at (-1.5,-0.5) {Vector};
            \node[draw, rectangle, fill=orange!20] at (1.5,-0.5) {Raster};
        \end{tikzpicture}
    \end{columns}
\end{frame}

\begin{frame}{Los Dos Paradigmas Fundamentales}
    \begin{columns}
        \column{0.5\textwidth}
        \textbf{Modelo Vectorial}
        \begin{itemize}
            \item Objetos discretos
            \item Fronteras definidas
            \item Coordenadas exactas
            \item Ejemplos: calles, edificios, límites
        \end{itemize}
        
        \column{0.5\textwidth}
        \textbf{Modelo Raster}
        \begin{itemize}
            \item Campos continuos
            \item Grilla regular
            \item Valores en celdas
            \item Ejemplos: temperatura, elevación, NDVI
        \end{itemize}
    \end{columns}
    
    \vspace{0.5cm}
    \begin{tcolorbox}[colframe=usachblue,colback=blue!5]
        \centering
        \small
        "La elección del modelo determina las operaciones posibles"
    \end{tcolorbox}
\end{frame}

\section{Datos Vectoriales: Puntos, Líneas, Polígonos}

\begin{frame}{Geometrías Vectoriales Básicas}
    \begin{tikzpicture}[scale=0.9]
        % Punto
        \node[circle, fill=red, minimum size=0.3cm] at (0,0) {};
        \node at (0,-0.5) {\textbf{Punto}};
        \node at (0,-0.9) {\tiny (x, y)};
        
        % Línea
        \draw[thick, blue] (3,0) -- (4,0.5) -- (5,0) -- (5.5,0.3);
        \node at (4.25,-0.5) {\textbf{Línea}};
        \node at (4.25,-0.9) {\tiny [(x₁,y₁), (x₂,y₂), ...]};
        
        % Polígono
        \draw[thick, fill=green!30] (7,0) -- (8,0.5) -- (8.5,0) -- (7.5,-0.3) -- cycle;
        \node at (7.75,-0.7) {\textbf{Polígono}};
        \node at (7.75,-1.1) {\tiny Anillo cerrado};
    \end{tikzpicture}
    
    \vspace{0.3cm}
    
    \begin{tabular}{|l|l|l|}
        \hline
        \textbf{Geometría} & \textbf{Dimensión} & \textbf{Ejemplo en Chile} \\
        \hline
        Punto & 0D & Estaciones de Metro \\
        Línea & 1D & Red vial, ríos \\
        Polígono & 2D & Comunas, lagos \\
        \hline
    \end{tabular}
\end{frame}

\begin{frame}[fragile]{Puntos: Localización Exacta}
    \begin{columns}
        \column{0.5\textwidth}
        \textbf{Características:}
        \begin{itemize}
            \item Coordenadas (x, y) o (lon, lat)
            \item Opcionalmente coordenada z
            \item Sin área ni longitud
            \item Atributos asociados
        \end{itemize}
        
        \textbf{Aplicaciones:}
        \begin{itemize}
            \item Ubicación de sensores
            \item Puntos de interés (POI)
            \item Eventos (delitos, accidentes)
            \item Muestras de campo
        \end{itemize}
        
        \column{0.5\textwidth}
        \begin{lstlisting}[language=Python, caption=Puntos en GeoPandas]
import geopandas as gpd
from shapely.geometry import Point

# Crear puntos de estaciones
estaciones = [
    Point(-70.6693, -33.4489),  # Santiago
    Point(-70.6483, -33.4372),  # La Moneda
    Point(-70.6506, -33.4183),  # Santa Ana
]

# Crear GeoDataFrame
gdf = gpd.GeoDataFrame(
    {'nombre': ['Santiago', 'La Moneda', 
                'Santa Ana'],
     'linea': [1, 1, 2]},
    geometry=estaciones,
    crs='EPSG:4326'
)

# Operaciones
print(gdf.distance(gdf.iloc[0].geometry))
buffer = gdf.buffer(0.001)  # 100m aprox
        \end{lstlisting}
    \end{columns}
\end{frame}

\begin{frame}[fragile]{Líneas: Conectividad y Redes}
    \begin{columns}
        \column{0.5\textwidth}
        \textbf{Características:}
        \begin{itemize}
            \item Secuencia ordenada de puntos
            \item Longitud medible
            \item Dirección (opcional)
            \item Topología de red
        \end{itemize}
        
        \textbf{Aplicaciones:}
        \begin{itemize}
            \item Redes de transporte
            \item Hidrografía
            \item Límites lineales
            \item Trayectorias GPS
        \end{itemize}
        
        \column{0.5\textwidth}
        \begin{lstlisting}[language=Python, caption=Líneas y redes]
from shapely.geometry import LineString
import osmnx as ox

# Crear línea simple
ruta = LineString([
    (-70.669, -33.449),
    (-70.665, -33.445),
    (-70.660, -33.440)
])

# Propiedades
print(f"Longitud: {ruta.length}")
print(f"Punto medio: {ruta.centroid}")

# Red vial con OSMnx
G = ox.graph_from_place(
    'Providencia, Chile',
    network_type='drive'
)

# Análisis de red
centrality = ox.betweenness_centrality(G)
shortest_path = ox.shortest_path(
    G, orig, dest, weight='length'
)
        \end{lstlisting}
    \end{columns}
\end{frame}

\begin{frame}[fragile]{Polígonos: Áreas y Regiones}
    \begin{columns}
        \column{0.5\textwidth}
        \textbf{Características:}
        \begin{itemize}
            \item Anillo exterior cerrado
            \item Posibles huecos internos
            \item Área y perímetro
            \item Relaciones topológicas
        \end{itemize}
        
        \textbf{Aplicaciones:}
        \begin{itemize}
            \item Límites administrativos
            \item Parcelas/predios
            \item Zonas de cobertura
            \item Áreas de influencia
        \end{itemize}
        
        \column{0.5\textwidth}
        \begin{lstlisting}[language=Python, caption=Polígonos y análisis]
from shapely.geometry import Polygon

# Crear polígono
manzana = Polygon([
    (-70.650, -33.440),
    (-70.648, -33.440),
    (-70.648, -33.438),
    (-70.650, -33.438),
    (-70.650, -33.440)
])

# Propiedades geométricas
print(f"Área: {manzana.area}")
print(f"Perímetro: {manzana.length}")

# Operaciones espaciales
edificio = Point(-70.649, -33.439)
print(manzana.contains(edificio))  # True

# Overlay de polígonos
interseccion = poligono1.intersection(poligono2)
union = poligono1.union(poligono2)
diferencia = poligono1.difference(poligono2)
        \end{lstlisting}
    \end{columns}
\end{frame}

\begin{frame}{Topología Vectorial}
    \begin{center}
    \begin{tikzpicture}[scale=1.2]
        % Polígonos adyacentes
        \draw[thick, fill=blue!20] (0,0) rectangle (2,2);
        \draw[thick, fill=green!20] (2,0) rectangle (4,2);
        \draw[thick, fill=red!20] (0,2) rectangle (2,4);
        \draw[thick, fill=yellow!20] (2,2) rectangle (4,4);
        
        % Nodos
        \node[circle, fill=black, minimum size=0.2cm] at (0,0) {};
        \node[circle, fill=black, minimum size=0.2cm] at (2,0) {};
        \node[circle, fill=black, minimum size=0.2cm] at (4,0) {};
        \node[circle, fill=black, minimum size=0.2cm] at (0,2) {};
        \node[circle, fill=black, minimum size=0.2cm] at (2,2) {};
        \node[circle, fill=black, minimum size=0.2cm] at (4,2) {};
        \node[circle, fill=black, minimum size=0.2cm] at (0,4) {};
        \node[circle, fill=black, minimum size=0.2cm] at (2,4) {};
        \node[circle, fill=black, minimum size=0.2cm] at (4,4) {};
        
        % Labels
        \node at (1,1) {A};
        \node at (3,1) {B};
        \node at (1,3) {C};
        \node at (3,3) {D};
        
        % Leyenda
        \node at (6,3) {\textbf{Reglas topológicas:}};
        \node[align=left] at (7.5,2.3) {\small • Sin gaps};
        \node[align=left] at (7.5,1.8) {\small • Sin overlaps};
        \node[align=left] at (7.5,1.3) {\small • Nodos compartidos};
        \node[align=left] at (7.5,0.8) {\small • Arcos compartidos};
    \end{tikzpicture}
    \end{center}
\end{frame}

\section{Datos Raster: Grillas y Resolución}

\begin{frame}{Modelo Raster: Campos Continuos}
    \begin{columns}
        \column{0.5\textwidth}
        \textbf{Estructura de grilla:}
        \begin{itemize}
            \item Matriz de celdas regulares
            \item Cada celda = un valor
            \item Posición implícita
            \item Resolución espacial fija
        \end{itemize}
        
        \vspace{0.3cm}
        \textbf{Componentes:}
        \begin{itemize}
            \item Origen (x₀, y₀)
            \item Tamaño de celda (res)
            \item Dimensiones (rows, cols)
            \item Sistema de referencia
        \end{itemize}
        
        \column{0.5\textwidth}
        \begin{tikzpicture}[scale=0.35]
            % Grilla raster
            \foreach \x in {0,...,7} {
                \foreach \y in {0,...,7} {
                    \pgfmathparse{20+int(rand*60)}
                    \edef\val{\pgfmathresult}
                    \definecolor{cellcolor}{rgb}{\val/100,\val/120,0.2}
                    \fill[cellcolor] (\x,\y) rectangle (\x+1,\y+1);
                    \draw[gray, thin] (\x,\y) rectangle (\x+1,\y+1);
                }
            }
            
            % Ejes
            \draw[thick, ->] (-0.5,0) -- (8.5,0) node[right] {x};
            \draw[thick, ->] (0,-0.5) -- (0,8.5) node[above] {y};
            
            % Etiquetas
            \node at (4,-1.5) {\small Columnas};
            \node[rotate=90] at (-1.5,4) {\small Filas};
        \end{tikzpicture}
    \end{columns}
\end{frame}

\begin{frame}{Resolución Espacial}
    \begin{center}
    \begin{tikzpicture}[scale=0.6]
        % Alta resolución
        \node at (0,4) {\textbf{Alta Resolución}};
        \foreach \x in {0,...,15} {
            \foreach \y in {0,...,15} {
                \pgfmathparse{int(rand*2)}
                \ifnum\pgfmathresult=0
                    \fill[green!60] (\x*0.2,\y*0.2) rectangle (\x*0.2+0.2,\y*0.2+0.2);
                \else
                    \fill[brown!40] (\x*0.2,\y*0.2) rectangle (\x*0.2+0.2,\y*0.2+0.2);
                \fi
            }
        }
        \node at (1.5,-0.5) {\tiny 1m/píxel};
        
        % Media resolución
        \node at (5,4) {\textbf{Media Resolución}};
        \foreach \x in {0,...,7} {
            \foreach \y in {0,...,7} {
                \pgfmathparse{int(rand*3)}
                \ifnum\pgfmathresult=0
                    \fill[green!60] (5+\x*0.4,\y*0.4) rectangle (5+\x*0.4+0.4,\y*0.4+0.4);
                \else
                    \fill[brown!40] (5+\x*0.4,\y*0.4) rectangle (5+\x*0.4+0.4,\y*0.4+0.4);
                \fi
            }
        }
        \node at (6.5,-0.5) {\tiny 10m/píxel};
        
        % Baja resolución
        \node at (10,4) {\textbf{Baja Resolución}};
        \foreach \x in {0,...,3} {
            \foreach \y in {0,...,3} {
                \pgfmathparse{int(rand*3)}
                \ifnum\pgfmathresult=0
                    \fill[green!60] (10+\x*0.8,\y*0.8) rectangle (10+\x*0.8+0.8,\y*0.8+0.8);
                \else
                    \fill[brown!40] (10+\x*0.8,\y*0.8) rectangle (10+\x*0.8+0.8,\y*0.8+0.8);
                \fi
            }
        }
        \node at (11.5,-0.5) {\tiny 30m/píxel};
    \end{tikzpicture}
    \end{center}
    
    \vspace{0.3cm}
    
    \begin{tabular}{|l|c|c|c|}
        \hline
        \textbf{Satélite} & \textbf{Resolución} & \textbf{Cobertura} & \textbf{Aplicación} \\
        \hline
        WorldView-3 & 0.3m & Local & Catastro urbano \\
        Sentinel-2 & 10m & Regional & Agricultura \\
        MODIS & 250-1000m & Global & Clima \\
        \hline
    \end{tabular}
\end{frame}

\begin{frame}[fragile]{Bandas Espectrales}
    \begin{columns}
        \column{0.5\textwidth}
        \textbf{Raster Multibanda:}
        \begin{itemize}
            \item Cada banda = longitud de onda
            \item Stack de matrices
            \item Información espectral
            \item Índices derivados
        \end{itemize}
        
        \vspace{0.3cm}
        \textbf{Bandas comunes:}
        \begin{itemize}
            \item Azul (450-520 nm)
            \item Verde (520-600 nm)
            \item Rojo (630-690 nm)
            \item NIR (760-900 nm)
            \item SWIR (1550-2300 nm)
        \end{itemize}
        
        \column{0.5\textwidth}
        \begin{lstlisting}[language=Python, caption=Trabajo con rasters]
import rasterio
import numpy as np

# Abrir imagen multiespectral
with rasterio.open('sentinel2.tif') as src:
    # Metadatos
    print(f"CRS: {src.crs}")
    print(f"Res: {src.res}")
    print(f"Bandas: {src.count}")
    
    # Leer bandas
    red = src.read(4)   # B4
    nir = src.read(8)   # B8
    
    # Calcular NDVI
    ndvi = (nir - red) / (nir + red + 0.0001)
    
    # Estadísticas
    print(f"NDVI medio: {ndvi.mean():.3f}")
    
    # Guardar resultado
    profile = src.profile
    profile.update(count=1, dtype='float32')
    
    with rasterio.open('ndvi.tif', 'w', 
                       **profile) as dst:
        dst.write(ndvi, 1)
        \end{lstlisting}
    \end{columns}
\end{frame}

\begin{frame}[fragile]{Operaciones Raster}
    \begin{columns}
        \column{0.5\textwidth}
        \textbf{Álgebra de mapas:}
        \begin{itemize}
            \item Operaciones celda a celda
            \item Funciones locales
            \item Funciones focales (kernel)
            \item Funciones zonales
            \item Funciones globales
        \end{itemize}
        
        \vspace{0.3cm}
        \textbf{Aplicaciones:}
        \begin{itemize}
            \item Análisis de terreno
            \item Detección de cambios
            \item Clasificación
            \item Modelado hidrológico
        \end{itemize}
        
        \column{0.5\textwidth}
        \begin{lstlisting}[language=Python, caption=Análisis de terreno]
import richdem as rd
from scipy.ndimage import generic_filter

# Cargar DEM
dem = rd.LoadGDAL("srtm_chile.tif")

# Derivados del terreno
slope = rd.TerrainAttribute(dem, 'slope_degrees')
aspect = rd.TerrainAttribute(dem, 'aspect')
curvature = rd.TerrainAttribute(dem, 'curvature')

# Hillshade
hillshade = rd.hillshade(dem, azimuth=315)

# Filtro focal (media móvil 3x3)
kernel_size = 3
dem_smooth = generic_filter(
    dem, np.mean, size=kernel_size
)

# Detección de crestas
def detect_ridges(dem):
    # Laplaciano
    laplacian = generic_filter(
        dem, lambda x: x[4] - x.mean(), 
        size=(3,3)
    )
    return laplacian > threshold
        \end{lstlisting}
    \end{columns}
\end{frame}

\section{Formatos de Archivos Comunes}

\begin{frame}{Formatos Vectoriales}
    \begin{tabular}{|l|p{3cm}|p{3cm}|p{3cm}|}
        \hline
        \textbf{Formato} & \textbf{Características} & \textbf{Ventajas} & \textbf{Limitaciones} \\
        \hline
        \textbf{Shapefile} (.shp) & 
        Múltiples archivos, estándar ESRI & 
        Universal, compatible & 
        Límite 2GB, sin topología \\
        \hline
        \textbf{GeoJSON} & 
        Texto JSON, web-friendly & 
        Legible, simple & 
        Gran tamaño, sin índices \\
        \hline
        \textbf{GeoPackage} (.gpkg) & 
        SQLite, OGC estándar & 
        Un archivo, con índices & 
        Menos compatible \\
        \hline
        \textbf{KML/KMZ} & 
        XML, Google Earth & 
        Visualización, estilos & 
        Solo WGS84 \\
        \hline
        \textbf{PostGIS} & 
        PostgreSQL extensión & 
        Consultas SQL, multiusuario & 
        Requiere servidor \\
        \hline
    \end{tabular}
    
    \vspace{0.3cm}
    
    \begin{tcolorbox}[colframe=usachblue,colback=blue!5]
        \small
        \textbf{Recomendación 2025:} GeoPackage para archivos locales, PostGIS para producción
    \end{tcolorbox}
\end{frame}

\begin{frame}{Formatos Raster}
    \begin{tabular}{|l|p{3cm}|p{3cm}|p{3cm}|}
        \hline
        \textbf{Formato} & \textbf{Características} & \textbf{Ventajas} & \textbf{Uso típico} \\
        \hline
        \textbf{GeoTIFF} & 
        TIFF con georreferencia & 
        Estándar, compresión & 
        Imágenes satelitales \\
        \hline
        \textbf{NetCDF} & 
        Multidimensional, científico & 
        Series temporales & 
        Datos climáticos \\
        \hline
        \textbf{HDF5} & 
        Jerárquico, big data & 
        Eficiente, metadatos & 
        MODIS, Sentinel \\
        \hline
        \textbf{COG} & 
        Cloud Optimized GeoTIFF & 
        Streaming, web & 
        Servicios cloud \\
        \hline
        \textbf{Zarr} & 
        Array comprimido, chunks & 
        Paralelo, cloud-native & 
        Big data analysis \\
        \hline
    \end{tabular}
    
    \vspace{0.3cm}
    
    \begin{tcolorbox}[colframe=usachorange,colback=orange!5]
        \small
        \textbf{Tendencia:} Formatos cloud-native (COG, Zarr) para análisis escalable
    \end{tcolorbox}
\end{frame}

\begin{frame}[fragile]{Conversión entre Formatos}
    \begin{lstlisting}[language=Python, caption=Conversión de formatos con Python]
import geopandas as gpd
import rasterio
from osgeo import gdal, ogr

# VECTORIAL
# Shapefile a GeoPackage
gdf = gpd.read_file('comunas.shp')
gdf.to_file('comunas.gpkg', driver='GPKG')

# GeoJSON a PostGIS
gdf = gpd.read_file('puntos.geojson')
gdf.to_postgis('puntos', con=engine, if_exists='replace')

# RASTER
# GeoTIFF a COG (Cloud Optimized GeoTIFF)
gdal.Translate('output_cog.tif', 'input.tif',
               format='GTiff',
               creationOptions=['COMPRESS=LZW', 'TILED=YES',
                               'COPY_SRC_OVERVIEWS=YES'])

# NetCDF a GeoTIFF
ds = gdal.Open('NETCDF:"climate.nc":temperature')
gdal.Translate('temperature.tif', ds)

# Raster a Vector (poligonizar)
src_ds = gdal.Open('classified.tif')
srs = osr.SpatialReference()
srs.ImportFromWkt(src_ds.GetProjection())
gdal.Polygonize(src_ds.GetRasterBand(1), None, dst_layer)
    \end{lstlisting}
\end{frame}

\section{Atributos y Geometrías}

\begin{frame}{Relación Atributos-Geometría}
    \begin{center}
    \begin{tikzpicture}[scale=0.8]
        % Tabla de atributos
        \node[draw, rectangle] at (0,2) {
            \begin{tabular}{|c|c|c|}
                \hline
                \textbf{ID} & \textbf{Nombre} & \textbf{Población} \\
                \hline
                1 & Santiago & 404,495 \\
                2 & Providencia & 142,079 \\
                3 & Las Condes & 294,838 \\
                \hline
            \end{tabular}
        };
        \node at (0,0.5) {\textbf{Atributos}};
        
        % Geometrías
        \draw[thick, fill=blue!20] (6,3) -- (7,3.5) -- (7.5,2.5) -- (6.5,2) -- cycle;
        \node at (6.75,2.75) {1};
        
        \draw[thick, fill=green!20] (8,3) -- (9,3.2) -- (8.8,2) -- (8,2.2) -- cycle;
        \node at (8.5,2.6) {2};
        
        \draw[thick, fill=red!20] (7,1) -- (8,1.2) -- (8.2,0.2) -- (7.2,0) -- cycle;
        \node at (7.6,0.6) {3};
        
        \node at (7.5,-0.5) {\textbf{Geometrías}};
        
        % Flecha de relación
        \draw[<->, thick, dashed] (2.5,2) -- (5,2);
        \node at (3.75,2.3) {\small 1:1};
    \end{tikzpicture}
    \end{center}
    
    \vspace{0.3cm}
    
    \begin{columns}
        \column{0.5\textwidth}
        \textbf{Atributos:}
        \begin{itemize}
            \item Datos tabulares
            \item Tipos: texto, número, fecha
            \item Operaciones SQL
        \end{itemize}
        
        \column{0.5\textwidth}
        \textbf{Geometrías:}
        \begin{itemize}
            \item Representación espacial
            \item Coordenadas
            \item Operaciones geométricas
        \end{itemize}
    \end{columns}
\end{frame}

\begin{frame}[fragile]{Consultas Espaciales vs Atributos}
    \begin{lstlisting}[language=Python, caption=Tipos de consultas]
import geopandas as gpd
from shapely.geometry import Point

# Cargar datos
comunas = gpd.read_file('comunas_chile.gpkg')
hospitales = gpd.read_file('hospitales.gpkg')

# CONSULTAS POR ATRIBUTOS
# Filtrar por poblacion
comunas_grandes = comunas[comunas['poblacion'] > 100000]

# Filtrar por nombre
providencia = comunas[comunas['nombre'] == 'Providencia']

# CONSULTAS ESPACIALES
# Buffer de 2km alrededor de hospitales
areas_cobertura = hospitales.geometry.buffer(2000)

# Comunas que intersectan con areas de cobertura
comunas_cubiertas = comunas[
    comunas.intersects(areas_cobertura.unary_union)
]

# CONSULTAS MIXTAS
# Comunas grandes sin hospital cercano
grandes_sin_hospital = comunas_grandes[
    ~comunas_grandes.intersects(areas_cobertura.unary_union)
]

print(f"Comunas sin cobertura hospitalaria: {len(grandes_sin_hospital)}")
    \end{lstlisting}
\end{frame}

\begin{frame}{Índices Espaciales}
    \begin{columns}
        \column{0.5\textwidth}
        \textbf{Problema:}
        \begin{itemize}
            \item Búsquedas O(n²) sin índice
            \item Lento con muchas geometrías
        \end{itemize}
        
        \vspace{0.3cm}
        \textbf{Solución - R-tree:}
        \begin{itemize}
            \item Divide el espacio en rectángulos
            \item Estructura jerárquica
            \item Búsquedas O(log n)
            \item Filtra candidatos rápidamente
        \end{itemize}
        
        \column{0.5\textwidth}
        \begin{tikzpicture}[scale=0.6]
            % R-tree visualization
            \draw[thick] (0,0) rectangle (6,6);
            
            % Level 1
            \draw[dashed, blue] (0,0) rectangle (3,3);
            \draw[dashed, blue] (3,0) rectangle (6,3);
            \draw[dashed, blue] (0,3) rectangle (3,6);
            \draw[dashed, blue] (3,3) rectangle (6,6);
            
            % Geometrías
            \fill[red!50] (0.5,0.5) circle (0.2);
            \fill[red!50] (1.5,1) circle (0.2);
            \fill[red!50] (4,1) circle (0.2);
            \fill[red!50] (5,2) circle (0.2);
            \fill[red!50] (1,4) circle (0.2);
            \fill[red!50] (2,5) circle (0.2);
            \fill[red!50] (4.5,4.5) circle (0.2);
            
            % Query
            \draw[thick, green] (3.5,0.5) rectangle (5.5,2.5);
            \node at (4.5,-0.5) {\small Query};
        \end{tikzpicture}
        
        \vspace{0.3cm}
        \small
        Solo examina geometrías en rectángulos que intersectan con la consulta
    \end{columns}
\end{frame}

\section{Vector vs Raster: ¿Cuándo usar cada uno?}

\begin{frame}{Criterios de Selección}
    \begin{center}
    \begin{tabular}{|l|c|c|}
        \hline
        \textbf{Criterio} & \textbf{Vector} & \textbf{Raster} \\
        \hline
        Fenómenos discretos & \checkmark & \\
        Fenómenos continuos & & \checkmark \\
        Precisión geométrica & \checkmark & \\
        Análisis de superficies & & \checkmark \\
        Topología/redes & \checkmark & \\
        Imágenes/teledetección & & \checkmark \\
        Almacenamiento eficiente & \checkmark & \\
        Álgebra de mapas & & \checkmark \\
        Consultas por atributos & \checkmark & \\
        Modelado 3D terreno & & \checkmark \\
        \hline
    \end{tabular}
    \end{center}
    
    \vspace{0.3cm}
    
    \begin{columns}
        \column{0.5\textwidth}
        \begin{tcolorbox}[colframe=blue!50,colback=blue!5,title=Usa Vector]
            \small
            \begin{itemize}
                \item Catastro
                \item Redes de transporte
                \item Límites administrativos
                \item POIs
            \end{itemize}
        \end{tcolorbox}
        
        \column{0.5\textwidth}
        \begin{tcolorbox}[colframe=orange!50,colback=orange!5,title=Usa Raster]
            \small
            \begin{itemize}
                \item Elevación (DEM)
                \item Temperatura
                \item Precipitación
                \item NDVI/cobertura
            \end{itemize}
        \end{tcolorbox}
    \end{columns}
\end{frame}

\begin{frame}[fragile]{Integración Vector-Raster}
    \begin{lstlisting}[language=Python, caption=Análisis combinado]
import geopandas as gpd
import rasterio
from rasterio import features
from rasterstats import zonal_stats

# Escenario: Analizar NDVI promedio por comuna

# 1. Cargar vector (comunas)
comunas = gpd.read_file('comunas.gpkg')

# 2. Cargar raster (NDVI)
with rasterio.open('ndvi_santiago.tif') as src:
    ndvi = src.read(1)
    transform = src.transform

# 3. Estadisticas zonales (raster -> vector)
stats = zonal_stats(
    comunas.geometry, 
    ndvi, 
    transform=transform,
    stats=['mean', 'std', 'min', 'max']
)

# 4. Agregar resultados al GeoDataFrame
comunas['ndvi_mean'] = [s['mean'] for s in stats]
comunas['ndvi_std'] = [s['std'] for s in stats]

# 5. Rasterizar vector (vector -> raster)
comunas['zone_id'] = range(len(comunas))
zones = features.rasterize(
    [(geom, value) for geom, value in zip(comunas.geometry, comunas.zone_id)],
    out_shape=ndvi.shape,
    transform=transform
)
    \end{lstlisting}
\end{frame}

\section{Ejercicio Práctico}

\begin{frame}{Caso de Estudio: Análisis de Accesibilidad}
    \textbf{Problema:} Evaluar accesibilidad a áreas verdes en Santiago
    
    \vspace{0.3cm}
    
    \textbf{Datos disponibles:}
    \begin{itemize}
        \item Vector: Parques (polígonos), Población por manzana (polígonos)
        \item Raster: Imagen Sentinel-2, Modelo Digital de Elevación
    \end{itemize}
    
    \vspace{0.3cm}
    
    \textbf{Pasos del análisis:}
    \begin{enumerate}
        \item Calcular NDVI desde Sentinel-2 (raster)
        \item Identificar áreas verdes: NDVI > 0.4 (raster → vector)
        \item Combinar con parques oficiales (vector)
        \item Crear buffers de 300m, 500m, 1000m (vector)
        \item Calcular población con acceso (overlay vectorial)
        \item Generar mapa de calor de accesibilidad (vector → raster)
    \end{enumerate}
    
    \begin{tcolorbox}[colframe=green!50,colback=green!5]
        \centering
        Este ejercicio combina ambos modelos de datos
    \end{tcolorbox}
\end{frame}

\begin{frame}[fragile]{Implementación del Ejercicio}
    \begin{lstlisting}[language=Python, caption=Solución integrada]
# 1. Calcular NDVI
red = rasterio.open('B04.tif').read(1)
nir = rasterio.open('B08.tif').read(1)
ndvi = (nir - red) / (nir + red + 0.0001)

# 2. Vectorizar areas verdes
mask = ndvi > 0.4
shapes = features.shapes(mask.astype(np.int16), transform=transform)
verde_detectado = gpd.GeoDataFrame.from_features(shapes)

# 3. Combinar con parques oficiales
parques = gpd.read_file('parques_oficiales.gpkg')
areas_verdes = gpd.GeoDataFrame(
    pd.concat([parques, verde_detectado], ignore_index=True)
)

# 4. Buffers de accesibilidad
buffers = {
    300: areas_verdes.buffer(300),
    500: areas_verdes.buffer(500),
    1000: areas_verdes.buffer(1000)
}

# 5. Poblacion con acceso
manzanas = gpd.read_file('manzanas_censo.gpkg')
for dist, buffer in buffers.items():
    acceso = manzanas[manzanas.intersects(buffer.unary_union)]
    print(f"Poblacion con parque a {dist}m: {acceso['poblacion'].sum():,}")
    \end{lstlisting}
\end{frame}

\section{Resumen y Próximos Pasos}

\begin{frame}{Conceptos Clave de Hoy}
    \begin{columns}
        \column{0.5\textwidth}
        \textbf{Modelo Vectorial:}
        \begin{itemize}
            \item Puntos, líneas, polígonos
            \item Geometrías explícitas
            \item Topología
            \item GeoPackage, PostGIS
            \item GeoPandas, Shapely
        \end{itemize}
        
        \column{0.5\textwidth}
        \textbf{Modelo Raster:}
        \begin{itemize}
            \item Grillas regulares
            \item Resolución espacial
            \item Bandas espectrales
            \item GeoTIFF, COG
            \item Rasterio, GDAL
        \end{itemize}
    \end{columns}
    
    \vspace{0.5cm}
    
    \begin{center}
    \begin{tikzpicture}[scale=0.8]
        % Vector
        \node[draw, circle, fill=blue!20] at (0,0) {Vector};
        % Raster
        \node[draw, circle, fill=orange!20] at (4,0) {Raster};
        % Intersección
        \draw (0,0) circle (1.5);
        \draw (4,0) circle (1.5);
        \node at (2,0) {\small Integración};
    \end{tikzpicture}
    \end{center}
\end{frame}

\begin{frame}{Para Profundizar}
    \textbf{Lecturas recomendadas:}
    \begin{itemize}
        \item Geocomputation with Python - Cap. 2 y 3
        \item PostGIS in Action - Cap. 1-3
        \item Remote Sensing and GIS - Cap. 4
    \end{itemize}
    
    \vspace{0.3cm}
    
    \textbf{Ejercicios para practicar:}
    \begin{enumerate}
        \item Descargar comunas de Chile desde el IDE
        \item Calcular área y perímetro con GeoPandas
        \item Descargar Sentinel-2 y calcular NDVI
        \item Convertir Shapefile a GeoPackage
        \item Hacer overlay de dos capas vectoriales
    \end{enumerate}
    
    \vspace{0.3cm}
    
    \textbf{Próxima clase:} 
    \begin{tcolorbox}[colframe=usachblue,colback=blue!5]
        Jueves: Sistemas de Referencia Espacial (CRS) + Lab 1
    \end{tcolorbox}
\end{frame}

\begin{frame}{Cierre}
    \begin{center}
        \Large{¿Preguntas?}
        
        \vspace{1cm}
        
        \faIcon{envelope} francisco.parra.o@usach.cl
        
        \vspace{0.5cm}
        
        \faIcon{clock} Horario de consulta: Jueves después del lab
        
        \vspace{1cm}
        
        \textbf{¡Nos vemos el jueves con el primer laboratorio!}
    \end{center}
\end{frame}

\end{document}