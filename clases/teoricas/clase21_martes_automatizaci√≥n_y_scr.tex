\documentclass[10pt]{beamer}
\usetheme{metropolis}
\usepackage{FiraSans}
\usefonttheme{professionalfonts}

\usepackage{graphicx}
\usepackage{tikz}
\usepackage[spanish]{babel}
\usepackage{tcolorbox}
\usepackage{ragged2e}
\usepackage{pgfplots}
\pgfplotsset{compat=1.18}
\usepackage{listings}
\usepackage{xcolor}

\definecolor{codegreen}{rgb}{0,0.6,0}
\definecolor{codegray}{rgb}{0.5,0.5,0.5}
\definecolor{codepurple}{rgb}{0.58,0,0.82}
\definecolor{backcolour}{rgb}{0.95,0.95,0.92}

\lstdefinestyle{mystyle}{
    backgroundcolor=\color{backcolour},   
    commentstyle=\color{codegreen},
    keywordstyle=\color{magenta},
    numberstyle=\tiny\color{codegray},
    stringstyle=\color{codepurple},
    basicstyle=\ttfamily\footnotesize,
    breakatwhitespace=false,         
    breaklines=true,                 
    captionpos=b,                    
    keepspaces=true,                 
    numbers=left,                    
    numbersep=5pt,                  
    showspaces=false,                
    showstringspaces=false,
    showtabs=false,                  
    tabsize=2
}

\lstset{style=mystyle}

\newcommand{\examplebox}[2]{
\begin{tcolorbox}[colframe=darkcardinal,colback=boxgray,title=#1]
#2
\end{tcolorbox}
}

% Personalización del pie de página
\setbeamertemplate{footline}{%
  \begin{beamercolorbox}[wd=\paperwidth,sep=2ex]{footline}%
    \usebeamerfont{structure}\textbf{Geoinformática - Clase 21} \hfill Profesor: Francisco Parra O. \hfill \textbf{Semestre 2, 2025}
  \end{beamercolorbox}%
}

\title{Clase 21: Automatización y scripting}
\subtitle{Programación para geoprocesamiento}
\author{Profesor: Francisco Parra O.}
\institute{USACH - Ingeniería Civil en Informática}
\date{\today}

\titlegraphic{%
  \begin{tikzpicture}[overlay, remember picture]
    \node[anchor=north east, yshift=0cm] at (current page.north east) {
      \includegraphics[width=1.5cm]{Logo-Color-Usach-Web.jpg}
    };
  \end{tikzpicture}
}

\begin{document}

\maketitle

\begin{frame}{Agenda}
    \tableofcontents
\end{frame}


\section{Funciones personalizadas}

\begin{frame}{Funciones personalizadas}
    % Contenido a desarrollar en clase
    \begin{itemize}
        \item Concepto clave 1
        \item Concepto clave 2
        \item Concepto clave 3
        \item Ejemplo práctico
    \end{itemize}
\end{frame}

\begin{frame}{Profundización}
    % Detalles adicionales del tema
    \begin{itemize}
        \item Aspecto detallado 1
        \item Aspecto detallado 2
        \item Consideraciones importantes
    \end{itemize}
\end{frame}

\section{Manejo de errores}

\begin{frame}{Manejo de errores}
    % Contenido a desarrollar en clase
    \begin{itemize}
        \item Concepto clave 1
        \item Concepto clave 2
        \item Concepto clave 3
        \item Ejemplo práctico
    \end{itemize}
\end{frame}

\begin{frame}{Profundización}
    % Detalles adicionales del tema
    \begin{itemize}
        \item Aspecto detallado 1
        \item Aspecto detallado 2
        \item Consideraciones importantes
    \end{itemize}
\end{frame}

\section{Logging y debugging}

\begin{frame}{Logging y debugging}
    % Contenido a desarrollar en clase
    \begin{itemize}
        \item Concepto clave 1
        \item Concepto clave 2
        \item Concepto clave 3
        \item Ejemplo práctico
    \end{itemize}
\end{frame}

\begin{frame}{Profundización}
    % Detalles adicionales del tema
    \begin{itemize}
        \item Aspecto detallado 1
        \item Aspecto detallado 2
        \item Consideraciones importantes
    \end{itemize}
\end{frame}

\section{Buenas prácticas}

\begin{frame}{Buenas prácticas}
    % Contenido a desarrollar en clase
    \begin{itemize}
        \item Concepto clave 1
        \item Concepto clave 2
        \item Concepto clave 3
        \item Ejemplo práctico
    \end{itemize}
\end{frame}

\begin{frame}{Profundización}
    % Detalles adicionales del tema
    \begin{itemize}
        \item Aspecto detallado 1
        \item Aspecto detallado 2
        \item Consideraciones importantes
    \end{itemize}
\end{frame}


\begin{frame}{Cierre}
    \begin{center}
        \Large{¿Preguntas?}
        
        \vspace{1cm}
        
        Próxima clase: Jueves (teórica + laboratorio)
    \end{center}
\end{frame}

\end{document}