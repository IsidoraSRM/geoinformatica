\documentclass[10pt]{beamer}
\usetheme{metropolis}
\usepackage{FiraSans}
\usefonttheme{professionalfonts}

\usepackage{graphicx}
\usepackage{tikz}
\usepackage[spanish]{babel}
\usepackage{tcolorbox}
\usepackage{ragged2e}
\usepackage{pgfplots}
\pgfplotsset{compat=1.18}
\usepackage{listings}
\usepackage{xcolor}

\definecolor{codegreen}{rgb}{0,0.6,0}
\definecolor{codegray}{rgb}{0.5,0.5,0.5}
\definecolor{codepurple}{rgb}{0.58,0,0.82}
\definecolor{backcolour}{rgb}{0.95,0.95,0.92}

\lstdefinestyle{mystyle}{
    backgroundcolor=\color{backcolour},   
    commentstyle=\color{codegreen},
    keywordstyle=\color{magenta},
    numberstyle=\tiny\color{codegray},
    stringstyle=\color{codepurple},
    basicstyle=\ttfamily\footnotesize,
    breakatwhitespace=false,         
    breaklines=true,                 
    captionpos=b,                    
    keepspaces=true,                 
    numbers=left,                    
    numbersep=5pt,                  
    showspaces=false,                
    showstringspaces=false,
    showtabs=false,                  
    tabsize=2
}

\lstset{style=mystyle}

\newcommand{\examplebox}[2]{
\begin{tcolorbox}[colframe=darkcardinal,colback=boxgray,title=#1]
#2
\end{tcolorbox}
}

% Personalización del pie de página
\setbeamertemplate{footline}{%
  \begin{beamercolorbox}[wd=\paperwidth,sep=2ex]{footline}%
    \usebeamerfont{structure}\textbf{Geoinformática - Laboratorio 8} \hfill Profesor: Francisco Parra O. \hfill \textbf{Semestre 2, 2025}
  \end{beamercolorbox}%
}

\title{Laboratorio 08: Integración raster-vector}
\subtitle{Combinando tipos de datos}
\author{Profesor: Francisco Parra O.}
\institute{USACH - Ingeniería Civil en Informática}
\date{\today}

\titlegraphic{%
  \begin{tikzpicture}[overlay, remember picture]
    \node[anchor=north east, yshift=0cm] at (current page.north east) {
      \includegraphics[width=1.5cm]{Logo-Color-Usach-Web.jpg}
    };
  \end{tikzpicture}
}

\begin{document}

\maketitle

\begin{frame}{Agenda}
    \tableofcontents
\end{frame}


\section{Objetivos del laboratorio}

\begin{frame}{Objetivos de hoy}
    \begin{itemize}
        \item Extracción de valores a puntos
        \item Conversión entre formatos
        \item Análisis combinado
        \item Ejercicio con datos ambientales
    \end{itemize}
\end{frame}

\section{Configuración inicial}

\begin{frame}{Preparación del entorno}
    \begin{itemize}
        \item Cargar librerías necesarias
        \item Configurar directorio de trabajo
        \item Descargar datos de ejemplo
        \item Verificar instalaciones
    \end{itemize}
\end{frame}

\begin{frame}[fragile]{Código inicial}
    \begin{lstlisting}[language=R]
# Cargar librerías
library(sf)
library(terra)
library(tmap)
library(tidyverse)

# Configurar directorio
setwd("~/geoinformatica/lab8")

# Verificar
sf::sf_extSoftVersion()
    \end{lstlisting}
\end{frame}

\section{Ejercicios prácticos}

\begin{frame}{Ejercicio 1}
    \textbf{Objetivo:} Aplicar conceptos de la clase teórica
    
    \begin{enumerate}
        \item Paso 1: Preparación de datos
        \item Paso 2: Aplicación de técnicas
        \item Paso 3: Análisis de resultados
        \item Paso 4: Visualización
    \end{enumerate}
\end{frame}

\begin{frame}[fragile]{Implementación}
    \begin{lstlisting}[language=R]
# Código del ejercicio
# A completar durante el laboratorio
    \end{lstlisting}
\end{frame}

\begin{frame}{Ejercicio 2}
    \textbf{Objetivo:} Práctica avanzada
    
    \begin{itemize}
        \item Descripción del problema
        \item Datos disponibles
        \item Resultado esperado
        \item Criterios de evaluación
    \end{itemize}
\end{frame}

\section{Proyecto semestral}

\begin{frame}{Trabajo en proyecto}
    \textbf{Tiempo dedicado al proyecto:} 30 minutos
    
    \begin{itemize}
        \item Aplicar lo aprendido al proyecto
        \item Consultas con el profesor
        \item Trabajo en equipo
        \item Avance incremental
    \end{itemize}
\end{frame}

\section{Cierre y tareas}

\begin{frame}{Para la próxima clase}
    \begin{itemize}
        \item Completar ejercicios pendientes
        \item Revisar material complementario
        \item Avanzar en el proyecto según calendario
        \item Preparar dudas para consultar
    \end{itemize}
\end{frame}


\begin{frame}{Cierre}
    \begin{center}
        \Large{¿Preguntas?}
        
        \vspace{1cm}
        
        Próxima semana: Clase teórica 17
    \end{center}
\end{frame}

\end{document}