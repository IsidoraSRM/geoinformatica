\documentclass[12pt,a4paper]{article}
\usepackage[utf8]{inputenc}
\usepackage[spanish,es-tabla]{babel}
\usepackage{geometry}
\geometry{margin=2.5cm}
\usepackage{graphicx}
\usepackage{float}
\usepackage{amsmath,amssymb,amsthm}
\usepackage{enumerate}
\usepackage{listings}
\usepackage{xcolor}
\usepackage{hyperref}
\usepackage{tcolorbox}
\usepackage{booktabs}
\usepackage{longtable}
\usepackage{array}
\usepackage{multirow}
\usepackage{fancyhdr}
\usepackage{tikz}
\usetikzlibrary{shapes,arrows,positioning,calc}
\usepackage{dirtytalk}

% Configuración de código
\lstset{
    language=Python,
    basicstyle=\small\ttfamily,
    keywordstyle=\color{blue!80!black},
    commentstyle=\color{green!60!black},
    stringstyle=\color{red!80!black},
    numbers=left,
    numberstyle=\tiny\color{gray},
    breaklines=true,
    frame=single,
    backgroundcolor=\color{gray!10},
    captionpos=b
}

% Configuración de encabezados
\pagestyle{fancy}
\fancyhf{}
\lhead{Manual del Profesor - Clase 06}
\rhead{Geoestadística y Análisis Espacial}
\cfoot{\thepage}

% Comandos personalizados
\newtcolorbox{conceptbox}[2][]{
    colback=blue!5!white,
    colframe=blue!75!black,
    title=#2,
    fonttitle=\bfseries,
    #1
}

\newtcolorbox{alertbox}[1][]{
    colback=red!5!white,
    colframe=red!75!black,
    fonttitle=\bfseries,
    #1
}

\newtcolorbox{ejemplo}[2][]{
    colback=green!5!white,
    colframe=green!75!black,
    title=#2,
    fonttitle=\bfseries,
    #1
}

\newtcolorbox{reflexion}[1][]{
    colback=yellow!5!white,
    colframe=yellow!75!black,
    title=Para reflexionar,
    fonttitle=\bfseries,
    #1
}

\newtcolorbox{analogia}[2][]{
    colback=purple!5!white,
    colframe=purple!75!black,
    title=#2,
    fonttitle=\bfseries,
    #1
}

\title{
    \vspace{-2cm}
    \Large{UNIVERSIDAD DE SANTIAGO DE CHILE} \\
    \large{Facultad de Ingeniería} \\
    \large{Departamento de Ingeniería Informática} \\
    \vspace{1cm}
    \LARGE{\textbf{Manual del Profesor}} \\
    \Large{\textbf{Clase 06: Geoestadística y Análisis Espacial}} \\
    \vspace{0.5cm}
    \large{Curso: Desarrollo de Aplicaciones Geoinformáticas}
}

\author{
    Prof. Francisco Parra O. \\
    \texttt{francisco.parra.o@usach.cl}
}

\date{Versión: \today}

\begin{document}

\maketitle
\thispagestyle{empty}
\newpage

\tableofcontents
\newpage

%==============================================================================
\section{Introducción y Filosofía de la Geoestadística}
%==============================================================================

\subsection{¿Qué es la Geoestadística y Por Qué Importa?}

\begin{conceptbox}{Definición Fundamental}
La geoestadística es la rama de la estadística que reconoce que el espacio importa. Mientras la estadística clásica asume que las observaciones son independientes, la geoestadística abraza y modela la dependencia espacial como una fuente de información valiosa, no como un problema a resolver.
\end{conceptbox}

Imaginemos dos escenarios para entender la diferencia fundamental:

\textbf{Escenario 1 - Estadística Clásica:} Un investigador mide la altura de 100 personas seleccionadas aleatoriamente en Santiago. Cada medición es independiente; conocer la altura de una persona no nos dice nada sobre la altura de otra.

\textbf{Escenario 2 - Geoestadística:} Un investigador mide la concentración de contaminantes en 100 puntos de Santiago. Aquí, conocer el valor en un punto SÍ nos da información sobre puntos cercanos. Si hay alta contaminación en Pudahuel, es probable que áreas adyacentes también la tengan.

\subsection{La Intuición Detrás de la Dependencia Espacial}

\begin{analogia}{El Café Derramado}
Imagine que derrama café sobre una mesa. El líquido no se distribuye aleatoriamente - forma un patrón continuo donde las áreas mojadas están conectadas. Si toca la mesa en un punto y está mojada, es muy probable que los puntos inmediatamente adyacentes también lo estén. A medida que se aleja del punto inicial, la probabilidad de encontrar café disminuye gradualmente. Este es el principio fundamental de la autocorrelación espacial.
\end{analogia}

Esta intuición se formaliza en la Primera Ley de Geografía de Waldo Tobler (1970):

\begin{conceptbox}{Primera Ley de Geografía}
\say{Todo está relacionado con todo lo demás, pero las cosas cercanas están más relacionadas que las cosas distantes.}

Esta ley no es una ley física como la gravedad, sino un principio empírico que observamos consistentemente en fenómenos geográficos. Es la base filosófica de toda la geoestadística.
\end{conceptbox}

\subsection{Implicaciones Profundas de la Ley de Tobler}

\subsubsection{1. El Espacio Como Información}

Cuando sabemos que existe dependencia espacial, la ubicación se convierte en información predictiva. Considere estos ejemplos:

\begin{itemize}
    \item \textbf{Precios inmobiliarios:} Una casa lujosa probablemente está rodeada de otras casas de similar valor
    \item \textbf{Tipos de suelo:} El suelo arcilloso no cambia abruptamente a arenoso; hay transiciones graduales
    \item \textbf{Temperatura:} No pasa de 30°C a 0°C en un metro; los gradientes son suaves
    \item \textbf{Comportamiento social:} Las preferencias políticas tienden a agruparse geográficamente
\end{itemize}

\subsubsection{2. El Problema de la Pseudo-replicación}

En estadística clásica, más datos generalmente significa más información. En geoestadística, esto no siempre es cierto:

\begin{ejemplo}{La Paradoja del Muestreo Espacial}
Imagine que quiere estimar la temperatura promedio de Santiago. Tiene dos opciones:
\begin{enumerate}
    \item Tomar 100 mediciones, todas en Las Condes
    \item Tomar 20 mediciones distribuidas por toda la ciudad
\end{enumerate}

Aunque la opción 1 tiene más datos, la opción 2 probablemente dará una mejor estimación. Los 100 puntos en Las Condes son pseudo-réplicas: parecen ser 100 observaciones independientes, pero en realidad contienen mucha información redundante debido a su proximidad.
\end{ejemplo}

\subsubsection{3. La Falacia Ecológica y el MAUP}

\begin{alertbox}
\textbf{Falacia Ecológica:} Inferir características individuales desde datos agregados espacialmente.

Ejemplo: Si una comuna tiene ingreso promedio alto, NO significa que todos sus residentes sean ricos.
\end{alertbox}

\begin{alertbox}
\textbf{MAUP (Modifiable Areal Unit Problem):} Los resultados del análisis pueden cambiar dramáticamente según cómo agreguemos los datos espacialmente.

Ejemplo: La segregación racial puede parecer alta o baja dependiendo de si analizamos por manzanas, barrios o comunas.
\end{alertbox}

%==============================================================================
\section{Conceptos Fundamentales de Autocorrelación Espacial}
%==============================================================================

\subsection{¿Qué es la Autocorrelación Espacial?}

La autocorrelación espacial mide qué tan similar es un valor en una ubicación con los valores en ubicaciones vecinas. Es \say{auto} porque correlacionamos una variable consigo misma, pero en diferentes lugares del espacio.

\begin{analogia}{Las Ondas en el Agua}
Tire una piedra a un estanque. Las ondas se propagan desde el punto de impacto, creando un patrón donde la altura del agua en cualquier punto está relacionada con la altura en puntos cercanos. La autocorrelación espacial es similar: un \say{impacto} en un lugar (como una fábrica contaminante) crea efectos que se propagan espacialmente.
\end{analogia}

\subsection{Tipos de Patrones Espaciales}

\subsubsection{Autocorrelación Positiva}

\textbf{Concepto:} Valores similares tienden a estar cerca.

\begin{ejemplo}{Barrios Socioeconómicos}
Los barrios ricos tienden a estar cerca de otros barrios ricos (Las Condes, Vitacura, La Dehesa), mientras que los barrios de menores ingresos también se agrupan (La Pintana, San Ramón, La Granja). Este no es un accidente sino el resultado de procesos sociales, económicos e históricos que crean segregación espacial.
\end{ejemplo}

\textbf{¿Por qué ocurre?}
\begin{itemize}
    \item \textbf{Procesos de difusión:} Las enfermedades se propagan a vecinos
    \item \textbf{Factores ambientales compartidos:} Áreas con mismo clima tienen vegetación similar
    \item \textbf{Interacciones sociales:} Las personas influyen en sus vecinos
    \item \textbf{Políticas espaciales:} Zonificación urbana crea áreas homogéneas
\end{itemize}

\subsubsection{Autocorrelación Negativa}

\textbf{Concepto:} Valores diferentes tienden a estar cerca.

\begin{ejemplo}{Competencia Territorial}
En la naturaleza, los territorios de depredadores muestran autocorrelación negativa: si hay un león en un área, es menos probable encontrar otro león inmediatamente al lado (competencia). En ciudades, negocios similares a veces se dispersan para evitar competencia directa (aunque otras veces se agrupan para beneficiarse de economías de aglomeración).
\end{ejemplo}

\subsubsection{Patrón Aleatorio}

\textbf{Concepto:} No hay relación entre valores y ubicación.

\begin{reflexion}
Los patrones verdaderamente aleatorios son raros en geografía. Casi siempre hay algún proceso espacial operando. Cuando encontramos aleatoriedad aparente, puede significar:
\begin{itemize}
    \item Los procesos operan a una escala diferente a la que estamos midiendo
    \item Múltiples procesos se cancelan mutuamente
    \item Nuestra muestra es insuficiente para detectar el patrón
\end{itemize}
\end{reflexion}

%==============================================================================
\section{Matrices de Pesos Espaciales: El Corazón del Análisis}
%==============================================================================

\subsection{¿Qué Son y Por Qué Importan?}

Las matrices de pesos espaciales ($W$) son la forma matemática de definir \say{vecindad} o \say{cercanía}. Son fundamentales porque toda la geoestadística depende de cómo definimos qué observaciones están relacionadas espacialmente.

\begin{conceptbox}{Definición Intuitiva}
Una matriz de pesos espaciales es como una red social del espacio: define quién es \say{amigo} (vecino) de quién y qué tan fuerte es esa amistad (peso).
\end{conceptbox}

\subsection{Tipos de Vecindad y Sus Implicaciones}

\subsubsection{Contigüidad: Los Vecinos que se Tocan}

\begin{analogia}{El Tablero de Ajedrez}
\begin{itemize}
    \item \textbf{Vecindad Queen (Reina):} Como la reina en ajedrez, considera vecinos a todos los que comparten borde o vértice. Una celda interior tiene 8 vecinos.
    \item \textbf{Vecindad Rook (Torre):} Como la torre, solo considera vecinos a los que comparten borde. Una celda interior tiene 4 vecinos.
\end{itemize}
\end{analogia}

\textbf{¿Cuándo usar cada una?}

\begin{itemize}
    \item \textbf{Queen:} Cuando los procesos pueden propagarse en cualquier dirección (enfermedades, información)
    \item \textbf{Rook:} Cuando la propagación requiere contacto directo sustancial (flujo de agua entre parcelas)
\end{itemize}

\subsubsection{Distancia: Cuando el Espacio es Continuo}

\textbf{K-Vecinos Más Cercanos}

Define como vecinos a los $k$ lugares más cercanos, independientemente de la distancia absoluta.

\begin{ejemplo}{Análisis de Competencia}
Una tienda considera competidores a las 5 tiendas similares más cercanas, sin importar si están a 100m o 10km. Esto es útil cuando la densidad varía mucho (urbano vs rural).
\end{ejemplo}

\textbf{Umbral de Distancia}

Todos los lugares dentro de distancia $d$ son vecinos.

\begin{ejemplo}{Propagación de Contaminación}
Una fábrica contamina hasta 2km a la redonda. Todos los puntos dentro de ese radio son afectados, sin importar cuántos sean.
\end{ejemplo}

\subsection{La Importancia de la Estandarización}

\begin{conceptbox}{Estandarización por Fila}
Hacer que cada fila de la matriz $W$ sume 1. Esto convierte los pesos en promedios ponderados, facilitando la interpretación.
\end{conceptbox}

\textbf{Intuición:} Sin estandarización, lugares con más vecinos tendrían más influencia total, lo cual puede no ser deseable.

%==============================================================================
\section{El Índice de Moran: Midiendo la Autocorrelación Global}
%==============================================================================

\subsection{La Intuición del Índice de Moran}

El Índice de Moran $I$ es como un coeficiente de correlación, pero para el espacio. Mide si los lugares cercanos tienen valores más similares de lo esperado por azar.

\begin{analogia}{La Metáfora de la Fiesta}
Imagine una fiesta donde cada persona tiene una edad. El Índice de Moran pregunta: \say{¿Las personas de edades similares tienden a conversar entre sí (agruparse), o la mezcla es aleatoria?}
\begin{itemize}
    \item $I > 0$: Los jóvenes hablan con jóvenes, los mayores con mayores (agrupamiento)
    \item $I = 0$: Conversaciones aleatorias respecto a edad
    \item $I < 0$: Los jóvenes buscan hablar con mayores y viceversa (dispersión)
\end{itemize}
\end{analogia}

\subsection{Interpretación Matemática y Práctica}

La fórmula del Índice de Moran:

$$I = \frac{n}{\sum_{i,j} w_{ij}} \cdot \frac{\sum_{i,j} w_{ij}(x_i - \bar{x})(x_j - \bar{x})}{\sum_i (x_i - \bar{x})^2}$$

\textbf{Descomposición intuitiva:}

\begin{enumerate}
    \item $(x_i - \bar{x})(x_j - \bar{x})$: Producto de desviaciones. Positivo si ambos valores están del mismo lado de la media
    \item $w_{ij}$: Solo consideramos este producto si $i$ y $j$ son vecinos
    \item La suma captura la covarianza espacial total
    \item La normalización permite comparar entre diferentes variables y escalas
\end{enumerate}

\subsection{Significancia Estadística: ¿Es Real el Patrón?}

\begin{conceptbox}{La Pregunta Clave}
No basta con que $I \neq 0$. Debemos preguntarnos: ¿Qué tan probable es obtener este valor de $I$ si los datos fueran espacialmente aleatorios?
\end{conceptbox}

\textbf{Aproximación por Permutaciones:}
\begin{enumerate}
    \item Mantener las ubicaciones fijas
    \item Permutar aleatoriamente los valores entre ubicaciones
    \item Calcular $I$ para cada permutación
    \item Comparar $I$ observado con la distribución de permutaciones
\end{enumerate}

\begin{ejemplo}{Interpretación Práctica}
Si hacemos 999 permutaciones y nuestro $I$ observado es mayor que todos los $I$ permutados, entonces $p < 0.001$. Esto sugiere fuertemente que el patrón no es aleatorio.
\end{ejemplo}

%==============================================================================
\section{LISA: Descomponiendo Patrones Globales en Locales}
%==============================================================================

\subsection{La Necesidad del Análisis Local}

El Índice de Moran global es un promedio que puede ocultar variación local importante.

\begin{analogia}{La Temperatura Promedio}
Decir que la temperatura promedio de Chile es 15°C oculta que el norte es desértico y caluroso mientras el sur es frío y lluvioso. Similarmente, un Moran global de 0 podría ocultar hot spots y cold spots que se cancelan mutuamente.
\end{analogia}

\subsection{Los Cuatro Tipos de Asociación Local}

LISA (Local Indicators of Spatial Association) identifica cuatro tipos de relaciones locales:

\begin{conceptbox}{Tipología LISA}
\begin{enumerate}
    \item \textbf{High-High (HH):} Valores altos rodeados de valores altos - \say{Hot spots}
    \item \textbf{Low-Low (LL):} Valores bajos rodeados de valores bajos - \say{Cold spots}
    \item \textbf{High-Low (HL):} Valor alto rodeado de valores bajos - \say{Outlier espacial}
    \item \textbf{Low-High (LH):} Valor bajo rodeado de valores altos - \say{Outlier espacial}
\end{enumerate}
\end{conceptbox}

\begin{ejemplo}{Aplicación en Criminalidad}
\begin{itemize}
    \item \textbf{HH:} Zonas de alta criminalidad (requieren intervención policial)
    \item \textbf{LL:} Zonas seguras (pueden servir como modelo)
    \item \textbf{HL:} Punto caliente aislado (¿presencia de factor criminógeno específico?)
    \item \textbf{LH:} Oasis de seguridad en zona peligrosa (¿qué lo hace diferente?)
\end{itemize}
\end{ejemplo}

\subsection{Significancia vs Magnitud}

\begin{alertbox}
Un cluster LISA puede ser estadísticamente significativo pero prácticamente irrelevante, o viceversa. Siempre considere:
\begin{itemize}
    \item \textbf{Significancia estadística:} ¿Es improbable por azar?
    \item \textbf{Magnitud del efecto:} ¿Es lo suficientemente grande para importar?
    \item \textbf{Contexto sustantivo:} ¿Tiene sentido en el mundo real?
\end{itemize}
\end{alertbox}

%==============================================================================
\section{Interpolación Espacial: Prediciendo lo Desconocido}
%==============================================================================

\subsection{El Problema Fundamental}

En geografía, frecuentemente necesitamos estimar valores en ubicaciones no muestreadas. Los mapas de temperatura, precipitación, contaminación, todos requieren interpolación porque es imposible medir en todos los puntos.

\begin{conceptbox}{Principio de Interpolación}
La interpolación espacial se basa en la Primera Ley de Tobler: usamos valores conocidos en ubicaciones cercanas para estimar valores desconocidos, asumiendo que la cercanía implica similitud.
\end{conceptbox}

\subsection{Métodos Determinísticos vs Geoestadísticos}

\subsubsection{Métodos Determinísticos}

Usan fórmulas matemáticas fijas sin modelo estadístico subyacente.

\textbf{IDW (Inverse Distance Weighting):}

\begin{analogia}{La Influencia que Decae}
Imagine que cada punto de medición es una estufa que calienta su entorno. El calor (influencia) disminuye con la distancia. IDW formaliza esto: puntos más cercanos tienen más peso en la estimación.
\end{analogia}

Ventajas:
\begin{itemize}
    \item Simple e intuitivo
    \item No requiere análisis previo
    \item Rápido de calcular
\end{itemize}

Desventajas:
\begin{itemize}
    \item No proporciona medida de incertidumbre
    \item Crea \say{ojos de buey} alrededor de puntos de datos
    \item El parámetro de potencia es arbitrario
\end{itemize}

\subsubsection{Métodos Geoestadísticos (Kriging)}

Usan un modelo estadístico de la variabilidad espacial.

\textbf{La Filosofía del Kriging:}

\begin{conceptbox}{BLUE - Best Linear Unbiased Estimator}
Kriging busca el mejor estimador lineal insesgado:
\begin{itemize}
    \item \textbf{Best:} Minimiza la varianza del error
    \item \textbf{Linear:} Es una combinación lineal de los datos
    \item \textbf{Unbiased:} En promedio, no sobreestima ni subestima
    \item \textbf{Estimator:} Predice valores desconocidos
\end{itemize}
\end{conceptbox}

%==============================================================================
\section{El Semivariograma: La Huella Digital de la Variabilidad Espacial}
%==============================================================================

\subsection{¿Qué es un Semivariograma?}

El semivariograma es una función que describe cómo la diferencia entre valores cambia con la distancia. Es la herramienta fundamental para entender la estructura de dependencia espacial.

\begin{analogia}{La Conversación a Distancia}
Imagine medir qué tan diferentes son las conversaciones entre personas según la distancia que las separa:
\begin{itemize}
    \item A 1 metro (misma mesa): Conversaciones muy similares (mismo tema)
    \item A 10 metros (mesas diferentes): Conversaciones algo diferentes
    \item A 100 metros (diferentes salones): Conversaciones completamente independientes
\end{itemize}
El semivariograma cuantifica esta relación.
\end{analogia}

\subsection{Componentes Clave del Semivariograma}

\begin{conceptbox}{Anatomía del Semivariograma}
\begin{enumerate}
    \item \textbf{Nugget (Pepita):} Variabilidad a distancia cero
    \begin{itemize}
        \item Representa error de medición
        \item O variabilidad a escala más fina que el muestreo
    \end{itemize}
    
    \item \textbf{Sill (Meseta):} Varianza total donde se estabiliza
    \begin{itemize}
        \item Representa la varianza total del proceso
        \item Indica independencia espacial alcanzada
    \end{itemize}
    
    \item \textbf{Range (Rango):} Distancia donde se alcanza el sill
    \begin{itemize}
        \item Define la zona de influencia
        \item Más allá del rango, no hay correlación espacial
    \end{itemize}
\end{enumerate}
\end{conceptbox}

\begin{ejemplo}{Interpretación Práctica - Concentración de Minerales}
En una mina:
\begin{itemize}
    \item \textbf{Nugget alto:} Gran variabilidad local (vetas delgadas)
    \item \textbf{Range corto:} Cambios abruptos en mineralización
    \item \textbf{Range largo:} Depósitos extensos y continuos
\end{itemize}
\end{ejemplo}

%==============================================================================
\section{Análisis de Patrones de Puntos}
%==============================================================================

\subsection{La Naturaleza de los Patrones Puntuales}

Los eventos puntuales (crímenes, enfermedades, tiendas, árboles) pueden distribuirse de tres formas fundamentales:

\begin{conceptbox}{Complete Spatial Randomness (CSR)}
CSR es la hipótesis nula en análisis de patrones: los eventos ocurren independientemente con intensidad constante en el espacio. Es el equivalente espacial de lanzar dardos con los ojos vendados.
\end{conceptbox}

\subsection{Función K de Ripley: Analizando a Múltiples Escalas}

\begin{analogia}{El Zoom Progresivo}
La función K es como hacer zoom out progresivamente en un mapa:
\begin{itemize}
    \item A escala pequeña: ¿Los puntos están agrupados localmente?
    \item A escala media: ¿Hay patrones de clusters?
    \item A escala grande: ¿Cuál es la estructura general?
\end{itemize}
\end{analogia}

\textbf{Interpretación de la Función L (K normalizada):}
\begin{itemize}
    \item $L(r) > 0$: Agrupamiento a distancia $r$
    \item $L(r) = 0$: Aleatoriedad a distancia $r$
    \item $L(r) < 0$: Dispersión regular a distancia $r$
\end{itemize}

\begin{ejemplo}{Ecología Urbana}
Los árboles en un parque natural vs un parque diseñado:
\begin{itemize}
    \item \textbf{Natural:} Agrupamiento a escala pequeña (reproducción local), aleatorio a gran escala
    \item \textbf{Diseñado:} Regular a todas las escalas (plantación uniforme)
\end{itemize}
\end{ejemplo}

\subsection{Getis-Ord G*: Identificando Hot y Cold Spots}

El estadístico $G^*$ identifica dónde se concentran valores altos o bajos.

\begin{conceptbox}{Diferencia con LISA}
\begin{itemize}
    \item \textbf{LISA:} Identifica cualquier tipo de cluster (HH, LL, HL, LH)
    \item \textbf{G*:} Solo identifica concentraciones de valores altos o bajos
    \item \textbf{LISA:} Énfasis en la diferencia con vecinos
    \item \textbf{G*:} Énfasis en la magnitud absoluta
\end{itemize}
\end{conceptbox}

%==============================================================================
\section{Regresión Espacial: Cuando el Espacio Afecta las Relaciones}
%==============================================================================

\subsection{Por Qué OLS Falla con Datos Espaciales}

\begin{ejemplo}{El Problema del Contagio}
Imagine estudiar el efecto de la educación en el ingreso usando OLS. Si vecinos educados tienden a vivir juntos (segregación) y se influencian mutuamente (spillovers), entonces:
\begin{itemize}
    \item Los errores están correlacionados espacialmente
    \item El efecto de una observación \say{contamina} a sus vecinas
    \item Los errores estándar están subestimados
    \item Los tests de hipótesis son inválidos
\end{itemize}
\end{ejemplo}

\subsection{Modelos Espaciales: Incorporando la Geografía}

\subsubsection{Spatial Lag Model (SAR)}

\textbf{Intuición:} El valor en un lugar depende de los valores en lugares vecinos.

$$y_i = \rho \sum_j w_{ij} y_j + X_i\beta + \varepsilon_i$$

\begin{ejemplo}{Precios Inmobiliarios}
El precio de una casa depende no solo de sus características (metros cuadrados, habitaciones) sino también del precio promedio del barrio. Si los vecinos remodelan y aumentan sus precios, tu casa también vale más.
\end{ejemplo}

\subsubsection{Spatial Error Model (SEM)}

\textbf{Intuición:} Los errores están espacialmente correlacionados.

$$y_i = X_i\beta + u_i$$
$$u_i = \lambda \sum_j w_{ij} u_j + \varepsilon_i$$

\begin{ejemplo}{Rendimiento Agrícola}
El modelo incluye lluvia y fertilizante, pero omite calidad del suelo. Como la calidad del suelo varía suavemente en el espacio, los errores de parcelas cercanas están correlacionados.
\end{ejemplo}

\subsection{GWR: Cuando las Relaciones Varían en el Espacio}

\begin{conceptbox}{Geographically Weighted Regression}
GWR permite que cada lugar tenga sus propios coeficientes de regresión. Es como ajustar una regresión diferente para cada punto, pero usando información de lugares cercanos.
\end{conceptbox}

\begin{ejemplo}{Factores de Obesidad}
La relación entre ingreso y obesidad puede ser:
\begin{itemize}
    \item \textbf{Negativa en zonas urbanas ricas:} Mayor ingreso = comida más saludable
    \item \textbf{Positiva en zonas rurales:} Mayor ingreso = más consumo calórico
    \item \textbf{Neutra en zonas de transición}
\end{itemize}
GWR captura esta heterogeneidad espacial.
\end{ejemplo}

%==============================================================================
\section{Aplicaciones Prácticas y Casos de Estudio}
%==============================================================================

\subsection{Caso 1: Epidemiología Espacial}

\begin{ejemplo}{COVID-19 en Santiago}
\textbf{Problema:} Identificar factores de propagación y zonas de riesgo.

\textbf{Metodología:}
\begin{enumerate}
    \item \textbf{ESDA:} Moran's I reveló fuerte autocorrelación positiva (I = 0.65)
    \item \textbf{LISA:} Identificó clusters en comunas de alta densidad y movilidad
    \item \textbf{Regresión espacial:} SAR mostró que densidad poblacional y uso de transporte público explicaban 70\% de variación
    \item \textbf{GWR:} Reveló que el efecto del transporte público variaba: más fuerte en comunas periféricas
\end{enumerate}

\textbf{Implicaciones de política:}
\begin{itemize}
    \item Focalizar recursos en hot spots identificados
    \item Estrategias diferenciadas según características locales
    \item Considerar efectos de derrame entre comunas
\end{itemize}
\end{ejemplo}

\subsection{Caso 2: Valoración Inmobiliaria}

\begin{ejemplo}{Impacto del Metro en Precios}
\textbf{Problema:} Medir cómo una nueva estación de metro afecta los precios inmobiliarios.

\textbf{Desafíos espaciales:}
\begin{itemize}
    \item Efecto decae con distancia (no lineal)
    \item Spillovers entre propiedades
    \item Anticipación del mercado
\end{itemize}

\textbf{Solución:}
\begin{itemize}
    \item Kriging para interpolar precios pre y post metro
    \item Diferencia de superficies para identificar impacto
    \item Considerar anisotropía (efecto más fuerte a lo largo de la línea)
\end{itemize}
\end{ejemplo}

%==============================================================================
\section{Guía Pedagógica para el Profesor}
%==============================================================================

\subsection{Estrategias de Enseñanza}

\subsubsection{Secuencia Recomendada}

\begin{enumerate}
    \item \textbf{Comenzar con intuición}
    \begin{itemize}
        \item Use ejemplos cotidianos antes de formalizar
        \item Mapas de calor de temperatura son intuitivos
        \item Todos entienden que casas cercanas tienen precios similares
    \end{itemize}
    
    \item \textbf{Construir gradualmente}
    \begin{itemize}
        \item Patrón visual → Medición → Significancia → Modelado
        \item No introducir fórmulas hasta que la intuición esté clara
    \end{itemize}
    
    \item \textbf{Enfatizar interpretación sobre cálculo}
    \begin{itemize}
        \item El software calcula; los humanos interpretan
        \item ¿Qué significa este resultado en el mundo real?
        \item ¿Cambia alguna decisión basada en este análisis?
    \end{itemize}
\end{enumerate}

\subsubsection{Actividades Interactivas}

\begin{enumerate}
    \item \textbf{Ejercicio del Mapa Mental}
    \begin{itemize}
        \item Dar a estudiantes un mapa con puntos
        \item Pedirles que identifiquen visualmente patrones
        \item Comparar con resultados estadísticos
        \item Discutir discrepancias
    \end{itemize}
    
    \item \textbf{Simulación de Procesos Espaciales}
    \begin{itemize}
        \item Simular difusión (enfermedad, información)
        \item Variar parámetros y observar cambios
        \item Conectar con medidas de autocorrelación
    \end{itemize}
    
    \item \textbf{Debate de Definición de Vecindad}
    \begin{itemize}
        \item Mismo dataset, diferentes matrices W
        \item Comparar resultados
        \item Discutir cuál es \say{correcta} (no hay respuesta única)
    \end{itemize}
\end{enumerate}

\subsection{Errores Conceptuales Comunes}

\begin{alertbox}
\textbf{Errores a anticipar y corregir:}

\begin{enumerate}
    \item \textbf{\say{Correlación implica causalidad espacial}}
    \begin{itemize}
        \item Correlación espacial puede ser espuria
        \item Variables omitidas espacialmente correlacionadas
        \item Necesidad de teoría, no solo estadística
    \end{itemize}
    
    \item \textbf{\say{Un método es siempre mejor}}
    \begin{itemize}
        \item IDW vs Kriging depende del contexto
        \item Diferentes matrices W para diferentes procesos
        \item No hay \say{talla única} en geoestadística
    \end{itemize}
    
    \item \textbf{\say{Significancia estadística = importancia práctica}}
    \begin{itemize}
        \item Con muchos datos, todo es significativo
        \item Magnitud del efecto importa
        \item Contexto sustantivo es clave
    \end{itemize}
    
    \item \textbf{\say{Los clusters son fijos}}
    \begin{itemize}
        \item Clusters dependen de escala
        \item Pueden cambiar con el tiempo
        \item Son sensibles a definición de vecindad
    \end{itemize}
\end{enumerate}
\end{alertbox}

\subsection{Evaluación del Aprendizaje}

\subsubsection{Evaluación Conceptual}

\begin{enumerate}
    \item \textbf{Preguntas de Interpretación}
    \begin{itemize}
        \item Dado un Moran's I = 0.7, ¿qué significa?
        \item ¿Por qué elegir Queen vs Rook?
        \item ¿Cuándo falla Kriging?
    \end{itemize}
    
    \item \textbf{Análisis de Casos}
    \begin{itemize}
        \item Presentar resultado y pedir interpretación
        \item Identificar problemas en análisis dados
        \item Proponer mejoras a estudios existentes
    \end{itemize}
\end{enumerate}

\subsubsection{Evaluación Práctica}

\begin{enumerate}
    \item \textbf{Proyecto Integrador}
    \begin{itemize}
        \item Dataset real de su interés
        \item Análisis completo desde ESDA hasta modelado
        \item Énfasis en decisiones metodológicas justificadas
        \item Interpretación sustantiva de resultados
    \end{itemize}
    
    \item \textbf{Presentación de Paper}
    \begin{itemize}
        \item Elegir paper reciente de geoestadística
        \item Explicar métodos a compañeros
        \item Crítica constructiva
        \item Proponer extensiones
    \end{itemize}
\end{enumerate}

%==============================================================================
\section{Recursos y Referencias}
%==============================================================================

\subsection{Lecturas Esenciales}

\begin{enumerate}
    \item \textbf{Fundamentos}
    \begin{itemize}
        \item Tobler, W. (1970). \say{A computer movie simulating urban growth in the Detroit region}
        \item Anselin, L. (1995). \say{Local indicators of spatial association—LISA}
    \end{itemize}
    
    \item \textbf{Libros de Texto}
    \begin{itemize}
        \item Bivand, Pebesma \& Gómez-Rubio (2013). \say{Applied Spatial Data Analysis with R}
        \item Rey, Arribas-Bel \& Wolf (2020). \say{Geographic Data Science with Python}
    \end{itemize}
    
    \item \textbf{Aplicaciones}
    \begin{itemize}
        \item Fotheringham et al. (2002). \say{Geographically Weighted Regression}
        \item Cressie (1993). \say{Statistics for Spatial Data}
    \end{itemize}
\end{enumerate}

\subsection{Herramientas de Software}

\textbf{Python:}
\begin{itemize}
    \item PySAL: Suite completa de análisis espacial
    \item GeoPandas: Manipulación de datos espaciales
    \item scikit-gstat: Semivariogramas
\end{itemize}

\textbf{R:}
\begin{itemize}
    \item spdep: Dependencia espacial
    \item gstat: Geoestadística
    \item GWmodel: GWR
\end{itemize}

\subsection{Datasets de Práctica}

\begin{itemize}
    \item Datos INE Chile: Censos y encuestas georreferenciadas
    \item OpenStreetMap: Datos vectoriales libres
    \item SINCA: Datos ambientales de Chile
    \item Portal de Datos Abiertos: Múltiples fuentes gubernamentales
\end{itemize}

%==============================================================================
\section{Reflexiones Finales}
%==============================================================================

\begin{conceptbox}{El Arte y la Ciencia de la Geoestadística}
La geoestadística es tanto arte como ciencia. La ciencia proporciona las herramientas matemáticas y estadísticas rigurosas. El arte está en:
\begin{itemize}
    \item Elegir la escala apropiada de análisis
    \item Definir vecindades significativas
    \item Interpretar resultados en contexto
    \item Comunicar hallazgos efectivamente
    \item Reconocer las limitaciones
\end{itemize}
\end{conceptbox}

\begin{reflexion}
Como profesor, su rol no es solo enseñar técnicas, sino desarrollar intuición espacial. Los estudiantes deben aprender a \say{pensar espacialmente}: ver patrones, cuestionar la aleatoriedad, entender que la geografía importa. Las fórmulas se olvidan; la intuición espacial permanece.
\end{reflexion}

\end{document}