\documentclass[11pt,a4paper]{article}
\usepackage[utf8]{inputenc}
\usepackage[spanish]{babel}
\usepackage{geometry}
\geometry{margin=2.5cm}
\usepackage{graphicx}
\usepackage{xcolor}
\usepackage{tcolorbox}
\usepackage{listings}
\usepackage{hyperref}
\usepackage{enumitem}
\usepackage{fancyhdr}
\usepackage{titlesec}
\usepackage{array}
\usepackage{tikz}
\usepackage{mdframed}
\usepackage{amssymb}

% Colores personalizados
\definecolor{usachblue}{RGB}{0,121,192}
\definecolor{usachred}{RGB}{239,51,64}
\definecolor{codegreen}{rgb}{0,0.6,0}
\definecolor{codegray}{rgb}{0.5,0.5,0.5}
\definecolor{codepurple}{rgb}{0.58,0,0.82}
\definecolor{backcolour}{rgb}{0.95,0.95,0.92}
\definecolor{timecolor}{RGB}{150,50,50}

% Configuración de listings
\lstdefinestyle{mystyle}{
    backgroundcolor=\color{backcolour},   
    commentstyle=\color{codegreen},
    keywordstyle=\color{magenta},
    numberstyle=\tiny\color{codegray},
    stringstyle=\color{codepurple},
    basicstyle=\ttfamily\footnotesize,
    breakatwhitespace=false,         
    breaklines=true,                 
    captionpos=b,                    
    keepspaces=true,                 
    numbers=left,                    
    numbersep=5pt,                  
    showspaces=false,                
    showstringspaces=false,
    showtabs=false,                  
    tabsize=2,
    frame=single,
    literate={á}{{\'a}}1 {é}{{\'e}}1 {í}{{\'i}}1 {ó}{{\'o}}1 {ú}{{\'u}}1 {ñ}{{\~n}}1
}
\lstset{style=mystyle}

% Configuración de secciones
\titleformat{\section}[block]{\normalfont\Large\bfseries\color{usachblue}}{\thesection}{1em}{}
\titleformat{\subsection}[block]{\normalfont\large\bfseries\color{usachred}}{\thesubsection}{1em}{}

% Encabezado y pie
\pagestyle{fancy}
\fancyhf{}
\fancyhead[L]{\small Manual del Profesor - Clase 02}
\fancyhead[R]{\small Geoinformática 2025}
\fancyfoot[C]{\thepage}

% Comandos personalizados
\newcommand{\tiempo}[1]{\textcolor{timecolor}{\textbf{[#1]}}}
\newcommand{\decir}[1]{\begin{tcolorbox}[colback=blue!5,colframe=usachblue,title={DECIR}]#1\end{tcolorbox}}
\newcommand{\hacer}[1]{\begin{tcolorbox}[colback=green!5,colframe=green!50!black,title={HACER}]#1\end{tcolorbox}}
\newcommand{\nota}[1]{\begin{tcolorbox}[colback=yellow!10,colframe=orange,title={NOTA}]#1\end{tcolorbox}}
\newcommand{\alerta}[1]{\begin{tcolorbox}[colback=red!5,colframe=red,title={ALERTA}]#1\end{tcolorbox}}

\title{{\Huge \textbf{Manual del Profesor}}\\[0.5cm]
{\Large Clase 02: Fundamentos de Geocomputación}\\[0.3cm]
{\large Historia, herramientas y ecosistema de desarrollo}}
\author{Prof. Francisco Parra O.\\
Geólogo, PhD en Informática\\
\texttt{francisco.parra.o@usach.cl}}
\date{Semestre 2, 2025\\
Duración: 80 minutos}

\begin{document}

\maketitle
\thispagestyle{empty}
\newpage

\tableofcontents
\newpage

\section{Introducción}

Este manual proporciona una guía detallada para dictar la Clase 02 del curso de Geoinformática. Está diseñado para ser completamente autocontenido, permitiendo que cualquier profesor pueda conducir la clase exitosamente.

\subsection{Objetivos de Aprendizaje}

Al finalizar esta clase de 80 minutos, los estudiantes serán capaces de:

\begin{enumerate}
    \item \textbf{Comprender} la evolución histórica desde los SIG tradicionales hasta la Geocomputación moderna
    \item \textbf{Identificar} las principales herramientas de software geoespacial y sus casos de uso
    \item \textbf{Comparar} las fortalezas y debilidades de Python vs R para análisis espacial
    \item \textbf{Ejecutar} un primer script geoespacial básico en Python
    \item \textbf{Planificar} su ambiente de trabajo para el laboratorio
\end{enumerate}

\section{Preparación Previa (15 minutos antes de clase)}

\subsection{Checklist de Materiales}

\begin{mdframed}[backgroundcolor=gray!10]
\textbf{Hardware y Software:}
\begin{itemize}[label=$\square$]
    \item Computador con Python (Anaconda) y R (RStudio) instalados
    \item Presentación PDF cargada en modo presentación
    \item Conexión a internet estable (tener hotspot móvil como backup)
    \item Proyector/pantalla funcionando correctamente
    \item Adaptadores de video necesarios
\end{itemize}

\textbf{Materiales de Clase:}
\begin{itemize}[label=$\square$]
    \item Pizarra y al menos 3 plumones de colores diferentes
    \item Mouse presentador o puntero láser
    \item Botella de agua para hidratación
    \item Lista de asistencia (si aplica)
\end{itemize}

\textbf{Archivos Digitales:}
\begin{itemize}[label=$\square$]
    \item santiago\_comunas.shp - Shapefile de comunas
    \item hospitales.csv - Datos de ubicación
    \item demo\_script.py - Script Python funcional
    \item demo\_script.R - Script R funcional
    \item Links de recursos en documento compartido
\end{itemize}
\end{mdframed}

\subsection{Configuración del Ambiente}

\begin{lstlisting}[language=bash, caption=Terminal 1: Preparar Python]
# Activar ambiente Python
cd ~/demos_clase02
conda activate geo  # o source geo_env/bin/activate

# Verificar instalaciones
python --version
python -c "import geopandas; print('OK')"

# Abrir Jupyter
jupyter notebook
\end{lstlisting}

\begin{lstlisting}[language=R, caption=Terminal 2: Preparar R]
# En RStudio, pre-cargar bibliotecas
library(sf)
library(tmap)
library(dplyr)

# Verificar datos
file.exists("santiago_comunas.shp")
\end{lstlisting}

\newpage

\section{Desarrollo de la Clase}

\subsection{Apertura y Bienvenida \tiempo{0:00-0:02}}

\decir{Buenos días a todos. Bienvenidos a nuestra segunda clase de Geoinformática. Hoy profundizaremos en los fundamentos de la Geocomputación, su historia fascinante, y las herramientas que usaremos durante todo el semestre.}

\hacer{
\begin{itemize}
    \item Hacer contacto visual con varios estudiantes
    \item Verificar que todos pueden ver la presentación
    \item Ajustar iluminación si es necesario
    \item Confirmar funcionamiento del audio
\end{itemize}
}

\nota{El tono debe ser entusiasta pero profesional. Es jueves primera hora, los estudiantes pueden estar cansados.}

\subsection{Repaso de Clase Anterior \tiempo{0:03-0:06}}

\decir{Antes de comenzar, hagamos un breve repaso. En nuestra primera clase definimos qué es la Geocomputación. ¿Alguien recuerda la diferencia entre datos vectoriales y raster?}

\hacer{
\begin{itemize}
    \item Esperar 30 segundos para respuestas voluntarias
    \item Si no hay respuestas, señalar a un estudiante específicamente
    \item Agradecer cada participación: "Exacto, muy bien"
    \item Complementar las respuestas si es necesario
\end{itemize}
}

\textbf{Respuesta esperada:} Vectorial = puntos, líneas, polígonos con coordenadas precisas. Raster = grilla de celdas/píxeles con valores.

\alerta{Si menos del 30\% instaló software, dedicar 5 minutos extra en el laboratorio para instalación.}

\newpage

\section{Sección 1: Historia y Evolución \tiempo{20 minutos totales}}

\subsection{Los Precursores de la Geocomputación \tiempo{0:06-0:09}}

\subsubsection{Historia de John Snow (1854)}

\decir{La Geocomputación no comienza con computadores. Comienza en 1854 con un médico llamado John Snow en Londres, durante una epidemia de cólera.}

\hacer{
\begin{itemize}
    \item Dibujar en la pizarra un mapa simple:
    \begin{itemize}
        \item Rectángulos para manzanas de casas
        \item X para marcar casos de cólera
        \item Círculo para la bomba de agua
    \end{itemize}
    \item Usar colores diferentes para enfatizar el patrón
\end{itemize}
}

\textbf{Elementos clave para mencionar:}
\begin{itemize}
    \item Teoría predominante: "miasmas" o aire contaminado
    \item Método de Snow: mapeo sistemático de casos
    \item Descubrimiento: clustering alrededor de bomba de Broad Street
    \item Solución: remover manija de la bomba
    \item Resultado: caída dramática de casos
\end{itemize}

\nota{Esta historia conecta emocionalmente a los estudiantes con el poder del análisis espacial.}

\subsubsection{Conexión con el Presente}

\decir{¿Les suena familiar? Es exactamente lo que hicimos con COVID-19 en 2020, pero ahora con millones de datos procesados en tiempo real. La pregunta fundamental no ha cambiado: ¿dónde están los casos y qué patrón revelan?}

\subsection{La Era de los SIG (1960-1990) \tiempo{0:09-0:13}}

\subsubsection{CGIS - El Primer SIG}

\decir{1963 marca el nacimiento del primer SIG operacional: CGIS - Canada Geographic Information System.}

\textbf{Contexto importante:}
\begin{itemize}
    \item Canadá: 2do país más grande del mundo (10 millones km²)
    \item Necesidad: inventario de recursos naturales
    \item Imposible mapear manualmente millones de hectáreas
    \item Costo: \$15 millones CAD (1960) = \$150 millones actuales
\end{itemize}

\textbf{Capacidad revolucionaria de CGIS:}

\decir{CGIS podía superponer capas de información - algo revolucionario. Imaginen: una capa de suelos, otra de vegetación, otra de hidrología. La pregunta "muéstrame áreas con suelo tipo A, bosque de coníferas, cerca de un río" tomaba horas en mainframes del tamaño de esta sala.}

\subsubsection{Nacimiento de ESRI}

\decir{1969: Jack y Laura Dangermond fundan ESRI en su garage en California. Hoy es una empresa de 2 billones de dólares. ArcGIS sigue siendo el estándar de la industria.}

\subsection{El Nacimiento de la Geocomputación \tiempo{0:13-0:17}}

\subsubsection{Cambio de Paradigma (1996)}

\decir{En los 90s, los académicos sentían que los SIG se habían estancado. Eran buenos para mapear y almacenar, pero ¿para entender procesos? ¿Para predecir? ¿Para simular?}

\textbf{La cita fundamental de Stan Openshaw:}

\begin{mdframed}[backgroundcolor=blue!10]
\textit{"Geocomputation is about using the various different types of geodata and about developing relevant geo-tools within an 'intelligent' IT framework"}
\end{mdframed}

\subsubsection{Diferencia Fundamental}

\hacer{Escribir en la pizarra una tabla comparativa:}

\begin{center}
\begin{tabular}{|l|l|}
\hline
\textbf{SIG Tradicional} & \textbf{Geocomputación} \\
\hline
¿Dónde está el hospital más cercano? & ¿Dónde construir el próximo hospital? \\
¿Cuántos incendios hubo? & ¿Dónde será el próximo incendio? \\
¿Cuál es la ruta más corta? & ¿Cómo evolucionará el tráfico? \\
Responde preguntas & Hace predicciones \\
\hline
\end{tabular}
\end{center}

\textbf{Nuevas técnicas introducidas:}
\begin{itemize}
    \item Autómatas celulares (física)
    \item Redes neuronales (IA)
    \item Algoritmos genéticos (biología)
    \item Simulación basada en agentes
\end{itemize}

\subsection{Era Web GIS y Big Data \tiempo{0:17-0:26}}

\subsubsection{La Revolución de Google Maps (2005)}

\decir{El 8 de febrero de 2005 cambió todo. Google lanza Google Maps. De repente, cualquier persona podía hacer análisis espacial sin saberlo.}

\textbf{Contraste dramático:}
\begin{itemize}
    \item ANTES: Software \$5000, semanas de training, datos caros
    \item DESPUÉS: Navegador web, gratis, intuitivo, datos incluidos
\end{itemize}

\subsubsection{Caso Chile: Terremoto 27F}

\decir{El terremoto del 27F en 2010 fue uno de los primeros desastres mapeados colaborativamente. Voluntarios mundiales usaron OpenStreetMap para mapear zonas afectadas en tiempo real. El gobierno chileno usó estos mapas para coordinar ayuda.}

\subsubsection{Big Data Geoespacial Actual}

\textbf{Crecimiento exponencial de datos:}
\begin{itemize}
    \item 2010: 0.5 petabytes/año
    \item 2025: 200 petabytes/año (400x más)
    \item Contexto: 1 petabyte = 31 años si 1GB = 1 segundo
\end{itemize}

\textbf{Fuentes de datos actuales:}
\begin{enumerate}
    \item Sentinel-2: fotografía toda la Tierra cada 5 días (1.5 TB/día)
    \item 5 billones de smartphones con GPS
    \item IoT: Santiago tiene 500+ sensores de calidad del aire
    \item Un vuelo LiDAR: 100 GB/hora
\end{enumerate}

\newpage

\section{Sección 2: Software para Análisis Geoespacial \tiempo{20 minutos}}

\subsection{Panorama del Software \tiempo{0:26-0:29}}

\hacer{Mostrar y explicar el diagrama de dispersión en la diapositiva}

\textbf{Explicación del diagrama:}
\begin{itemize}
    \item Eje X: Facilidad de uso
    \item Eje Y: Capacidad analítica
\end{itemize}

\textbf{Analogía útil:}

\decir{Es como herramientas de un mecánico:
\begin{itemize}
    \item Google Earth = Destornillador básico
    \item QGIS = Caja de herramientas completa
    \item Python/R = Taller profesional
    \item PostGIS = Fábrica de herramientas
\end{itemize}}

\subsection{QGIS vs ArcGIS \tiempo{0:29-0:36}}

\subsubsection{Historia de QGIS}

\decir{2002: Gary Sherman, frustrado con el costo de ArcGIS, crea QGIS. Lo libera gratis. Hoy tiene millones de usuarios. Un desarrollador cambió la industria.}

\subsubsection{Comparación Objetiva}

\begin{center}
\begin{tabular}{|l|c|c|}
\hline
\textbf{Característica} & \textbf{QGIS} & \textbf{ArcGIS Pro} \\
\hline
Costo & Gratis & \$700-3000/año \\
Sistema Operativo & Win/Mac/Linux & Solo Windows \\
Curva aprendizaje & Moderada & Empinada \\
Documentación & Comunitaria & Profesional \\
Python & Sí & Sí \\
3D & Básico & Avanzado \\
\hline
\end{tabular}
\end{center}

\nota{Ser honesto: ArcGIS es más pulido, pero QGIS hace 95\% de las tareas igual de bien.}

\subsection{Plataformas Cloud \tiempo{0:36-0:40}}

\subsubsection{Google Earth Engine}

\decir{Google Earth Engine procesa 40 años de imágenes satelitales en segundos. Es procesamiento distribuido masivo.}

\alerta{Si hay internet estable, considerar demo de 1 minuto en code.earthengine.google.com}

\subsubsection{Otras Plataformas}

\begin{itemize}
    \item \textbf{Microsoft Planetary Computer:} Enfoque en datos abiertos
    \item \textbf{AWS:} Poder computacional crudo
    \item \textbf{Mapbox:} APIs de mapas y navegación
\end{itemize}

\textbf{Problema del cloud:} Vendor lock-in y costos ocultos a largo plazo.

\subsection{Bases de Datos Espaciales \tiempo{0:40-0:46}}

\subsubsection{¿Por qué PostGIS?}

\decir{Imaginen Uber: "Encuentra el conductor más cercano". Con 10,000 conductores activos, calcular distancia a cada uno tomaría segundos. Inaceptable. PostGIS usa índices R-tree: encuentra el más cercano en milisegundos.}

\textbf{Ejemplo de query espacial:}

\begin{lstlisting}[language=SQL]
SELECT nombre, ST_Area(geom)/10000 as hectareas
FROM parcelas
WHERE ST_Within(geom, 
    (SELECT geom FROM comunas WHERE nombre='Santiago')
);
\end{lstlisting}

\newpage

\section{Sección 3: Ecosistema Python y R \tiempo{25 minutos}}

\subsection{¿Por qué Programar para GIS? \tiempo{0:46-0:49}}

\subsubsection{El Argumento de Eficiencia}

\decir{Caso real: Cliente necesita análisis mensual de 100 comunas. Manual: 2 días. Con Python: 10 minutos. Después de 3 meses, el script ahorró 35 días de trabajo.}

\hacer{Escribir en la pizarra la regla de oro:}

\begin{center}
\fbox{\Large Si lo haces más de 3 veces → AUTOMATÍZALO}
\end{center}

\textbf{Ventajas de programar:}
\begin{enumerate}
    \item Reproducibilidad
    \item Automatización
    \item Escalabilidad
    \item Versionado (Git)
    \item Compartir código
\end{enumerate}

\subsection{Ecosistema Python \tiempo{0:49-0:58}}

\subsubsection{Bibliotecas Principales}

\textbf{GeoPandas - El corazón:}

\decir{GeoPandas es pandas + geografía. Si saben pandas, ya saben 70\% de GeoPandas.}

\begin{lstlisting}[language=Python]
# Pandas normal
df[df['poblacion'] > 100000]

# GeoPandas - igual pero con superpoderes
gdf[gdf['poblacion'] > 100000].plot()  # Mapa!
\end{lstlisting}

\textbf{Otras bibliotecas clave:}
\begin{itemize}
    \item \textbf{Shapely:} Geometrías puras
    \item \textbf{Rasterio:} Datos raster
    \item \textbf{Folium:} Mapas web (3 líneas = mapa interactivo)
    \item \textbf{Contextily:} Mapas base
\end{itemize}

\subsubsection{Ejemplo Completo Python}

\begin{lstlisting}[language=Python]
import geopandas as gpd
from shapely.geometry import Point

# Leer comunas
comunas = gpd.read_file('santiago_comunas.shp')

# Crear hospitales
hospitales = gpd.GeoDataFrame(
    {'nombre': ['Hospital 1', 'Hospital 2'],
     'geometry': [Point(-70.65, -33.45), 
                  Point(-70.60, -33.42)]},
    crs='EPSG:4326')

# Buffer 2km y análisis
areas = hospitales.buffer(0.02)
servidas = comunas[comunas.intersects(areas.unary_union)]

print(f"Comunas con cobertura: {len(servidas)}")
\end{lstlisting}

\subsection{Ecosistema R \tiempo{0:58-1:07}}

\subsubsection{Filosofía de R}

\decir{R fue creado por estadísticos, para estadísticos. Trata todo como datos para analizar. Python trata todo como objetos para procesar.}

\textbf{Paquetes principales:}
\begin{itemize}
    \item \textbf{sf:} Simple Features - revolucionó R espacial
    \item \textbf{terra:} Raster, 10x más rápido que el antiguo 'raster'
    \item \textbf{tmap:} Mapas de calidad publicación
    \item \textbf{leaflet:} Mapas interactivos
\end{itemize}

\subsubsection{Ejemplo R}

\begin{lstlisting}[language=R]
library(sf)
library(tmap)

# Leer datos
comunas <- st_read("santiago_comunas.shp")

# Pipe operator para flujo de trabajo
comunas %>%
  filter(poblacion > 100000) %>%
  tm_shape() +
  tm_polygons("poblacion", palette = "Blues")
\end{lstlisting}

\subsection{Python vs R: La Verdad \tiempo{1:07-1:11}}

\decir{La pregunta del millón: ¿Python o R? Mi respuesta: DEPENDE.}

\textbf{Usa Python cuando:}
\begin{itemize}
    \item Integración con aplicaciones web
    \item Deep Learning (TensorFlow, PyTorch)
    \item Pipelines de producción
    \item Trabajas con ingenieros
\end{itemize}

\textbf{Usa R cuando:}
\begin{itemize}
    \item Análisis estadístico espacial complejo
    \item Visualización para publicación académica
    \item Investigación
    \item Trabajas con estadísticos
\end{itemize}

\nota{Secreto profesional: Los expertos usan ambos. Python para producción, R para exploración.}

\newpage

\section{Sección 4: Ejemplos Prácticos \tiempo{15 minutos}}

\subsection{Flujo de Trabajo Típico \tiempo{1:11-1:14}}

\hacer{Señalar el diagrama de flujo en la diapositiva}

\textbf{Los 6 pasos del flujo:}

\begin{enumerate}
    \item \textbf{Obtención de datos:} APIs, web scraping, archivos
    \item \textbf{Limpieza:} 80\% del tiempo se va aquí
    \item \textbf{Análisis espacial:} La parte divertida
    \item \textbf{Visualización:} Un mapa vale más que mil tablas
    \item \textbf{Modelado:} Predicción y optimización
    \item \textbf{Compartir:} GitHub, web apps, reportes
\end{enumerate}

\subsection{Caso COVID-19 en Santiago \tiempo{1:14-1:19}}

\decir{Marzo 2020. Primer caso COVID en Chile. Necesitamos entender la propagación. Este es un caso real donde la geocomputación salvó vidas.}

\textbf{Pasos del análisis:}

\begin{enumerate}
    \item Join espacial de datos MINSAL con comunas
    \item Normalización: casos por 100,000 habitantes
    \item Autocorrelación espacial: Moran's I = 0.7
    \item Identificación de clusters: Las Condes, Vitacura, Providencia
    \item Predicción con modelo SIR espacial
\end{enumerate}

\textbf{Resultado:} 8 de 10 comunas predichas correctamente para siguientes brotes.

\subsection{Configuración del Ambiente \tiempo{1:19-1:22}}

\hacer{Demostración en vivo de configuración}

\begin{lstlisting}[language=bash]
# Verificar Python
python --version  # Debe ser 3.8+

# Crear ambiente virtual
python -m venv geo_env
source geo_env/bin/activate  # Linux/Mac
# o
geo_env\Scripts\activate  # Windows

# Instalar bibliotecas
pip install geopandas folium matplotlib jupyter
\end{lstlisting}

\alerta{Windows puede dar problemas con Fiona. Solución: usar Anaconda}

\subsection{Primer Script Completo \tiempo{1:22-1:26}}

\decir{Este script es su plantilla base para todo el semestre. Guárdenlo.}

\begin{lstlisting}[language=Python]
import geopandas as gpd
import matplotlib.pyplot as plt

# Cargar datos del mundo
world = gpd.read_file(
    gpd.datasets.get_path('naturalearth_lowres'))

# Filtrar Sudamérica
sudamerica = world[world['continent'] == 'South America']

# Crear visualización
fig, ax = plt.subplots(figsize=(10, 8))
sudamerica.plot(column='pop_est', 
                ax=ax,
                legend=True,
                cmap='YlOrRd')
ax.set_title('Población de Sudamérica')
plt.show()
\end{lstlisting}

\newpage

\section{Cierre y Preparación para Laboratorio \tiempo{10 minutos}}

\subsection{Objetivos del Laboratorio \tiempo{1:26-1:29}}

\decir{En 10 minutos comenzamos el laboratorio. Será hands-on, práctico, posiblemente frustrante. Es normal.}

\textbf{Qué haremos:}
\begin{enumerate}
    \item Instalación completa de software
    \item Verificación con script de prueba
    \item Generar primer mapa de Chile
    \item Configurar Git para entregas
\end{enumerate}

\textbf{Problemas anticipados:}
\begin{itemize}
    \item Windows: Permisos de administrador
    \item Mac: Xcode Command Line Tools (2GB)
    \item Linux: Generalmente sin problemas
\end{itemize}

\subsection{Recursos para Profundizar \tiempo{1:29-1:32}}

\textbf{Libros esenciales (gratuitos):}
\begin{itemize}
    \item Geocomputation with Python - py.geocompx.org
    \item Geocomputation with R - r.geocompx.org
    \item Geographic Data Science - geographicdata.science/book
\end{itemize}

\textbf{Comunidades:}
\begin{itemize}
    \item Stack Overflow tag 'gis'
    \item r/gis en Reddit
    \item GitHub: awesome-gis
\end{itemize}

\subsection{Ideas para Proyecto \tiempo{1:32-1:34}}

\hacer{Actividad rápida: 30 segundos cada estudiante con su compañero para compartir una idea de proyecto}

\textbf{Ideas comerciales:}
\begin{itemize}
    \item Optimización de rutas delivery
    \item Análisis inmobiliario
    \item Segmentación de clientes
\end{itemize}

\textbf{Ideas científicas:}
\begin{itemize}
    \item Contaminación del aire
    \item Segregación urbana
    \item Riesgo de incendios
\end{itemize}

\subsection{Tareas y Cierre \tiempo{1:34-1:40}}

\textbf{Tareas (no negociables):}
\begin{enumerate}
    \item URGENTE: Instalar software ANTES del lab
    \item Leer Capítulo 1 de Geocomputation with Python (30 min)
    \item Pensar en un problema espacial de interés
\end{enumerate}

\decir{¿Preguntas finales?}

\hacer{Responder 2-3 preguntas máximo}

\decir{En 4 meses estarán haciendo análisis que hoy parecen magia. Confíen en el proceso. 10 minutos de break. Nos vemos aquí para el laboratorio.}

\newpage

\section{Anexos}

\subsection{Preguntas Frecuentes}

\begin{enumerate}
\item \textbf{¿Python o R primero?}\\
R: Si vienen de programación, Python. Si no programan, pueden empezar con cualquiera.

\item \textbf{¿Necesito GPU?}\\
R: No para el curso. GPU ayuda solo para deep learning avanzado.

\item \textbf{¿Por qué no ArcGIS?}\\
R: Costo, filosofía de enseñanza, y empleabilidad. El que programa GIS aprende ArcGIS rápido.

\item \textbf{¿Datos de Chile?}\\
R: IDE Chile tiene mucho. INE tiene shapefiles buenos. Les compartiré repositorio curado.

\item \textbf{¿Proyecto sobre mi comuna?}\\
R: ¡Absolutamente! Incentivo proyectos locales con contexto conocido.

\item \textbf{¿ChatGPT para código?}\\
R: Sí, pero para aprender, no copiar. Deben entender cada línea.

\item \textbf{¿Empleabilidad?}\\
R: Alta demanda. Junior: 1.2-1.5M CLP. Con experiencia: 2-3M CLP.

\item \textbf{¿PostGIS es necesario?}\\
R: Opcional. Muy poderoso pero requiere SQL. Puedo dar taller extra si hay interés.

\item \textbf{¿Imágenes satelitales?}\\
R: Sí, unidad 2. Sentinel-2 (10m, gratis) y Landsat (30m, histórico).

\item \textbf{¿Cuánto programación previa?}\\
R: Básico: variables, loops, funciones. Si pueden hacer FizzBuzz, están listos.
\end{enumerate}

\subsection{Actividades Interactivas}

\begin{tcolorbox}[colback=blue!5,colframe=blue!50!black,title=Actividad 1: Encuentra el Patrón]
\textbf{Tiempo:} 2 minutos\\
\textbf{Cuándo:} Al mostrar mapa de John Snow\\
\textbf{Qué hacer:} Mostrar puntos, preguntar por patrón\\
\textbf{Resultado esperado:} Identifican clustering
\end{tcolorbox}

\begin{tcolorbox}[colback=green!5,colframe=green!50!black,title=Actividad 2: QGIS vs ArcGIS]
\textbf{Tiempo:} 3 minutos\\
\textbf{Cuándo:} Sección de software\\
\textbf{Qué hacer:} Parejas debaten pros/contras\\
\textbf{Resultado esperado:} Comprenden trade-offs
\end{tcolorbox}

\begin{tcolorbox}[colback=yellow!5,colframe=orange,title=Actividad 3: Idea de Proyecto]
\textbf{Tiempo:} 1 minuto\\
\textbf{Cuándo:} Antes del cierre\\
\textbf{Qué hacer:} Compartir idea con compañero\\
\textbf{Resultado esperado:} Ideas fluyendo
\end{tcolorbox}

\subsection{Troubleshooting Común}

\textbf{Problema:} ImportError en Python\\
\textbf{Solución:} 
\begin{lstlisting}[language=bash]
# Windows con Anaconda
conda install -c conda-forge geopandas

# Mac/Linux
pip install --upgrade pip
pip install geopandas
\end{lstlisting}

\textbf{Problema:} Internet caído\\
\textbf{Solución:} Tener demos pre-grabados, screenshots, código en USB

\textbf{Problema:} Proyector falla\\
\textbf{Solución:} Compartir pantalla por Teams/Zoom, usar pizarra más

\subsection{Checklist Post-Clase}

\begin{itemize}[label=$\square$]
\item Subir slides al repositorio
\item Compartir links de recursos
\item Preparar datos para laboratorio
\item Responder emails (24h máximo)
\item Anotar qué mejorar
\item Backup de demos utilizados
\end{itemize}

\subsection{Métricas de Éxito}

\begin{center}
\begin{tabular}{|l|c|}
\hline
\textbf{Indicador} & \textbf{Meta} \\
\hline
Software instalado al final del lab & >90\% \\
Preguntas durante clase & >5 \\
Participación en actividades & >80\% \\
Satisfacción (si hay encuesta) & >4/5 \\
\hline
\end{tabular}
\end{center}

\vspace{2cm}

\begin{center}
\Large{\textbf{¡Éxito en tu clase!}}

\normalsize
Para consultas: francisco.parra.o@usach.cl
\end{center}

\end{document}