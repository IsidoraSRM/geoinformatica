\documentclass[12pt,a4paper]{article}
\usepackage[utf8]{inputenc}
\usepackage[spanish,es-tabla]{babel}
\usepackage{geometry}
\geometry{margin=2.5cm}
\usepackage{graphicx}
\usepackage{float}
\usepackage{amsmath,amssymb,amsthm}
\usepackage{enumerate}
\usepackage{listings}
\usepackage{xcolor}
\usepackage{hyperref}
\usepackage{tcolorbox}
\usepackage{booktabs}
\usepackage{longtable}
\usepackage{array}
\usepackage{multirow}
\usepackage{fancyhdr}
\usepackage{tikz}
\usetikzlibrary{shapes,arrows,positioning,calc}
\usepackage{dirtytalk}

% Configuración de código
\lstset{
    language=Python,
    basicstyle=\small\ttfamily,
    keywordstyle=\color{blue!80!black},
    commentstyle=\color{green!60!black},
    stringstyle=\color{red!80!black},
    numbers=left,
    numberstyle=\tiny\color{gray},
    breaklines=true,
    frame=single,
    backgroundcolor=\color{gray!10},
    captionpos=b
}

% Configuración de encabezados
\pagestyle{fancy}
\fancyhf{}
\lhead{Manual del Profesor - Clase 07}
\rhead{Machine Learning Geoespacial}
\cfoot{\thepage}

% Comandos personalizados
\newtcolorbox{conceptbox}[2][]{
    colback=blue!5!white,
    colframe=blue!75!black,
    title=#2,
    fonttitle=\bfseries,
    #1
}

\newtcolorbox{alertbox}[1][]{
    colback=red!5!white,
    colframe=red!75!black,
    fonttitle=\bfseries,
    #1
}

\newtcolorbox{ejemplo}[2][]{
    colback=green!5!white,
    colframe=green!75!black,
    title=#2,
    fonttitle=\bfseries,
    #1
}

\newtcolorbox{reflexion}[1][]{
    colback=yellow!5!white,
    colframe=yellow!75!black,
    title=Para reflexionar,
    fonttitle=\bfseries,
    #1
}

\newtcolorbox{analogia}[2][]{
    colback=purple!5!white,
    colframe=purple!75!black,
    title=#2,
    fonttitle=\bfseries,
    #1
}

\title{
    \vspace{-2cm}
    \Large{UNIVERSIDAD DE SANTIAGO DE CHILE} \\
    \large{Facultad de Ingeniería} \\
    \large{Departamento de Ingeniería Informática} \\
    \vspace{1cm}
    \LARGE{\textbf{Manual del Profesor}} \\
    \Large{\textbf{Clase 07: Machine Learning Geoespacial}} \\
    \vspace{0.5cm}
    \large{Curso: Desarrollo de Aplicaciones Geoinformáticas}
}

\author{
    Prof. Francisco Parra O. \\
    \texttt{francisco.parra.o@usach.cl}
}

\date{Versión: \today}

\begin{document}

\maketitle
\thispagestyle{empty}
\newpage

\tableofcontents
\newpage

%==============================================================================
\section{Introducción: La Convergencia de IA y Geografía}
%==============================================================================

\subsection{¿Por qué Machine Learning en el Análisis Geoespacial?}

El análisis geoespacial tradicional ha servido bien a la geografía durante décadas, pero enfrenta limitaciones ante la explosión de datos espaciales modernos. Satélites que generan terabytes diarios, sensores IoT urbanos, y rastros digitales de millones de usuarios crean un volumen de información que supera la capacidad humana de análisis manual.

\begin{conceptbox}{La Revolución del ML Geoespacial}
Machine Learning no reemplaza el análisis espacial tradicional; lo potencia. Permite descubrir patrones en datos masivos, modelar relaciones no lineales complejas, y automatizar tareas que tomarían años de trabajo manual. Es la evolución natural del análisis espacial en la era del big data.
\end{conceptbox}

\begin{analogia}{El Microscopio Digital}
Si los SIG tradicionales son como mapas y brújulas para navegar el territorio, el Machine Learning Geoespacial es como un microscopio digital que revela patrones invisibles al ojo humano. Puede ver a través de múltiples capas de complejidad, identificar relaciones sutiles, y predecir cambios antes de que ocurran.
\end{analogia}

\subsection{Los Desafíos Únicos del ML Geoespacial}

El Machine Learning aplicado a datos geoespaciales no es simplemente ML tradicional con coordenadas añadidas. Los datos espaciales tienen propiedades únicas que requieren consideración especial:

\subsubsection{1. Autocorrelación Espacial: La Bendición y la Maldición}

\begin{reflexion}
Tobler nos enseñó que \say{todo está relacionado con todo lo demás, pero las cosas cercanas están más relacionadas}. En ML, esto significa que nuestros datos nunca son verdaderamente independientes. Un modelo que ignora esta realidad está condenado a sobreestimar su propia precisión.
\end{reflexion}

La autocorrelación espacial viola el supuesto fundamental de independencia de observaciones en ML tradicional. Un modelo entrenado con puntos de Santiago puede tener 95\% de precisión en Santiago, pero fallar completamente en Valparaíso. Esto no es un bug; es la naturaleza de los datos espaciales.

\subsubsection{2. El Problema de Escala}

Los fenómenos espaciales operan a múltiples escalas simultáneamente. Un bosque es:
\begin{itemize}
    \item A escala microscópica: interacciones entre hojas y luz
    \item A escala local: competencia entre árboles individuales
    \item A escala paisaje: patrones de distribución de especies
    \item A escala regional: respuesta a gradientes climáticos
    \item A escala global: parte del sistema climático planetario
\end{itemize}

Un modelo de ML debe capturar estas múltiples escalas o arriesgarse a perder información crítica.

\subsubsection{3. Heterogeneidad Espacial}

\begin{ejemplo}{El Dilema del Modelo Global}
Imagine entrenar un modelo para predecir rendimiento agrícola. Las variables importantes en el Valle Central de California (irrigación, temperatura) pueden ser irrelevantes en Iowa (precipitación, tipo de suelo). Un modelo global promedia estas diferencias, perdiendo precisión local. Modelos locales capturan especificidad pero pierden poder estadístico. Este es el dilema central del ML geoespacial.
\end{ejemplo}

%==============================================================================
\section{Fundamentos de Machine Learning para Geografía}
%==============================================================================

\subsection{Tipos de Aprendizaje en Contexto Espacial}

\subsubsection{Aprendizaje Supervisado Espacial}

El aprendizaje supervisado en geografía tiene aplicaciones naturales:

\textbf{Clasificación:}
\begin{itemize}
    \item Clasificación de cobertura terrestre desde imágenes satelitales
    \item Identificación de tipos de edificios en imágenes aéreas
    \item Detección de enfermedades en cultivos
    \item Clasificación de zonas de riesgo
\end{itemize}

\textbf{Regresión:}
\begin{itemize}
    \item Predicción de valores inmobiliarios
    \item Estimación de rendimientos agrícolas
    \item Interpolación de variables ambientales
    \item Predicción de demanda de transporte
\end{itemize}

\subsubsection{Aprendizaje No Supervisado Espacial}

\begin{conceptbox}{Descubriendo Estructura Espacial}
El aprendizaje no supervisado revela patrones ocultos en datos espaciales sin necesidad de etiquetas. Es especialmente valioso cuando no sabemos qué buscar o cuando las categorías tradicionales no capturan la complejidad real.
\end{conceptbox}

\textbf{Clustering espacial:} Agrupa observaciones similares considerando proximidad:
\begin{itemize}
    \item Identificación de barrios socioeconómicos
    \item Detección de hot spots criminales
    \item Zonificación automática urbana
    \item Segmentación de mercados geográficos
\end{itemize}

\subsubsection{Aprendizaje por Refuerzo Espacial}

Aunque menos común, el RL tiene aplicaciones fascinantes:
\begin{itemize}
    \item Optimización de rutas de delivery
    \item Gestión adaptativa de tráfico
    \item Planificación urbana simulada
    \item Estrategias de conservación
\end{itemize}

\subsection{Feature Engineering: El Arte de Representar el Espacio}

\begin{alertbox}
En ML geoespacial, el 80\% del éxito viene del feature engineering. Los algoritmos son importantes, pero la forma en que representamos el espacio determina qué pueden aprender.
\end{alertbox}

\subsubsection{Features Espaciales Fundamentales}

\textbf{1. Coordenadas y Transformaciones:}

Las coordenadas brutas (lat/lon) rara vez son óptimas. Considere transformaciones:

\begin{itemize}
    \item \textbf{Coordenadas cartesianas:} Para distancias euclidianas locales
    \item \textbf{Coordenadas polares:} Para patrones radiales desde un centro
    \item \textbf{Splines espaciales:} Para capturar tendencias suaves
    \item \textbf{Embeddings espaciales:} Representaciones aprendidas del espacio
\end{itemize}

\textbf{2. Distancias y Proximidad:}

\begin{analogia}{El Vecindario Digital}
En el mundo físico, definimos vecindario por proximidad. En ML geoespacial, expandimos esta noción: vecindario puede ser similitud espectral, conectividad de red, o accesibilidad temporal. Cada definición captura diferentes aspectos de la relación espacial.
\end{analogia}

Tipos de distancia útiles:
\begin{itemize}
    \item \textbf{Euclidiana:} Línea recta, útil para espacios abiertos
    \item \textbf{Manhattan:} Distancia en grilla urbana
    \item \textbf{Haversine:} Distancia en esfera terrestre
    \item \textbf{Network distance:} Distancia real por red vial
    \item \textbf{Cost distance:} Incorpora fricción del terreno
\end{itemize}

\textbf{3. Contexto Espacial:}

El contexto enriquece cada observación con información de su entorno:

\begin{itemize}
    \item \textbf{Estadísticas focales:} Media, varianza en ventana
    \item \textbf{Densidad kernel:} Intensidad suavizada de eventos
    \item \textbf{Spatial lag:} Valor promedio de vecinos
    \item \textbf{Local indicators:} LISA, Getis-Ord local
\end{itemize}

%==============================================================================
\section{Algoritmos de ML Adaptados al Espacio}
%==============================================================================

\subsection{Random Forest Espacial: El Caballo de Batalla}

Random Forest es excepcionalmente popular en aplicaciones geoespaciales por buenas razones:

\begin{conceptbox}{¿Por qué RF domina el ML geoespacial?}
\begin{enumerate}
    \item Maneja naturalmente relaciones no lineales
    \item Robusto a outliers espaciales
    \item No requiere normalización de features
    \item Feature importance interpretable
    \item Maneja bien alta dimensionalidad
    \item Paralelizable para grandes datasets
\end{enumerate}
\end{conceptbox}

\subsubsection{Adaptaciones Espaciales de RF}

\textbf{1. Geographical Random Forest (GRF):}

GRF adapta RF para heterogeneidad espacial local:
\begin{itemize}
    \item Entrena un RF local para cada ubicación
    \item Pondera observaciones por distancia
    \item Captura variación espacial en relaciones
\end{itemize}

\textbf{2. Spatial Random Forest:}

Incorpora autocorrelación directamente:
\begin{itemize}
    \item Añade coordenadas como features
    \item Incluye spatial lag variables
    \item Usa spatial cross-validation
\end{itemize}

\begin{ejemplo}{RF para Predicción de Incendios Forestales}
Un modelo RF para predecir riesgo de incendios considera:
\begin{itemize}
    \item \textbf{Features estáticas:} Elevación, pendiente, aspecto
    \item \textbf{Features dinámicas:} Temperatura, humedad, viento
    \item \textbf{Features espaciales:} Distancia a caminos, densidad de vegetación
    \item \textbf{Features temporales:} Días desde última lluvia
\end{itemize}

El modelo aprende interacciones complejas: alta temperatura + baja humedad + vegetación densa + proximidad a caminos = alto riesgo.
\end{ejemplo}

\subsection{Support Vector Machines en el Espacio}

SVMs son poderosas para clasificación de imágenes satelitales:

\textbf{Ventajas:}
\begin{itemize}
    \item Efectivas en espacios de alta dimensión (bandas espectrales)
    \item Kernel trick captura relaciones no lineales
    \item Buena generalización con datos limitados
\end{itemize}

\textbf{Kernels espaciales especializados:}
\begin{itemize}
    \item \textbf{RBF espacial:} Incorpora distancia en kernel
    \item \textbf{Spectral angle mapper:} Para datos hiperespectrales
    \item \textbf{Graph kernels:} Para datos en redes
\end{itemize}

\subsection{Gradient Boosting Espacial}

XGBoost/LightGBM dominan competencias de ML por su precisión:

\begin{reflexion}
Gradient boosting es como un equipo de expertos especializados. Cada árbol aprende de los errores del anterior, enfocándose en las áreas problemáticas. En contexto espacial, esto significa que automáticamente presta más atención a regiones difíciles de predecir.
\end{reflexion}

%==============================================================================
\section{Deep Learning: La Nueva Frontera Geoespacial}
%==============================================================================

\subsection{CNNs: Los Ojos de la IA en el Territorio}

Las Redes Neuronales Convolucionales han revolucionado el análisis de imágenes satelitales y aéreas.

\begin{conceptbox}{La Magia de las Convoluciones}
Las CNNs imitan el sistema visual humano. Las primeras capas detectan bordes y texturas, las intermedias reconocen formas y patrones, las profundas entienden objetos y contextos. Es como enseñar a la computadora a \say{ver} el territorio como lo haría un fotointérprete experto, pero a velocidad y escala sobrehumanas.
\end{conceptbox}

\subsubsection{Arquitecturas Clave para Geoespacial}

\textbf{1. U-Net: La Reina de la Segmentación}

U-Net es ubicua en segmentación de imágenes satelitales:

\begin{analogia}{El Reloj de Arena}
U-Net tiene forma de U (o reloj de arena). La rama descendente comprime la imagen a su esencia, capturando el \say{qué}. La rama ascendente reconstruye detalles, recuperando el \say{dónde}. Las conexiones skip preservan información espacial fina. Es como destilar la esencia de una imagen y luego reconstruirla con comprensión profunda.
\end{analogia}

Aplicaciones exitosas:
\begin{itemize}
    \item Delineación de edificios
    \item Mapeo de carreteras
    \item Segmentación de parcelas agrícolas
    \item Detección de cuerpos de agua
    \item Mapeo de daños post-desastre
\end{itemize}

\textbf{2. ResNet: Profundidad con Estabilidad}

ResNets permiten redes muy profundas mediante conexiones residuales:

\begin{itemize}
    \item ResNet-50/101 para clasificación de escenas
    \item Transfer learning desde ImageNet
    \item Fine-tuning con datos satelitales
\end{itemize}

\textbf{3. YOLO/Faster R-CNN: Detección de Objetos}

Para identificar y localizar objetos específicos:
\begin{itemize}
    \item Conteo de vehículos
    \item Detección de barcos
    \item Identificación de infraestructura
    \item Monitoreo de construcción
\end{itemize}

\subsubsection{Desafíos Únicos de CNNs en Imágenes Satelitales}

\textbf{1. Multiescala y Multiresolución:}

Objetos de interés varían desde metros (autos) hasta kilómetros (ciudades). Soluciones:
\begin{itemize}
    \item Feature Pyramid Networks (FPN)
    \item Entrenamiento multiescala
    \item Arquitecturas con múltiples resoluciones
\end{itemize}

\textbf{2. Escasez de Datos Etiquetados:}

Etiquetar imágenes satelitales es costoso. Estrategias:
\begin{itemize}
    \item Transfer learning agresivo
    \item Semi-supervised learning
    \item Active learning para etiquetado eficiente
    \item Weak supervision con etiquetas ruidosas
\end{itemize}

\textbf{3. Dominios Espectrales Diferentes:}

Imágenes satelitales tienen más bandas que RGB:
\begin{itemize}
    \item Adaptación de arquitecturas para N bandas
    \item Fusión de información multiespectral
    \item Índices espectrales como features adicionales
\end{itemize}

\subsection{Graph Neural Networks: Modelando Relaciones Espaciales}

\begin{conceptbox}{El Espacio como Grafo}
No todo dato espacial es una imagen. Ciudades son redes de calles, ecosistemas son redes tróficas, economías son redes comerciales. Graph Neural Networks modelan estas relaciones complejas directamente.
\end{conceptbox}

\subsubsection{GNNs en Aplicaciones Geoespaciales}

\textbf{1. Predicción de Tráfico:}

El tráfico es inherentemente un problema de grafos:
\begin{itemize}
    \item Nodos: Intersecciones o segmentos viales
    \item Edges: Conexiones viales
    \item Features: Flujo histórico, eventos, clima
    \item Target: Flujo futuro o tiempo de viaje
\end{itemize}

GNNs capturan cómo la congestión se propaga por la red, superando modelos tradicionales.

\textbf{2. Análisis de Redes Urbanas:}

\begin{ejemplo}{Propagación de Innovación Urbana}
Un GNN puede modelar cómo las innovaciones (bike sharing, food trucks) se difunden por una ciudad:
\begin{itemize}
    \item Nodos: Barrios
    \item Edges: Proximidad o similitud socioeconómica
    \item Proceso: Message passing propaga información
    \item Predicción: Qué barrios adoptarán siguiente
\end{itemize}
\end{ejemplo}

\subsubsection{Ventajas de GNNs sobre CNNs}

\begin{itemize}
    \item Manejan datos irregulares (no grillas)
    \item Incorporan relaciones explícitas
    \item Invariantes a permutaciones
    \item Eficientes para datos sparse
\end{itemize}

%==============================================================================
\section{Validación Espacial: El Talón de Aquiles}
%==============================================================================

\subsection{Por Qué la Validación Tradicional Falla}

\begin{alertbox}
\textbf{El Error Más Común en ML Geoespacial:}
Usar random train-test split en datos espaciales. Esto crea data leakage espacial: el modelo \say{memoriza} el paisaje local y sobreestima dramáticamente su capacidad de generalización.
\end{alertbox}

\begin{analogia}{El Examen con las Respuestas}
Random split en datos espaciales es como hacer un examen donde las preguntas pares tienen las respuestas de las impares al lado. El modelo no aprende principios generales; aprende a mirar al costado.
\end{analogia}

\subsection{Estrategias de Validación Espacial}

\subsubsection{1. Spatial Cross-Validation}

Divide el espacio, no los datos:

\textbf{Block CV:}
\begin{itemize}
    \item Divide área en bloques rectangulares
    \item Entrena en algunos, valida en otros
    \item Previene contaminación espacial
\end{itemize}

\textbf{Spatial K-Fold:}
\begin{itemize}
    \item Clustering espacial para crear folds
    \item Cada fold es espacialmente compacto
    \item Simula predicción en nuevas áreas
\end{itemize}

\subsubsection{2. Buffered Leave-One-Out}

Para cada punto test:
\begin{enumerate}
    \item Crear buffer alrededor del punto
    \item Excluir todos los puntos dentro del buffer del training
    \item Entrenar modelo y predecir punto test
    \item Repetir para todos los puntos
\end{enumerate}

El buffer elimina autocorrelación entre train y test.

\subsubsection{3. Environmental Blocking}

Divide datos por espacio ambiental, no geográfico:
\begin{itemize}
    \item Útil para modelos de distribución de especies
    \item Test en condiciones ambientales no vistas
    \item Evalúa verdadera capacidad de extrapolación
\end{itemize}

\begin{reflexion}
La validación espacial siempre dará métricas \say{peores} que random split. Esto no es un problema del método; es una evaluación honesta de la capacidad real del modelo. Es mejor saber la verdad durante desarrollo que descubrirla en producción.
\end{reflexion}

%==============================================================================
\section{Interpretabilidad en ML Geoespacial}
%==============================================================================

\subsection{¿Por Qué la Interpretabilidad Importa Más en Geografía?}

Los modelos geoespaciales informan decisiones críticas:
\begin{itemize}
    \item Planificación urbana
    \item Gestión de desastres
    \item Política ambiental
    \item Inversión en infraestructura
\end{itemize}

Un modelo \say{caja negra} que no puede explicar sus predicciones es inaceptable para stakeholders.

\subsection{Técnicas de Interpretación Espacial}

\subsubsection{1. Feature Importance Espacializada}

No solo \say{qué features importan} sino \say{dónde importan}:

\begin{ejemplo}{Importancia Variable de Features}
En predicción de precios inmobiliarios:
\begin{itemize}
    \item Centro ciudad: Proximidad a metro domina
    \item Suburbios: Tamaño del terreno importa más
    \item Costa: Vista al mar es crítica
    \item Industrial: Distancia a fábricas clave
\end{itemize}

Un mapa de feature importance muestra esta heterogeneidad espacial.
\end{ejemplo}

\subsubsection{2. SHAP Values Geográficos}

SHAP (SHapley Additive exPlanations) descompone predicciones:
\begin{itemize}
    \item Contribución de cada feature a cada predicción
    \item Mapeable para ver patrones espaciales
    \item Identifica interacciones locales
\end{itemize}

\subsubsection{3. Counterfactual Maps}

\say{¿Qué pasaría si...?} en el espacio:
\begin{itemize}
    \item Si construimos una estación de metro aquí?
    \item Si el clima cambia 2°C?
    \item Si cambiamos zonificación?
\end{itemize}

Los mapas counterfactuales visualizan escenarios alternativos.

%==============================================================================
\section{Casos de Estudio Detallados}
%==============================================================================

\subsection{Caso 1: Agricultura de Precisión con ML}

\subsubsection{Contexto y Desafío}

Una cooperativa agrícola maneja 10,000 hectáreas de cultivos diversos. Necesitan optimizar rendimientos minimizando insumos (agua, fertilizantes, pesticidas).

\subsubsection{Datos Disponibles}

\begin{itemize}
    \item Imágenes Sentinel-2 cada 5 días
    \item Datos meteorológicos históricos
    \item Mapas de suelo
    \item Registros históricos de rendimiento
    \item Datos de sensores IoT en campo
\end{itemize}

\subsubsection{Pipeline de ML Implementado}

\textbf{1. Preprocesamiento:}
\begin{itemize}
    \item Corrección atmosférica de imágenes
    \item Cálculo de índices vegetacionales (NDVI, EVI, NDWI)
    \item Interpolación de datos faltantes
    \item Armonización temporal
\end{itemize}

\textbf{2. Feature Engineering:}
\begin{itemize}
    \item Estadísticas zonales por parcela
    \item Features fenológicas (días desde siembra)
    \item Agregados temporales (tendencias NDVI)
    \item Variables meteorológicas acumuladas
\end{itemize}

\textbf{3. Modelado:}
\begin{itemize}
    \item Random Forest para predicción de rendimiento
    \item CNN para detección de enfermedades
    \item LSTM para predicción de necesidades de riego
\end{itemize}

\textbf{4. Validación:}
\begin{itemize}
    \item Leave-one-year-out (temporal)
    \item Leave-one-field-out (espacial)
    \item Validación cruzada espacio-temporal
\end{itemize}

\subsubsection{Resultados e Impacto}

\begin{itemize}
    \item 15\% aumento en rendimiento promedio
    \item 20\% reducción en uso de agua
    \item 25\% reducción en pesticidas
    \item ROI de 300\% en primer año
\end{itemize}

\subsubsection{Lecciones Aprendidas}

\begin{reflexion}
El éxito no vino del algoritmo más sofisticado, sino de:
\begin{enumerate}
    \item Entender profundamente el dominio
    \item Feature engineering cuidadoso
    \item Validación rigurosa
    \item Interpretabilidad para ganar confianza
    \item Integración con flujos de trabajo existentes
\end{enumerate}
\end{reflexion}

\subsection{Caso 2: Detección de Asentamientos Informales}

\subsubsection{Problema Social}

Ciudades en desarrollo necesitan identificar y mapear asentamientos informales para:
\begin{itemize}
    \item Provisión de servicios básicos
    \item Planificación de mejoramiento
    \item Respuesta a emergencias
    \item Monitoreo de crecimiento urbano
\end{itemize}

\subsubsection{Desafíos Técnicos}

\begin{itemize}
    \item Alta heterogeneidad de asentamientos
    \item Cambios rápidos
    \item Datos etiquetados limitados
    \item Consideraciones éticas
\end{itemize}

\subsubsection{Solución Deep Learning}

\textbf{Arquitectura:}
\begin{itemize}
    \item U-Net modificada para segmentación
    \item Transfer learning desde modelo preentrenado
    \item Data augmentation extensiva
    \item Post-procesamiento morfológico
\end{itemize}

\textbf{Innovaciones:}
\begin{itemize}
    \item Multi-task learning (segmentación + clasificación)
    \item Incorporación de datos OSM
    \item Active learning para etiquetado eficiente
    \item Uncertainty quantification
\end{itemize}

\subsubsection{Consideraciones Éticas}

\begin{alertbox}
Mapear asentamientos informales tiene implicaciones éticas profundas:
\begin{itemize}
    \item Privacidad de residentes vulnerables
    \item Riesgo de desalojos forzados
    \item Estigmatización de comunidades
    \item Uso dual de la tecnología
\end{itemize}

El desarrollo debe involucrar a las comunidades afectadas y garantizar que la tecnología se use para mejorar vidas, no para vigilancia o exclusión.
\end{alertbox}

%==============================================================================
\section{Herramientas y Recursos Prácticos}
%==============================================================================

\subsection{Stack de Software Recomendado}

\subsubsection{Nivel Básico: Python Ecosystem}

\textbf{Manipulación de Datos:}
\begin{itemize}
    \item \textbf{GeoPandas:} DataFrames espaciales
    \item \textbf{Rasterio:} Lectura/escritura raster
    \item \textbf{Shapely:} Operaciones geométricas
    \item \textbf{Fiona:} I/O de vectores
\end{itemize}

\textbf{Machine Learning:}
\begin{itemize}
    \item \textbf{Scikit-learn:} Algoritmos clásicos
    \item \textbf{XGBoost/LightGBM:} Gradient boosting
    \item \textbf{PySAL:} ML espacial especializado
\end{itemize}

\textbf{Deep Learning:}
\begin{itemize}
    \item \textbf{TensorFlow/Keras:} Framework general
    \item \textbf{PyTorch:} Flexibilidad investigación
    \item \textbf{TorchGeo:} Datasets y modelos geo
\end{itemize}

\subsubsection{Nivel Intermedio: Plataformas Cloud}

\textbf{Google Earth Engine:}
\begin{itemize}
    \item Petabytes de imágenes satelitales
    \item Procesamiento en la nube
    \item JavaScript/Python APIs
    \item Gratis para investigación
\end{itemize}

\textbf{AWS SageMaker:}
\begin{itemize}
    \item Training distribuido
    \item Deployment automático
    \item Integración con S3
    \item Auto-scaling
\end{itemize}

\subsubsection{Nivel Avanzado: Frameworks Especializados}

\textbf{Raster Vision:}
\begin{itemize}
    \item Pipeline end-to-end
    \item Manejo de imágenes grandes
    \item Evaluación estandarizada
    \item Deployment production-ready
\end{itemize}

\textbf{GRASS GIS + ML:}
\begin{itemize}
    \item Integración con R
    \item Algoritmos espaciales nativos
    \item Procesamiento paralelo
    \item Decades de desarrollo
\end{itemize}

\subsection{Datasets para Práctica}

\begin{conceptbox}{Datasets Benchmark}
\begin{itemize}
    \item \textbf{SpaceNet:} Edificios y carreteras etiquetados
    \item \textbf{EuroSAT:} Clasificación uso del suelo
    \item \textbf{BigEarthNet:} Multi-label classification
    \item \textbf{SEN12MS:} Multimodal (SAR + optical)
    \item \textbf{LandCover.ai:} Segmentación alta resolución
\end{itemize}
\end{conceptbox}

%==============================================================================
\section{Tendencias Futuras y Reflexiones}
%==============================================================================

\subsection{Foundation Models Geoespaciales}

La nueva frontera son modelos masivos preentrenados:

\begin{conceptbox}{El ChatGPT de las Imágenes Satelitales}
Imagine un modelo entrenado con todas las imágenes satelitales del mundo, capaz de entender el planeta visualmente. Podríamos hacer preguntas como \say{encuentra todos los nuevos almacenes construidos este año} o \say{identifica áreas en riesgo de inundación} sin entrenamiento específico.
\end{conceptbox}

Ejemplos emergentes:
\begin{itemize}
    \item IBM/NASA Prithvi
    \item Clay Foundation Model
    \item SatMAE
\end{itemize}

\subsection{AutoML Espacial}

Automatización del pipeline completo:
\begin{itemize}
    \item Feature engineering automático
    \item Selección de algoritmos
    \item Hyperparameter tuning
    \item Validación espacial
\end{itemize}

Esto democratizará el ML geoespacial, permitiendo a no-expertos construir modelos sofisticados.

\subsection{Física-Informada y Modelos Híbridos}

\begin{reflexion}
El futuro no es reemplazar modelos físicos con ML, sino combinarlos. Physics-Informed Neural Networks (PINNs) incorporan leyes físicas como constraints, garantizando que las predicciones respeten principios fundamentales mientras capturan complejidad empírica.
\end{reflexion}

\subsection{Consideraciones Éticas y Sociales}

El poder del ML geoespacial conlleva responsabilidades:

\textbf{Sesgo y Equidad:}
\begin{itemize}
    \item Modelos entrenados en ciudades ricas fallan en pobres
    \item Datasets históricos perpetúan desigualdades
    \item Necesidad de auditorías de equidad espacial
\end{itemize}

\textbf{Privacidad y Vigilancia:}
\begin{itemize}
    \item Resolución satelital permite identificar individuos
    \item Tracking de movimientos y comportamientos
    \item Balance entre beneficio social y privacidad
\end{itemize}

\textbf{Soberanía de Datos:}
\begin{itemize}
    \item ¿Quién controla datos satelitales?
    \item Dependencia de plataformas extranjeras
    \item Necesidad de capacidades locales
\end{itemize}

%==============================================================================
\section{Guía Pedagógica}
%==============================================================================

\subsection{Secuencia de Enseñanza Recomendada}

\subsubsection{Semana 1: Fundamentos}
\begin{enumerate}
    \item Introducción: Por qué ML en geografía
    \item Repaso ML básico con ejemplos espaciales
    \item Feature engineering espacial
    \item Primer modelo: clasificación uso del suelo
\end{enumerate}

\subsubsection{Semana 2: Algoritmos y Validación}
\begin{enumerate}
    \item Random Forest espacial detallado
    \item Validación espacial (práctica intensiva)
    \item Interpretabilidad
    \item Proyecto: predicción espacial
\end{enumerate}

\subsubsection{Semana 3: Deep Learning}
\begin{enumerate}
    \item CNNs para imágenes satelitales
    \item Transfer learning práctica
    \item Segmentación con U-Net
    \item Proyecto: detección de objetos
\end{enumerate}

\subsubsection{Semana 4: Tópicos Avanzados}
\begin{enumerate}
    \item Graph Neural Networks
    \item Temporal + espacial
    \item AutoML y cloud platforms
    \item Presentación proyectos finales
\end{enumerate}

\subsection{Ejercicios Prácticos Clave}

\begin{ejemplo}{Ejercicio 1: El Dilema del Overfitting Espacial}
Dar a estudiantes un dataset de puntos con fuerte autocorrelación. Pedir que:
\begin{enumerate}
    \item Entrenen modelo con random split (alta precisión)
    \item Entrenen con spatial CV (baja precisión)
    \item Expliquen la diferencia
    \item Propongan mejoras
\end{enumerate}

Este ejercicio es revelador y memorable.
\end{ejemplo}

\begin{ejemplo}{Ejercicio 2: Feature Engineering Creativo}
Dar imagen satelital y pedir crear 20 features diferentes para predecir ingreso promedio por zona. Estudiantes descubren que features como \say{regularidad de calles}, \say{tamaño de techos}, \say{verdor} son predictivos. Desarrolla intuición sobre qué es visible desde el espacio.
\end{ejemplo}

\subsection{Errores Comunes a Anticipar}

\begin{alertbox}
\textbf{Los 5 Errores Mortales del ML Geoespacial Estudiantil:}
\begin{enumerate}
    \item Usar lat/lon directamente como features
    \item Random train-test split
    \item Ignorar proyecciones y CRS
    \item No normalizar distancias
    \item Interpretar correlation como causación espacial
\end{enumerate}

Dedique tiempo a cada uno con ejemplos de consecuencias reales.
\end{alertbox}

\subsection{Recursos de Evaluación}

\subsubsection{Proyecto Final Integrador}

Estudiantes eligen problema real local:
\begin{itemize}
    \item Identificar problema con componente espacial
    \item Recolectar/preparar datos
    \item Implementar pipeline completo
    \item Validar rigurosamente
    \item Interpretar resultados
    \item Presentar a stakeholder simulado
\end{itemize}

Evaluar:
\begin{itemize}
    \item Correctitud técnica (40\%)
    \item Validación apropiada (20\%)
    \item Interpretabilidad (20\%)
    \item Impacto potencial (20\%)
\end{itemize}

%==============================================================================
\section{Conclusiones y Reflexiones Finales}
%==============================================================================

\begin{conceptbox}{El Futuro es Espacialmente Inteligente}
Machine Learning Geoespacial no es una moda; es una evolución inevitable. Así como los SIG transformaron la geografía hace 30 años, el ML la está transformando hoy. La diferencia es la velocidad: lo que tomó décadas ahora toma años.
\end{conceptbox}

\begin{reflexion}
Como educadores, nuestra responsabilidad no es solo enseñar técnicas, sino formar profesionales que entiendan:
\begin{itemize}
    \item El poder y las limitaciones del ML espacial
    \item La importancia de la validación rigurosa
    \item Las implicaciones éticas de sus modelos
    \item Que el ML es herramienta, no solución mágica
    \item Que el conocimiento del dominio sigue siendo crucial
\end{itemize}

El mejor geógrafo del futuro no será quien domine más algoritmos, sino quien sepa cuándo y cómo aplicarlos para resolver problemas reales del mundo.
\end{reflexion}

\subsection{Mensaje Final para Estudiantes}

El Machine Learning Geoespacial está en su adolescencia: suficientemente maduro para aplicaciones reales, suficientemente joven para innovación radical. Ustedes tienen la oportunidad única de formar este campo.

No se limiten a aplicar recetas. Cuestionen, experimenten, fallen, aprendan. El territorio es complejo, dinámico, hermosamente caótico. Nuestros modelos deben respetarlo, no simplificarlo excesivamente.

Recuerden siempre: detrás de cada píxel, cada punto, cada predicción, hay personas, ecosistemas, historias. Usen el poder del ML geoespacial con sabiduría y compasión.

\end{document}