\documentclass[11pt,a4paper]{article}
\usepackage[utf8]{inputenc}
\usepackage[spanish]{babel}
\usepackage{geometry}
\geometry{margin=2.5cm}
\usepackage{graphicx}
\usepackage{xcolor}
\usepackage{tcolorbox}
\usepackage{listings}
\usepackage{hyperref}
\usepackage{fontawesome5}
\usepackage{enumitem}
\usepackage{fancyhdr}
\usepackage{titlesec}
\usepackage{longtable}
\usepackage{array}
\usepackage{multirow}
\usepackage{tikz}
\usepackage{mdframed}
\usepackage{float}

% Colores personalizados
\definecolor{usachblue}{RGB}{0,121,192}
\definecolor{usachred}{RGB}{239,51,64}
\definecolor{codegreen}{rgb}{0,0.6,0}
\definecolor{codegray}{rgb}{0.5,0.5,0.5}
\definecolor{codepurple}{rgb}{0.58,0,0.82}
\definecolor{backcolour}{rgb}{0.95,0.95,0.92}
\definecolor{timecolor}{RGB}{150,50,50}

% Configuración de listings
\lstdefinestyle{mystyle}{
    backgroundcolor=\color{backcolour},   
    commentstyle=\color{codegreen},
    keywordstyle=\color{magenta},
    numberstyle=\tiny\color{codegray},
    stringstyle=\color{codepurple},
    basicstyle=\ttfamily\footnotesize,
    breakatwhitespace=false,         
    breaklines=true,                 
    captionpos=b,                    
    keepspaces=true,                 
    numbers=left,                    
    numbersep=5pt,                  
    showspaces=false,                
    showstringspaces=false,
    showtabs=false,                  
    tabsize=2,
    frame=single
}
\lstset{style=mystyle}

% Configuración de secciones
\titleformat{\section}[block]{\normalfont\Large\bfseries\color{usachblue}}{\thesection}{1em}{}
\titleformat{\subsection}[block]{\normalfont\large\bfseries\color{usachred}}{\thesubsection}{1em}{}

% Encabezado y pie
\pagestyle{fancy}
\fancyhf{}
\fancyhead[L]{\small Manual del Profesor - Clase 02}
\fancyhead[R]{\small Geoinformática 2025}
\fancyfoot[C]{\thepage}

% Comandos personalizados
\newcommand{\tiempo}[1]{\textcolor{timecolor}{\faIcon{clock} \textbf{[#1]}}}
\newcommand{\decir}[1]{\begin{tcolorbox}[colback=blue!5,colframe=usachblue,title={\faIcon{microphone} DECIR}]#1\end{tcolorbox}}
\newcommand{\hacer}[1]{\begin{tcolorbox}[colback=green!5,colframe=green!50!black,title={\faIcon{hand-point-right} HACER}]#1\end{tcolorbox}}
\newcommand{\nota}[1]{\begin{tcolorbox}[colback=yellow!10,colframe=orange,title={\faIcon{sticky-note} NOTA}]#1\end{tcolorbox}}
\newcommand{\alerta}[1]{\begin{tcolorbox}[colback=red!5,colframe=red,title={\faIcon{exclamation-triangle} ALERTA}]#1\end{tcolorbox}}

\title{{\Huge \textbf{Manual del Profesor}}\\[0.5cm]
{\Large Clase 02: Fundamentos de Geocomputación}\\[0.3cm]
{\large Historia, herramientas y ecosistema de desarrollo}}
\author{Prof. Francisco Parra O.\\
Geólogo, PhD en Informática\\
\texttt{francisco.parra.o@usach.cl}}
\date{Semestre 2, 2025\\
Duración: 80 minutos}

\begin{document}

\maketitle
\thispagestyle{empty}
\newpage

\tableofcontents
\newpage

\section{Preparación Previa (15 minutos antes)}

\subsection{Checklist de Materiales}

\begin{mdframed}[backgroundcolor=gray!10]
\begin{itemize}[label=$\square$]
    \item Presentación PDF cargada y probada en modo presentación
    \item Computador con Python (Anaconda) y R (RStudio) instalados
    \item Conexión a internet estable (tener hotspot móvil como backup)
    \item Pizarra y plumones (3 colores mínimo)
    \item Adaptadores de video necesarios
    \item Mouse presentador/puntero láser
    \item Botella de agua
    \item Links de recursos en un documento para compartir rápido
\end{itemize}
\end{mdframed}

\subsection{Configuración Técnica}

\begin{lstlisting}[language=bash, title=Terminal 1: Ambiente Python]
# Preparar ambiente Python
cd ~/demos_clase02
source geo_env/bin/activate  # o conda activate geo

# Verificar instalaciones
python --version  # Debe ser 3.8+
python -c "import geopandas; print('GeoPandas OK')"

# Abrir Jupyter para demos
jupyter notebook
\end{lstlisting}

\begin{lstlisting}[language=R, title=Terminal 2: Ambiente R]
# En RStudio o terminal R
# Pre-cargar bibliotecas para ahorrar tiempo en demos
library(sf)
library(tmap)
library(dplyr)

# Verificar datos de ejemplo
file.exists("santiago_comunas.shp")
\end{lstlisting}

\subsection{Archivos de Ejemplo}

\nota{Tener estos archivos en la carpeta \texttt{demos\_clase02}:
\begin{itemize}
    \item \texttt{santiago\_comunas.shp} - Shapefile de comunas de Santiago
    \item \texttt{hospitales.csv} - Ubicaciones de hospitales
    \item \texttt{demo\_completo.py} - Script Python funcionando
    \item \texttt{demo\_completo.R} - Script R funcionando
\end{itemize}}

\newpage

\section{Apertura de la Clase}

\subsection{Inicio y Bienvenida \tiempo{0:00-0:02}}

\decir{Buenos días a todos. Bienvenidos a nuestra segunda clase de Geoinformática. Hoy vamos a profundizar en los fundamentos de la Geocomputación, su historia, y las herramientas que usaremos durante todo el semestre.}

\hacer{
\begin{itemize}
    \item Sonreír y hacer contacto visual con varios estudiantes
    \item Verificar que todos pueden ver la presentación
    \item Ajustar luces si es necesario
    \item Confirmar que el audio funciona si usas micrófono
\end{itemize}
}

\subsection{Agenda \tiempo{0:02-0:03}}

\textbf{Slide 2: Agenda}

\decir{Tenemos 80 minutos muy bien aprovechados. Vamos a cubrir 4 grandes temas:
\begin{enumerate}
    \item La historia y evolución de la geocomputación - de dónde venimos
    \item El panorama actual del software - qué herramientas existen
    \item Los ecosistemas de Python y R - nuestras herramientas principales
    \item Ejemplos prácticos - manos a la obra
\end{enumerate}
}

\nota{Señalar cada sección en la diapositiva mientras la mencionas}

\subsection{Repaso Clase Anterior \tiempo{0:03-0:06}}

\textbf{Slide 3: Repaso}

\decir{Antes de comenzar, hagamos un breve repaso. En nuestra primera clase definimos qué es la Geocomputación. ¿Alguien recuerda la diferencia entre datos vectoriales y raster?}

\hacer{
\begin{itemize}
    \item Esperar 30 segundos para respuestas
    \item Si nadie responde, señalar a un estudiante específico
    \item Agradecer la participación: "Exacto, muy bien"
\end{itemize}
}

\decir{Perfecto. Vectorial son puntos, líneas y polígonos con coordenadas precisas - como las calles en Google Maps. Raster son grillas de celdas, como píxeles - piensen en una imagen satelital. Esto es fundamental porque cada tipo requiere diferentes herramientas y técnicas de análisis.}

\textbf{Pregunta interactiva:}

\decir{Levanten la mano: ¿Quién ya instaló Python o R?}

\hacer{Contar visualmente las manos levantadas}

\decir{[Si pocas manos] OK, veo que necesitamos trabajar en eso. Recuerden que es urgente para el laboratorio de hoy.

[Si muchas manos] Excelente, veo que están preparados. Los que no han instalado, aprovechen el break antes del lab.}

\alerta{Si menos del 30\% tiene software instalado, considera dedicar 5 minutos extra en el lab para instalación}

\newpage

\section{Sección 1: Historia y Evolución de la Geocomputación}

\tiempo{Total: 20 minutos}

\subsection{Los Precursores \tiempo{0:06-0:09}}

\textbf{Slide 4: Precursores}

\decir{La historia de la Geocomputación no comienza con computadores. Comienza mucho antes, con personas tratando de entender patrones espaciales. El ejemplo más famoso es de 1854, con un médico llamado John Snow en Londres.}

\hacer{Caminar hacia un lado del salón, usar las manos para narrar}

\textbf{Narración de la historia de John Snow:}

\decir{Londres, 1854. La ciudad está sufriendo una epidemia de cólera devastadora. La teoría médica predominante era que se transmitía por 'miasmas' - básicamente, aire contaminado. Pero John Snow, un médico anestesista, tenía otra hipótesis.}

\hacer{Dibujar en la pizarra un esquema simple:
\begin{itemize}
    \item Rectángulo = manzana de casas
    \item X = casos de cólera
    \item Círculo = bomba de agua
\end{itemize}}

\decir{Snow hizo algo revolucionario para la época: creó un mapa. Marcó cada muerte por cólera con un punto negro. ¿Y qué descubrió? Que todos los casos se agrupaban alrededor de una bomba de agua específica en Broad Street.}

\hacer{Rodear con un círculo las X cerca del pozo}

\decir{Convenció a las autoridades de quitar la manija de la bomba. Los casos cayeron dramáticamente. Este fue el primer análisis espacial documentado de la historia. Sin computadores, sin GPS, solo un mapa y observación cuidadosa, Snow demostró que el espacio importa, que la ubicación puede revelar patrones invisibles de otra forma.}

\textbf{Conexión con la actualidad:}

\decir{¿Les suena familiar? Es exactamente lo que hicimos con COVID-19 en 2020, pero ahora con millones de datos procesados en tiempo real. La pregunta es la misma: ¿dónde están los casos? ¿hay un patrón? La diferencia es la escala y velocidad.}

\subsection{La Era de los SIG \tiempo{0:09-0:13}}

\textbf{Slide 5: Era de los SIG}

\hacer{Señalar el timeline en la diapositiva}

\decir{Miren esta línea de tiempo. 1963 es un año clave: nace CGIS en Canadá, el primer Sistema de Información Geográfica operacional del mundo.}

\textbf{Contexto histórico:}

\decir{¿Por qué Canadá? Piénsenlo: tienen el segundo país más grande del mundo - casi 10 millones de km². Necesitaban inventariar sus recursos naturales: bosques, minerales, agricultura. Imaginen tratar de mapear millones de hectáreas de bosque a mano... imposible.}

\nota{Pausa dramática aquí para efecto}

\decir{CGIS - Canada Geographic Information System - podía hacer algo revolucionario para la época: superponer capas de información. Una capa de tipos de suelo, otra de vegetación, otra de hidrología, y encontrar las intersecciones. "Muéstrame todas las áreas con suelo tipo A, bosque de coníferas, y cerca de un río". Esto que hoy hacemos en segundos con una query, tomaba horas en mainframes del tamaño de esta sala.}

\textbf{Dato curioso para engagement:}

\decir{¿Saben cuánto costó el sistema? 15 millones de dólares canadienses de 1960. Ajustado por inflación, serían como 150 millones de dólares de hoy. Por eso solo gobiernos y grandes empresas petroleras podían tener SIG.}

\textbf{ESRI y la comercialización:}

\decir{En 1969, algo importante pasa en California. Jack Dangermond y su esposa Laura fundan ESRI - Environmental Systems Research Institute. Empezaron literalmente en su garage, haciendo mapas para el departamento de parques. Hoy, ESRI es una empresa de 2 billones de dólares y su producto, ArcGIS, sigue siendo el estándar de la industria. Una historia de emprendimiento tecnológico antes de que existiera Silicon Valley como lo conocemos.}

\subsection{El Nacimiento de la Geocomputación \tiempo{0:13-0:17}}

\textbf{Slide 6: Nacimiento de la Geocomputación}

\hacer{Cambiar tono de voz a más reflexivo/filosófico}

\decir{Ahora, aquí viene lo interesante. En los 90s, había una sensación entre los académicos de que los SIG se habían estancado. Sí, eran buenos para mapear, para almacenar datos, para hacer queries. Pero ¿y para entender procesos? ¿Para predecir? ¿Para simular? Los SIG eran como una biblioteca muy bien organizada, pero sin la capacidad de crear conocimiento nuevo.}

\textbf{Leer la cita de Openshaw:}

\decir{Stan Openshaw, un geógrafo británico brillante y algo excéntrico, lo expresó mejor en la conferencia de Leeds en 1996:}

\hacer{Leer con énfasis}

\decir{"Geocomputation is about using the various different types of geodata and about developing relevant geo-tools within an 'intelligent' IT framework"}

\decir{Noten la palabra clave: 'intelligent'. No solo procesar datos, sino hacerlo inteligentemente.}

\textbf{Explicar la diferencia fundamental:}

\hacer{Escribir en la pizarra dos columnas: SIG vs GEOCOMPUTACIÓN}

\begin{center}
\begin{tabular}{|l|l|}
\hline
\textbf{SIG Tradicional} & \textbf{Geocomputación} \\
\hline
¿Dónde está el hospital más cercano? & ¿Dónde construir el próximo hospital? \\
¿Cuántos incendios hubo? & ¿Dónde será el próximo incendio? \\
¿Cuál es la ruta más corta? & ¿Cómo evolucionará el tráfico? \\
\hline
\end{tabular}
\end{center}

\decir{¿Ven la diferencia? SIG responde preguntas. Geocomputación hace predicciones y optimizaciones.}

\textbf{Técnicas nuevas (mencionar brevemente):}

\decir{La Geocomputación trajo técnicas de otras disciplinas:
\begin{itemize}
    \item Autómatas celulares de física: cada celda evoluciona según sus vecinos
    \item Redes neuronales de IA: el computador aprende patrones espaciales
    \item Algoritmos genéticos de biología: las soluciones 'evolucionan' hacia lo óptimo
    \item Simulación basada en agentes: miles de agentes virtuales interactuando
\end{itemize}
}

\subsection{La Revolución Web GIS \tiempo{0:17-0:21}}

\textbf{Slide 7: Revolución Web GIS}

\hacer{Tono entusiasta, este es un momento "wow"}

\decir{El 8 de febrero de 2005 cambió todo. Google lanza Google Maps. De repente, mi abuela podía hacer análisis espacial sin saberlo. Buscaba la farmacia más cercana, y Google calculaba rutas óptimas considerando tráfico en tiempo real.}

\hacer{Mostrar con gestos de manos el contraste}

\decir{Piensen en el cambio:
\begin{itemize}
    \item ANTES: Software de \$5000, semanas de training, comprar datos por miles de dólares
    \item DESPUÉS: Abrir navegador, gratis, intuitivo, datos incluidos
\end{itemize}
}

\textbf{Impacto en Chile:}

\decir{Les voy a contar algo fascinante sobre Chile. El terremoto del 27F en 2010 fue uno de los primeros desastres naturales mapeados colaborativamente en tiempo real. Voluntarios de todo el mundo, desde sus casas, usaron OpenStreetMap para mapear zonas afectadas usando imágenes satelitales. El gobierno chileno y ONGs usaron esos mapas para coordinar la ayuda. Personas en Alemania o Japón estaban mapeando Constitución y Talcahuano para ayudar a los rescatistas.}

\textbf{Tecnología clave:}

\decir{La magia técnica detrás fue AJAX - Asynchronous JavaScript and XML. Permitió cargar 'tiles' - pequeños cuadrados del mapa - sin recargar toda la página. Parece simple ahora, pero cambió todo. De repente, los mapas web eran fluidos, rápidos, interactivos.}

\subsection{Era Actual: Big Data y AI \tiempo{0:21-0:26}}

\textbf{Slide 8: Era actual}

\hacer{Señalar el gráfico de crecimiento exponencial}

\decir{Miren este crecimiento. Es exponencial, no linear. En 2010 generábamos 0.5 petabytes de datos geoespaciales al año. Para 2025, serán 200 petabytes. ¡400 veces más en 15 años!}

\textbf{Contextualizar el volumen:}

\decir{¿Qué es un petabyte? Les doy perspectiva: si un gigabyte fuera un segundo, un petabyte serían 31 años. O piénsenlo así: toda la música en Spotify cabe en unos 4 petabytes. Estamos generando 50 Spotifys de datos geoespaciales al año.}

\textbf{Fuentes de datos:}

\decir{¿De dónde vienen estos datos?}

\hacer{Enumerar con los dedos}

\decir{
\begin{enumerate}
    \item \textbf{Satélites:} Sentinel-2 fotografía toda la Tierra cada 5 días. 1.5 terabytes diarios.
    \item \textbf{Smartphones:} 5 billones de dispositivos con GPS generando ubicaciones constantemente.
    \item \textbf{IoT:} Santiago tiene más de 500 sensores de calidad del aire transmitiendo cada minuto.
    \item \textbf{Drones:} Un solo vuelo LiDAR de 1 hora puede generar 100 gigabytes.
    \item \textbf{Redes sociales:} Cada foto en Instagram tiene ubicación. Millones por día.
\end{enumerate}
}

\textbf{Ejemplo Chile:}

\decir{La plataforma IDE Chile - Infraestructura de Datos Espaciales - es impresionante. Integra más de 50 servicios: desde mapas geológicos del SERNAGEOMIN hasta zonas de riesgo de tsunami de la ONEMI, catastro de propiedad del SII, todo integrado. Hace 10 años, conseguir estos datos tomaba semanas de papeleo, idas a oficinas, CDs por correo. Hoy, 3 clicks.}

\textbf{Provocar reflexión:}

\decir{Pero aquí viene el desafío: la pregunta ya no es '¿dónde consigo datos?' - están por todas partes. La pregunta es '¿cómo proceso tanta información?' '¿cómo encuentro la señal en el ruido?' Y ahí es donde entran las herramientas que veremos ahora...}

\hacer{Transición suave a la siguiente sección}

\newpage

\section{Sección 2: Software para Análisis Geoespacial}

\tiempo{Total: 20 minutos}

\subsection{Panorama del Software \tiempo{0:26-0:29}}

\textbf{Slide 9: Panorama del software}

\hacer{Señalar el diagrama de dispersión en la diapositiva}

\decir{Este diagrama es fundamental para entender el ecosistema. En el eje X tenemos facilidad de uso - qué tan rápido puedes empezar a usarlo. En el eje Y, capacidad analítica - qué tan complejos pueden ser tus análisis.}

\hacer{Señalar cada software en el diagrama mientras explicas}

\decir{
\begin{itemize}
    \item \textbf{Google Earth} [señalar]: Súper fácil, pero limitado. Tu mamá puede usarlo, pero no puedes hacer análisis serios. Perfecto para exploración inicial.
    \item \textbf{QGIS} [señalar]: El sweet spot. Buen balance entre usabilidad y poder. Por eso lo usaremos como complemento visual.
    \item \textbf{Python/R} [señalar]: Difícil al principio - hay que programar - pero poder ilimitado. Puedes hacer literalmente cualquier cosa.
    \item \textbf{PostGIS} [señalar]: Para los valientes. Máximo poder, máxima complejidad. Es programar + bases de datos + geografía.
\end{itemize}
}

\textbf{Analogía para clarificar:}

\decir{Es como herramientas de un mecánico:
\begin{itemize}
    \item Google Earth = Destornillador básico del cajón de la cocina
    \item QGIS = Caja de herramientas completa del garage
    \item Python/R = Taller profesional con todas las máquinas
    \item PostGIS = Eres el que fabrica las herramientas para otros
\end{itemize}
}

\subsection{Software Desktop GIS \tiempo{0:29-0:33}}

\textbf{Slide 10: Software Desktop}

\hacer{Pregunta para activar participación}

\decir{Levanten la mano: ¿Quién ha escuchado de QGIS? [contar manos] ¿Y de ArcGIS? [contar manos]}

\textbf{Historia de QGIS:}

\decir{Les cuento la historia de QGIS porque es inspiradora. 2002, Gary Sherman, un desarrollador en Montana, USA, estaba frustrado. Necesitaba un visor de datos GIS para Linux y todo costaba miles de dólares. ¿Qué hizo? Lo programó él mismo. Lo liberó gratis en internet. Hoy, 20 años después, QGIS tiene millones de usuarios en todo el mundo. Un tipo, frustrado, cambió la industria.}

\textbf{Ventajas de QGIS para el curso:}

\decir{¿Por qué usaremos QGIS en el curso? Cuatro razones fundamentales:}

\hacer{Enumerar con los dedos}

\decir{
\begin{enumerate}
    \item \textbf{Gratis:} No hay excusa para no practicar en casa. No van a perder acceso cuando se gradúen.
    \item \textbf{Multiplataforma:} Windows, Mac, Linux, da igual. Funciona en todos.
    \item \textbf{Plugins:} Más de 1000. Hay un plugin para TODO. ¿Quieres hacer análisis de redes? Hay plugin. ¿Morfología urbana? Hay plugin.
    \item \textbf{Python integration:} Todo en QGIS se puede automatizar con Python. Es un puente perfecto.
\end{enumerate}
}

\textbf{Sobre ArcGIS (ser honesto):}

\decir{No les voy a mentir. ArcGIS Pro es más pulido, más estable, mejor documentado. ESRI tiene 50 años de experiencia y se nota. Pero - y es un gran pero - cuesta \$700 al año para estudiantes, \$3000 al año para profesionales. Y solo funciona en Windows. Si tienen Mac o Linux, olvídenlo.}

\textbf{Consejo profesional:}

\decir{En mi experiencia consultoría y academia, 95\% de los análisis se pueden hacer igual en QGIS. El 5\% restante son cosas muy específicas de ciertas industrias. Probablemente no lo necesiten hasta ser especialistas senior. Y para entonces, la empresa paga la licencia.}

\subsection{Comparación QGIS vs ArcGIS \tiempo{0:33-0:36}}

\textbf{Slide 11: Tabla comparativa}

\hacer{Repasar la tabla línea por línea}

\decir{Veamos la comparación punto por punto...}

\begin{center}
\begin{tabular}{|l|c|c|}
\hline
\textbf{Característica} & \textbf{QGIS} & \textbf{ArcGIS Pro} \\
\hline
Costo & \textcolor{green}{Gratis} & \textcolor{red}{\$700-3000/año} \\
Sistema Operativo & \textcolor{green}{Win/Mac/Linux} & \textcolor{red}{Solo Windows} \\
Curva aprendizaje & Moderada & Empinada \\
Documentación & Comunitaria & \textcolor{green}{Profesional} \\
Python & \checkmark & \checkmark \\
3D & Básico & \textcolor{green}{Avanzado} \\
Cloud & Limitada & \textcolor{green}{ArcGIS Online} \\
\hline
\end{tabular}
\end{center}

\textbf{Anécdota personal:}

\decir{Les cuento: yo trabajé 5 años con ArcGIS en mi doctorado. Cuando cambié a QGIS por temas de presupuesto, esperaba limitaciones graves, frustraciones constantes. ¿Saben qué encontré? Que QGIS hace algunas cosas MEJOR. Por ejemplo, maneja formatos open source nativamente. ArcGIS a veces se complica con GeoJSON o formatos web modernos. QGIS los come crudos.}

\textbf{Punto importante sobre el cloud:}

\decir{Vean 'Cloud Integration'. ArcGIS Online es impresionante, lo admito. Publicar mapas web es casi automático. Pero cuesta. QGIS + GitHub + Leaflet puede lograr resultados similares gratis, aunque requiere más trabajo manual.}

\textbf{Predicción:}

\decir{Mi predicción personal: en 5 años, la diferencia será mínima. El open source está cerrando la brecha rápidamente. Cada versión de QGIS incorpora features que antes eran exclusivas de ArcGIS.}

\subsection{Plataformas Cloud \tiempo{0:36-0:40}}

\textbf{Slide 12: Plataformas Cloud}

\hacer{Comenzar con impacto}

\decir{Google Earth Engine puede procesar 40 años de imágenes satelitales de todo el planeta en segundos. En mi doctorado, eso hubiera tomado meses y un cluster de servidores. Hoy, 10 líneas de código.}

\textbf{Explicar Google Earth Engine:}

\decir{Earth Engine no es solo almacenamiento de imágenes. Es procesamiento distribuido masivo. Tú escribes 10 líneas de JavaScript o Python, le das 'run', y Google moviliza miles de servidores para tu análisis. Es como tener un supercomputador personal.}

\alerta{Si hay internet estable, considerar hacer demo en vivo de 1 minuto}

\hacer{[OPCIONAL] Abrir code.earthengine.google.com}

\decir{[Si haces demo] Miren, voy a calcular la pérdida de bosque nativo en toda la Región de Los Lagos en los últimos 10 años... [escribir código]... 5 segundos. Ahí está. 10 años de análisis en 5 segundos.}

\textbf{Microsoft Planetary Computer:}

\decir{Microsoft está siendo muy agresivo en este espacio. Planetary Computer es su respuesta a Google Earth Engine. La diferencia: Microsoft se enfoca en datos abiertos. Tienen TODA la constelación Sentinel europea gratis, preprocesada, lista para análisis. Es la estrategia de Microsoft: abraza el open source, agrégale servicios.}

\textbf{Amazon Web Services:}

\decir{AWS es diferente. No te dan herramientas específicas para GIS. Te dan poder computacional crudo. Es como la diferencia entre comprar un auto armado y comprar todas las piezas para construirlo tú mismo. Más trabajo, más flexibilidad, potencialmente más barato a gran escala.}

\textbf{Pregunta reflexiva:}

\decir{¿Cuál creen que es el problema del cloud computing?}

\hacer{Esperar 15 segundos para respuestas}

\decir{Exacto: dependencia y costos ocultos. Google Earth Engine es gratis... para académicos... por ahora. Cuando tu proyecto crece, cuando necesitas más procesamiento, empiezan a cobrar. Y si construiste todo tu workflow ahí, estás atrapado. Vendor lock-in, se llama.}

\newpage

\section{Sección 3: Ecosistema Python y R para Geodatos}

\tiempo{Total: 25 minutos}

\subsection{¿Por qué Programar para GIS? \tiempo{0:46-0:49}}

\textbf{Slide 15: Por qué programar}

\hacer{Pregunta provocadora para empezar}

\decir{Pregunta honesta: ¿Cuántos clicks hicieron en el último trabajo que entregaron? ¿100? ¿500? ¿Perdieron la cuenta?}

\hacer{Esperar algunas respuestas o risas}

\textbf{Ejemplo dramático:}

\decir{Les cuento un caso real de consultoría. Cliente: una empresa de delivery. Necesidad: análisis mensual de tiempos de entrega para 100 comunas de Santiago. 

Método manual:
\begin{itemize}
    \item Cargar datos de cada comuna
    \item Calcular estadísticas
    \item Generar mapa
    \item Exportar reporte
    \item Repetir 100 veces
    \item Tiempo: 2 días completos
\end{itemize}

Con Python:
\begin{itemize}
    \item Un script de 50 líneas
    \item Tiempo: 10 minutos incluyendo el café
\end{itemize}

Después de 3 meses, el script había ahorrado 35 días de trabajo. Ese es un mes y medio de sueldo.}

\hacer{Mostrar el diagrama de comparación en la diapositiva}

\decir{Pero el tiempo no es lo mejor...}

\hacer{Pausa dramática}

\decir{Lo mejor es la REPRODUCIBILIDAD. Mes 4, el cliente dice: "Oops, los datos de enero estaban mal, ¿pueden recalcular?" 

Método manual: [Fingir llorar dramáticamente] "Nooooo, otros 2 días..."

Con script: "Claro, dame 10 minutos." [Click] Listo.}

\textbf{Regla de oro:}

\hacer{Escribir en la pizarra en letras grandes}

\begin{center}
\fbox{\Large Si lo haces más de 3 veces → AUTOMATÍZALO}
\end{center}

\decir{No es pereza. Es eficiencia. Es profesionalismo. Es respeto por tu tiempo y salud mental.}

\subsection{Ecosistema Python para Geoinformática \tiempo{0:49-0:54}}

\textbf{Slide 16: Ecosistema Python}

\hacer{Señalar el diagrama de ecosistema}

\decir{Python es como LEGO. Cada biblioteca es una pieza que encaja perfectamente con las demás. Vean este diagrama: en el centro está Python, y alrededor todas las bibliotecas especializadas.}

\textbf{Recorrer las bibliotecas principales:}

\subsubsection{GeoPandas - El corazón}

\decir{GeoPandas es la estrella. Es pandas + geografía. Si saben pandas - y deberían para cualquier análisis de datos - ya saben 70\% de GeoPandas.}

\hacer{Escribir ejemplo conceptual en la pizarra}

\begin{lstlisting}[language=Python]
# Pandas normal
df[df['poblacion'] > 100000]

# GeoPandas - exactamente igual, pero...
gdf[gdf['poblacion'] > 100000].plot()  # BOOM! Mapa filtrado
\end{lstlisting}

\decir{¿Ven? La misma sintaxis, pero ahora con superpoderes espaciales.}

\subsubsection{Shapely - La geometría pura}

\decir{Shapely maneja geometrías puras. No sabe de proyecciones, no sabe de archivos, solo sabe de formas. Es como la matemática detrás de todo.}

\begin{lstlisting}[language=Python]
from shapely.geometry import Point, Polygon
punto = Point(0, 0)
circulo = punto.buffer(10)  # Buffer de 10 unidades
\end{lstlisting}

\subsubsection{Rasterio - Los píxeles}

\decir{Para imágenes satelitales, datos de elevación, cualquier cosa en formato raster. El nombre es gracioso: 'Raster I/O' = raster-io = rasterio. Programadores y sus juegos de palabras...}

\subsubsection{Folium - Los mapas web}

\decir{De Python a mapa web interactivo en 3 líneas. No exagero, literalmente 3 líneas:}

\begin{lstlisting}[language=Python]
import folium
m = folium.Map(location=[-33.45, -70.65])  # Santiago
m.save('mapa.html')  # Listo, mapa web interactivo
\end{lstlisting}

\textbf{Consejo práctico:}

\decir{No intenten aprender todas las bibliotecas de una vez. Es como intentar aprender todos los instrumentos de una orquesta simultáneamente. Empiecen con GeoPandas + Folium. Con eso pueden hacer 80\% de los análisis típicos. Luego agreguen según necesidad.}

\subsection{Ejemplo Completo en Python \tiempo{0:54-0:58}}

\textbf{Slide 17: Ejemplo Python}

\hacer{Abrir terminal/Jupyter y ejecutar código en vivo}

\nota{Si falla internet o el código, tener screenshots listos como backup}

\decir{Voy a ejecutar este código en vivo. Si falla, culpen al WiFi de la universidad, no a Python.}

\begin{lstlisting}[language=Python, title=Ejecutar línea por línea]
import geopandas as gpd
import matplotlib.pyplot as plt
from shapely.geometry import Point

# Explicar mientras escribes
print("Importando bibliotecas... como sacar herramientas del cajón")

# Leer datos
comunas = gpd.read_file('santiago_comunas.shp')
print(f"Tenemos {len(comunas)} comunas")
print(f"CRS: {comunas.crs}")  # Mostrar sistema de coordenadas

# Ver primeras filas
comunas.head()
\end{lstlisting}

\decir{Miren, con una línea leímos un shapefile complejo. GeoPandas automáticamente reconoció la geometría, los atributos, el sistema de coordenadas, todo.}

\begin{lstlisting}[language=Python, title=Continuar con el análisis]
# Crear hospitales ficticios
hospitales = gpd.GeoDataFrame(
    {'nombre': ['Hospital Salvador', 'Hospital UC'],
     'geometry': [Point(-70.65, -33.45), 
                  Point(-70.60, -33.42)]},
    crs='EPSG:4326')  # WGS84, el sistema de GPS

print("Creamos 2 hospitales desde cero")

# Buffer de 2km (0.02 grados aproximadamente)
areas_servicio = hospitales.buffer(0.02)
print("Areas de servicio creadas")

# Análisis espacial: ¿qué comunas tienen cobertura?
comunas_servidas = comunas[comunas.intersects(
    areas_servicio.unary_union)]

print(f"Comunas con cobertura: {len(comunas_servidas)}")
print(f"Comunas: {list(comunas_servidas['nombre'])}")
\end{lstlisting}

\hacer{Si hay tiempo, generar visualización}

\begin{lstlisting}[language=Python]
# Visualizar resultado
fig, ax = plt.subplots(figsize=(10, 10))
comunas.plot(ax=ax, color='lightgray', edgecolor='black')
comunas_servidas.plot(ax=ax, color='green', alpha=0.5)
hospitales.plot(ax=ax, color='red', markersize=100, marker='+')
plt.title('Cobertura Hospitalaria en Santiago')
plt.show()
\end{lstlisting}

\decir{¡Miren! En 15 líneas de código hicimos un análisis de cobertura hospitalaria que en software tradicional tomaría 50 clicks.}

\subsection{Ecosistema R para Geoinformática \tiempo{0:58-1:03}}

\textbf{Slide 18: Ecosistema R}

\hacer{Transición comparativa}

\decir{Ahora R. R es diferente. Fue creado por estadísticos, para estadísticos. La filosofía es distinta: R trata todo como datos para analizar. Python trata todo como objetos para procesar. Sutil pero importante diferencia.}

\textbf{Paquete sf - La revolución:}

\decir{El paquete 'sf' - Simple Features - revolucionó R espacial. Antes, con 'sp', era un desastre. Código ilegible, estructuras de datos confusas. 'sf' lo cambió todo. Ahora los datos espaciales son data frames normales con una columna especial 'geometry'. Genial para análisis estadístico.}

\hacer{Mostrar estructura conceptual en pizarra}

\begin{center}
\begin{tabular}{|l|l|l|l|}
\hline
id & nombre & población & geometry \\
\hline
1 & Santiago & 400000 & POLYGON(...) \\
2 & Providencia & 150000 & POLYGON(...) \\
3 & Las Condes & 300000 & POLYGON(...) \\
\hline
\end{tabular}
\end{center}

\decir{¿Ven? Es una tabla normal, pero la última columna tiene las geometrías. Pueden hacer todo lo que hacen con datos normales, más operaciones espaciales.}

\textbf{Terra vs Raster:}

\decir{Para datos raster, 'terra' reemplazó a 'raster'. Es 10 veces más rápido, no es exageración, literalmente 10x. Si encuentran tutoriales viejos con 'raster', búsquenlos con 'terra'. Mismo concepto, mejor implementación.}

\textbf{Tmap - La joya de la corona:}

\decir{'tmap' es increíble para mapas temáticos. Hace mapas nivel publicación académica con código mínimo. Si van a publicar papers con mapas, tmap es su mejor amigo.}

\subsection{Ejemplo en R \tiempo{1:03-1:07}}

\textbf{Slide 19: Ejemplo R}

\hacer{Cambiar a RStudio o terminal R}

\begin{lstlisting}[language=R, title=Código R equivalente]
library(sf)
library(tmap)
library(dplyr)

# Comparación con Python
print("Noten: R usa <- para asignación, Python usa =")

# Leer datos
comunas <- st_read("santiago_comunas.shp")
print(paste("Tenemos", nrow(comunas), "comunas"))

# El pipe operator %>% es mágico
comunas %>%
  filter(poblacion > 100000) %>%
  select(nombre, poblacion) %>%
  head()
\end{lstlisting}

\decir{El operador \%>\% (pipe) es mágico. Lee el código como una receta de cocina: toma comunas, LUEGO filtra las que tienen más de 100,000 habitantes, LUEGO selecciona solo nombre y población. Es muy intuitivo una vez que te acostumbras.}

\begin{lstlisting}[language=R, title=Crear el análisis]
# Crear hospitales
hospitales <- data.frame(
  nombre = c("Hospital Salvador", "Hospital UC"),
  lon = c(-70.65, -70.60),
  lat = c(-33.45, -33.42)
) %>%
  st_as_sf(coords = c("lon", "lat"), crs = 4326)

# Buffer de 2km (transformar a métrico primero)
hospitales_utm <- st_transform(hospitales, 32719)  # UTM 19S
areas_servicio <- st_buffer(hospitales_utm, 2000)  # 2000 metros

# Análisis
comunas_utm <- st_transform(comunas, 32719)
comunas_servidas <- comunas_utm[areas_servicio, ]

print(paste("Comunas con cobertura:", nrow(comunas_servidas)))
\end{lstlisting}

\textbf{Visualización con tmap:}

\begin{lstlisting}[language=R]
# Mapa estático profesional
tm_shape(comunas) +
  tm_polygons("poblacion", 
              palette = "Blues",
              title = "Población") +
tm_shape(comunas_servidas) +
  tm_borders(col = "green", lwd = 3) +
tm_shape(hospitales) +
  tm_dots(col = "red", size = 0.5) +
tm_layout(title = "Cobertura Hospitalaria Santiago",
          legend.outside = TRUE)
\end{lstlisting}

\decir{Miren la calidad del mapa. Listo para publicación en revista académica. Eso es lo que R hace mejor: visualización estadística de calidad profesional.}

\subsection{Python vs R - La Verdad \tiempo{1:07-1:11}}

\textbf{Slide 20: Comparación Python vs R}

\decir{La pregunta del millón: ¿Python o R? Mi respuesta honesta: DEPENDE. Y no es diplomacia, es realidad.}

\hacer{Revisar la tabla comparativa}

\textbf{Cuándo usar Python:}

\decir{Python cuando:
\begin{itemize}
    \item Necesitas integrar con aplicaciones web (Django, Flask)
    \item Vas a hacer Deep Learning (TensorFlow, PyTorch dominan)
    \item Ya sabes Python de otros proyectos
    \item Necesitas pipelines de producción robustos
    \item Trabajas con ingenieros de software
\end{itemize}}

\textbf{Cuándo usar R:}

\decir{R cuando:
\begin{itemize}
    \item El análisis estadístico espacial es complejo (geoestadística, modelos bayesianos)
    \item Necesitas visualizaciones para publicación académica
    \item Trabajas en investigación
    \item Tu equipo son estadísticos o científicos
    \item Necesitas reportes reproducibles (RMarkdown es excelente)
\end{itemize}}

\textbf{Mi recomendación para el curso:}

\decir{En este curso usaremos 60\% Python, 30\% R, 10\% QGIS. ¿Por qué más Python? Porque como ingenieros informáticos, Python les será más útil en el mundo laboral. Pero aprenderán suficiente R para no tenerle miedo.}

\textbf{El secreto profesional:}

\decir{Les voy a contar un secreto: los profesionales usamos ambos. Yo uso Python para producción y pipelines automatizados. Uso R para exploración y análisis estadístico. Son complementarios, no competidores. Es como tener un destornillador Y un martillo. ¿Para qué limitarse a una herramienta?}

\newpage

\section{Sección 4: Primeros Ejemplos Prácticos}

\tiempo{Total: 15 minutos}

\subsection{Flujo de Trabajo Típico \tiempo{1:11-1:14}}

\textbf{Slide 21: Flujo de trabajo}

\hacer{Señalar el diagrama de flujo paso a paso}

\decir{Este diagrama es su hoja de ruta para CADA proyecto que harán. Memorícenlo.}

\textbf{Explicar cada paso:}

\subsubsection{Paso 1: Obtención de Datos}

\decir{Todo empieza con datos. Fuentes típicas:
\begin{itemize}
    \item APIs REST - ejemplo: datos COVID del MINSAL tienen API
    \item Web scraping - cuando no hay API
    \item Archivos tradicionales - shapefiles, CSVs
    \item Sensores IoT - tiempo real
\end{itemize}}

\subsubsection{Paso 2: Limpieza y Preparación}

\decir{Regla universal: 80\% del tiempo se va aquí. OCHENTA POR CIENTO. Los datos vienen sucios:
\begin{itemize}
    \item Fechas en formato gringo (MM/DD/YYYY)
    \item Coordenadas en sistema incorrecto
    \item Valores nulos, duplicados, inconsistentes
    \item Comunas escritas de 10 formas diferentes: "Stgo", "Santiago", "SANTIAGO"
\end{itemize}

Es tedioso pero crítico. Datos sucios = análisis basura.}

\subsubsection{Paso 3: Análisis Espacial}

\decir{La parte divertida. Aquí aplican geocomputación: buffers, intersecciones, clustering, interpolación. Aquí es donde el curso brilla.}

\subsubsection{Paso 4: Visualización}

\decir{Un mapa vale más que mil tablas. En serio, he visto ejecutivos ignorar 50 páginas de reportes y entender todo con un mapa de calor.}

\subsubsection{Paso 5: Modelado y Predicción}

\decir{Opcional pero poderoso. Machine learning espacial, optimización, simulación. Ejemplo: predecir dónde será el próximo foco de dengue.}

\subsubsection{Paso 6: Compartir y Publicar}

\decir{Código sin documentación es inútil. GitHub para código, web app para usuarios finales, reporte PDF para jefes old school.}

\subsection{Caso Práctico: Análisis COVID-19 \tiempo{1:14-1:19}}

\textbf{Slide 22: Caso COVID-19}

\hacer{Tono serio, este es un ejemplo real e importante}

\decir{Marzo 2020. Primer caso confirmado de COVID en Chile. El país entra en pánico. Necesitamos entender cómo se propagará. Este es un caso real donde la geocomputación salvó vidas.}

\textbf{Contexto del problema:}

\decir{El MINSAL publicaba datos diarios de casos por comuna. Problema: solo números, sin contexto espacial. 100 casos en Las Condes no es lo mismo que 100 casos en Cerro Navia. Diferente densidad, diferente demografía, diferente acceso a salud.}

\textbf{Solución paso a paso:}

\hacer{Escribir en la pizarra mientras explicas}

\decir{
\begin{enumerate}
    \item \textbf{Join espacial:} Unir datos COVID con shapefile de comunas
    \begin{lstlisting}[language=Python]
comunas_covid = comunas.merge(datos_covid, on='codigo_comuna')
    \end{lstlisting}
    
    \item \textbf{Normalización:} Casos por 100,000 habitantes
    \begin{lstlisting}[language=Python]
comunas_covid['tasa'] = (comunas_covid['casos'] / 
                         comunas_covid['poblacion']) * 100000
    \end{lstlisting}
    
    \item \textbf{Autocorrelación espacial:} ¿Hay clusters?
    \begin{lstlisting}[language=Python]
from esda.moran import Moran
moran = Moran(comunas_covid['tasa'], weights)
print(f"Moran's I: {moran.I}")  # 0.7 = alta correlación
    \end{lstlisting}
    
    \item \textbf{Identificar clusters:} Las Condes, Vitacura, Providencia
    
    \item \textbf{Predicción:} Modelo SIR espacial predijo expansión a Ñuñoa
\end{enumerate}
}

\textbf{Resultado real:}

\decir{El modelo predijo correctamente 8 de 10 comunas donde aparecerían brotes las siguientes 2 semanas. Esta información fue crucial para asignar recursos: ventiladores, personal médico, tests. No exagero: la geocomputación puede salvar vidas. Literalmente.}

\textbf{Lección importante:}

\decir{Este no es un ejercicio académico abstracto. Lo que aprenderán tiene aplicaciones reales, importantes, urgentes. Pueden hacer diferencia.}

\subsection{Demo: Configuración del Ambiente \tiempo{1:19-1:22}}

\textbf{Slide 23: Configuración}

\hacer{Abrir terminal, mostrar proceso real}

\decir{Ahora les muestro exactamente lo que haremos en el laboratorio. Presten atención porque esto puede dar problemas.}

\textbf{Verificar Python:}

\begin{lstlisting}[language=bash]
python --version
# Debe mostrar: Python 3.8 o superior
# Si no: houston, tenemos un problema
\end{lstlisting}

\textbf{Crear ambiente virtual:}

\decir{SIEMPRE usen ambientes virtuales. SIEMPRE. Es como tener una caja aislada para cada proyecto. No contamina tu Python del sistema.}

\begin{lstlisting}[language=bash]
# Crear ambiente
python -m venv geo_env

# Activar (Linux/Mac)
source geo_env/bin/activate

# Activar (Windows)
geo_env\Scripts\activate

# Deberían ver (geo_env) en el prompt
\end{lstlisting}

\textbf{Instalar bibliotecas:}

\begin{lstlisting}[language=bash]
pip install geopandas folium matplotlib jupyter
# Esto tomará 2-5 minutos, sean pacientes
\end{lstlisting}

\alerta{Windows suele dar problemas con Fiona (dependencia de GeoPandas). Solución: usar Anaconda en lugar de pip:
\texttt{conda install -c conda-forge geopandas}}

\subsection{Primer Script Completo \tiempo{1:22-1:26}}

\textbf{Slide 24: Script completo}

\decir{Les voy a compartir este script. Es su plantilla base para todo el semestre. Guárdenlo como oro.}

\hacer{Ejecutar el script completo, mostrando resultados intermedios}

\begin{lstlisting}[language=Python, title=mi primer mapa.py]
# archivo: mi_primer_mapa.py
import geopandas as gpd
import matplotlib.pyplot as plt

# 1. Descargar datos de ejemplo
print("Cargando datos del mundo...")
world = gpd.read_file(gpd.datasets.get_path('naturalearth_lowres'))
print(f"Países cargados: {len(world)}")

# 2. Filtrar América del Sur
sudamerica = world[world['continent'] == 'South America']
print(f"Países sudamericanos: {len(sudamerica)}")
print(list(sudamerica['name']))

# 3. Calcular centroides (punto central de cada país)
sudamerica['centroid'] = sudamerica.geometry.centroid

# 4. Crear figura con 2 mapas
fig, (ax1, ax2) = plt.subplots(1, 2, figsize=(15, 8))

# 5. Mapa 1: Población
sudamerica.plot(column='pop_est', 
                ax=ax1, 
                legend=True,
                legend_kwds={'label': 'Población',
                            'orientation': 'horizontal'},
                cmap='YlOrRd',  # Yellow-Orange-Red
                edgecolor='black',
                linewidth=0.5)
ax1.set_title('Población de Sudamérica', fontsize=16)
ax1.set_axis_off()  # Quitar ejes

# 6. Mapa 2: PIB
sudamerica.plot(column='gdp_md_est',
                ax=ax2,
                legend=True,
                legend_kwds={'label': 'PIB (millones USD)',
                            'orientation': 'horizontal'}, 
                cmap='Greens',
                edgecolor='black',
                linewidth=0.5)
ax2.set_title('PIB de países sudamericanos', fontsize=16)
ax2.set_axis_off()

# 7. Ajustar y guardar
plt.tight_layout()
plt.savefig('sudamerica_analisis.png', dpi=300, bbox_inches='tight')
print("Mapa guardado como 'sudamerica_analisis.png'")
plt.show()
\end{lstlisting}

\textbf{Explicar elementos clave:}

\decir{
Noten estos detalles importantes:
\begin{itemize}
    \item \texttt{get\_path('naturalearth\_lowres')} - GeoPandas incluye datos de ejemplo
    \item \texttt{figsize=(15,8)} - tamaño en pulgadas, ajusten según necesidad
    \item \texttt{cmap='YlOrRd'} - ColorMap secuencial, bueno para datos ordenados
    \item \texttt{tight\_layout()} - evita que se corten etiquetas
    \item \texttt{dpi=300} - calidad de publicación
\end{itemize}
}

\textbf{Mostrar resultado:}

\hacer{Mostrar la imagen generada}

\decir{¡Miren! Dos mapas temáticos profesionales en 40 líneas de código. ¿Qué patrones ven? ¿Por qué Brasil tiene tanto PIB total pero no necesariamente per cápita? ¿Por qué Chile aparece con PIB medio pero sabemos que tiene el PIB per cápita más alto?}

\newpage

\section{Preparación para el Laboratorio y Cierre}

\tiempo{Total: 10 minutos}

\subsection{Laboratorio 1: Qué Haremos \tiempo{1:26-1:29}}

\textbf{Slide 25: Laboratorio 1}

\decir{En 10 minutos comenzamos el laboratorio. Va a ser hands-on, práctico, posiblemente frustrante. Las instalaciones siempre dan problemas. Es normal.}

\textbf{Expectativas del laboratorio:}

\decir{
Objetivos del laboratorio:
\begin{enumerate}
    \item \textbf{Instalación completa:} Todos salen con software funcionando
    \item \textbf{Verificación:} Ejecutar script de prueba sin errores
    \item \textbf{Primer mapa:} Cada uno genera un mapa de Chile
    \item \textbf{Configurar Git:} Para entregar tareas
\end{enumerate}
}

\textbf{Problemas comunes y soluciones:}

\decir{
Anticipo estos problemas:
\begin{itemize}
    \item \textbf{Windows:} Van a necesitar permisos de administrador. Si no los tienen, hablen conmigo.
    \item \textbf{Mac:} Puede pedir instalar Xcode Command Line Tools. Son 2GB, demora.
    \item \textbf{Linux:} Generalmente sin problemas. Linux users, ayuden a sus compañeros.
\end{itemize}
}

\textbf{Estrategia de trabajo:}

\decir{Trabajen en parejas. Si uno tiene problemas instalando, el otro puede avanzar y compartir pantalla. No se frustren si algo no funciona inmediato. Es parte del proceso.}

\textbf{Meta clara:}

\decir{Todos salen del laboratorio con un mapa de Chile funcionando. No acepto menos. Si es necesario, me quedo después de clase.}

\subsection{Recursos para Profundizar \tiempo{1:29-1:32}}

\textbf{Slide 26: Recursos}

\decir{Les voy a compartir recursos de calidad. No pierdan tiempo con tutoriales random de YouTube.}

\textbf{Libros esenciales (gratuitos):}

\decir{
\begin{itemize}
    \item \textbf{Geocomputation with Python} - py.geocompx.org - LEAN ESTE
    \item \textbf{Geocomputation with R} - r.geocompx.org - Para los valientes
    \item \textbf{Geographic Data Science} - geographicdata.science/book - Más avanzado
\end{itemize}

Todos gratis, todos online, todos excelentes. El primero es lectura obligatoria, capítulo 1 para la próxima semana.}

\textbf{Comunidades activas:}

\decir{
Cuando se atoren - y se van a atorar:
\begin{itemize}
    \item \textbf{Stack Overflow} tag 'gis' - 50,000+ preguntas respondidas
    \item \textbf{GIS Stack Exchange} - más específico
    \item \textbf{r/gis en Reddit} - comunidad amigable
    \item \textbf{Twitter \#gischat} - profesionales compartiendo
\end{itemize}
}

\textbf{Canales YouTube recomendados:}

\decir{
\begin{itemize}
    \item \textbf{GeoDelta Labs} - Excelente contenido en español
    \item \textbf{Matt Forrest} - Visualización de datos
    \item \textbf{Qiusheng Wu} - Google Earth Engine
\end{itemize}
}

\textbf{Consejo de aprendizaje:}

\decir{No intenten aprender todo de una vez. Es como ir al gimnasio: mejor 30 minutos diarios que 5 horas el domingo. Un concepto nuevo por día. En un mes, serán 30 conceptos. Consistencia gana.}

\subsection{Ideas para el Proyecto \tiempo{1:32-1:34}}

\textbf{Slide 27: Proyecto - Ideas}

\decir{Semana 4 formarán grupos. Empiecen a pensar desde ya. El mejor proyecto es el que les apasiona o resuelve un problema real que conocen.}

\textbf{Ideas comerciales:}

\decir{
Piensen en problemas reales de empresas:
\begin{itemize}
    \item Optimización de rutas para Rappi/Uber Eats
    \item Análisis de localización para nueva sucursal
    \item Segmentación de clientes por zona
    \item Predicción de demanda espacial
\end{itemize}
}

\textbf{Ideas científicas:}

\decir{
¿Qué les preocupa de Chile?
\begin{itemize}
    \item Sequía y disponibilidad de agua
    \item Contaminación del aire en Santiago
    \item Segregación urbana y acceso a servicios
    \item Riesgo de incendios forestales
    \item Cambio climático y agricultura
\end{itemize}
}

\hacer{Actividad rápida de 1 minuto}

\decir{Giren a su compañero de al lado. 30 segundos cada uno: ¿qué problema espacial les gustaría resolver? GO!}

\hacer{Cronometrar 1 minuto, luego recuperar atención}

\decir{Excelente. Ya tienen ideas fluyendo. Anótenlas. En semana 4 las necesitarán.}

\subsection{Resumen y Conceptos Clave \tiempo{1:34-1:36}}

\textbf{Slide 28: Resumen}

\hacer{Ritmo rápido, son recordatorios}

\decir{
Repaso rápido de lo cubierto:

\textbf{Historia:}
\begin{itemize}
    \item 1854: John Snow inventa el análisis espacial
    \item 1960s: Primeros SIG digitales
    \item 1996: Nace término Geocomputación
    \item 2005: Google Maps democratiza mapas
    \item Hoy: Big Data + AI + Cloud
\end{itemize}

\textbf{Software:}
\begin{itemize}
    \item Desktop: QGIS (gratis) vs ArcGIS (caro)
    \item Cloud: Google Earth Engine es poder puro
    \item Bases de datos: PostGIS para los valientes
    \item CLI: GDAL/OGR salva vidas
\end{itemize}

\textbf{Programación:}
\begin{itemize}
    \item Python: geopandas + folium = 80\% necesidades
    \item R: sf + tmap = publicación académica
    \item Ambos son válidos, ambos son útiles
    \item Reproducibilidad > clicks manuales
\end{itemize}
}

\subsection{Tareas y Preparación \tiempo{1:36-1:37}}

\textbf{Slide 29: Tarea}

\hacer{Tono serio, esto es importante}

\decir{
Tres tareas, no negociables:

\begin{enumerate}
    \item \textbf{URGENTE - Instalación:} Si no tienen Python y R instalados, háganlo AHORA. No durante el lab. El lab es para verificar y practicar, no para instalar desde cero.
    
    \item \textbf{Lectura:} Capítulo 1 de Geocomputation with Python. Son 30 páginas, 30 minutos máximo. py.geocompx.org/01-introduction
    
    \item \textbf{Reflexión:} Piensen en un problema espacial que les interese. Anótenlo. Investiguen si hay datos disponibles.
\end{enumerate}
}

\subsection{Cierre y Preguntas \tiempo{1:37-1:40}}

\textbf{Slide 30: Cierre}

\decir{¿Preguntas? ¿Dudas? ¿Comentarios? ¿Quejas? ¿Sugerencias?}

\hacer{Responder 2-3 preguntas, máximo 2 minutos}

\textbf{Motivación final:}

\decir{En 4 meses, estarán haciendo análisis que hoy parecen magia. Créanme. He visto estudiantes pasar de "¿qué es un shapefile?" a ganar hackatones de datos geoespaciales. La clave es práctica constante. Confíen en el proceso.}

\textbf{Logística:}

\decir{10 minutos de break. Vayan al baño, tomen agua, estiren las piernas. Nos vemos aquí mismo para el laboratorio. Traigan sus computadores y paciencia.}

\textbf{Último recordatorio:}

\decir{Si no han descargado NADA, usen estos 10 minutos para al menos bajar los instaladores:
\begin{itemize}
    \item Anaconda: anaconda.com/download
    \item R: r-project.org
    \item RStudio: posit.co
\end{itemize}

Los veo en 10 minutos. ¡A descansar!}

\hacer{
\begin{itemize}
    \item Apagar proyector
    \item Preparar material del laboratorio
    \item Estar disponible para preguntas individuales durante el break
\end{itemize}
}

\newpage

\section{Anexos}

\subsection{Preguntas Frecuentes y Respuestas}

\begin{enumerate}
\item \textbf{P: ¿Es mejor aprender Python o R primero?}\\
\textbf{R:} Si vienen de programación, Python será más natural - sintaxis familiar, paradigma conocido. Si vienen de estadística o no programan, R puede ser más intuitivo para análisis. Mi consejo: empiecen con Python porque es más versátil fuera del análisis de datos. Pueden hacer web, automatización, IoT, todo.

\item \textbf{P: ¿Necesito una GPU para el curso?}\\
\textbf{R:} No. Todo lo que haremos funciona en cualquier laptop de los últimos 5 años. 4GB RAM mínimo, 8GB ideal. GPU ayuda para deep learning con imágenes satelitales, pero eso es opcional/avanzado. Si les interesa, hay Google Colab gratis con GPU.

\item \textbf{P: ¿Por qué no usamos ArcGIS si es el estándar de la industria?}\\
\textbf{R:} Tres razones: 
\begin{itemize}
    \item Costo: No todos pueden pagarlo después
    \item Filosofía: Quiero que entiendan QUÉ hacen, no solo clicks
    \item Empleabilidad: El que sabe programar GIS puede aprender ArcGIS en una semana. Al revés, no.
\end{itemize}

\item \textbf{P: ¿Los datos de Chile son fáciles de conseguir?}\\
\textbf{R:} Sí y no. IDE Chile tiene mucho, pero disperso y a veces desactualizado. El INE tiene shapefiles buenos. MINVU tiene datos urbanos. Les compartiré un repositorio con datos limpios y listos. Pro tip: el Centro UC de Cambio Global tiene datos climáticos excelentes.

\item \textbf{P: ¿Cuánto Python/R necesito saber antes?}\\
\textbf{R:} Lo básico: variables, loops, funciones, listas/vectores. Si pueden hacer un FizzBuzz, están listos. Si no saben qué es FizzBuzz, Khan Academy tiene un curso Python gratuito excelente. 10 horas y están listos.

\item \textbf{P: ¿El proyecto puede ser sobre mi comuna/región?}\\
\textbf{R:} ¡Absolutamente! De hecho, lo incentivo. Conocen el contexto, tienen interés personal, pueden conseguir datos locales únicos. Proyectos de Valparaíso sobre incendios, de Antofagasta sobre minería, de Chiloé sobre acuicultura. Todos bienvenidos.

\item \textbf{P: ¿Vamos a trabajar con imágenes satelitales?}\\
\textbf{R:} Sí, en la unidad 2. Usaremos principalmente Sentinel-2 (10m resolución, gratis, cada 5 días) y Landsat (30m, histórico desde 1972). Si hay interés, podemos ver Planet (3m resolución, pago) o datos de drones.

\item \textbf{P: ¿Qué tan difícil es PostGIS?}\\
\textbf{R:} Requiere saber SQL primero. Es opcional para el curso. Si les interesa, puedo dar una clase extra o un taller sabatino. Es MUY poderoso para proyectos grandes o si quieren trabajar en startups tech.

\item \textbf{P: ¿Sirve para conseguir trabajo?}\\
\textbf{R:} Absolutamente. Demanda altísima, poca oferta. Un junior con Python + GIS parte en 1.2-1.5M CLP. Con 2 años experiencia, 2-3M. Startups de delivery, proptech, consultoras ambientales, gobierno digital, todos buscan este perfil.

\item \textbf{P: ¿Podemos usar ChatGPT/Copilot para el código?}\\
\textbf{R:} Sí, PERO... úsenlo para aprender, no para copiar ciegamente. Si no entienden el código que genera, no sirve. En el proyecto, deben poder explicar cada línea. ChatGPT es excelente para debugging y sintaxis, malo para lógica espacial específica.
\end{enumerate}

\subsection{Actividades Interactivas Detalladas}

\begin{tcolorbox}[colback=blue!5,colframe=blue!50!black,title=Actividad 1: Encuentra el Patrón (Slide 4)]
\textbf{Tiempo:} 2 minutos\\
\textbf{Objetivo:} Demostrar análisis espacial intuitivo\\
\textbf{Instrucciones:}
\begin{enumerate}
    \item Mostrar mapa con puntos
    \item Preguntar: "¿Qué patrón ven?"
    \item Esperar respuestas (30 seg)
    \item Preguntar: "¿Dónde pondrían el próximo punto?"
    \item Revelar que es el mapa de Snow
\end{enumerate}
\textbf{Resultado esperado:} Estudiantes identifican clustering alrededor del pozo
\end{tcolorbox}

\begin{tcolorbox}[colback=green!5,colframe=green!50!black,title=Actividad 2: Software Speed Dating (Slide 9)]
\textbf{Tiempo:} 3 minutos\\
\textbf{Objetivo:} Reflexionar sobre trade-offs de software\\
\textbf{Instrucciones:}
\begin{enumerate}
    \item Formar parejas
    \item Asignar: uno defiende QGIS, otro ArcGIS
    \item 1 minuto cada uno para argumentar
    \item Cambiar roles
    \item Compartir conclusión con la clase (30 seg)
\end{enumerate}
\textbf{Resultado esperado:} Comprenden que no hay "mejor" absoluto
\end{tcolorbox}

\begin{tcolorbox}[colback=yellow!5,colframe=orange,title=Actividad 3: Pseudo-código Espacial (Slide 21)]
\textbf{Tiempo:} 3 minutos\\
\textbf{Objetivo:} Pensar algorítmicamente sobre problemas espaciales\\
\textbf{Instrucciones:}
\begin{enumerate}
    \item Problema: "Encontrar el café más cercano"
    \item Escribir pasos en español (no código)
    \item Compartir con compañero
    \item Voluntario comparte con clase
\end{enumerate}
\textbf{Ejemplo esperado:}
\begin{enumerate}
    \item Obtener mi ubicación actual
    \item Buscar todos los cafés en radio de 1km
    \item Calcular distancia a cada uno
    \item Ordenar por distancia
    \item Retornar el primero
\end{enumerate}
\end{tcolorbox}

\subsection{Troubleshooting Técnico}

\begin{mdframed}[backgroundcolor=red!5]
\textbf{Problema:} "ImportError: No module named geopandas"

\textbf{Solución Windows:}
\begin{lstlisting}[language=bash]
conda install -c conda-forge geopandas
\end{lstlisting}

\textbf{Solución Mac/Linux:}
\begin{lstlisting}[language=bash]
pip install --upgrade pip
pip install wheel
pip install geopandas
\end{lstlisting}

\textbf{Si persiste:} Usar Google Colab como backup
\end{mdframed}

\begin{mdframed}[backgroundcolor=red!5]
\textbf{Problema:} Proyector no muestra colores correctamente

\textbf{Solución:}
\begin{itemize}
    \item Tener versión alto contraste del PDF
    \item Usar pizarra para diagramas importantes
    \item Compartir pantalla por Teams/Zoom a los estudiantes
\end{itemize}
\end{mdframed}

\begin{mdframed}[backgroundcolor=red!5]
\textbf{Problema:} Internet caído durante demo

\textbf{Solución:}
\begin{itemize}
    \item Tener demos pre-grabados como GIF
    \item Screenshots de resultados esperados
    \item Hotspot móvil como backup
    \item Código completo en USB
\end{itemize}
\end{mdframed}

\subsection{Checklist Post-Clase}

\begin{itemize}[label=$\square$]
    \item Subir slides actualizados al repositorio
    \item Compartir links de recursos en foro/Teams
    \item Preparar datos para el laboratorio
    \item Verificar que lab tenga computadores funcionando
    \item Responder emails pendientes (máximo 24 horas)
    \item Anotar qué funcionó bien y qué mejorar
    \item Preparar material extra para estudiantes avanzados
    \item Actualizar calendario con fechas importantes
    \item Revisar asistencia
    \item Backup de demos y códigos utilizados
\end{itemize}

\subsection{Métricas de Éxito}

\begin{center}
\begin{tabular}{|l|c|}
\hline
\textbf{Indicador} & \textbf{Meta} \\
\hline
Estudiantes con software instalado al final del lab & >90\% \\
Preguntas durante la clase & >5 \\
Participación en actividades & >80\% \\
Comprensión (mini-quiz al final) & >70\% \\
Satisfacción (encuesta anónima) & >4/5 \\
\hline
\end{tabular}
\end{center}

\vspace{1cm}

\begin{center}
\large{\textbf{¡Éxito en tu clase!}}
\end{center}

\end{document}